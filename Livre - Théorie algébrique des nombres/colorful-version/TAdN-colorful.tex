% LTeX: language=fr
\documentclass[11pt, %useosf,
  title in boldface,
  theorem in new line,
  theorem numbering = section,
  number theorems separately,
  simple name,
%   paperstyle = lightyellow,
]{beaulivre}
\usepackage[quaternion]{XL-kit}
\usepackage{tabularx,arydshln,hanging}
\usepackage[
    lang     = {french},
    type     = {CC},
    modifier = {by-sa},
    version  = {4.0},
    hyperxmp = false,
]{doclicense}

\setcounter{tocdepth}{1}

\UseLanguage { French }

\SwitchTheoremNumbering { example , remark }
\counterwithin{equation}{section}


\begin{document}


\frontmatter

\TitlePage [ style = academic ]
  {
    , title     = { \fontspec{SourceSansPro}[WordSpace={1.25,1,1}]\addfontfeature{LetterSpace=2.0} Théorie algébrique des nombres }
    , info      = {
                    Ce texte, intitulé \textquote{Théorie algébrique des nombres}, est dû à \textsc{Pierre Samuel}.
                  }
    , license   = { \doclicenseThis }
    , license page name = { Info }
  }

\tableofcontents

\chapter{Introduction}

    La Théorie des Nombres, ou Arithmétique, est souvent qualifiée de \textquote{Reine des Mathématiques}. La simplicité de son objet (les nombres entiers et leurs généralisations), l'élégance et la diversité de ses méthodes, les nombreux problèmes non résolus qu'elle contient, exercent en effet un incontestable attrait sur de nombreux mathématiciens, qu'ils soient des débutants, des arithméticiens professionnels, ou des spécialistes d'autres branches. On ne s'étonnera donc pas trop de voir que l'auteur de ce livre est un géomètre-algébriste, qui n'a aucune publication originale dans le domaine de l'Arithmétique proprement~dite.

    Ce livre ne se distinguera donc, ni par sa profondeur, ni par son étendue. Bien plus, il ne contient qu'un seul des point de vue par lesquels on peut aborder la Théorie des Nombres, à savoir le point de vue algébrique. À part un résultat élémentaire de Minkowski sur les réseaux de \( \R^n \), on n'y verra aucune des belles et fécondes méthodes analytiques.

    La primauté donnée au point de vue algébrique me semble toutefois justifiée par plusieurs raisons. Il permet tout d'abord de se placer rapidement dans le cadre où les problèmes se posent le plus naturellement, même s'ils ne concernent que les nombres entiers naturels. On verra, par exemple, que la recherche des solutions en nombres entiers de l'équation de Pell-Fermat \( x^2 - d y^2 = \pm 1 \) (\( d \) : entier donné, sans facteurs carrés) est un problème qui concerne essentiellement le corps quadratique \( \Q(\sqrt{d}) \). Pour la \textquote{grande} équation de Fermat \( x^n + y^n = z^n \), c'est le corps des racines \( n \)‑ièmes de l'unité qui joue le rôle décisif. Pour décomposer un entier en somme de deux (resp. quatre) carrés, on verra qu'il y a grand avantage à se placer dans l'anneau \( \Z[\i] \) des entiers de Gauss (resp. dans un anneau de quaternions convenablement choisi). La loi de réciprocité quadratique fait intervenir les corps quadratiques et les racines de l'unité. Dans tout ceci apparaissent des corps plus généraux que \( \Q \)\,, des anneaux plus généraux que \( \Z \)\,, ainsi que les corps et anneaux quotients de ces derniers, c'est-à-dire les corps finis et les algèbres sur ceux-ci.

    Ainsi, sans épuiser la Théorie des Nombres, la méthode algébrique amène toutefois rapidement à des résultats substantiels. En poussant dans la même direction, on arriverait à des théorèmes plus profonds, comme ceux de la théorie du corps de~classes.

    D'autre part, celui qui aime les méthodes analytiques s'aperçoit qu'elles n'acquièrent leur pleine efficacité que si on les applique à des corps de nombres algébriques, et pas seulement à \( \Q \). Par exemple il est regrettable d'avoir à étudier la seule fonction \( \zeta(s) \) sans pouvoir traiter en même temps de la fonction \( \zeta_K(s) \) d'un corps de nombres \( K \), ni des nombreuses \textquote{séries~\( L \)}.

    Pour l'étudiant enfin, le développement de la méthode algébrique a l'avantage de fournir de très nombreux exemples illustrant les notions introduites en cours d'algèbre : groupes, anneaux, corps, idéaux, anneaux et corps quotients, homomorphismes et isomorphismes, modules et espaces vectoriels. Un autre avantage est qu'il voit introduites en chemin plusieurs notions algébriques nouvelles qui sont fondamentales, non seulement pour l'arithmétique, mais aussi pour d'autres branches des Mathématiques, la Géométrie Algébrique en particulier : par exemple, les éléments entiers sur un anneau, les extensions de corps, la théorie de Galois, les modules sur les anneaux principaux, les anneaux et modules nœthériens, les anneaux de Dedekind et les anneaux de fractions.

    Ce qui précède a implicitement décrit ce que le lecteur pourra trouver dans ce livre, et mentionné ce qu'il n'y trouvera pas. J'ai supposé qu'il connaît l'algèbre de base : notions élémentaires sur les groupes, anneaux, corps, polynômes, espaces vectoriels, --- maniement des sous-objets, des objets-quotients et des objets-produits, --- mécanisme du passage au quotient par un idéal ou un sous-module, --- notions diverses d'homomorphisme et d'isomorphisme. Il trouvera tout ce qui est nécessaire sur ces questions dans un livre élémentaire d'algèbre dite \textquote{moderne}, par exemple les excellents \textquote{Cours d'Algèbre} de R.~Godement et \textquote{Algebra} de S.~Lang\footnote{Bien entendu la portée de ces deux ouvrages est nettement plus grande.}. J'utiliserai donc sans aucune hésitation ce langage et ces résultats, et j'espère montrer au lecteur qu'ils sont fort efficaces pour arriver rapidement à des théorèmes substantiels d'arithmétique. Par contre, bien que ce soit assez souvent traité en cours d'algèbre, j'ai pensé qu'il serait plus commode pour le lecteur de trouver ici ce qui lui sera nécessaire au sujet des éléments entiers sur un anneau, des extensions algébriques de corps, de la théorie de Galois, des modules et anneaux nœthériens, et des anneaux de fractions. J'ai tenté de le faire sans ambages, mais aussi sans inutile sophistication.


\chapter{Rappel de notations, de définitions et de résultats}

    On utilise les notations classiques de la théorie des ensembles : \( \in \), \( \subset \), \( \cup \), \( \cap \). Le complémentaire d'une partie \( B \) d'un ensemble \( A \) est noté \( A \setminus B \). Le cardinal (ou puissance, ou nombre d'éléments) d'un ensemble \( A \) est noté \( \card(A) \) ; si \( A \) est un groupe on dit aussi l'ordre de ce groupe.

    On suppose connues les notions de groupe, anneau, corps et espace vectoriel, ainsi que la théorie élémentaire des espaces vectoriels (appelée aussi \textquote{algèbre linéaire}). Dans ce livre, à l'exception du \cref{sec:théorème des quatre carrés}, \textquote{anneau} (resp. \textquote{corps}) veut dire anneau (resp. corps) \emph{commutatif à élément unité}.

    Étant donnés un groupe fini et un sous-groupe \( H \) de \( G \), on rappelle que \( \card(H) \) divise \( \card(G) \) ; le quotient \( \card(G)/\card(H) \) s'appelle l'indice de \( H \) dans \( G \) et se note \( (G:H) \).

    Étant données deux parties \( A , B \) d'un groupe \( G \) noté additivement, \( A+B \) désigne l'ensemble des sommes \( a+b \) où \( a \in A \) et \( b \in B \).

    Étant donné un anneau \( A \), on désigne par \( A[X] \) ou par \( A[Y] \) (lettre majuscule) l'anneau des polynômes (formels) à une variable sur \( A \) ; notations \( A[X_1, \cdots, X_n] \) pour les polynômes à \( n \) variables et \( A\lBrack X \rBrack \) pour les séries formelles.

    Par convention, un sous-anneau \( A \) d'un anneau \( B \) contient l'élément unité de \( B \). Étant donnés un anneau \( B \), un sous-anneau \( A \) de \( B \), et un élément \( x \in B \), on désigne par \( A[x] \) le sous-anneau de \( B \) engendré par \( A \) et \( x \), c'est-à-dire l'intersection des sous-anneaux de \( B \) qui contiennent \( A \) et \( x \) ; c'est l'ensemble des sommes de la forme \( a_0 + a_1 x + \cdots + a_n x^n \) (\( a_i \in A \)) ; notation \( A[x_1, \cdots, x_n] \) et résultats analogues pour le sous-anneau de \( B \) engendré par \( A \) et une famille finie \( (x_1, \cdots, x_n) \) d'éléments de~\( B \).

    Un anneau \( A \) est dit \emph{intègre} (ou sans diviseurs de zéro) si le produit de deux éléments non nuls quelconques de \( A \) est non nul, et si \( A \) n'est pas réduit à~\( 0 \).

    Un idéal \( \ideal{b} \) d'un anneau \( A \) est un sous-groupe additif tel que \( x \in \ideal{b} \) et \( a \in A \) impliquent \( ax \in \ideal{b} \). L'anneau tout entier et l'ensemble réduit à \( 0 \) (et noté \( (0) \)) sont des idéaux, parfois qualifiés de \textquote{triviaux}. Un corps n'en a pas d'autres, et ceci caractérise les corps parmi les anneaux. Étant donnée une famille \( (b_i) \) d'éléments d'un anneau \( A \), l'intersection des idéaux de \( A \) contenant les \( b_i \) est un idéal de \( A \), appelé idéal engendré par les \( b_i \) ; c'est l'ensemble des sommes finies \( \sum_i a_i b_i \) avec \( a_i \in A \). Un idéal engendré par un élément \( b \) est dit principal ; notation \( Ab \) ou~\( (b) \).

    Étant donnés un anneau \( A \) et un idéal \( \ideal{b} \) de \( A \), les classes d'équivalence \( a+\ideal{b} \) (\( a \in A \)) forment un anneau, appelé anneau quotient de \( A \) par \( \ideal{b} \) et noté \( A/\ideal{b} \). Les idéaux de \( A/\ideal{b} \) sont de la forme \( \ideal{b}'/\ideal{b} \) où \( \ideal{b}' \) parcourt l'ensemble des idéaux de~\( A \) contenant \( \ideal{b} \). Pour que \( A/\ideal{b} \) soit un corps il faut et il suffit que \( \ideal{b} \) soit maximal parmi les idéaux de~\( A \) distincts de~\( A \) ; on dit alors que \( \ideal{b} \) est maximal. Un idéal \( \ideal{p} \) est dit premier si \( A/\ideal{p} \) est~intègre.

    Étant donnés deux anneaux \( A, A' \), d'éléments unités \( e \) et \( e' \), un homomorphisme \( f \colon A \to A' \) est une application \( f \) de \( A \) dans \( A' \) telle que :
    \[
        f(a+b) = f(a)+f(b), \quad f(ab) = f(a)f(b), \quad f(e) = e'.
    \]

    Étant donné un anneau \( A \), une \emph{\( A \)‑algèbre} est un anneau \( B \) muni d'un homomorphisme \( \varphi \colon A \to B \). Si \( A \) est un corps, \( \varphi \) est injectif, et on identifie souvent alors \( A \) à son image \( \varphi(A) \) (qui est un sous-anneau de~\( B \)).

    Étant donnés un corps \( L \) et un sous-corps \( K \) de \( L \), on dit souvent que \( L \) est une extension de~\( K \).

    L'élément unité d'un anneau \( A \) sera le plus souvent noté~\( 1 \).

    La notion de \emph{module} sur un anneau \( A \) (ou de \( A \)‑module), est la généralisation directe de la notion d'espace vectoriel sur un corps. Un \( A \)‑module \( M \) est un groupe abélien (noté additivement) muni d'une application \( A \times M \to M \) (notée multiplicativement) telle que \( a(x+y) = ax + ay \), \( (a+b)x = ax+bx \), \( a(bx) = (ab)x \), \( 1 x = x \) (\( a,b \in A \), \( x,y \in M \)). On a les notions de sous-module et de module quotient. Étant donnés deux \( A \)‑modules \( M \) et \( M' \), un homomorphisme (ou application \( A \)‑linéaire) de \( M \) dans \( M' \) est une application \( f \colon M \to M' \) telle que
    \[
        f(x+y) = f(x)+f(y), \quad f(ax) = af(x) \quad (a \in A, ~ x,y \in M).
    \]
    Étant donné un homomorphisme \( f \colon X \to X' \) (de groupes, d'anneaux ou de modules), on appelle \emph{noyau} de \( f \) et on note \( \Ker(f) \)~\footnote{À cause de l'anglais \textquote{kernel}, ou de l'allemand \textquote{kern}.} l'image réciproque de l'élément neutre de \( X' \) par \( f \) ; c'est un sous-groupe invariant (ou un idéal, ou un sous-module) de \( X \) ; pour que \( f \) soit injectif il faut et il suffit que \( \Ker(f) \) soit réduit à l'élément neutre de \( X \). On~appelle \emph{image} de \( f \) la partie \( f(X) \) de \( X' \) ; c'est un sous-groupe (ou un sous-anneau, ou un sous-module) de~\( X' \).

\enlargethispage*{\baselineskip}
    Étant donnés deux ensembles \( X, X' \), une application \( f \) de \( X \) dans \( X' \), est souvent désignée par la notation \( f \colon X \to X' \) (flèche droite). Lorsqu'une application \( f \colon X \to X' \) est décrite par la valeur qu'elle prend en un élément arbitraire \( x \) de \( X \), on emploie la notation \( x \mapsto f(x) \). Ainsi la fonction sinus, \( \sin \colon \R \to \R \) peut être définie par
    \[
        x \mapsto \sum_{n=0}^{\infty} (-1)^n \frac{x^{2n+1}}{(2n+1)!}.
    \]

    Nous emploierons les notations classiques des objets mathématiques courants :

    \noindent
    \begin{tabularx}{\textwidth}{l@{\hspace{1ex}}X}
        \( \mathbb{N} \) : & ensemble des entiers naturels \( (0, 1, 2, \cdots, n, \cdots) \) \par\quad (N pour \textquote{nombre}).\\
        \( \mathbb{Z} \) : & anneau des entiers rationnels ou relatifs (entiers naturels et leurs opposés) \par\quad (Z pour \textquote{Zahlen}).\\
        \( \mathbb{Q} \) : & corps des nombres rationnels (quotients d'éléments de \( \mathbb{Z} \)) \par\quad (Q pour \textquote{quotients}).\\
        \( \mathbb{R} \) : & corps des nombres réels \par\quad (R pour \textquote{réels}).\\
        \( \mathbb{C} \) : & corps des nombres complexes \par\quad (C pour \textquote{complexes}).\\
        \( \mathbb{F}_q \) : & corps fini à \( q \) éléments \par\quad (F pour \textquote{fini} ou \textquote{field}).
    \end{tabularx}



\mainmatter

\chapter{Anneaux principaux}\label{chap:anneaux principaux}

\section{Relation de divisibilité dans les anneaux principaux}

    Soient \( A \) un anneau intègre, \( K \) son corps des fractions, \( x \) et \( y \) des éléments de~\( K \). On dit que \emph{\( x \) divise \( y \)} s'il existe \( a \in A \) tel que \( y = ax \). Expressions synonymes : \( x \) est diviseur de \( y \), \( y \) est multiple de~\( x \) ; notation \( x \mid y \). Cette relation entre éléments de~\( K \) dépend essentiellement de l'anneau~\( A \) ; s'il faut préciser, on dit qu'il s'agit de la relation de divisibilité dans \( K \) \emph{par rapport à~\( A \)}.

    Étant donné \( x \in K \), l'ensemble des multiples de~\( x \) est \( Ax \), avec la notation classique. Ainsi \( x \mid y \) s'écrit aussi \( y \in Ax \), ou encore \( Ay \subset Ax \). L'ensemble \( Ax \) s'appelle un \emph{idéal fractionnaire principal} de~\( K \) par rapport à~\( A \) ; si \( x \in A \), \( Ax \) est l'idéal principal\index{principal (ideal)@principal (idéal)} (ordinaire) de~\( A \) engendré par~\( x \). Comme la relation de divisibilité \( x \mid y \) équivaut à la relation d'\emph{ordre} \( Ay \subset Ax \), elle a les deux propriétés suivantes des relations d'ordre.
    \begin{equation}
        x \mid x  ~;
        \text{ si } x \mid y \text{ et } y \mid z, \text{ alors } x \mid z.
    \end{equation}
    Par contre, si \( x \mid y \) et \( y \mid x \), on ne peut pas en général conclure que \( x = y \) ; on a seulement \( Ax = Ay \), ce qui (si~\( y \neq 0 \)) veut dire que le quotient \( x y^{-1} \) est un élément \emph{inversible} de~\( A \) ; deux tels éléments sont dits \emph{associés}\index{associes (elements)@associés (éléments)} ; ils sont indistinguables du point de vue de la divisibilité.

    \begin{quote}
        \textsc{\textbf{Exemple}}.
        Les éléments de \( K \) associés à \( 1 \) sont les éléments inversibles de \( A \) ; on les appelle souvent les \emph{unités}\index{unites (d'un anneau)@unités (d'un anneau)} de \( A \) ; ils forment un groupe pour la multiplication, que nous noterons \( A^\times \). La détermination des unités d'un anneau \( A \) est un problème intéressant, que nous traiterons dans le cas où \( A \) est l'anneau des entiers d'un corps de nombres (voir chapitre \cref{chap:classes d'idéaux;théorème des unités}). Voici quelques exemples simples :
        \begin{enumerate}
            \item Si \( A \) est un corps, \( A^\times \) est l'ensemble des éléments non nuls de \( A \).
            \item Si \( A = \Z \)\,, \( A^\times \) se compose de \( +1 \) et de \( -1 \).
            \item Les unités de l'anneau de polynômes \( B = A[X_1, \cdots, X_n] \) sont, lorsque \( A \) est intègre, les constantes inversibles ; autrement dit, \( B^\times = A^\times \).
            \item Les unités de l'anneau de séries formelles \( A\lBrack X_1, \cdots, X_n \rBrack \) sont les séries formelles dont le terme constant est inversible.
        \end{enumerate}
    \end{quote}

    \begin{definition}[anneau principal]
        Un anneau \( A \) est dit \emph{principal}\index{principal (anneau)} s'il est intègre et si tout idéal de \( A \) est principal.
    \end{definition}

    On sait que l'anneau \( \Z \) est principal (Rappel : étant donné un idéal \( \ideal{a} \neq (0) \) de \( \Z \)\,, il contient un plus petit entier \( b > 0 \) ; par division euclidienne de \( x \in \ideal{a} \) par \( b \), on voit que \( x \) est un multiple de \( b \)). Si \( k \) est un corps, on sait que l'anneau \( k[X] \) des polynômes à \emph{une} variable sur \( k \) est principal (même méthode ; prendre un polynôme non nul \( b(X) \) de plus petit degré de l'idéal donné \( \ideal{a} \), et utiliser la division euclidienne, c'est-à-dire suivant les puissances décroissantes, par \( b(X) \)). Cette méthode se généralise aux anneaux qu'on appelle \textquote{euclidiens} (\cite{bourbaki1}, chap.\,VIII, \S\,1, exercices ; ou \cite{zariski1958}, chap.\,I). Si~\( k \) est un corps, on voit facilement que tout idéal non nul de l'anneau de séries formelles \( A = k\lBrack X \rBrack \) est de la forme \( AX^n \) avec \( n \geqslant 0 \), de sorte que \( A = k\lBrack X \rBrack \) est~principal.

    La divisibilité dans le corps des fractions \( K \) d'un anneau \emph{principal} \( A \) a une forme particulièrement simple. Comme c'est la généralisation immédiate du cours d'Arithmétique, nous serons très brefs :
    \begin{enumerate}[itemsep=.3\baselineskip]
        \item \label{divisibilité dans le corps des fractions;pgcd} Deux éléments quelconques \( u,v \) de \( K \) ont un \emph{plus grand commun diviseur} (p.g.c.d.\index{p.g.c.d.}), c'est-à-dire un élément \( d \) tel que les relations
        \begin{equation}
            \text{\textquote{\( x \mid u \) et \( x \mid v \)}\quad et \quad \textquote{\( x \mid d \)}}
        \end{equation}
        soient équivalentes. En effet il revient au même de dire que \( Au \) et \( Av \) ont une \emph{borne supérieure} dans l'ensemble ordonné des idéaux fractionnaires principaux ; or celle-ci est \( Au+Av \), qui est un idéal fractionnaire principal car l'anneau \( A \) est principal (clair pour \( u,v \in A \) ; on se ramène à ce cas en multipliant \( u \) et \( v \) par un dénominateur commun). Nous obtenons une information supplémentaire (\textquote{l'\emph{identité de Bézout}}\index{Bezout (identite de)@Bézout (identité de)}\index{identite de Bezout@identité de Bézout}) : il existe des éléments \( a,b \) de \( A \) tels que le p.g.c.d. \( d \) de \( u \) et \( v \) s'écrive
        \begin{equation}
            d = au+bv. \label{eq:bezout}
        \end{equation}
        Le p.g.c.d. de \( u \) et \( v \) est déterminé à un élément inversible près de \( A \).
        \item Deux éléments quelconques \( u,v \) de \( K \) ont un \emph{plus petit commun multiple} (p.p.c.m.\index{p.p.c.m.}), c'est-à-dire un élément \( m \) tel que les relations
        \begin{equation}
            \text{\textquote{\( u \mid x \) et \( v \mid x \)}\quad et \quad \textquote{\( m \mid x \)}}
        \end{equation}
        soient équivalentes. On peut le voir en remarquant que le passage à l'inverse \( t \mapsto t^{-1} \) renverse les relations de divisibilité, ce qui nous ramène au p.g.c.d. ; cette méthode nous apporte aussitôt le sous-produit suivant :
        \begin{equation}
            \ppcm(u,v) = \pgcd(u^{-1},v^{-1})^{-1} \quad (\text{pour } u,v \neq 0),
        \end{equation}
        d'où on déduit sans peine la formule classique
        \begin{equation}
            \pgcd(u,v) \cdot \ppcm(u,v) = uv.
        \end{equation}
        On peut aussi procéder comme dans \ref{divisibilité dans le corps des fractions;pgcd} et remarquer que l'existence du p.p.c.m. de \( u \) et \( v \) équivaut à celle d'une \emph{borne inférieure} de \( Au \) et \( Av \) dans l'ensemble ordonné des idéaux fractionnaires principaux ; or celle-ci est \( Au \cap Av \).
        \item Deux éléments \( a,b \) de \( A \) sont dits \emph{étrangers}\index{etrangers (elements)@étrangers (éléments)} (ou \emph{premiers entre eux}\index{premiers entre eux}) si \( 1 \) est un de leurs p.g.c.d. Rappelons l'important \textsc{Lemme d'Euclide}. \emph{Soient \( a,b,c \) des éléments d'un anneau principal \( A \) ; si \( a \) divise \( bc \) et est étranger à \( b \), alors \( a \) divise \( c \)}.
        \begin{quote}
            Démonstration succincte :
            par Bézout \eqref{eq:bezout}, on a \( a' \) et \( b' \in A \) tels que \( 1 = a'a + b'b \) ; d'où \( c = a'a c + b'b c \) ; comme \( a \) divise chaque terme du second membre, il divise~\( c \).
        \end{quote}
        \item Enfin on a l'importante \textquote{décomposition en facteurs premiers} :
        \begin{theorem*}
            Étant donnés un anneau principal \( A \) et son corps des fractions \( K \), il existe une partie \( P \) de \( A \) telle que tout \( x \in K \) s'écrive de façon unique
            \begin{equation}
                x = u \prod_{p \in P} p^{v_p(x)}
            \end{equation}
            où \( u \) est un élément inversible de \( A \), et où les exposants \( v_p(x) \) sont des éléments de \( \Z \)\,, tous nuls sauf un nombre fini d'entre eux.
        \end{theorem*}
    \end{enumerate}

    \begin{quote}
        Pour un exposé plus systématique de ces questions, nous renvoyons le lecteur à \cite{bourbaki1}~\emph{Algèbre}, chap.\,VI, \S\,1 et chap.\,VII, \S\,1. Une partie de la théorie (plus précisément tout ce qui ne tourne pas autour de l'identité de Bézout) s'étend à des anneaux plus généraux que les anneaux principaux, à savoir les anneaux \emph{factoriels} ; voir~\cite{samuel1964}, ou \cite{bourbaki2}~\emph{Algèbre commutative}, chap.\,VII, \S\,3.
    \end{quote}

\section{Un exemple : les équations \texorpdfstring{\( x^2 + y^2 = z^2 \)}{x\textasciicircum 2+y\textasciicircum 2=z\textasciicircum 2} and \texorpdfstring{\( x^4 + y^4 = z^4 \)}{x\textasciicircum 4+y\textasciicircum 4=z\textasciicircum 4}}\label{sec:l'équation de Fermat;cas 2 et 4}

    Un des parties les plus attirantes de la Théorie des Nombres est l'étude des \emph{équations diophantiennes}\index{diophantienne (equation)@diophantienne (équation)}\index{equation diophantienne@équation diophantienne}. Il s'agit d'équations polynomiales \( P(x_1, \cdots, x_n) = 0 \) à coefficients dans \( \Z \) (resp. dans \( \Q \)), dont on cherche les solutions \( (x_i) \) en nombres entiers (resp. en nombres rationnels). On peut remplacer \( \Z \) (resp. \( \Q \)) par des anneaux \( A \) (resp. des corps \( K \)) plus généraux ; nous en verrons un exemple plus tard (\cref{sec:racine de l'unité dans un corps}).

    Nous allons étudier ici deux cas particuliers de la fameuse \emph{équation de Fermat}\index{equation de Fermat@équation de Fermat}\index{Fermat (equation de)@Fermat (équation de)} :
    \begin{equation}
        x^n+y^n=z^n.\label{eq:fermat}
    \end{equation}
    Fermat a affirmé avoir démontré que, pour \( n \geqslant 3 \), cette équation n'a pas de solution \( (x,y,z) \) en nombres entiers tous non-nuls ; sa démonstration n'a pas été retrouvée. De~très nombreux mathématiciens ont, depuis, intensément travaillé sur ce problème, et montré que l'affirmation de Fermat est vraie pour un grand nombre de valeurs de l'exposant \( n \) ; ce n'est qu'en 1994 que A. J. Wiles a finalement trouvé une preuve générale (\cf.~\cite{wiles1994}).

    \begin{quote}
        L'opinion aujourd'hui la plus courante est que, dans sa \textquote{démonstration}, Fermat avait commis une erreur, mais une erreur digne de ce mathématicien de premier ordre. Par exemple il aurait pu avoir l'idée (géniale pour son époque) d'opérer dans l'anneau des entiers du corps des racines \( n \)‑ièmes de l'unité, et avoir cru que cet anneau est toujours principal. En effet, on sait démontrer l'assertion de Fermat pour tout exposant \( n \) tel que cet anneau soit principal ; mais il ne l'est pas pour tout \( n \) ; bien plus, pour \( n \) premier, cet anneau n'est principal que pour un nombre fini de valeurs de \( n \)~\footnote{Voir C. L. Siegel -- \textquote{Gesammelte Werke}, t.\,III, p.\,436-442.}.
    \end{quote}

    Pour \( n = 2 \), l'équation \eqref{eq:fermat} a des solutions entières, par exemple \( (3,4,5) \). On peut en donner une description complète :

    \begin{theorem}\label{thm:eq.fermat;n=2}
        Si \( x,y,z \) sont des entiers \( \geqslant 1 \) tels que \( x^2+y^2=z^2 \), il existe un entier \( d \) et des entiers étrangers \( u,v \) tels que (à une permutation près de \( x \) et \( y \)) on ait :
        \begin{equation}\label{eq:solution de x2+y2=z2}
            x = d(u^2-v^2), \quad
            y = 2duv, \quad
            z = d(u^2+v^2).
        \end{equation}
    \end{theorem}
    \begin{proof}
        Un calcul facile montre que les formules \eqref{eq:solution de x2+y2=z2} donnent des solutions de \( x^2+y^2=z^2 \). Réciproquement, soient \( x,y,z \) des entiers \( \geqslant 1 \) tels que \( x^2+y^2=z^2 \). Quitte à diviser \( x,y,z \) par leur p.g.c.d., on peut les supposer étrangers dans leur ensemble ; ils sont alors étrangers deux à deux, car, si, par exemple, \( x \) et \( z \) ont un facteur premier commun \( p \), alors \( p \) divise \( y^2 = z^2-x^2 \) et donc \( y \). En particulier deux des nombres \( x,y,z \) sont impairs, et le troisième est nécessairement pair. Les nombres \( x \) et \( y \) ne peuvent pas être tous deux impairs, car sinon, on aurait \( x^2 \equiv 1 \pmod*{4} \), \( y^2 \equiv 1 \pmod*{4} \) d'où \( z^2 \equiv 2 \pmod*{4} \) contrairement au fait que \( z^2 \) est un carré. On a donc, après échange éventuel de \( x \) et~\( y \),
        \begin{equation}\label{eq:fermat;n=2;pair et impair}
            \text{\( x \) impair, \( y \) pair, \( z \) impair.}
        \end{equation}

        Écrivons l'équation
        \begin{equation}
            y^2 = z^2-x^2 = (z-x)(z+x).
        \end{equation}
        Comme le p.g.c.d. de \( 2x \) et de \( 2z \) est \( 2 \), et que \( 2x = (z+x)-(z-x) \) et \( 2z = (z+x)+(z-x) \), le p.g.c.d. de \( z-x \) et \( z+x \) ne peut être que \( 2 \). Posons \( y = 2y' \), \( z+x = 2x' \), \( z-x = 2z' \), où \( y',x',z' \) sont des entiers, car \( y, z+x, z-x \) sont pairs par \eqref{eq:fermat;n=2;pair et impair}. On a alors \( y'\,^2=x'z' \). Comme \( x' \) et \( z' \) sont étrangers, la décomposition en facteurs premiers de \( y'\,^2 \) montre que \( x' \) et \( z' \) sont des \emph{carrés} \( u^2 \) et \( v^2 \) : en effet tout facteur premier de \( y'\,^2 \) va, avec son exposant pair, soit tout entier dans \( x' \), soit tout entier dans \( z' \). On a donc \( z+x = 2u^2 \), \( z-x = 2v^2 \), \( y^2 = 2u^2 \cdot 2v^2 \), d'où \( x = u^2-v^2 \), \( y = 2uv \), \( z = u^2+v^2 \). Ici \( u \) et \( v \) sont étrangers, sinon \( x,y,z \) auraient un facteur premier commun. On en déduit \eqref{eq:solution de x2+y2=z2} en remultipliant par \( d \) le p.g.c.d.
    \end{proof}

    \begin{theorem}\label{thm:eq.fermat;n=4}
        L'équation \( x^4+y^4=z^2 \) n'a pas de solution en nombres entiers \( x,y,z \geqslant 1 \).
    \end{theorem}
    \begin{proof}
        Raisonnons par l'absurde. On a alors une solution \( (x,y,z) \) où \( z \) est \emph{minimal}. Pour celle-ci, \( x \), \( y \) et \( z \) sont étrangers deux à deux : en effet, si par exemple, \( x \) et \( y \) avaient un facteur premier commun \( p \), alors \( p^4 \) diviserait \( z^2 \), donc \( p^2 \) diviserait \( z \), et \( \Bigl(\frac{x\vphantom{y}}{p}, \frac{y}{p}, \frac{z\vphantom{y}}{p^2}\Bigr) \) serait une solution contredisant la minimalité de \( z \) ; les deux autres cas sont analogues et plus faciles. Comme notre équation s'écrit \( (x^2)^2+(y^2)^2 = z^2 \), on peut lui appliquer \cref{thm:eq.fermat;n=2} : après permutation éventuelle de \( x \) et~\( y \), on voit qu'on a des entiers \( u,v \geqslant 1 \) et étrangers tels que
        \begin{equation}\label{eq:fermat;n=4;aux1}
            x^2 = u^2-v^2, \quad
            y^2 = 2uv, \quad
            z = u^2+v^2.
        \end{equation}
        Comme \( 4 \mid y^2 \), la relation \( y^2 = 2uv \) montre que l'un des deux nombres \( u \) et \( v \) est pair ; l'autre est nécessairement impair ; la répartition \textquote{\( u \) pair, \( v \) impair} donne \( u^2 \equiv 0 \pmod*{4} \), \( v^2 \equiv 1 \pmod*{4} \), d'où \( x^2 = u^2-v^2 \equiv -1 \pmod*{4} \), ce qui est absurde ; donc \( u \) est impair et \( v = 2v' \). La relation \( y^2 = 4uv' \) et le fait que \( u \) et \( v' \) sont étrangers montrent que \( u \) et \( v' \) sont des carrés \( a^2 \) et \( b^2 \). Appliquons encore \cref{thm:eq.fermat;n=2}, cette fois à l'équation \( x^2+v^2=u^2 \) (\cf.~\eqref{eq:fermat;n=4;aux1}) ; comme \( x \) et \( u \) sont impairs, \( v \) pair, et \( x,v,u \) étrangers deux à deux, on a des entiers étrangers \( c, d \geqslant 1 \) tels que :
        \begin{equation}\label{eq:fermat;n=4;aux2}
            x = c^2-d^2, \quad
            v = 2cd, \quad
            u = c^2+d^2.
        \end{equation}
        Or, de \( v = 2v' = 2b^2 \), on déduit \( cd = b^2 \), de sorte que \( c \) et \( d \) sont encore des carrés \( x'\,^2 \) et \( y'\,^2 \), car ils sont étrangers. Comme \( u = a^2 \), la dernière équation \eqref{eq:fermat;n=4;aux2} s'écrit
        \begin{equation}
            a^2 = x'\,^4 + y'\,^4
        \end{equation}
        et a la \emph{même forme} que l'équation donnée. Mais, on a, par \eqref{eq:fermat;n=4;aux1}, \( z = u^2 + v^2 = a^4+4b^4 >\nobreak a^4 \), d'où \( z > a \), ce qui contredit le caractère minimal de~\( z \). Notre assertion est donc~démontrée.
    \end{proof}

    \begin{quote}
        Une légère variante de notre démonstration montre que, étant donnée une solution \( (x,y,z) \) en entiers \( \geqslant 1 \) de \( x^4+y^4=z^2 \), on construit une suite \( (x_n,y_n,z_n) \) de telles solutions, où la suite \( (z_n) \) est strictement décroissante, ce qui est absurde. Ceci est la méthode de \emph{descente infinie}\index{descente infinie}, due à Fermat.
    \end{quote}

    \begin{corollary}
        L'équation \( x^4+y^4=z^4 \) n'a pas de solution en nombres entiers \( x,y,z \geqslant 1 \).
    \end{corollary}
    \begin{proof}
        En effet cette équation s'écrit \( x^4+y^4=(z^2)^2 \), et on applique \cref{thm:eq.fermat;n=4}.
    \end{proof}

\section{Quelques lemmes sur les idéaux ; l'indicateur d'Euler}

    Soit \( n \geqslant 1 \) un entier naturel. On appelle \emph{indicateur d'Euler}\index{Euler (indicateur d')}\index{indicateur d'Euler} de \( n \), et on note \( \varphi(n) \) le nombre des entiers \( q \) premiers à \( n \) et tels que \( 0 \leqslant q \leqslant n \) (il revient au même d'écrire \( 1 \leqslant q \leqslant n-1 \), car \( 0 \) et \( n \) ne sont pas premiers à \( n \)). Si \( p \) est un nombre premier, il est clair que :
    \begin{equation}
        \varphi(p) = p-1.
    \end{equation}
    Pour \( n = p^s \), puissance de nombre premier, les entiers premiers à \( p^s \) sont les entiers non multiples de \( p \) ; or il y a \( p^{s-1} \) multiples de \( p \) entre \( 1 \) et \( p^s \) ; on a donc :
    \begin{equation}\label{eq:indicateur d'Euler;puissance de nombre premier}
        \varphi(p^s) = p^s-p^{s-1} = p^{s-1}(p-1).
    \end{equation}
    À partir de là nous nous proposons de calculer \( \varphi(n) \) en utilisant la décomposition de \( n \) en facteurs premiers. Pour cela nous aurons besoin de caractérisations de \( \varphi(n) \), et de lemmes sur les idéaux qui serons encore utilisés par la suite.

    \begin{proposition}\label{prop:indicateur d'Euler et Z/nZ}
        Soit \( n \geqslant 1 \) un entier naturel. L'indicateur d'Euler \( \varphi(n) \) est égal au nombre des éléments de \( \Z/n\Z \) qui engendrent ce groupe, et aussi au nombre des éléments inversibles de l'anneau \( \Z/n\Z \).
    \end{proposition}
    \begin{proof}
        Rappelons que chaque classe de congruence modulo \( n\Z \) contient un entier \( q \) et un seul tel que \( 0 \leqslant q \leqslant n-1 \) ; pour un tel entier \( q \), notons \( \overline{q} \) sa classe modulo~\( n\Z \). Il suffit de démontrer, en cercle, les implications : \( q \) premier à \( n \) \( \Rightarrow \) \( \overline{q} \) inversible \( \Rightarrow \) \( \overline{q} \) engendre \( \Z/n\Z \) \( \Rightarrow \) \( q \) premier à \( n \). Si \( q \) est premier à \( n \), l'identité de Bézout (éq.~\eqref{eq:bezout}) montre qu'on a des entiers \( x \) et \( y \) tels que \( qx+ny=1 \) ; d'où \( \overline{q} \cdot \overline{x} = \overline{1} \), et \( \overline{q} \) est inversible. Si \( \overline{q} \) est inversible, notons \( x \) un entier tel que \( \overline{q} \cdot \overline{x} = \overline{1} \) ; si \( \overline{a} \) est un élément quelconque de \( \Z/n\Z \) et si \( a \) est un représentant de \( \overline{a} \), on a \( \overline{a} = \overline{a} \overline{x} \overline{q} \) (dans l'anneau \( \Z/n\Z \)), d'où \( \overline{a} = (ax) \overline{q} \) (dans le groupe additif \( \Z/n\Z \)) ; donc \( \overline{q} \) engendre le groupe \( \Z/n\Z \). Enfin si \( \overline{q} \) engendre \( \Z/n\Z \)\,, on a un entier \( x \) tel que \( x \cdot \overline{q} = \overline{1} \), donc tel que \( xq \equiv 1 \pmod*{n} \) ; ainsi il existe un entier \( y \) tel que \( xq-1 = yn \), d'où \( 1 = xq-yn \) ; ceci est une identité de Bézout qui montre que \( q \) est premier à~\( n \).
    \end{proof}

    \begin{lemma}\label{lem:reste chinois;r=2}
        Soient \( A \) un anneau, \( \ideal{a} \) et \( \ideal{b} \) des idéaux de \( A \) tels que \( \ideal{a}+\ideal{b}=A \). Alors \( \ideal{a} \cap \ideal{b} = \ideal{a}\ideal{b} \) et l'homomorphisme canonique \( \varphi \colon A \to A/\ideal{a} \times A/\ideal{b} \) définit un isomorphisme \( \theta \colon A/\ideal{a}\ideal{b} \to A/\ideal{a} \times A/\ideal{b} \).
    \end{lemma}
    \begin{quote}
        Rappelons que l'homomorphisme \( \varphi \) associe à tout \( x \in A \) le couple formé de la classe de \( x \) modulo~\( \ideal{a} \), et de la classe de \( x \) modulo~\( \ideal{b} \).
    \end{quote}
    \begin{proof}
        On sait qu'on a, en général, \( \ideal{a}\ideal{b} \subset \ideal{a} \), \( \ideal{a}\ideal{b} \subset \ideal{b} \), d'où \( \ideal{a}\ideal{b} \subset \ideal{a} \cap \ideal{b} \) ; soit alors \( x \in \ideal{a} \cap \ideal{b} \) ; comme \( \ideal{a} + \ideal{b} = A \) on dispose d'éléments \( a \in \ideal{a} \) et \( b \in \ideal{b} \) tels que \( a+b = 1 \) ; alors \( x = ax+xb \) est somme de deux éléments de \( \ideal{a}\ideal{b} \), d'où \( x \in \ideal{a}\ideal{b} \) et \( \ideal{a} \cap \ideal{b} \subset \ideal{a}\ideal{b} \). Donc \( \ideal{a} \cap \ideal{b} = \ideal{a}\ideal{b} \).

        Il est clair que le noyau de \( \varphi \) est \( \ideal{a} \cap \ideal{b} \) ; comme \( \ideal{a} \cap \ideal{b} = \ideal{a}\ideal{b} \), \( \varphi \) est constante sur chaque classe modulo \( \ideal{a}\ideal{b} \), d'où l'application \( \theta \colon A/\ideal{a}\ideal{b} \to A/\ideal{a} \times A/\ideal{b} \) ; c'est évidemment un homomorphisme ; comme \( \varphi^{-1}(0) = \ideal{a}\ideal{b} \), on a \( \theta^{-1}(0) = (0) \), et \( \theta \) est injective. Reste à montrer que \( \theta \) est surjective.

        \begin{quote}
            Nous avons un peu développé ce raisonnement de \textquote{passage au quotient} à titre d'entraînement. Nous serons désormais bien plus brefs pour des raisonnements analogues.
        \end{quote}

        Pour montrer la surjectivité de \( \theta \) (ou de \( \varphi \), ce qui revient au même), il s'agit de construire un élément \( x \) de \( A \) dont les classes modulo \( \ideal{a} \) et modulo \( \ideal{b} \) sont arbitrairement données ; soit \( y \) et \( z \) des représentants de ces classes. Or on dispose d'éléments \( a \in \ideal{a} \) et \( b \in \ideal{b} \) tels que \( a+b = 1 \). On prend \( x = az+by \). Modulo \( \ideal{a} \), on a \( x \equiv by \equiv (1-a)y \equiv y-ay \equiv y \) ; par échange des rôles, on en déduit \( x \equiv z \pmod*{\ideal{b}} \).
    \end{proof}

    \begin{lemma}\label{lem:reste chinois}
        Soient \( A \) un anneau, et \( (\ideal{a_i})_{1 \leqslant i \leqslant r} \) une famille finie d'idéaux de \( A \) tels que \( \ideal{a_i} + \ideal{a_j} = A \) pour \( i \neq j \). On a alors un isomorphisme canonique de \( A/\ideal{a_1} \cdots \ideal{a_r} \) sur \( \prod_{i=1}^r A/\ideal{a_i} \).
    \end{lemma}
    \begin{proof}
        \Cref{lem:reste chinois;r=2} est le cas \( r=2 \) \cref[de]{lem:reste chinois}. Nous procéderons à partir de là par récurrence sur \( r \). Posons \( \ideal{b} = \ideal{a_2} \cdots \ideal{a_r} \). Montrons que \( \ideal{a_1} + \ideal{b} = A \).

        En effet, pour \( i \geqslant 2 \), on a \( \ideal{a_1}+\ideal{a_i} = A \) et on dispose donc d'éléments \( c_i \in \ideal{a_1} \) et \( a_i \in \ideal{a_i} \) tels que \( c_i + a_i = 1 \). Multiplions membre à membre : il vient \( c + a_2 \cdots a_n = 1 \), où \( c \) est une somme de termes dont chacun contient au moins un \( c_i \) en facteur ; on a donc \( c \in \ideal{a_1} \). Comme \( a_2 \cdots a_r \in \ideal{b} \), on a bien \( \ideal{a_1} + \ideal{b} = A \).

        Par \cref{lem:reste chinois;r=2}, on a un isomorphisme \( A/\ideal{a_1}\ideal{b} \simeq A/\ideal{a_1} \times A/\ideal{b} \). D'après l'hypothèse de récurrence on a un isomorphisme
        \[
            A/\ideal{b} = A/\ideal{a_2} \cdots \ideal{a_r} \simeq (A/\ideal{a_2}) \times \cdots \times (A/\ideal{a_r}).
        \]
        On compose ces isomorphismes et notre assertion s'ensuit.
    \end{proof}

\pagebreak
    Appliquons maintenant ces lemmes à l'anneau \( \Z \) :

    \begin{proposition}\label{prop:reste chinois;Z;r=2}
        Soient \( n \) et \( n' \) des entiers premiers entre eux ;  alors l'anneau \( \Z/nn'\Z \) est isomorphe à l'anneau produit \( \Z/n\Z \times \Z/n'\Z \).
    \end{proposition}
    \begin{proof}
        Ceci est un cas particulier \cref[de]{lem:reste chinois;r=2}, l'hypothèse \( n\Z + n'\Z = \Z \) étant l'identité de Bézout.
    \end{proof}

    \begin{corollary}\label{cor:multiplicativité de l'indicateur d'Euler}
        Si \( n \) et \( n' \) sont des entiers \( \geqslant 1 \) premiers entre eux, on a \( \varphi(nn') = \varphi(n)\varphi(n') \).
    \end{corollary}
    \begin{proof}
        En effet, \( \varphi(nn') \) est le nombre des éléments inversibles de \( \Z/nn'\Z \) (\cref{prop:indicateur d'Euler et Z/nZ}), qui est isomorphe à \( \Z/n\Z \times \Z/n'\Z \). Or un élément \( (\alpha,\beta) \) d'un anneau produit est inversible si et seulement si chacune de ses composantes \( \alpha,\beta \) est inversible. D'où notre assertion, en appliquant encore \cref{prop:indicateur d'Euler et Z/nZ}.
    \end{proof}

    \begin{corollary}
        Soient \( n \) un entier \( \geqslant 1 \), et \( n = p_1^{\alpha_1} \cdots p_r^{\alpha_r} \) sa décomposition en facteurs premiers. Alors
        \begin{equation}
            \varphi(n) = n \left(1 - \frac{1}{p_1}\right) \cdots \left(1 - \frac{1}{p_r}\right).
        \end{equation}
    \end{corollary}
    \begin{proof}
        Par \cref{cor:multiplicativité de l'indicateur d'Euler}, on a \( \varphi(n) = \varphi(p_1^{\alpha_1}) \cdots \varphi(p_r^{\alpha_r}) \). Or, par \eqref{eq:indicateur d'Euler;puissance de nombre premier}, on a \( \varphi(p_i^{\alpha_i}) = p_i^{\alpha_i-1}(p_i-1) = p_i^{\alpha_i} \Bigl(1-\frac{1}{p_i}\Bigr) \). Notre formule s'ensuit par multiplication.
    \end{proof}

\section{Quelques préliminaires sur les modules}\label{sec:quelques préliminaires sur les modules}

    En vue d'étudier les modules sur un anneau principal, quelques préliminaires nous seront nécessaires.

    Étant donnés un anneau \( A \) et un ensemble \( I \), on note \( A^{(I)} \) l'ensemble des familles \( (a_i)_{i \in I} \)\,, indexées par \( I \), d'éléments de \( A \) telles que \( a_i = 0 \) sauf pour un nombre \emph{fini} d'indices \( i \in I \) ;  ainsi \( A^{(I)} \) est une partie de l'ensemble produit \( A^I \), et aussi un sous-module de \( A^I \) si on munit \( A^I \) de la structure de \( A \)‑module définie par composantes.

    Si \( I \) est fini, on a \( A^{(I)} = A^I \).

    Pour \( j \in I \), la famille \( (\delta_{ji})_{i \in I} \) telle que \( \delta_{jj} = 1 \) et \( \delta_{ji} = 0 \) pour \( i \neq j \) est un élément \( e_j \) de \( A^{(I)} \). Tout élément \( (a_j)_{j \in I} \) de \( A^{(I)} \) s'écrit, d'une façon et d'une seule, comme combinaison linéaire (finie) des \( e_j \) ; plus présisément :
    \begin{equation}
        (a_j)_{j \in I} = \sum_{j \in I} a_j e_j
    \end{equation}
    (noter que, dans la sommation du second membre, tous les termes sont nuls sauf un nombre fini, de sorte que la sommation a un sens). On dit que \( (e_j)_{j \in I} \) est la \emph{base canonique}\index{base canonique} de \( A^{(I)} \).

    Soient \( A \) un anneau, \( M \) un \( A \)‑module, et \( (x_i)_{i \in I} \) une famille d'éléments de \( M \). À tout élément \( (a_i)_{i \in I} \) de \( A^{(I)} \) associons l'élément \( \sum_i a_i x_i \) de \( M \) (comme ci-dessus la sommation a un sens). On a ainsi défini une application \( \varphi \colon A^{(I)} \to M \), qui est évidemment \emph{linéaire} ; si \( (e_i)_{i \in I} \) est la base canonique de \( A^{(I)} \), on a \( \varphi(e_i) = x_i \) pour tout \( i \in I \). Les équivalences suivantes sont immédiates :
    \begin{enumerate}
        \item les \( x_i \) sont linéairement indépendants \( \iff \) \( \varphi \) est injective.
        \item \( (x_i)_{i \in I} \) est un système générateur \( \iff \) \( \varphi \) est surjective.
    \end{enumerate}
    Si \( \varphi \) est \emph{bijective} on dit que \( (x_i)_{i \in I} \) est une \emph{base}\index{base d'un module} de \( M \) ; ceci veut dire que tout élément \( x \) de \( M \) s'écrit, \emph{d'une façon et d'une seule}, comme combinaison linéaire des \( x_i \). Un module \( M \) qui admet une base est appelé un \emph{module libre}\index{libre (module)}\index{module!libre}.

    \begin{quote}
        Contrairement aux espaces vectoriels sur les corps, un module sur un anneau n'admet pas nécessairement de base. Par exemple, le \( \Z \)‑module \( \Z/n\Z \) où \( n \neq 0,1 \). Dans la suite nous démontrerons que certains modules sont libres ; ce sera rarement trivial.
    \end{quote}

    Un module sera dit \emph{de type fini}\index{module!de type fini}\index{type fini (module de)} s'il admet un système générateur fini. \nameCref{thm:nœthérien} suivant est à la base de l'étude des anneaux et modules nœthériens, que nous développerons un peu plus au chapitre \cref{chap:anneaux nœthériens;anneaux de Dedekind}.

    \begin{theorem}\label{thm:nœthérien}
        Soient \( A \) un anneau, \( M \) un \( A \)‑module. Les conditions suivantes sont équivalents :
        \begin{enumerate}
            \item \label{thm:nœthérien;item1} Toute famille non vide de sous-modules de \( M \) possède un élément maximal (pour la relation d'inclusion) ;
            \item \label{thm:nœthérien;item2} Toute suite croissante \( (M_n)_{n \geqslant 0} \) (pour la relation d'inclusion) de sous-modules de \( M \) est stationnaire (c'est-à-dire qu'il existe \( n_0 \) tel que \( M_n = M_{n_0} \) pour tout \( n \geqslant n_0 \)) ;
            \item \label{thm:nœthérien;item3} Tout sous-module de \( M \) est de type fini.
        \end{enumerate}
    \end{theorem}
    \begin{proof}
        Montrons que \ref{thm:nœthérien;item1} implique \ref{thm:nœthérien;item3}. Soient \( E \) un sous-module de \( M \) et \( \Phi \) la famille des sous-modules de type fini de~\( E \) ; elle n'est pas vide car \( \set{0} \in \Phi \). Par \ref{thm:nœthérien;item1}, \( \Phi \) admet un élément maximal \( F \). Pour \( x \in E \), \( F + Ax \) est un sous-module de type fini de~\( E \) (il est engendré par la réunion de \( \set{x} \) et d'un système générateur fini de \( F \)). On~a donc \( F + Ax = F \) car \( F + Ax \supset F \) et car \( F \) est maximal. D'où \( x \in F \), \( E \subset F \), \( E = F \), et \( E \) est de type fini.

        Prouvons maintenant que \ref{thm:nœthérien;item3} implique \ref{thm:nœthérien;item2}. Soit \( (M_n)_{n \geqslant 0} \) une suite croissante de sous-modules de \( M \). Alors \( E = \bigcup_{n \geqslant 0} M_n \) est un sous-module de \( M \). Par \ref{thm:nœthérien;item3}, il admet un système générateur fini \( (x_1, \cdots, x_q) \). Pour tout \( i \), il y a un indice \( n(i) \) tel que \( x_i \in M_{n(i)} \). Soit \( n_0 \) le plus grand des \( n(i) \). On a \( x_i \in M_{n_0} \) pour tout \( i \), d'où \( E \subset M_{n_0} \) et \( E = M_{n_0} \). Pour \( n \geqslant n_0 \)\,, les inclusions \( M_{n_0} \subset M_n \subset E \) et l'égalité \( M_{n_0} = E \) montrent que \( M_{n_0} = M_n \). Donc la suite \( (M_n) \) est stationnaire à partir de~\( n_0 \).

        Reste à montrer que \ref{thm:nœthérien;item2} implique \ref{thm:nœthérien;item1}. Or l'équivalence de \ref{thm:nœthérien;item1} et \ref{thm:nœthérien;item2} est un cas particulier \cref[de]{lem:ensembles ordonnés} suivant sur les ensembles ordonnés.
    \end{proof}

    \begin{lemma}\label{lem:ensembles ordonnés}
        Soit \( T \) un ensemble ordonné. Les conditions suivantes sont équivalentes :
        \begin{enumerate}
            \item \label{lem:ensembles ordonnés;item1} Toute famille non vide d'éléments de \( T \) admet un élément maximal ;
            \item \label{lem:ensembles ordonnés;item2} Toute suite croissante \( (t_n)_{n \geqslant 0} \) d'éléments de \( T \) est stationnaire.
        \end{enumerate}
    \end{lemma}
    \begin{proof}
        \ref{lem:ensembles ordonnés;item1}~\( \Rightarrow \)~\ref{lem:ensembles ordonnés;item2} : Soit \( t_q \) un élément maximal de la suite croissante \( (t_n) \). Pour \( n \geqslant q \) on a \( t_n \geqslant t_q \) (croissante), donc \( t_n = t_q \) (maximalité).

        \ref{lem:ensembles ordonnés;item2}~\( \Rightarrow \)~\ref{lem:ensembles ordonnés;item1} : Supposons qu'on ait une partie non vide \( S \) de \( T \) sans élément maximal. Alors, pour \( x \in S \), l'ensemble des éléments de \( S \) strictement supérieurs à \( x \) est non vide. Par l'axiome de choix, il existe une application \( f \colon S \to S \) telle que \( f(x) > x \) pour tout \( x \in S \). Comme \( S \) est non vide, on choisit \( t_0 \in S \), et on définit par récurrence la suite \( (t_n)_{n \geqslant 0} \) au moyen de \( t_{n+1} = f(t_n) \). Cette suite est strictement croissante, donc n'est pas stationnaire. Ainsi l'implication \ref{lem:ensembles ordonnés;item2}~\( \Rightarrow \)~\ref{lem:ensembles ordonnés;item1} est démontrée par l'absurde.
    \end{proof}

    \begin{corollary}[{\cref[de]{thm:nœthérien}}]\label{cor:anneau principal;famille d'idéaux;élément maximal}
        Dans un anneau principal \( A \), toute famille non vide d'idéaux de \( A \) admet un élément maximal.
    \end{corollary}
    \begin{proof}
        En effet, si l'on considère \( A \) comme un module sur lui-même, ses sous-modules ne sont autres que ses idéaux. Comme ceux-ci sont principaux, ce sont des \( A \)‑modules à un générateur, donc de type fini. On applique alors l'implication \ref{thm:nœthérien;item3}~\( \Rightarrow \)~\ref{thm:nœthérien;item1} \cref[de]{thm:nœthérien}.
    \end{proof}

\section{Modules sur les anneaux principaux}

    Soient \( A \) un anneau intègre et \( K \) son corps des fractions. Un \( A \)‑module libre, donc isomorphe à un \( A^{(I)} \), peut se plonger dans un espace vectoriel sur \( K \) (\( K^{(I)} \) dans le cas de \( A^{(I)} \)). Il en est donc de même de tout sous-module \( M \) d'un \( A \)‑module libre. La dimension du sous-espace engendré par \( M \) est appelée le \emph{rang}\index{rang (d'un module)} de \( M \) ; c'est le nombre maximum d'éléments linéairement indépendants de \( M \). Si \( M \) est lui-même libre et admet une base ayant \( n \) éléments, alors le rang de \( M \) est égal à~\( n \).

    \begin{theorem}\label{thm:base adaptée}
        Soient \( A \) un anneau principal, \( M \) un \( A \)‑module libre de rang fini \( n \), et \( M' \) un sous-module de \( M \). Alors :
        \begin{enumerate}
            \item \label{thm:base adaptée;item1} \( M' \) est libre, de rang \( \leqslant n \) ;
            \item \label{thm:base adaptée;item2} Il existe une base \( (e_1, \cdots, e_n) \) de \( M \), un entier \( q \leqslant n \), et des éléments non nuls \( a_1, \cdots, a_q \) de \( A \) tels que \( (a_1 e_1, \cdots, a_q e_q) \) soit une base de \( M' \), et que \( a_i \) divise \( a_{i+1} \) pour \( 1 \leqslant i \leqslant q-1 \).
        \end{enumerate}
    \end{theorem}
    \begin{proof}
        \nameCref{thm:base adaptée} étant trivial pour \( M' = \set{0} \), on peut supposer \( M' \neq \set{0} \). Soit \( \L(M,A) \) l'ensemble des formes linéaires sur \( M \). Pour \( u \in \L(M,A) \), \( u(M') \) est un sous-\( A \)‑module de \( A \), donc un idéal de \( A \) ; nous écrirons \( u(M') = A a_u \) avec \( a_u \in A \), car cet idéal est principal. Soit \( u \in \L(M,A) \) tel que \( A a_u \) soit \emph{maximal} parmi les \( A a_v \) (\( v \in \L(M,A) \)) (\cref{cor:anneau principal;famille d'idéaux;élément maximal}). Prenons une base \( (x_1, \cdots, x_n) \) de \( M \), qui identifie \( M \) à \( A^n \) ; soit \( \pr_i \colon M \to A \) la forme coordonnée d'indice \( i \), définie par \( \pr_i(x_j) = \delta_{ij} \). Comme \( M' \neq \set{0} \), l'un des \( \pr_i(M') \) est \( \neq \set{0} \) ; on a donc \( a_u \neq 0 \). Par construction, il existe \( e' \in M' \) tel que \( u(e') = a_u \). Montrons que, \emph{pour tout \( v \in \L(M,A) \), \( a_u \) divise \( v(e') \)}. En effet, si \( d \) est le p.g.c.d. de \( a_u \) et de \( v(e') \), on a \( d = b a_u + c v(e') \) avec \( b,c \in A \), d'où \( d  = (bu+cv)(e') \) ; or \( bu + cv \) est une forme linéaire \( w \) sur \( M \) ; on a ainsi \( A a_u \subset Ad \subset w(M') \). Le caractère maximal de \( A a_u \) montre qu'on a \( Ad = A a_u \)\,, de sorte que \( a_u \) divise bien \( v(e') \).

        En particulier \( a_u \) divise les \( \pr_i(e') \), soit \( \pr_i(e') = a_u b_i \) avec \( b_i \in A \). Posons \( e = \sum_{i=1}^n b_i x_i \) ; on a \( e' = a_u e \). Comme \( u(e') = a_u = a_u \cdot u(e) \), on en déduit \( u(e) = 1 \) (noter que \( a_u \neq 0 \)). \emph{Montrons qu'on a}
        \begin{enumerate-roman}
            \item \label{thm:base adaptée;aux;item1} \( M = Ae + \Ker(u) \)
            \item \label{thm:base adaptée;aux;item2} \( M' = Ae' + (M' \cap \Ker(u)) \) \quad (où \( e' = a_u e \))
        \end{enumerate-roman}
        \emph{les sommes étant directes}. En effet, tout \( x \in M \) s'écrit \( x = u(x) e + (x-u(x)e) \), et on a \( u(x-u(x)e) = u(x) - u(x)u(e) = 0 \), ce qui démontre \ref{thm:base adaptée;aux;item1}. Pour \( y \in M' \) on a \( u(y) = b a_u \) avec \( b \in A \), et donc \( y = b a_u e + (y-u(y)e) = b e' + (y-u(y)e) \) ; on a encore \( y - u(y)e \in \Ker(u) \) et aussi \( y - u(y)e = y - b e' \in M' \) ; ceci démontre \ref{thm:base adaptée;aux;item2}. Enfin, pour montrer que les sommes sont directes, il suffit de voir que \( Ae \cap \Ker(u) = \set{0} \) ; or si \( x = ce \) est un élément de \( Ae \) (\( c \in A \)) et si \( u(x) = 0 \), on a \( c = c u(e) = u(ce) = u(x) = 0 \), d'où \( x = 0 \).

        Ces préliminaires étant établis, nous allons démontrer \ref{thm:base adaptée;item1} par récurrence sur le rang \( q \) de \( M' \). Si \( q = 0 \), on a \( M' = \set{0} \) et c'est trivial. Si \( q > 0 \), \( M' \cap \Ker(u) \) est de rang \( q-1 \) d'après \ref{thm:base adaptée;aux;item2}, et est donc libre d'après l'hypothèse de récurrence. Comme, dans~\ref{thm:base adaptée;aux;item2}, la somme est directe, on obtient une base de \( M' \) en adjoignant \( e' \) à une base de \( M' \cap \Ker(u) \). Ainsi \( M' \) est libre, et \ref{thm:base adaptée;item1} est vraie.

        Ceci étant, on va démontrer \ref{thm:base adaptée;item2} par récurrence sur le rang \( n \) de \( M \). C'est trivial pour \( n = 0 \). Par \ref{thm:base adaptée;item1}, \( \Ker(u) \) est libre, et est de rang \( n-1 \) car, dans \ref{thm:base adaptée;aux;item1}, la somme est directe. Appliquons l'hypothèse de récurrence au module libre \( \Ker(u) \) et à son sous-module \( M' \cap \Ker(u) \) : il existe \( q \leqslant n \), une base \( (e_2, \cdots, e_n) \) de \( \Ker(u) \) et des éléments non nuls \( a_2, \cdots, a_q \) de \( A \) tels que \( (a_2 e_2, \cdots, a_q e_q) \) soit une base de \( M' \cap \Ker(u) \), et que \( a_i \) divise \( a_{i+1} \) pour \( 2 \leqslant i \leqslant q-1 \). Avec les notations ci-dessus, on pose \( a_1 = a_u \) et \( e_1 = e \) ; alors \( (e_1, e_2, \cdots, e_n) \) est une base de \( M \) d'après \ref{thm:base adaptée;aux;item1}, et \( (a_1 e_1, \cdots, a_q e_q) \) est une base de \( M' \) (d'après \ref{thm:base adaptée;aux;item2} et le fait que \( e' = a_1 e_1 \)). Reste à montrer l'assertion de divisibilité \( a_1 \mid a_2 \). Or soit \( v \) la forme linéaire sur \( M \) définie par \( v(e_1) = v(e_2) = 1 \), \( v(e_i) = 0 \) pour \( i \geqslant 3 \) ; on a \( a_1 = a_u = v(a_u e_1) = v(e') \in v(M') \), d'où \( A a_u \subset v(M') \) ; d'après le caractère maximal de \( A a_u \) on en déduit \( v(M') = A a_u = A a_1 \) ; comme \( a_2 = v(a_2 e_2) \in v(M') \), on a \( a_2 \in A a_1 \)\,, c'est-à-dire \( a_1 \mid a_2 \).
    \end{proof}

    \begin{quote}
        Les idéaux \( A a_i \) \cref[de]{thm:base adaptée} s'appellent les \emph{facteurs invariants}\index{facteurs invariants} de \( M' \) dans~\( M \). On peut démontrer qu'ils sont uniquement déterminés par la donnée de \( M \) et~\( M' \) (\cite{bourbaki1}, chap.\,VII, \S\,3).
    \end{quote}

    \begin{corollary}\label{cor:structure d'un module de type fini sur un anneau principal}
        Soient \( A \) un anneau principal, et \( E \) un \( A \)‑module de type fini. Alors \( E \) est isomorphe à un produit \( (A/\ideal{a_1}) \times (A/\ideal{a_2}) \times \cdots \times (A/\ideal{a_n}) \), où les \( \ideal{a_i} \) sont des idéaux de \( A \) tels que \( \ideal{a_1} \supset \ideal{a_2} \supset \cdots \supset \ideal{a_n} \).
    \end{corollary}
    \begin{proof}
        Soit, en effet, \( (x_1, \cdots, x_n) \) un système générateur de \( E \). D'après le début du \cref{sec:quelques préliminaires sur les modules}, on a un homomorphisme surjectif \( \varphi \colon A^n \to E \), de sorte que \( E \) est isomorphe à \( A^n/\Ker(\varphi) \). Par \cref{thm:base adaptée}, on a une base \( (e_1, \cdots, e_n) \) de \( A^n \), un entier \( q \leqslant n \), et des éléments non nuls \( a_1, \cdots, a_q \) de \( A \) tels que \( (a_1 e_1, \cdots, a_q e_q) \) soit une base de \( \Ker(\varphi) \) et que \( a_i \) divise \( a_{i+1} \) pour \( 1 \leqslant i \leqslant q-1 \). On pose \( a_p = 0 \) pour \( q+1 \leqslant p \leqslant n \). Alors \( A^n/\Ker(\varphi) \) est isomorphe au produit des \( A e_i / A a_i e_i \) (\( 1 \leqslant i \leqslant n \)), et \( A e_i / A a_i e_i \) est isomorphe à \( A / A a_i \). En posant \( \ideal{a_i} = A a_i \)\,, notre assertion s'ensuit.
    \end{proof}

    Nous dirons qu'un module \( E \) sur un anneau intègre \( A \) est \emph{sans torsion}\index{module!sans torsion}\index{torsion (module sans)} si la relation \( ax=0 \) (\( a \in A \), \( x \in E \)) implique \( a = 0 \) ou \( x = 0 \).

    \begin{corollary}
        Sur un anneau principal \( A \), tout module \( E \) sans torsion de type fini est libre.
    \end{corollary}
    \begin{proof}
        On applique \cref{cor:structure d'un module de type fini sur un anneau principal} : \( E \simeq (A/\ideal{a_1}) \times \cdots \times (A/\ideal{a_n}) \). En supprimant les facteurs nuls, on peut supposer que \( \ideal{a_i} \neq A \) pour tout \( i \). Si \( \ideal{a_1} \neq (0) \), si \( a \) est un élément non nul de \( \ideal{a_1} \)\,, si \( x_1 \) est un élément non nul de \( A/\ideal{a_1} \), et si \( x = (x_1, 0, \cdots, 0) \), on a \( ax = 0 \) contrairement au fait que \( E \) est sans torsion. Donc \( \ideal{a_1} = (0) \), \( \ideal{a_i} = (0) \) pour tout \( i \) (car \( \ideal{a_i} \subset \ideal{a_1} \)), et \( E \) est isomorphe à \( A^n \).
    \end{proof}

    \begin{quote}
        L'hypothèse que \( E \) est de type fini est essentielle : par exemple \( \Q \) est un \( \Z \)‑module sans torsion non libre.
    \end{quote}

    \begin{corollary}
        Sur un anneau principal \( A \), tout module \( E \) de type fini est isomorphe à un produit fini de modules \( M_i \)\,, où chaque \( M_i \) est égal à \( A \) ou à un quotient \( A/Ap^s \) avec \( p \) premier.
    \end{corollary}
    \begin{proof}
        On utilise \cref{cor:structure d'un module de type fini sur un anneau principal}, et on décompose chaque facteur \( A/Aa \) où \( a \neq 0 \) au moyen \cref[de]{lem:reste chinois} : si \( a = u p_1^{s_1} \cdots p_r^{s^r} \) est la décomposition en facteurs premiers de \( A \), \( A/Aa \) est isomorphe au produit des \( A / A p_i^{s_i} \).
    \end{proof}

    \begin{corollary}\label{cor:groupe commutatif fini;élément d'ordre ppcm}
        Soit \( G \) un groupe commutatif fini. Il existe \( x \in G \) dont l'ordre est le p.p.c.m. des ordres des éléments de~\( G \).
    \end{corollary}
    \begin{proof}
        Un groupe commutatif est un \( \Z \)‑module (si on le note additivement). D'après \cref{cor:structure d'un module de type fini sur un anneau principal}, on a \( G \simeq \Z/a_1\Z \times \cdots \times \Z/a_n\Z \) où \( a_1 \mid a_2 \mid \cdots \mid a_n \). Aucun des \( a_i \) n'est nul, sinon \( G \) serait infini. Notons \( y \) la classe de \( 1 \) dans \( \Z/a_n\Z \)\,, et posons \( x = (0, \cdots, 0, y) \). L'ordre de \( x \) est évidemment \( a_n \). Pour \( z = (z_1, \cdots, z_n) \in G \), on a \( a_n z = 0 \) car \( a_i \) divise \( a_n \) pour tout \( i \) ; donc \( a_n \) est multiple de l'ordre de \( z \). L'élément cherché est donc~\( x \).
    \end{proof}

\section{Racine de l'unité dans un corps}\label{sec:racine de l'unité dans un corps}

    \begin{theorem}\label{thm:sous-groupe fini du groupe multiplicatif d'un corps}
        Soit \( K \) un corps. Tout sous-groupe fini \( G \) du groupe multiplicatif \( K^\times \) est formé de racines de l'unité, et est cyclique.
    \end{theorem}
    \begin{proof}
        En effet, d'après \cref{cor:groupe commutatif fini;élément d'ordre ppcm} \cref[de]{thm:base adaptée}, il existe \( z \in G \) dont l'ordre \( n \) est tel que \( y^n = 1 \) pour tout \( y \in G \). Comme un polynôme de degré \( n \) sur un corps (par exemple \( X^n-1 \)) a au plus \( n \) racines dans ce corps, le nombre d'éléments de \( G \) est au plus \( n \). Or, comme \( z \) est d'ordre \( n \), \( G \) contient les \( n \) éléments \( z, z^2, \cdots, z^n=1 \), qui sont distincts. Donc \( G \) est formé par ces élément, et est cyclique.
    \end{proof}

    Si un corps \( K \) contient \( n \) racines \( n \)‑ièmes de l'unité, celles-ci forment donc un groupe cyclique d'ordre \( n \) (isomorphe à \( \Z/n\Z \)). Un générateur de ce groupe s'appelle une \emph{racine primitive \( n \)‑ième de l'unité}\index{primitive (racine primitive\IndexLinebreak de l'unité)}\index{racine primitive de l'unité} ; toute racine \( n \)‑ième de l'unité est alors une puissance d'une telle racine primitive. D'après \cref{prop:indicateur d'Euler et Z/nZ}, le nombre de ces racines primitives est \( \varphi(n) \).

\section{Corps finis}\label{sec:corps finis}

    Soit \( K \) un corps. On a un unique homomorphisme d'anneaux \( \varphi \colon \Z \to K \) (défini par \( \varphi(n) = 1+1+\cdots+1 \), \( n \) fois, pour \( n \geqslant 0 \) et par \( \varphi(-n) = -\varphi(n) \)).
    \begin{itemize}
        \item Si \( \varphi \) est injectif, il identifie \( \Z \) à un sous-anneau de \( K \) ; alors \( K \) contient aussi le corps des fractions \( \Q \) de \( \Z \) ; on dit que \( K \) est \emph{de caractéristique \( 0 \)}.\index{caracteristique (d'un corps)@caractéristique (d'un corps)}
        \item Si \( \varphi \) n'est pas injectif, son noyau est un idéal \( p\Z \) où \( p \geqslant 0 \) ; alors \( \Z/p\Z \) s'identifie à un sous-anneau de \( K \) ; il est donc intègre, de sorte que \( p \) est un \emph{nombre premier} ; on dit que \( K \) est \emph{de caractéristique \( p \)}. Dans ce cas \( \Z/p\Z \) est un corps, qu'on note \( \F_p \).\index{caracteristique (d'un corps)@caractéristique (d'un corps)}
    \end{itemize}

    \begin{quote}
        Le sous-corps, \( \Q \) ou \( \F_p \)\,, de \( K \) est le plus petit sous-corps de \( K \) ; on l'appelle le \emph{sous-corps premier}\index{premier (corps)} de \( K \).

        Pour tout nombre premier \( p \) il existe des corps de caractéristique \( p \), par exemple \( \F_p = \Z/p\Z \).
    \end{quote}

    \begin{proposition}\label{prop:(x+y)^p = x^p+y^p}
        Si \( K \) est un corps de caractéristique \( p \neq 0 \), on a \( px = 0 \) pour tout \( x \in K \), et \( (x+y)^p = x^p+y^p \) quels que soient \( x,y \in K \).
    \end{proposition}
    \begin{proof}
        Pour \( x \in K \), on a \( p \cdot x = (p \cdot 1) \cdot x = 0 \cdot x = 0 \). D'autre part, d'après la formule du binôme, on a \( (x+y)^p = x^p+y^p+\sum_{j=1}^{p-1}\binom{p}{j}x^j y^{p-j} \) ; or le coefficient binomial \( \binom{p}{j} \) est un entier qui vaut \( \frac{p!}{j!(p-j)!} \) ; comme le nombre premier \( p \) figure dans son numérateur et ne figure pas dans son dénominateur, \( \binom{p}{j} \) est multiple de \( p \) pour \( 1 \leqslant j \leqslant p-1 \) ; le terme correspondant est donc nul.

        Par récurrence sur \( n \), on a \( (x+y)^{p^n} = x^{p^n} + y^{p^n} \) pour tout \( n \geqslant 0 \).
    \end{proof}

    \begin{theorem}\label{thm:corps fini}
        Soit \( K \) un corps fini. Posons \( q = \card(K) \). Alors :
        \begin{enumerate}
            \item \label{thm:corps fini;item1} La caractéristique de \( K \) est un nombre premier \( p \), \( K \) est un espace vectoriel de dimension finie \( s \) sur \( \F_p \)\,, et on a \( q = p^s \).
            \item \label{thm:corps fini;item2} Le groupe multiplicatif \( K^\times \) est cyclique d'ordre \( q-1 \).
            \item \label{thm:corps fini;item3} On a \( x^{q-1} = 1 \) pour tout \( x \in K^\times \), et \( x^q = x \) pour tout \( x \in K \).
        \end{enumerate}
    \end{theorem}
    \begin{proof}
        En effet, comme \( \Z \) est infini, \( K \) ne peut être de caractéristique \( 0 \). Donc il contient \( \F_p \)\,, avec \( p \) premier. Ainsi \( K \) est un espace vectoriel sur \( \F_p \) ; sa dimension \( s \) est finie, sinon \( K \) serait infini. En tant qu'espace vectoriel, \( K \) est isomorphe à \( (\F_p)^s \), donc a \( p^s \) éléments. L'assertion \ref{thm:corps fini;item2} résulte \cref[de]{thm:sous-groupe fini du groupe multiplicatif d'un corps}. On en déduit aussitôt~\ref{thm:corps fini;item3}.
    \end{proof}

    \begin{example}
        Appliquons \ref{thm:corps fini;item2} à \( \F_p \)\,, où \( p \) est premier : il existe un entier \( x \in \Z \) tel que \( 0 \leqslant x \leqslant p-1 \) et que tout entier \( y \) non multiple de \( p \) soit congru modulo \( p \) à une puissance de \( x \). On dit alors que \( x \) est une \emph{racine primitive modulo \( p \)}\index{primitive (racine primitive\IndexLinebreak modulo \( p \))}\index{racine primitive modulo \( p \)}. La recherche des racines primitives modulo \( p \) n'est nullement triviale. Par exemple il y a \( \varphi(6) = 2 \) racines primitives modulo \( 7 \) ; ce sont \( 3 \) et \( 5 \) (en effet on a \( 1^2 \equiv 6^2 \equiv 1 \pmod*{7} \) et \( 2^3 \equiv 4^3\equiv 1 \pmod*{7} \) ; seuls restent \( 3 \) et \( 5 \)).
    \end{example}

    \begin{remark}
        Il résulte de \ref{thm:corps fini;item3} qu'un corps fini \( K \) à \( q \) éléments est l'ensemble des racines du polynôme \( X^q - X \) (qui n'a que \( q \) racines). On peut en déduire que deux corps finis à \( q \) éléments sont isomorphes. On note souvent \( \F_q \) un corps fini à \( q \) éléments.
    \end{remark}

    À titre d'exercice et d'intermède, nous allons démontrer un élégant théorème relatif aux équations diophantiennes sur un corps fini :

    \begin{theorem}[Chevalley]\label{thm:Chevalley;corps fini}
        Soient \( K \) un corps fini, et \( F(X_1, \cdots, X_n) \) un polynôme homogène de degré \( d \) sur \( K \). On suppose \( d < n \). Il existe alors un point \( (x_1, \cdots, x_n) \in K^n \) distinct de l'origine \( (0, \cdots, 0) \) tel que \( F(x_1, \cdots, x_n) = 0 \).
    \end{theorem}

    \begin{quote}
        Étant donnés un corps \( K \) et un entier \( j \), on dit que \( K \) est un \emph{corps \( \mathscr{C}_j \)} si tout polynôme homogène sur \( K \) de degré \( d \) à \( n \) variables, tel que \( n > d^j \), admet un zéro non trivial (\ie. distinct de l'origine) dans \( K^n \). Les corps \( \mathscr{C}_0 \) sont les corps algébriquement clos. Le théorème de Chevalley exprime que les corps finis sont \( \mathscr{C}_1 \) (on dit aussi \textquote{quasi-algébriquement clos}\index{quasi-algebriquement clos (corps)@quasi-algébriquement clos\IndexLinebreak (corps)}). On montre que, si \( K \) est un corps \( \mathscr{C}_j \)\,, le corps \( K(T) \) des fractions rationnelles à une variable sur \( K \) et le corps \( K(\!\!(T)\!\!) \) des séries formelles à une variable sur \( K \) sont des corps \( \mathscr{C}_{j+1} \) (\cite{lang1962}). On s'est longtemps demandé si les corps \( p \)‑adiques sont \( \mathscr{C}_2 \)\,, et il s'avère qu'il n'en est rien (\cite{terjanian1966}).
    \end{quote}

    \begin{proof}[{Démonstration \cref[de]{thm:Chevalley;corps fini}}]
        Notons \( q \) le cardinal de \( K \) et \( p \) sa caractéristique (de sorte que \( q = p^s \)). Soit \( V \subset K^n \) l'ensemble des zéros de \( F \), \ie. des points \( x \coloneqq (x_1, \cdots, x_n) \in K^n \) tels que \( F(x) = 0 \) (nous employons, ici et dans la suite, l'écriture vectorielle où \( x \) désigne un point \( (x_1, \cdots, x_n) \) de \( K^n \)). D'après \cref{thm:corps fini}, \ref{thm:corps fini;item3}, on a \( F(x)^{q-1} = 0 \) pour \( x \in V \), et \( F(x)^{q-1} = 1 \) pour \( x \in K^n \setminus V \) ; ainsi le polynôme \( G(x) = F(x)^{q-1} \) est une \emph{fonction caractéristique} de \( K^n \setminus V \), à valeurs dans \( \F_p \). Le nombre modulo \( p \) de points de \( K^n \setminus V \) sera donc donné par la somme \( \sum_{x \in K^n} G(x) \) ; nous allons calculer cette somme et montrer qu'elle est \emph{nulle}. Alors \( \card(K^n \setminus V) \) sera multiple de~\( p \) ; comme \( \card(K^n) = q^n = p^{ns} \) est aussi multiple de \( p \), \( \card(V) \) sera multiple de~\( p \) ; comme \( V \) contient déjà l'origine, il contiendra nécessairement d'autres points, car \( p \geqslant 2 \) ; \cref{thm:Chevalley;corps fini} sera ainsi démontré.

        Calculons donc \( \sum_{x \in K^n} G(x) \). Le polynôme \( G \) est combinaison linéaire de monômes \( M_{\alpha}(X) = X_1^{\alpha_1} \cdots X_n^{\alpha_n} \) ; et on est ramenés à calculer \( \sum_{x \in K^n} M_{\alpha}(x) = \sum_{x \in K^n} x_1^{\alpha_1} \cdots x_n^{\alpha_n} = \bigl(\sum_{x_1 \in K} x_1^{\alpha_1}\bigr) \cdots \bigl(\sum_{x_n \in K} x_n^{\alpha_n}\bigr) \). Il s'agit donc de calculer des sommes de la forme \( \sum_{z \in K} z^{\beta} \) (\( \beta \in \N \)).
        \begin{enumerate-alph}
            \item \label{thm:Chevalley;corps fini;aux;item1} Pour \( \beta = 0 \), on a \( z^{\beta} = 1 \) pour tout \( z \in K \), et la somme vaut \( q = 0 \) ;
            \item \label{thm:Chevalley;corps fini;aux;item2} Pour \( \beta > 0 \), le terme \( 0^{\beta} \) est nul, et la somme se réduit à \( \sum_{z \in K^\times} z^{\beta} \). Or \( K^\times \) est un groupe cyclique d'ordre \( q-1 \) (\cref{thm:corps fini}, \ref{thm:corps fini;item3}) ; soit \( \omega \) un générateur de celui-ci. Alors \( \sum_{z \in K^\times} z^{\beta} = \sum_{j=0}^{q-2} {\omega}^{\beta j} \), qui est la somme d'une progression géométrique. Donc :
            \begin{enumerate}
                \manualitem{\emph{b'}$)$} \label{thm:Chevalley;corps fini;aux;item2-1} Si la raison \( \omega^{\beta} \) est \( \neq 1 \), c'est-à-dire si \( \beta \) n'est pas multiple de \( q-1 \), on a
                \[
                    \sum_{j=0}^{q-2} {\omega}^{\beta j} = \frac{{\omega}^{\beta (q-1)}-1}{{\omega}^{\beta}-1} = 0
                \]
                (car \( \omega^{q-1} = 1 \)).
                \manualitem{\emph{b}$'')$} \label{thm:Chevalley;corps fini;aux;item2-2} Si \( \omega^{\beta} = 1 \), c'est-à-dire si \( \beta \) est multiple de \( q-1 \), on a
                \[
                    \sum_{j=0}^{q-2} {\omega}^{\beta j} = q-1.
                \]
            \end{enumerate}
        \end{enumerate-alph}
        Il résulte de \ref{thm:Chevalley;corps fini;aux;item1}, \ref{thm:Chevalley;corps fini;aux;item2-1} et \ref{thm:Chevalley;corps fini;aux;item2-2} que \( \sum_{x \in K^n} x_1^{\alpha_1} \cdots x_n^{\alpha_n} \) est nul \emph{sauf si} tous les \( \alpha_i \) sont \( > 0 \) et multiples de \( q-1 \). Le degré \( \alpha_1 + \cdots + \alpha_n \) du monôme est, dans ce cas, \( \geqslant (q-1)n \). Mais, comme \( G = F^{q-1} \), \( G \) est de degré \( (q-1)d \), et on a \( (q-1)d < (q-1)n \) d'après l'hypothèse. On a donc \( \sum_{x \in K^n} M_{\alpha}(x) = 0 \) pour tout monôme \( M_{\alpha}(x) \) qui figure dans~\( G \) avec coefficient non nul. D'où, par addition, \( \sum_{x \in K^n} G(x) = 0 \). Nous avons vu que cette relation entraîne notre conclusion.
    \end{proof}

    \begin{quote}
        On remarquera qu'il aurait, au lieu de supposer \( F \) homogène, suffi de supposer \( F \) sans terme constant. Naturellement l'inégalité \emph{strict} \( d < n \) entre degré et nombre de variables et essentielle. Par exemple la \emph{norme} de \( \F_{q^n} \) à \( \F_q \) (\cf.~\cref{sec:normes et traces}) fournit un polynôme homogène de degré \( n \) et à \( n \) variables sur \( \F_q \) qui n'a d'autre zéro que l'origine.

        \textsc{\textbf{Exemple}}.
        Une forme quadratique à \( 3 \) variables sur un corps \emph{fini} \( K \) \textquote{représente \( 0 \)}\index{represente zero@représente zéro} (\ie. a un zéro non trivial). En passant de \( K^3 \) au plan projectif \( \P_2(K) \), ceci veut dire qu'une \emph{conique} sur \( K \) admet un point rationnel sur \( K \) (\ie. dont les coordonnées homogènes peuvent être choisies dans \( K \)). L'exemple de la conique \( x^2+y^2+z^2 = 0 \) sur \( \R \) (resp. \( x^2+y^2-3z^2 = 0 \) sur \( \Q \) ; pour s'assurer que \( x^2+y^2-3z^2 = 0 \) n'a pas de solution non trivial dans \( \Q \)\,, on se ramène au cas où \( x,y,z \) sont des entiers premiers entre eux, et on réduit modulo \( 4 \)) montre qu'il ne s'agit pas d'une propriété vraie sur tout corps.
    \end{quote}


\chapter{Éléments entiers sur un anneau, éléments algébriques sur un corps}

Pour les nombres complexes, on va s'occuper dans ce livre des nombres \emph{algébriques}, c'est-à-dire ceux qui satisfont à une équation de la forme
\[
    x^n + a_{n-1} x^{n-1} + \cdots + a_1 x + a_0 = 0
\]
où les \( a_i \) sont des nombres rationnels. Lorsque les \( a_i \) sont des nombres entiers (\( a_i \in \Z \)) le nombre algébrique \( x \) est appelé un \emph{entier algébrique} ; ainsi \( \sqrt{2} \), \( \sqrt{3} \), \( \i \)\,, \( \e^{2\i\pi/5} \) sont des entiers algébriques. Il n'est pas évident à priori que des sommes ou des produits de nombres algébriques (resp. d'entiers algébriques) sont encore des nombres algébriques (resp. des entiers algébriques). Regardons l'exemple de \( x = \sqrt{2}+\sqrt{3} \) ; on calcule \( x^2 = 2+3+2\sqrt{6} \) ; on isole la racine carrée et il vient \( x^2-5 = 2\sqrt{6} \) ; encore une élévation au carré, \( (x^2-5)^2 = 24 \), ce qui montre que \( x \) est un entier algébrique. Le lecteur pourra s'exercer sur \( \sqrt[3]{5}+\sqrt[5]{7} \), et sera convaincu que la série d'astuces qui mène au résultat n'est pas facilement généralisable.

Pour surmonter cette difficulté, les algébristes des siècles passés, Dedekind en particulier, ont eu l'idée de \textquote{linéariser} le problème, c'est-à-dire d'y introduire des modules. C'est ce que nous allons faire. Le remplacement de \( \Z \) (ou \( \Q \)) par un anneau commutatif quelconque ne coûte aucun effort supplémentaire, et sera fort utile pour la suite. Nous commencerons par le cas général des éléments entiers sur un anneau, et particulariserons ensuite aux éléments algébriques sur un corps.

\section{Éléments entiers sur un anneau}

    \begin{theorem}\label{thm:conditions d'être entier}
        Soient \( R \) un anneau, \( A \) un sous-anneau de \( R \), et \( x \) un élément de \( R \). Les propriétés suivantes sont équivalentes :
        \begin{enumerate}
            \item \label{thm:conditions d'être entier;item1} Il existe \( a_0, \cdots, a_{n-1} \in A \) tels que
            \begin{equation}
                x^n + a_{n-1} x^{n-1} + \cdots + a_1 x + a_0 = 0
            \end{equation}
            (autrement dit \( x \) est racine d'un polynôme unitaire sur \( A \)).
            \item \label{thm:conditions d'être entier;item2} L'anneau \( A[x] \) est un \( A \)‑module de type~fini.
            \item \label{thm:conditions d'être entier;item3} Il existe un sous-anneau \( B \) de \( R \), contenant \( A \) et \( x \), et qui est un \( A \)‑module de type~fini.
        \end{enumerate}
    \end{theorem}
    \begin{proof}
        Montrons que \ref{thm:conditions d'être entier;item1} implique \ref{thm:conditions d'être entier;item2}. Notons \( M \) le sous-\( A \)‑module de \( R \) engendré par \( 1, x, \cdots, x^{n-1} \). Par \ref{thm:conditions d'être entier;item1} on a \( x^n \in M \). Par récurrence sur \( j \) montrons que \( x^{n+j} \in M \) ; en effet, par multiplication par \( x^j \), \ref{thm:conditions d'être entier;item1} donne \( x^{n+j} = - a_{n-1}x^{n+j-1} - \cdots - a_0 x^j \). Comme \( A[x] \) est le \( A \)‑module engendré par les \( x^k \) (\( k \geqslant 0 \)), on a \( A[x] = M \), ce qui démontre~\ref{thm:conditions d'être entier;item2}.

        L'implication \ref{thm:conditions d'être entier;item2}~\( \Rightarrow \)~\ref{thm:conditions d'être entier;item3} est triviale. Montrons que \ref{thm:conditions d'être entier;item3} implique \ref{thm:conditions d'être entier;item1}. Soit \( (y_1, \cdots, y_n) \) un système générateur fini du \( A \)‑module \( B \) ; on a ainsi \( B = Ay_1 + \cdots + Ay_n \). Comme \( x \in B \), \( y_i \in B \) et que \( B \) est un sous-anneau de \( R \), on a \( x y_i \in B \), de sorte qu'il existe des éléments \( a_{ij} \) de \( A \) tels que \( x y_i = \sum_{j=1}^n a_{ij}y_j \). Ceci s'écrit aussi
        \[
            \sum_{j=1}^n (\delta_{ij}x - a_{ij})y_j = 0 \quad \text{(\( i = 1, \cdots, n \))}.
        \]
        On obtient ainsi un système de \( n \) équations linéaires homogènes en \( (y_1, \cdots, y_n) \). Si on note \( d \) le déterminant \( \det(\delta_{ij}x - a_{ij}) \), le calcul menant aux formules de Cramer montre qu'on a \( d y_i = 0 \) pour tout \( i \). Comme \( B = \sum_i A y_i \)\,, on en déduit \( d B = 0 \), d'où \( d = d \cdot 1 = 0 \) car \( B \) a un élément unité. Or, si on développe le déterminant
        \[
            d = \det(\delta_{ij}x - a_{ij}),
        \]
        on obtient une équation de la forme \( P(x) = 0 \) où \( P \) est un polynôme de degré \( n \) sur \( A \) ; ce polynôme est unitaire car le terme en \( x^n \) provient uniquement du produit \( \prod_{i=1}^n (x-a_{ii}) \) des éléments de la diagonale principale ; ainsi \ref{thm:conditions d'être entier;item1} est vraie.
    \end{proof}

    \begin{definition}[élément entier ; équation de dépendance intégrale]
        Soient \( R \) un anneau et \( A \) un sous-anneau de \( R \). Un élément \( x \) de \( R \) est dit \emph{entier}\index{entier (element entier sur un anneau)@entier (élément entier\IndexLinebreak sur un anneau)} sur \( A \) s'il satisfait aux conditions équivalentes \ref{thm:conditions d'être entier;item1}, \ref{thm:conditions d'être entier;item2}, \ref{thm:conditions d'être entier;item3} \cref[de]{thm:conditions d'être entier}. Soit \( P \in A[X] \) un polynôme unitaire tel que \( P(x) = 0 \) (polynôme dont l'existence est affirmée par~\ref{thm:conditions d'être entier;item1}) ; la relation \( P(x) = 0 \) est appelée une \emph{équation de dépendance intégrale}\index{dependance integrale (equation de)@dépendance intégrale\IndexLinebreak (équation de)}\index{equation de dependance integrale@équation de dépendance\IndexLinebreak intégrale} de \( x \) sur~\( A \).
    \end{definition}

    \begin{quote}
        \textsc{\textbf{Exemple}}.
        L'élément \( x = \sqrt{2} \) de \( \R \) est entier sur \( \Z \) ; une équation de dépendance intégrale est donnée par \( x^2-2 = 0 \).
    \end{quote}

    \begin{proposition}\label{prop:entier;tower}
        Soient \( R \) un anneau, \( A \) un sous-anneau de \( R \), et \( (x_i)_{1 \leqslant i \leqslant n} \) une famille finie d'éléments de \( R \). Si, pour tout \( i \), \( x_i \) est entier sur \( A[x_1, \cdots, x_{i-1}] \) (en particulier si tous les \( x_i \) sont entiers sur \( A \)), alors \( A[x_1, \cdots, x_n] \) est un \( A \)‑module de type~fini.
    \end{proposition}
    \begin{proof}
        Raisonnons par récurrence sur \( n \). Pour \( n=1 \) c'est l'assertion \ref{thm:conditions d'être entier;item2} \cref[de]{thm:conditions d'être entier}. Supposons la proposition vraie jusqu'à \( n-1 \). Alors \( B = A[x_1, \cdots, x_{n-1}] \) est un \( A \)‑module de type fini, soit \( B = \sum_{j=1}^p A b_j \). Par application du cas \( n=1 \), \( A[x_1, \cdots, x_n] = B[x_n] \) est un \( B \)‑module de type fini, soit \( \sum_{k=1}^q B c_k \). On~a~alors
        \[
            A[x_1, \cdots, x_n] = \sum_{k=1}^q B c_k = \sum_{k=1}^q \left( \sum_{j=1}^p A b_j \right) c_k = \sum_{j,k} A b_j c_k.
        \]
        Ainsi \( (b_j c_k) \) est un système générateur fini du \( A \)‑module \( A[x_1, \cdots, x_n] \).
    \end{proof}

    \begin{corollary}\label{cor:éléments entiers forment un anneau;pré}
        Soient \( R \) un anneau, \( A \) un sous-anneau de \( R \), \( x \) et \( y \) des éléments de \( R \) entiers sur \( A \). Alors \( x+y \), \( x-y \) et \( xy \) sont entiers sur~\( A \).
    \end{corollary}
    \begin{proof}
        En effet on a \( x+y, x-y, xy \in A[x,y] \). D'après \cref{prop:entier;tower}, \( A[x,y] \) est un \( A \)‑module de type fini. Donc, d'après le \ref{thm:conditions d'être entier;item3} \cref[de]{thm:conditions d'être entier}, \( x+y \), \( x-y \) et \( xy \) sont entiers sur~\( A \).
    \end{proof}

    \begin{corollary}\label{cor:éléments entiers forment un anneau}
        Soient \( R \) un anneau, \( A \) un sous-anneau de \( R \). L'ensemble \( A' \) des éléments de \( R \) qui sont entiers sur \( A \) est un sous-anneau de \( R \), qui contient~\( A \).
    \end{corollary}
    \begin{proof}
        En effet, \( A' \) est un sous-anneau de \( R \) d'après \cref{cor:éléments entiers forment un anneau;pré} ; il contient \( A \) car tout \( a \in A \) est racine du polynôme unitaire \( X-a \), donc est entier sur~\( A \).
    \end{proof}

    \begin{definition}[fermeture intégrale ; clôture intégrale ; anneau entier]\index{integrale (fermeture, cloture)@intégrale (fermeture, clôture)}
        Soient \( R \) un anneau, \( A \) un sous-anneau de \( R \) ; l'anneau \( A' \) des éléments de \( R \) entiers sur \( A \) s'appelle la \emph{fermeture intégrale}\index{fermeture integrale@fermeture intégrale} de \( A \) dans \( R \). Soient \( A \) un anneau intègre et \( K \) son corps des fractions ; la fermeture intégrale de \( A \) dans \( K \) s'appelle la \emph{clôture intégrale}\index{cloture integrale@clôture intégrale} de~\( A \). Soient \( B \) un anneau et \( A \) un sous-anneau de \( B \) ; on dit que \( B \) est entier sur \( A \) si tout élément de \( B \) est \emph{entier}\index{entier (anneau entier\IndexLinebreak sur un autre)} sur \( A \) (autrement dit, si la fermeture intégrale de~\( A \) dans~\( B \) est \( B \) lui-même).
    \end{definition}

    \begin{proposition}[de transitivité]\label{prop:anneau entier;transitivité}
        Soient \( C \) un anneau, \( B \) un sous-anneau de \( C \) et \( A \) un sous-anneau de \( B \). Si \( B \) est entier sur \( A \) et \( C \) est entier sur \( B \), alors \( C \) est entier sur~\( A \).
    \end{proposition}
    \begin{proof}
        En effet, soit \( x \in C \). Il est entier sur \( B \), et on a donc une équation de dépendance intégrale \( x^n + b_{n-1}x^{n-1} + \cdots + b_0 = 0 \) avec \( b_i \in B \). Posons \( B' = A[b_0, \cdots, b_{n-1}] \) ; alors \( x \) est aussi entier sur \( B' \). Comme \( B \) est entier sur \( A \), les \( b_i \) sont entiers sur \( A \) ; donc, d'après \cref{prop:entier;tower}, \( B'[x] = A[b_0, \cdots, b_{n-1}, x] \) est un \( A \)‑module de type fini. Par le \ref{thm:conditions d'être entier;item3} \cref[de]{thm:conditions d'être entier}, on en conclut que \( x \) est entier sur \( A \). Donc \( C \) est entier sur \( A \).
    \end{proof}

    \begin{proposition}\label{prop:anneau entier et corps}
        Soient \( B \) un anneau intègre et \( A \) un sous-anneau de \( B \), tel que \( B \) soit entier sur \( A \). Pour que \( B \) soit un corps, il faut et il suffit que \( A \) soit un corps.
    \end{proposition}
    \begin{proof}
        Supposons que \( A \) soit un corps, et soit \( b \in B \), \( b \neq 0 \). Alors \( A[b] \) est un espace vectoriel de dimension \emph{finie} sur \( A \) (\cref{thm:conditions d'être entier}, \ref{thm:conditions d'être entier;item2}). D'autre part \( y \mapsto by \) est une application \( A \)‑linéaire de \( A[b] \) dans lui-même, qui est injective car \( A[b] \) est intègre et car \( b \neq 0 \). Elle est donc surjective, et il existe \( b' \in A[b] \) tel que \( bb' = 1 \). Donc \( b \) est inversible dans \( B \) et \( B \) est un corps\footnote{Le même type de raisonnement, utilisant l'homothétie \( y \mapsto by \), montre que tout anneau intègre \emph{fini} est un corps.}.

        Inversement, supposons que \( B \) soit un corps, et soit \( a \in A \), \( a \neq 0 \). Alors \( a \) admet un inverse \( a^{-1} \in B \), qui satisfait à une équation de dépendance intégrale
        \[
            a^{-n} + a_{n-1} a^{-n+1} + \cdots + a_1 a^{-1} + a_0 = 0, \quad a_i \in A.
        \]
        En multipliant par \( a^{n-1} \), on obtient \( a^{-1} = -(a_{n-1} + \cdots + a_1 a^{n-2} + a_0 a^{n-1}) \), d'où \( a^{-1} \in A \), de sorte que \( A \) est un corps.
    \end{proof}

\section{Anneaux intégralement clos}

    \begin{definition}[anneaux intégralement clos]
        On dit qu'un anneau \( A \) est \emph{intégralement clos}\index{integralement clos (anneau)@intégralement clos (anneau)} s'il est intègre et si sa clôture intégrale est \( A \) lui-même.
    \end{definition}

    Autrement dit, tout élément \( x \) du corps des fractions \( K \) de \( A \), qui est entier sur \( A \), est élément de~\( A \).

    \begin{example*}
        Soient \( A \) un anneau intègre et \( K \) son corps des fractions.  Alors la clôture intégrale \( A' \) de \( A \) (c'est-à-dire la fermeture intégrale de \( A \) dans \( K \)) est un anneau intégralement clos. En effet la clôture intégrale de \( A' \) est entière sur \( A' \), donc sur \( A \) (\cref{prop:anneau entier;transitivité}) ; elle est donc égale à~\( A' \).
    \end{example*}

\enlargethispage{-\baselineskip}
    \begin{example*}\label{exp:anneau principal est intégralement clos}
        Tout anneau principal est intégralement clos.
    \end{example*}
    \begin{proof}
        En effet un anneau principal \( A \) est intègre par définition. Soit \( x \) un élément entier sur \( A \) de son corps des fractions ; on a une équation de dépendance intégrale
        \begin{equation}\label{eq:anneau principal est intégralement clos;aux}
            x^n + a_{n-1} x^{n-1} + \cdots + a_1 x + a_0 = 0, \quad a_i \in A.
        \end{equation}
        Or on peut écrire \( x = a/b \), avec \( a,b \in A \) \emph{premiers entre eux}. D'où, en reportant dans \eqref{eq:anneau principal est intégralement clos;aux} et en multipliant par \( b^n \) :
        \[
            a^n + b(a_{n-1} a^{n-1} + \cdots + a_1 a b^{n-2} + a_0 b^{n-1}) = 0.
        \]
        Ainsi \( b \) divise \( a^n \) ; comme il est premier avec \( a \), l'application répétée du lemme d'Euclide montre qu'il divise \( a \) ; donc \( x = a/b \in A \), et \( A \) est intégralement clos.
    \end{proof}

    \begin{quote}
        On notera qu'on a seulement utilisé les propriétés multiplicatives des anneaux principaux (éléments premiers entre eux, lemme d'Euclide). Le raisonnement montre aussi que tout anneau \emph{factoriel} est intégralement clos.
    \end{quote}

\section{Éléments algébriques sur un corps ; extensions algébriques}\label{sec:éléments algébriques sur un corps;extensions algébriques}

    \begin{definition}[élément algébrique sur un corps]\label{def:élément algébrique sur un corps}\index{algebrique (element algebrique sur un corps)@algébrique (élément algébrique sur\IndexLinebreak un corps)}
        Soient \( R \) un anneau, et \( K \) un sous-corps de \( R \). On dit qu'un élément \( x \) de \( R \) est \emph{algébrique} sur \( K \) s'il existe des éléments non tous nuls \( a_0, \cdots, a_n \) de \( K \) tels que \( a_n x^n + \cdots + a_1 x + a_0 = 0 \).
    \end{definition}

    Il revient au même de dire que les monômes \( (x^j)_{j \in \N} \) sont linéairement dépendants sur \( K \). Un élément non algébrique sur \( K \) est dit \emph{transcendant}\index{transcendant (element transcendant sur un corps)@transcendant (élément transcendant\IndexLinebreak sur un corps)} sur \( K \) ; ceci veut dire que les monômes \( (x^j)_{j \in \N} \) sont linéairement indépendants sur~\( K \).

    Dans la relation \cref[de]{def:élément algébrique sur un corps}, on peut supposer \( a_n \) non nul ; il admet alors un inverse \( a_n^{-1} \) car \( K \) est un corps ; en multipliant par \( a_n^{-1} \) on obtient une équation de dépendance intégrale. Donc :
    \begin{enumerate-Roman}\itshape
        \item Sur un corps, algébrique \( = \) entier.
    \end{enumerate-Roman}
    On peut donc appliquer la théorie des éléments premiers ; par exemple, pour \( K \subset R \) et \( x \in R \), le \ref{thm:conditions d'être entier;item2} \cref[de]{thm:conditions d'être entier} donne :
    \begin{enumerate-Roman}[resume]\itshape
        \item \( x \) algébrique sur \( K \) \( \iff \) \( [K[x]:K] \) finie.
    \end{enumerate-Roman}

    On dit qu'un anneau \( R \) contenant un corps \( K \) est \emph{algébrique} sur \( K \) si tout élément de \( R \) est algébrique sur \( K \) ; si \( R \) lui-même est un corps, on dit alors que \( R \) est une \emph{extension algébrique}\index{algebrique (extension)@algébrique (extension)}\index{extension!algebrique@algébrique} de~\( K \).

    Étant donnés un corps \( L \) et un sous-corps \( K \) de \( L \), la dimension \( [L:K] \) s'appelle aussi le \emph{degré} de~\( L \) sur~\( K \). Le \ref{thm:conditions d'être entier;item3} \cref[de]{thm:conditions d'être entier} donne alors :
    \begin{enumerate-Roman}[resume]\itshape
        \item Si le degré de~\( L \) sur~\( K \) est fini, \( L \) est extension algébrique de~\( K \).
    \end{enumerate-Roman}

    On appelle \emph{corps de nombres algébriques} (ou \emph{corps de nombres})\index{corps de nombres, ou corps de nombres algebriques@corps de nombres, ou corps de\IndexLinebreak nombres algébriques} toute extension de degré fini de~\( \Q \).

    \begin{proposition}[multiplicativité des degrés]\label{prop:extension algébrique;multiplicativité des degrés}
        Soient \( K \) un corps, \( L \) une extension algébrique de \( K \) et \( M \) une extension algébrique de \( L \). Alors \( M \) est une extension algébrique de \( K \). De plus, \( [M:K] = [M:L] [L:K] \).
    \end{proposition}
    \begin{proof}
        La première assertion est un cas particulier \cref[de]{prop:anneau entier;transitivité}. De plus, si \( (x_i)_{i \in I} \) est une base de~\( L \) sur~\( K \) et \( (y_j)_{j \in J} \) une base de \( M \) sur \( L \), alors \( (x_i y_j)_{(i,j) \in I \times J} \) est une base de \( M \) sur \( K \) : en effet, c'est un système générateur, comme dans \cref{prop:entier;tower} ; d'autre part une relation \( \sum_{i,j} a_{ij} x_i y_j = 0 \) avec \( a_{ij} \in K \) donne \( \sum_j \bigl( \sum_i a_{ij} x_i \bigr) y_j = 0 \), d'où \( \sum_i a_{ij} x_i = 0 \) pour tout \( j \) (car \( \sum a_{ij} x_i \in L \)), et par conséquent \( a_{ij} = 0 \) pour tous \( i,j \). Ceci démontre la formule de multiplicativité des degrés.
    \end{proof}

    \begin{proposition}\label{prop:éléments algébriques forment un anneau}
        Soient \( R \) un anneau, et \( K \) un sous-corps de \( R \). Alors :
        \begin{enumerate}
            \item \label{prop:éléments algébriques forment un anneau;item1} L'ensemble \( K' \) des éléments de \( R \) algébriques sur \( K \) est un sous-anneau de \( R \) contenant \( K \).
            \item \label{prop:éléments algébriques forment un anneau;item2} Si \( R \) est intègre, \( K' \) est un sous-corps de \( R \).
        \end{enumerate}
    \end{proposition}
    \begin{proof}
        En effet \ref{prop:éléments algébriques forment un anneau;item1} est un cas particulier \cref[de]{cor:éléments entiers forment un anneau} \cref[de]{prop:entier;tower}, et \ref{prop:éléments algébriques forment un anneau;item2} résulte \cref[de]{prop:anneau entier et corps}.
    \end{proof}

    Nous allons maintenant étudier de plus près les éléments algébriques sur un corps. Soient \( R \) un anneau, \( K \) un sous-corps de \( R \) et \( x \) un élément de \( R \). Il existe un homomorphisme \( \varphi \) et un seule de l'anneau de polynômes \( K[X] \) dans \( R \) tel que \( \varphi(X) = x \), et que \( \varphi(a) = a \) pour tout \( a \in K \) ; l'image de \( \varphi \) est \( K[x] \). La définition des éléments algébriques se traduit par :
    \begin{enumerate-Roman}[resume]\itshape
        \item \label{élément algébrique;item4} \( x \) algébrique sur \( K \) \( \iff \) \( \Ker(\varphi) \neq (0) \).
    \end{enumerate-Roman}
    Si \( x \) est algébrique sur \( K \), l'idéal \( \Ker(\varphi) \) est un idéal \emph{principal} \( (F(X)) \) (car \( K[X] \) est un anneau principal), engendré par un polynôme non nul \( F(X) \), qu'on peut supposer \emph{unitaire} car \( K \) est un corps ; ce polynôme unitaire est déterminé de façon unique par \( K \) et \( x \), et on l'appelle le \emph{polynôme minimal}\index{minimal (polynome)@minimal (polynôme)}\index{polynome minimal@polynôme minimal} de \( x \) sur \( K \). Traduisons sa définition :
    \begin{enumerate-Roman}[resume]\itshape
        \item \label{élément algébrique;item5} Soient \( F(X) \) le polynôme minimal de \( x \) sur \( K \), et \( G(X) \in K[X] \) ; pour qu'on ait \( G(X) = 0 \), il faut et il suffit que \( F(X) \) divise \( G(X) \) dans \( K[X] \).
    \end{enumerate-Roman}
    De plus, par passage au quotient, on obtient un \emph{isomorphisme canonique} :
    \begin{enumerate-Roman}[resume]\itshape
        \item \label{élément algébrique;item6} \( K[X]/(F(X)) \bijection K[x] \).
    \end{enumerate-Roman}

    Avec les même notations, supposons toujours \( x \) algébrique sur \( K \), et soit \( F(X) \) son polynôme minimal ; en appliquant \ref{élément algébrique;item5} et \cref{prop:anneau entier et corps}, on obtient les équivalences :
    \begin{enumerate-Roman}[resume]\itshape
        \item \( K[x] \) corps \( \iff \) \( K[x] \) intègre \( \iff \) \( F(X) \) irréductible.
    \end{enumerate-Roman}
    Inversement, soient \( K \) un corps et \( F(X) \in K[X] \) un polynôme irréductible ; alors \( K[X]/(F(X)) \) est un corps contenant \( K \), et, en notant \( x \) la classe de \( X \) dans ce corps, on a \( F(x) = 0 \). Ainsi \( F(X) \) est divisible par \( X-x \) sur ce corps \( K[X] \). Plus généralement :
    \begin{proposition}\label{prop:existence d'une extension algébrique telle que le polynôme se décompose}
        Soient \( K \) un corps, et \( P(X) \in K[X] \) un polynôme non constant. Il existe une extension algébrique \( K' \) de degré fini de \( K \) telle que \( P(X) \) se décompose en facteurs du premier degré dans \( K'[X] \).
    \end{proposition}
    \begin{proof}
        On raisonne par récurrence sur le degré \( d \) de \( P(X) \). C'est évident pour \( d = 1 \). Supposons l'assertion démontrée jusqu'au degré \( d-1 \). Soit \( F(X) \) un facteur irréductible de \( P(X) \). On vient de voir qu'il existe une extension \( K'' \) de degré fini de \( K \) (à savoir \( K[X]/F(X) \)) et un élément \( x \in K'' \) tels que \( F(X) \) soit multiple de \( X-x \) dans \( K''[X] \). On a alors \( P(X) = (X-x)P_1(X) \) avec \( P_1(X) \in K''[X] \). D'après l'hypothèse de récurrence, \( P_1(X) \) se décompose en facteurs du premier degré sur une extension \( K' \) de degré fini de \( K'' \). Alors \( K' \) est une extension de degré fini de \( K \) (\cref{prop:extension algébrique;multiplicativité des degrés}), et \( P(X) \) se décompose en facteurs du premier degré dans~\( K'[X] \).
    \end{proof}

    \begin{remark}[corps algébriquement clos]\index{algebriquement clos (corps)@algébriquement clos (corps)}
        On dit qu'un corps \( K \) est \emph{algébriquement clos} si \emph{tout} polynôme non constant \( P(X) \in K[X] \) se décompose en facteurs du premier degré dans \( K[X] \) ; il suffit pour cela, par récurrence sur le degré, que tout polynôme non constant \( P(X) \in K[X] \) admette une racine \( x \in K \). Par application \textquote{transfinie} \cref[de]{prop:existence d'une extension algébrique telle que le polynôme se décompose} (c'est-à-dire en combinant \cref{prop:existence d'une extension algébrique telle que le polynôme se décompose} et le théorème de Zorn ; \cf.~\cite{bourbaki1}, chap.\,V et \cite{zariski1958}, chap.\,II), on démontre que tout corps est un sous-corps d'un corps algébriquement clos.

        On démontre en Analyse, par des méthodes variées\footnote{Par une démonstration utilisant les propriétés des fonctions continues sur un espace compact, voir \cite{choquet1963}. Pour une démonstration utilisant les propriétés des fonctions holomorphe d'une variable complexe, voir \cite{cartan1962}. Nous donnerons en appendice à ce chapitre une démonstration plus algébrique, mais n'utilisant d'autre Analyse que les propriétés les plus simples des nombres réels.}, que le corps \( \C \) des \emph{nombres complexes} est algébriquement clos. Cela suffira à nos besoins.
    \end{remark}

\section{Éléments conjugués, corps conjugués}\label{sec:éléments conjugués;corps conjugués}

    Étant donnés deux corps \( L \) et \( L' \) contenant un corps \( K \), on appelle \emph{\( K \)‑isomorphisme} de \( L \) sur \( L' \) tout isomorphisme \( \varphi \colon L \to L' \) tel que \( \varphi(a) = a \) pour tout \( a \in K \) ; dans ces conditions on dit que \( L \) et \( L' \) sont \emph{\( K \)‑isomorphes}, ou (s'ils sont algébriques sur \( K \)) sont des \emph{corps conjugués}\index{conjugues (corps)@conjugués (corps)} sur \( K \). Étant données deux extensions \( L, L' \) de \( K \), on dit que deux éléments \( x \in L \) et \( x' \in L' \) sont \emph{conjugués}\index{conjugues (elements)@conjugués (éléments)} sur \( K \) s'il existe un \( K \)‑isomorphisme \( \varphi \colon K(x) \to K(x') \) tel que \( \varphi(x) = x' \) ; alors \( \varphi \) est unique. Ceci signifie que, ou bien \( x \) et \( x' \) sont tous deux transcendant sur \( K \), ou bien \( x \) et \( x' \) sont tous deux algébriques sur \( K \) et admettent le même polynôme minimal (\cf.~\cref{sec:éléments algébriques sur un corps;extensions algébriques}, \ref{élément algébrique;item5}).

    \begin{example}
        Soit \( F(X) \) un polynôme \emph{irréductible} de degré \( n \) sur \( K \), et \( x_1, \cdots, x_n \) ses racines dans une extension \( K' \) de \( K \) (\cref{prop:existence d'une extension algébrique telle que le polynôme se décompose}). Alors les \( x_i \) sont deux à deux conjugués sur~\( K \) (\cref{sec:éléments algébriques sur un corps;extensions algébriques}, \ref{élément algébrique;item6}), et les corps \( K[x_i] \) sont aussi deux à deux conjugués sur~\( K \).
    \end{example}

    \begin{lemma}\label{lem:racines conjugués sont distinctes}
        Soient \( K \) un corps de caractéristique \( 0 \) ou un corps fini, \( F(X) \in K[X] \) un polynôme unitaire irréductible, et \( F(X) = \prod_{i = 1}^n (X-x_i) \) sa décomposition en facteurs du premier degré dans une extension \( K' \) de \( K \) (\cref{prop:existence d'une extension algébrique telle que le polynôme se décompose}). Alors les \( n \) racines \( x_1, \cdots, x_n \) de \( F(X) \) sont distinctes.
    \end{lemma}
    \begin{proof}
        Raisonnons par l'absurde. Dans le cas contraire, \( F(X) \) admettrait une racine \emph{multiple} \( x \), qui serait donc aussi racine du polynôme dérivé \( F'(X) \) ; alors \( F(X) \) diviserait \( F'(X) \) (\cref{sec:éléments algébriques sur un corps;extensions algébriques}, \ref{élément algébrique;item4}). Comme \( \deg F' < \deg F \), ceci implique que \( F'(X) \) est le polynôme nul. Or si
        \[
            F(X) = X^n + a_{n-1} X^{n-1} + \cdots + a_0, \quad a_i \in K,
        \]
        on a
        \[
            F'(X) = n X^{n-1} + (n-1) a_{n-1} X^{n-2} + \cdots + a_1.
        \]
        On a donc \( n \cdot 1 = 0 \) et \( j \cdot a_j = 0 \) pour \( j = 1, \cdots, n-1 \). Ceci est impossible en caractéristique~\( 0 \).

        En caractéristique \( p \neq 0 \), ceci implique que \( p \) divise \( n \), et qu'on a \( a_j = 0 \) pour \( j \) non multiple de \( p \) (on rappelle que \( p \) est un nombre premier). Ainsi \( F(X) \) est de la forme
        \[
            F(X) = X^{q p} + b_{q-1} X^{(q-1) p} + \cdots + b_1 X^p + b_0, \quad b_i \in K.
        \]
        Si chacun des \( b_i \) est une \emph{puissance \( p \)‑ième}, soit \( b_i = c_i^p \) avec \( c_i \in K \), on a \( F(X) = (X^q + c_{q-1} X^{q-1} + \cdots + c_0)^p \) (\cref{prop:(x+y)^p = x^p+y^p}), et \( F(X) \) n'est pas irréductible. Or, si \( K \) est un corps fini et si on note \( p \) (\( \neq 0 \)) sa caractéristique, l'application \( x \mapsto x^p \) de \( K \) dans \( K \) est injective (car \( x^p = y^p \) implique \( x^p - y^p = 0 \), \( (x-y)^p = 0 \) et \( x-y = 0 \)) ; elle est donc surjective car \( K \) est fini. On a donc une contradiction.
    \end{proof}

    \begin{quote}
        Les corps \( K \) de caractéristique \( p \neq 0 \) tels que \( x \mapsto x^p \) soit surjective (\ie. que tout élément de \( K \) soit une puissance \( p \)‑ième) sont appelés les corps \emph{parfaits}\index{corps parfait}\index{parfait (corps)} ; on vient de montrer que tout corps fini est parfait ; on convient qu'un corps de caractéristique \( 0 \) est parfait. Nous avons démontré que la conclusion \namecref[de]{lem:racines conjugués sont distinctes} reste vraie sous la seule hypothèse que \( K \) est un corps parfait. Le corps \( \F_p(T) \) des fractions rationnelles à une variable sur \( \F_p \) n'est pas parfait, car la variable \( T \) n'est pas une puissance \( p \)‑ième dans \( \F_p(T) \).
    \end{quote}

    \begin{theorem}\label{thm:nombre de prolongements et degré}
        Soient \( K \) un corps de caractéristique \( 0 \) ou fini, \( K' \) une extension de degré fini \( n \) de \( K \), et \( C \) un corps algébriquement clos contenant \( K \). Alors il existe \( n \) \( K \)‑isomorphismes distincts de \( K' \) dans~\( C \).
    \end{theorem}
    \begin{proof}
        Notre assertion est vraie pour une extension \emph{monogène}\index{monogene (extension)@monogène (extension)} \( K' \), c'est-à-dire de la forme \( K' = K[x] \) (\( x \in K' \)). En effet le polynôme minimal \( F(X) \) de \( x \) sur \( K \) est alors de degré \( n \). Il admet \( n \) racines \( x_1, \cdots, x_n \) dans \( C \), qui sont distinctes d'après \cref{lem:racines conjugués sont distinctes}. Pour \( i = 1, \cdots, n \), on a donc un \( K \)‑isomorphisme \( \sigma_i \colon K' \to C \) tel que \( \sigma_i(x) = x_i \).

        Nous procéderons alors par récurrence sur le degré \( n \) de \( K' \). Soit \( x \in K' \) ;  considérons les corps \( K \subset K[x] \subset K' \) et posons \( q = [K[x]:K] \) ; on peut supposer \( q > 1 \). D'après le cas monogène, on a \( q \) \( K \)‑isomorphismes distincts \( \sigma_1, \cdots, \sigma_q \) de \( K[x] \) dans~\( C \). Comme \( K[\sigma_i(x)] \) et \( K[x] \) sont isomorphes, on peut construire une extension \( K_i' \) de \( K[\sigma_i(x)] \) et un isomorphisme \( \tau_i \colon K' \to K_i' \) qui prolonge \( \sigma_i \). Or \( K[\sigma_i(x)] \) est un corps de caractéristique \( 0 \) ou fini. Comme
        {\[\setlength{\abovedisplayskip}{.5\baselineskip}%
        \setlength{\belowdisplayskip}{.5\baselineskip}%
            [K_i' : K[\sigma_i(x)]] = [K' : K[x]] = \frac{n}{q} < n,
        \]}%
        l'hypothèse de récurrence montre qu'on a \( \frac{n}{q} \) \( K[\sigma_i(x)] \)‑isomorphismes distincts \( \theta_{ij} \) de \( K_i' \) dans~\( C \). Alors les \( n \) composés \( \theta_{ij} \circ \tau_i \) fournissent \( q \cdot \frac{n}{q} = n \) \( K \)‑isomorphismes de \( K' \) dans~\( C \). Ils sont distincts car \( \theta_{ij} \circ \tau_i \) et \( \theta_{i'j'} \circ \tau_{i'} \) diffèrent sur \( K[x] \) si \( i \neq i' \), et, si \( i = i' \), \( \theta_{ij} \) et \( \theta_{ij'} \) diffèrent sur \( K_i' \).
    \end{proof}

    \begin{quote}
        \Cref{thm:nombre de prolongements et degré} s'étend à un corps parfait \( K \) : on montre en effet que toute extension algébrique d'un corps parfait (en particulier \( K[\sigma_i(x)] \)) est un corps parfait ; le reste de la démonstration est inchangé.
    \end{quote}

    \begin{corollary}[théorème de l'élément primitif]\label{cor:théorème de l'élément primitif}
        Soient \( K \) un corps fini ou de caractéristique \( 0 \), et \( K' \) une extension de \( K \) de degré fini~\( n \). Il existe alors un élément \( x \) de \( K' \) (dit \textquote{primitif}\index{primitif (element primitif d'une extension)@primitif (élément primitif\IndexLinebreak d'une extension)}) tels que \( K' = K[x] \).
    \end{corollary}
    \begin{proof}
        Si \( K \) est fini, \( K' \) est fini, et son groupe multiplicatif \( K'{}^\times \) est formé des puissances d'un même élément \( x \) (\cref{thm:corps fini}, \ref{thm:corps fini;item2}). On a alors \( K' = K[x] \).

        Supposons maintenant \( K \) de caractéristique \( 0 \), donc infini. D'après \cref{thm:nombre de prolongements et degré}, on a \( n \) \( K \)‑isomorphismes \( \sigma_i \) de~\( K' \) dans un corps algébriquement clos~\( C \) contenant \( K \). Pour \( i \neq j \) l'équation \( \sigma_i(y) = \sigma_j(y) \) (\( y \in K' \)) définit une partie \( V_{ij} \) de~\( K' \), qui est évidemment un sous-\( K \)‑espace vectoriel de~\( K' \), et qui est distincte de~\( K' \) car \( \sigma_i \neq \sigma_j \). Comme \( K \) est infini, l'algèbre linéaire montre que la réunion des \( V_{ij} \) est distincte de~\( K' \). Prenons \( x \) en dehors de cette réunion. Les \( \sigma_i(x) \) sont alors deux à deux distincts, de sorte que le polynôme minimal \( F(X) \) de~\( x \) sur~\( K \) a au moins \( n \)~racines distincts (les \( \sigma_i(x) \)) dans \( C \) ; on a donc \( \deg F \geqslant n \), c'est-à-dire \( [K[x]:K] \geqslant n \). Comme \( K[x] \subset K' \) et que \( [K':K] = n \), on en déduit \( K' = K[x] \).
    \end{proof}

\section{Entiers des corps quadratiques}\label{sec:entiers des corps quadratiques}

    Nous allons interrompe un moment la théorie générale pour donner un exemple.

    \begin{definition}[corps quadratique]\index{corps quadratique}\index{quadratique (corps)}
        On appelle corps quadratique toute extension de degré \( 2 \) du corps \( \Q \) des nombre rationnels.
    \end{definition}

    Si \( K \) est un corps quadratique, tout élément \( x \) de \( K \setminus \Q \) est de degré \( 2 \) sur \( \Q \)\,, donc est élément primitif de \( K \) (\ie. \( K = \Q[x] \), et \( (1,x) \) est une base de \( K \) sur \( \Q \)). Soit \( F(X) = X^2+bX+c \) (\( b,c \in \Q \)) le polynôme minimal d'un tel élément \( x \in K \). La résolution de l'équation du second degré \( x^2+bx+c = 0 \) donne \( 2x = -b \pm \sqrt{b^2-4c} \). Ainsi \( K = \Q\Bigl(\sqrt{b^2-4c}\Bigr) \)\,\footnote{Par \( \sqrt{b^2-4c} \) nous entendons l'un des deux éléments de \( K \) dont le carré est \( b^2-4c \).}. Or \( b^2-4c \) est un nombre rationnel \( \frac{u}{v} = \frac{uv}{v^2} \) avec \( u,v \in \Z \) ; on a donc aussi \( K = \Q(\sqrt{uv}) \) avec \( uv \in \Z \). Par le même procédé on voit qu'on peut enfin écrire \( K = \Q(\sqrt{d}) \) où \( d \) est un entier \emph{sans facteurs carrés}\index{sans facteurs carres (entier)@sans facteurs carrés (entier)} dans sa décomposition en facteurs premiers. On a ainsi prouvé :
    \begin{proposition}
        Tout corps quadratique est de la forme \( \Q(\sqrt{d}) \), où \( d \) est un entier sans facteurs carrés.
    \end{proposition}

    L'élément \( \sqrt{d} \) est une racine du polynôme irréductible \( X^2-d \). Il admet un \emph{conjugué} dans \( K \), à savoir \( -\sqrt{d} \). Il existe donc un automorphisme \( \sigma \) de \( K \) qui applique \( \sqrt{d} \) en \( -\sqrt{d} \) (\cref{sec:éléments conjugués;corps conjugués}). L'élément général de \( K \) est de la forme \( a+b\sqrt{d} \) avec \( a,b \in \Q \)\,, et on a
    \begin{equation}\label{eq:quadratique;automorphisme}
        \sigma(a+b\sqrt{d}) = a-b\sqrt{d}.
    \end{equation}

    Nous nous proposons d'étudier l'anneau \( A \) des \emph{entiers} de \( K \), c'est-à-dire l'ensemble des \( x \in K \) qui sont entiers sur \( \Z \) (\cref{cor:éléments entiers forment un anneau} \cref[de]{prop:entier;tower}). Si \( x \in A \), \( \sigma(x) \) est racine de même équation de dépendance intégrale que \( x \), donc \( \sigma(x) \in A \). On a donc \( x + \sigma(x) \in A \) et \( x \cdot \sigma(x) \in A \). Or, si \( x = a+b\sqrt{d} \) avec \( a, b \in \Q \)\,, on a, d'après \eqref{eq:quadratique;automorphisme},
    \begin{equation}
        x + \sigma(x) = 2a \in \Q, \quad x \cdot \sigma(x) = a^2-db^2 \in \Q.
    \end{equation}
    Comme \( \Z \) est principal, et donc intégralement clos (\cref{exp:anneau principal est intégralement clos}), on a donc
    \begin{equation}\label{eq:condition d'être entier dans un corps quadratique}
        2a \in \Z, \quad a^2-db^2 \in \Z.
    \end{equation}
    Ces conditions \eqref{eq:condition d'être entier dans un corps quadratique} sont nécessaires pour que \( x = a+b\sqrt{d} \) soit entier sur \( \Z \) ; elles sont aussi suffisantes car alors \( x \) est racine de
    \[
        x^2 - 2a x + a^2-db^2 = 0.
    \]
    D'après \eqref{eq:condition d'être entier dans un corps quadratique} on a \( (2a)^2-d(2b)^2 \in \Z \) ; comme \( 2a \in \Z \)\,, on a donc \( d(2b)^2 \in \Z \). Or \( d \) est sans facteurs carrés ; si \( 2b \) n'était pas entier, son dénominateur comporterait un facteur premier \( p \) ; ce facteur apparaîtrait sous la forme \( p^2 \) dans \( (2b)^2 \), et la multiplication par \( d \) ne pourrait pas le ramener dans \( \Z \) ; on a donc \( 2b \in \Z \).

    Bref on peut poser \( a = \frac{u}{2} \), \( b = \frac{v}{2} \) avec \( u,v \in \Z \) ; les conditions dans \eqref{eq:condition d'être entier dans un corps quadratique} devient alors :
    \begin{equation}\label{eq:condition d'être entier dans un corps quadratique;suivant}
        u^2 - d v^2 \in 4\Z.
    \end{equation}
    Si \( v \) est pair, \eqref{eq:condition d'être entier dans un corps quadratique;suivant} montre que \( u \) est pair aussi ; on a alors \( a, b \in \Z \). Si \( v \) est impair, on a \( v^2 \equiv 1 \pmod*{4} \) ; or la classe de \( u^2 \pmod*{4} \) est \( 0 \) ou \( 1 \) (écrire la table des carrés modulo \( 4 \)) ; comme \( d \) est sans facteurs carrés, il n'est pas multiple de \( 4 \) ; ainsi on a nécessairement \( u^2 \equiv 1 \pmod*{4} \) et \( d \equiv 1 \pmod*{4} \). On a donc prouvé ce qui suit :
    \begin{theorem}\label{thm:entiers dans un corps quadratique}
        Soit \( K = \Q(\sqrt{d}) \) un corps quadratique, avec \( d \in \Z \) sans facteurs carrés (et donc \( \not\equiv 0 \pmod*{4} \)).
        \begin{enumerate}
            \item \label{thm:entiers dans un corps quadratique;cas1} Si \( d \equiv 2 \) ou \( d \equiv 3 \pmod*{4} \), l'anneau \( A \) des entiers de \( K \) est l'ensemble des \( a+b\sqrt{d} \) avec \( a,b \in \Z \).
            \item \label{thm:entiers dans un corps quadratique;cas2} Si \( d \equiv 1 \pmod*{4} \), \( A \) est l'ensemble des \( \frac{1}{2}(u+v\sqrt{d}) \) avec \( u,v \in \Z \) de même parité.
        \end{enumerate}
    \end{theorem}

    Dans le cas \( d \equiv 2 \) ou \( d \equiv 3 \pmod*{4} \), une base du \( \Z \)‑module \( A \) est évidemment \( (1,\sqrt{d}) \). Dans le cas \( d \equiv 1 \pmod*{4} \), une base du \( \Z \)‑module \( A \) est \( \left(1, \frac{1}{2}(1+\sqrt{d})\right) \). En effet, par \ref{thm:entiers dans un corps quadratique;cas2}, les éléments \( 1 \) et \( \frac{1}{2}(1+\sqrt{d}) \) sont dans \( A \). Inversement, pour montrer que \( \frac{1}{2}(u+v\sqrt{d}) \) (avec \( u,v \in \Z \) de même parité) est combinaison \( \Z \)‑linéaire de \( 1 \) et \( \frac{1}{2}(1+\sqrt{d}) \), on se ramène au cas où \( u \) et \( v \) sont pairs par soustraction éventuelle de \( \frac{1}{2}(1+\sqrt{d}) \) ; dans ce cas on a \( \frac{1}{2}(u+v\sqrt{d}) = \left(\frac{u}{2}-\frac{v}{2}\right) \cdot 1 + v \cdot \frac{1}{2}(1+\sqrt{d}) \).

    Pour finir, un peu de \emph{terminologie}. Si \( d > 0 \), on dit que \( \Q(\sqrt{d}) \) est un \emph{corps quadratique réel}\index{corps quadratique!reel@réel}\index{reel (corps quadratique)@réel (corps quadratique)} (car il existe un sous-corps de \( \R \) conjugué de \( \Q(\sqrt{d}) \) sur \( \Q \)). Si \( d \leqslant 0 \), on dit que \( \Q(\sqrt{d}) \) est un \emph{corps quadratique imaginaire}\index{corps quadratique!imaginaire}\index{imaginaire (corps quadratique)}.

\section{Normes et traces}\label{sec:normes et traces}

\subsection{Rappels d'algèbre linéaire}
    Soient \( A \) un anneau, \( E \) un \( A \)‑module \emph{libre} de rang fini, et \( u \) un endomorphisme de~\( E \). On définit en algèbre linéaire la \emph{trace}, le \emph{déterminant}, et le \emph{polynôme caractéristique}\index{caracteristique (polynome)@caractéristique (polynôme)} de \( u \). Si une base \( (e_i) \) de \( E \) a été choisie et si \( (a_{ij}) \) est la matrice de \( u \) dans cette base, ces quantités valent respectivement :
    \begin{equation}\label{eq:tr;det;matrice}
        \Tr(u) = \sum_{i=1}^n a_{ii}, \quad
        \det(u) = \det(a_{ij}), \quad \text{ et } \quad
        \det(X\cdot \I_E - u).
    \end{equation}
    \begin{remark}
        Ces quantités sont indépendantes de la base choisie.
    \end{remark}

    Les formules \eqref{eq:tr;det;matrice} montrent qu'on a :
    \begin{align}\label{eq:tr;det;prop}
        \Tr(u+u') &= \Tr(u)+\Tr(u'),\nonumber\\
        \det(uu') &= \det(u)\det(u'),\\
        \det(X\cdot \I_E - u) &= X^n - (\Tr(u))X^{n-1} + \cdots + (-1)^n \det(u).\nonumber
    \end{align}

\subsection{Normes et traces dans une extension}
    Soient \( B \) un anneau, et \( A \) un sous-anneau de \( B \) tel que \( B \) soit un \( A \)‑module de rang fini \( n \) (par exemple \( A \) peut être un corps, et \( B \) une extension de degré \( n \) de~\( A \)). Pour \( x \in B \), la multiplication \( m_x \) par~\( x \) (soit \( y \mapsto xy \)) est un endomorphisme du \( A \)‑module~\( B \).

    \begin{definition}[trace, norme et polynôme caractéristique]\label{def:trace;norme;polynôme caractéristique}
        On appelle \emph{trace}\index{trace} (resp. \emph{norme}\index{norme}, \emph{polynôme caractéristique}\index{polynome caracteristique@polynôme caractéristique}) de \( x \in B \), relativement à \( B \) et \( A \), la \emph{trace} (resp. le \emph{déterminant}, le \emph{polynôme caractéristique}) de l'endomorphisme \( m_x \) de multiplication par~\( x \).
    \end{definition}

    La trace (resp. norme) de \( x \) est notée \( \Tr_{A/B}(x) \) (resp. \( \Norm_{A/B}(x) \)), ou \( \Tr(x) \) (resp. \( \Norm(x) \)) lorsqu'aucune confusion n'est à craindre ; ce sont des éléments de \( A \). Le polynôme caractéristique de \( x \) est un polynôme unitaire à coefficients dans~\( A \).

    Pour \( x,x' \in B \) et \( a \in A \) on a évidemment \( m_x + m_{x'} = m_{x+x'} \)\,, \( m_x \circ m_{x'} = m_{x x'} \) et \( m_{ax} = a m_x \) ; de plus la matrice de \( m_a \)\,, dans n'importe quelle base de \( B \) sur \( A \), est la matrice diagonale dont tous les éléments diagonaux sont égaux à~\( a \). Il résulte alors des formules \eqref{eq:tr;det;matrice} et \eqref{eq:tr;det;prop} qu'on a :
    \begin{equation}
        \begin{alignedat}{5}
            \Tr(x+x') &= \Tr(x) + \Tr(x'),\quad &\Tr(ax)& &&= a\Tr(x), \quad &\Tr(a)& &&= n \cdot a ;\\
            \Norm(xx') &= \Norm(x)\Norm(x'), \quad &\Norm(ax)& &&= a^n \Norm(x), \quad &\Norm(a)& &&= a^n.
        \end{alignedat}
    \end{equation}

    \begin{proposition}\label{prop:trace;norme;polynôme caractéristique;extension}
        Soient \( K \) un corps de caractéristique \( 0 \) ou fini, \( L \) une extension algébrique de degré~\( n \) de~\( K \), \( x \) un élément de \( L \), et \( x_1, \cdots, x_n \) les racines du polynôme minimal de \( x \) sur \( K \) (dans une extension convenable de \( K \) ; \cf.~\cref{prop:existence d'une extension algébrique telle que le polynôme se décompose}), chacune répétée \( [L:K[x]] \) fois. Alors \( \Tr_{L/K}(x) = x_1 + \cdots + x_n \)\,, \( \Norm_{L/K}(x) = x_1 \cdots x_n \) ; de plus le polynôme caractéristique de \( x \), relativement à \( L \) et \( K \), est \( (X-x_1) \cdots (X-x_n) \).
    \end{proposition}
    \begin{quote}
        Ainsi ce polynôme caractéristique est la puissance \( [L:K[x]] \)‑ième du polynôme minimal de \( x \) sur~\( K \).
    \end{quote}
    \begin{proof}
        Traitons d'adord le cas où est un \emph{élément primitif} de~\( L \) sur~\( K \) (\cf.~\cref{cor:théorème de l'élément primitif} \cref[de]{thm:nombre de prolongements et degré}). Soit \( F(X) \) le polynôme minimal de \( x \) sur~\( K \) ; alors \( L \) est \( K \)‑isomorphe à \( K[X]/(F(X)) \) (\cref{sec:éléments algébriques sur un corps;extensions algébriques}, \ref{élément algébrique;item5}), et \( (1, x, \cdots, x^{n-1}) \) est une base de~\( L \) sur~\( K \). Posons \( F(X) = X^n + a_{n-1} X^{n-1} + \cdots + a_1 X + a_0 \). La matrice de l'endomorphisme \( m_x \) dans cette base est :
        \[
            M =
            \begin{pmatrix}
                0 & 0 & \cdots & 0 & -a_0\\
                1 & 0 & \cdots & 0 & -a_1\\
                0 & 1 & \cdots & 0 & \svdots\\
                \svdots & 0 &  & \svdots & \svdots\\
                \svdots & \svdots &  & \svdots & \svdots\\
                0 & 0 & \cdots & 1 & -a_{n-1}
            \end{pmatrix}
            .
        \]
        Le déterminant de \( X \cdot \Id_L - m_x \) est donc :
        \[
            \det(X \cdot \I_n - M) =
            \begin{vmatrix}
                X & 0 & \cdots & 0 & a_0\\
                -1 & X & \cdots & 0 & a_1\\
                0 & -1 & \cdots & 0 & \svdots\\
                \svdots & \svdots &  & \svdots & \svdots\\
                \svdots & \svdots &  & X & a_{n-2}\\
                0 & 0 & \cdots & -1 & X+a_{n-1}
            \end{vmatrix}
            .
        \]
        En développant on obtient le polynôme caractéristique de \( x \), qui est donc égal au polynôme minimal \( X^n + a_{n-1} X^{n-1} + \cdots + a_1 X + a_0 \). D'après \eqref{eq:tr;det;prop} on en déduit \( \Tr(x) = -a_{n-1} \) et \( \Norm(x) = (-1)^n a_0 \). Or, comme \( x \) est primitif, on a \( F(X) = (X-x_1) \cdots (X-x_n) \) ; d'où, en comparant,
        \[
            \Tr(x) = x_1 + \cdots + x_n \quad \text{ et } \quad \Norm(x) = x_1 \cdots x_n.
        \]

        Passons maintenant au \emph{cas général}, et posons \( r = [L:K[x]] \). Il nous suffit de montrer que le polynôme caractéristique \( P(X) \) de \( x \) relativement  à \( L \) et \( K \) est égal à la puissance \( r \)‑ième du polynôme minimal de \( x \) sur \( K \). Or soient \( (y_i)_{i = 1, \cdots, q} \) une base de \( K[x] \) sur \( K \), et \( (z_j)_{j = 1, \cdots, r} \) une base de \( L \) sur \( K[x] \) ; alors \( (y_i z_j) \) est une base de~\( L \) sur~\( K \) et on a \( n = qr \) (\cref{prop:extension algébrique;multiplicativité des degrés}). Soit \( M = (a_{ih}) \) la matrice de la multiplication par \( x \) dans \( K[x] \) par rapport à la base \( (y_i) \) : ainsi \( xy_i = \sum_h a_{ih} y_h \). Si on ordonne lexicographiquement la base \( (y_i z_j) \) de~\( L \) sur~\( K \), on voit donc que la matrice \( M' \) de la multiplication par \( x \) dans~\( L \) par rapport à cette base se présente sous la forme d'un tableau diagonal de~matrices
        \[
            M' =
            \begin{pmatrix}
                M & 0 & \cdots & 0\\
                0 & M & \cdots & 0\\
                \svdots & \svdots & \cdots & \svdots\\
                0 & 0 & \cdots & M
            \end{pmatrix}
            .
        \]
        La matrice \( X \cdot \I_n - M' \) se présente donc comme tableau diagonal des matrices \( X \cdot \I_q - M \) ; d'où \( \det(X \cdot \I_n - M') = (\det(X \cdot \I_q - M))^r \). Or le premier membre est \( P(X) \), et \( \det(X \cdot \I_q - M) \) est le polynôme minimal de \( x \) sur \( K \) d'après la première partie.
    \end{proof}

    \medskip
    Donnons enfin un résultat sur les traces et normes d'éléments entiers.

    \begin{proposition}\label{prop:trace et norme d'élément entier}
        Soient \( A \) un anneau intègre, \( K \) sont corps des fractions, \( L \) une extension de degré fini de \( K \), et \( x \) un élément de \( L \) entier sur \( A \) ; on suppose \( K \) de caractéristique \( 0 \). Alors les coefficients du polynôme caractéristique \( P(X) \) de \( x \) relativement à \( L \) et \( K \), en particulier \( \Tr_{L/K}(x) \) et \( \Norm_{L/K}(x) \), sont entiers sur~\( A \).
    \end{proposition}
    \begin{proof}
        Utilisons \cref{prop:trace;norme;polynôme caractéristique;extension} : on a \( P(X) = (X-x_1) \cdots (X-x_n) \) ; les coefficients de \( P(x) \) sont donc, au signe près, des sommes de produits des \( x_i \) ; il suffit de montrer que les \( x_i \) sont entiers sur \( A \) (\cref{cor:éléments entiers forment un anneau;pré} \cref[de]{prop:entier;tower}). Or chaque \( x_i \) est un conjugué de \( x \) sur \( K \) (\cref{sec:éléments conjugués;corps conjugués}), et on a un \( K \)‑isomorphisme \( \sigma_i \colon K[x] \to K[x_i] \) tel que \( \sigma_i(x) = x_i \). En appliquant \( \sigma_i \) à une équation de dépendance intégrale de~\( x \) sur~\( A \), on obtient une équation de dépendance intégrale de~\( x_i \) sur~\( A \).
    \end{proof}

    \begin{corollary}\label{cor:trace et norme d'élément entier}
        Supposons de plus \( A \) intégralement clos. Alors les coefficients du polynôme caractéristique de \( x \), en particulier \( \Tr_{L/K}(x) \) et \( \Norm_{L/K}(x) \), sont éléments de~\( A \).
    \end{corollary}
    \begin{proof}
        En effet ces coefficients sont éléments de \( K \) par définition, et sont entiers sur \( A \) par \cref{prop:trace et norme d'élément entier}.
    \end{proof}

    \begin{quote}
        On remarquera que les quantités \( x + \sigma(x) \) et \( x \cdot \sigma(x) \) utilisées dans l'étude des corps quadratique (\cref{sec:entiers des corps quadratiques}) sont la trace et la norme de \( x \). On y a prouvé (l'éq.~\eqref{eq:condition d'être entier dans un corps quadratique}) un cas particulier du corollaire ci-dessus.
    \end{quote}

\section{Discriminant}\label{sec:discriminant}

    \begin{definition}[discriminant du système d'éléments]\label{def:disc}\index{discriminant}
        Soient \( B \) un anneau et \( A \) un sous-anneau de \( B \) tel que \( B \) soit un \( A \)‑module libre de rang fini \( n \). Pour \( (x_1, \cdots, x_n) \in B^n \), on appelle discriminant du système \( (x_1, \cdots, x_n) \) l'élément de \( A \) défini par
        \begin{equation}
            \D(x_1, \cdots, x_n) = \det(\Tr_{B/A}(x_i x_j)).
        \end{equation}
    \end{definition}

    \begin{proposition}\label{prop:disc;changement de base}
        Si \( (y_1, \cdots, y_n) \in B^n \) est un autre système d'éléments de \( B \) tel que \( y_i = \sum_{j=1}^n a_{ij} x_j \) avec \( a_{ij} \in A \), on a
        \begin{equation}
            \D(y_1, \cdots, y_n) = \det(a_{ij})^2 \D(x_1, \cdots, x_n).
        \end{equation}
    \end{proposition}
    \begin{proof}
        En effet on a \( \Tr(y_p y_q) = \Tr\left(\sum_{i,j}a_{pi}a_{qj}x_i x_j\right) = \sum_{i,j} a_{pi}a_{qj}\Tr(x_i x_j) \). D'où l'égalité entre matrices \( (\Tr(y_p y_q)) = (a_{pi}) (\Tr(x_i x_j)) \franspose(a_{qj}) \) (où \( \franspose{M} \) désigne la transportée de la matrice \( M \)). Il suffit de prendre les déterminants.
    \end{proof}

    Il résulte \cref[de]{prop:disc;changement de base} que les discriminants des bases de \( B \) sur \( A \) sont deux à deux \emph{associés} dans \( A \) : en effet la matrice de passage \( (a_{ij}) \) d'une base à une autre est inversible, donc a son déterminant inversible. On peut donc poser la ---
    \begin{definition}[idéal discriminant d'un anneau sur un sous-anneau]\label{def:discriminant d'un anneau sur un sous-anneau}\index{discriminant (ideal)@discriminant (idéal)}\index{ideal discriminant@idéal discriminant}
        Sous les hypothèses \cref[de]{def:disc}, on appelle \emph{(idéal) discriminant} de \( B \) sur \( A \), et on note \( \ideal{D}_{B/A} \)\,, l'idéal principal de \( A \) engendré par le discriminant de n'importe quelle base de \( B \) sur~\( A \).
    \end{definition}

    \begin{proposition}
        Supposons que \( \ideal{D}_{B/A} \) contienne un élément qui n'est pas diviseur de \( 0 \). Alors, pour qu'un système \( (x_1, \cdots, x_n) \in B^n \) soit une base de \( B \) sur \( A \), il faut et il suffit que \( \ideal{D}_{B/A} \) soit engendré par \( \D(x_1, \cdots, x_n) \).
    \end{proposition}
    \begin{proof}
        La nécessité a été démontrée plus haut. Supposons donc que \( d = \D(x_1, \cdots, x_n) \) engendre \( \ideal{D}_{B/A} \). Soit \( (e_1, \cdots, e_n) \) une base de \( B \) sur \( A \) ; posons \( d' = \D(e_1, \cdots, e_n) \) et \( x_i = \sum_{j=1}^n a_{ij} e_j \) avec \( a_{ij} \in A \). On a \( d = \det(a_{ij})^2 d' \). Par hypothèse on~a \( Ad = \ideal{D}_{B/A} = Ad' \). Il existe donc \( b \in A \) tel que \( d' = bd \) ; d'où \( d(1-b \det(a_{ij})^2) = 0 \). Or \( d \) n'est pas diviseur de~\( 0 \), car sinon tout élément de \( Ad = \det(a_{ij})^2 \) serait diviseur de~\( 0 \). On a donc \( 1-b \det(a_{ij})^2 = 0 \). Ceci montre que \( \det(a_{ij}) \) est inversible, donc aussi la matrice \( (a_{ij}) \) ; par conséquent \( (x_1, \cdots, x_n) \) est une base de \( B \) sur~\( A \).
    \end{proof}

    \begin{proposition}\label{prop:disc et det;K-isomorphismes}
        Soient \( K \) un corps fini ou de caractéristique \( 0 \), \( L \) une extension de degré fini \( n \) de \( K \), et \( \sigma_1, \cdots, \sigma_n \) les \( n \) \( K \)‑isomorphismes distincts de \( L \) dans un corps algébriquement clos \( C \) contenant \( K \) (\cref{thm:nombre de prolongements et degré}). Alors, si \( (x_1, \cdots, x_n) \) est une base de~\( L \) sur~\( K \), on a
        \begin{equation}
            \D(x_1, \cdots, x_n) = \det(\sigma_i(x_j))^2 \neq 0.
        \end{equation}
    \end{proposition}
    \begin{proof}
        La première égalité résulte d'un calcul simplet :
        \begin{align*}
            \D(x_1, \cdots, x_n) = \det(\Tr(x_i x_j)) &= \det\left(\sum_k \sigma_k(x_i x_j)\right) = \det\left(\sum_k \sigma_k(x_i) \sigma_k(x_j)\right)\\
            &= \det(\sigma_j(x_i)) \cdot \det(\sigma_i(x_j)) = \det(\sigma_i(x_j))^2.
        \end{align*}
        Reste à montrer qu'on a \( \det(\sigma_i(x_j))^2 \neq 0 \). Raisonnons par l'absurde. Si \mbox{\( \det(\sigma_i(x_j))^2 = 0 \)}, il existe \( u_1, \cdots, u_n \in C \), non tous nuls, tels que \( \sum_{i=1}^n u_i \sigma_i (x_j) = 0 \) pour tout \( j \). Par linéarité on en déduit \( \sum_{i=1}^n u_i \sigma_i (x) = 0 \) pour tout \( x \in L \). Or ceci contredit le \textquote{lemme de Dedekind} suivant.
    \end{proof}

    \smallskip
    \begin{theorem-with-name}{Lemme de Dedekind}
        Soient \( G \) un groupe, \( C \) un corps, et \( \sigma_1, \cdots, \sigma_n \) des homomorphismes distincts de \( G \) dans le groupe multiplicatif \( C^\times \). Alors les \( \sigma_i \) sont linéairement indépendants sur \( C \) (\ie, \( \sum_{i} u_i \sigma_i(g) = 0 \) pour tout \( g \in G \) implique que tous les \( u_i \) sont nuls).
    \end{theorem-with-name}
    \begin{proof}
        Si les \( \sigma_i \) sont linéairement dépendants, considérons une relation non triviale \( \sum_i u_i \sigma_i = 0 \) (\( u_i \in C \)) telle que le nombre \( q \) des \( u_i \) non nuls soit \emph{minimal}. Après renumérotation, on peut supposer que c'est
        \begin{equation}\label{eq:lemme de Dedekind;aux}
            u_1 \sigma_1(g) + \cdots + u_q \sigma_q(g) = 0 \quad \text{ pour tout \( g \in G \).}
        \end{equation}
        On a \( q \geqslant 2 \) car les \( \sigma_i \) sont non nuls. Pour \( g \) et \( h \) quelconques dans \( G \), on a
        \[
            u_1 \sigma_1(hg) + \cdots + u_q \sigma_q(hg) = u_1 \sigma_1(h)\sigma_1(g) + \cdots + u_q \sigma_q(h)\sigma_q(g) = 0.
        \]
        Multiplions \eqref{eq:lemme de Dedekind;aux} par \( \sigma_1(h) \) et soustrayons ; il vient
        \[
            u_2 (\sigma_1(h)-\sigma_2(h))\sigma_1(g) + \cdots + u_q (\sigma_1(h)-\sigma_q(h))\sigma_q(g) = 0.
        \]
        Comme ceci a lieu pour tout \( g \in G \) et que \( q \) a été choisi minimum, ceci implique \( u_2(\sigma_1(h)-\sigma_2(h)) = 0 \) ; d'où \( \sigma_1(h) = \sigma_2(h) \) car \( u_2 \neq 0 \) ; ceci contredit l'hypothèse que les \( \sigma_i \) sont distincts.
    \end{proof}

    \medskip
    \begin{remark}
        Sous les conditions \cref[de]{prop:disc et det;K-isomorphismes}, la relation \( \D(x_1, \cdots, x_n) \neq 0 \) exprime que la forme bilinéaire \( (x,y) \mapsto \Tr_{L/K}(xy) \) est \emph{non-dégénérée}, c'est-à-dire que \( \Tr_{L/K}(xy) = 0 \) pour tout \( y \in L \) implique \( x = 0 \). Ainsi l'application \( K \)‑linéaire qui, à \( x \in L \), fait correspondre la forme \( K \)‑linéaire, \( s_x \colon y \mapsto \Tr_{L/K}(xy) \), est une injection de \( L \) dans son dual \( \Hom_K(L,K) \) (pour la structure d'espace vectoriel sur \( K \)). Comme \( L \) et \( \Hom_K(L,K) \) sont de même dimension finie \( n \) sur \( K \), il en résulte que \( x \mapsto s_x \) est une bijection. L'existence de \textquote{\emph{base duales}}\index{bases duales (pour la trace)} sur un espace vectoriel et son dual montre alors que, pour toute base \( (x_1, \cdots, x_n) \) de~\( L \) sur~\( K \), il existe une autre base \( (y_1, \cdots, y_n) \) telle que
        \begin{equation}\label{eq:trace et base duales}
            \Tr_{L/K}(x_i y_j) = \delta_{ij} \quad \text{(\( 1 \leqslant i,j \leqslant n \)).}
        \end{equation}
    \end{remark}

    Cette remarque va nous être utile.

    \medskip
    \begin{theorem}\label{thm:structure de fermeture intégrale dans une extension fini}
        Soient \( A \) un anneau intégralement clos, \( K \) son corps des fractions, \( L \) une extension de degré fini \( n \) de \( K \) et \( A' \) la fermeture intégrale de \( A \) dans \( L \). On suppose \( K \) de caractéristique \( 0 \). Alors \( A' \) est un sous-\( A \)‑module d'un \( A \)‑module libre de rang~\( n \).
    \end{theorem}
    \begin{proof}
        Soit, en effet, \( (x_1, \cdots, x_n) \) une base de~\( L \) sur~\( K \). Chaque \( x_i \) est algébrique sur \( K \), d'où \( a_n x_i^n + a_{n-1} x_i^{n-1} + \cdots + a_0 = 0 \) avec \( a_j \in A \) pour \( j = 0, \cdots, n \) ; par multiplication par une puissance de \( x_i \)\,, on peut supposer \( a_n \neq 0 \) ; par multiplication par \( a_n^{n-1} \), on voit que \( a_n x_i \) est entier sur \( A \). Posons \( x_i' = a_n x_i \). Alors \( (x_1', \cdots, x_n') \) est une base de~\( L \) sur~\( K \) contenue dans~\( A' \).

        D'après la remarque ci-dessus, on a un autre base \( (y_1, \cdots, y_n) \) de~\( L \) sur~\( K \) telle que \( \Tr(x_i'y_j) = \delta_{ij} \) (l'éq.~\eqref{eq:trace et base duales}). Soit alors \( z \in A' \). Comme \( (y_1, \cdots, y_n) \) est une base de~\( L \) sur~\( K \), on peut écrire \( z = \sum_{j=1}^n b_j y_j \) avec \( b_j \in K \). Pour tout \( i \) on a \( x_i' z \in A' \) (car \( x_i' \in A' \)), d'où \( \Tr(x_i' z) \in A \) (\cref{cor:trace et norme d'élément entier} \cref[de]{prop:trace et norme d'élément entier}). Or
        \[
            \Tr(x' z) = \Tr\left(\sum_j b_j x_i' y_i\right) = \sum_j b_j \Tr(x_i' y_j) = \sum_j b_j \delta_{ij} = b_i.
        \]
        On a donc \( b_i \in A \) pour tout \( i \). Ainsi \( A' \) est contenu dans le \( A \)‑module libre \( \sum_{j=1}^n A y_i \).
    \end{proof}

    \begin{corollary}\label{cor:structure de fermeture intégrale dans une extension fini;cas principal}
        Avec les hypothèses \cref[de]{thm:structure de fermeture intégrale dans une extension fini}, supposons de plus \( A \) principal. Alors \( A' \) est un \( A \)‑module libre de rang~\( n \).
    \end{corollary}
    \begin{proof}
        En effet un sous-module d'un \( A \)‑module libre est alors libre (\cref{thm:base adaptée}, \ref{thm:base adaptée;item2}), et de rang \( \leqslant n \). D'autre part on a vu au cours de la démonstration \cref[de]{thm:structure de fermeture intégrale dans une extension fini} que \( A' \) contient une base de~\( L \) sur~\( K \), donc est de rang \( n \).
    \end{proof}

    \begin{quote}
        À titre d'exercice, le lecteur peu familier avec le contenu de la remarque précédent \cref{thm:structure de fermeture intégrale dans une extension fini} pourra chercher une démonstration plus calculatoire de ce théorème : avec les notations ci-dessus, il posera \( d = \D(x_1', \cdots, x_n') \), et montrera que, si \( z = \sum_i c_i x_i' \) (\( c_i \in K \)) est entier sur \( A \), alors \( d c_i \in A \) (calculer \( \Tr(z x_j') \) et utiliser les formules de Cramer).
    \end{quote}

\subsubsection*{Un exemple de calcul de discriminant}
    Soient \( K \) un corps fini ou de caractéristique \( 0 \), \( L = K[x] \) une extension de degré fini \( n \) de \( K \), et \( F(X) \) le polynôme minimal de \( x \) sur \( K \). Alors
    \begin{equation}\label{eq:disc;cas particulier}
        \D(1, x, \cdots, x^{n-1}) = (-1)^{\frac{n(n-1)}{2}} \Norm_{L/K}(F'(x))
    \end{equation}
    (où \( F'(X) \) désigne le polynôme dérivé de \( F(X) \)). En effet notons \( x_1, \cdots, x_n \) les racines de \( F(X) \) dans une extension de \( K \) ; ce sont les conjugués de \( x \) (\cref{prop:existence d'une extension algébrique telle que le polynôme se décompose}, et~\cref{sec:éléments conjugués;corps conjugués}). On~a
    \begingroup
    \allowdisplaybreaks
    \begin{align*}
        \D(1, x, \cdots, x^{n-1}) &= \det(\sigma_i(x^j))^2 \text{ (\cref{prop:disc et det;K-isomorphismes}) } = \det(x_i^j)^2\\
        (-1)^{\frac{n(n-1)}{2}} \det(x_i^j)^2 &= \prod_{i \neq j} (x_i - x_j) \text{ (Vandermonde) } = \prod_i\left(\prod_{j \neq i} (x_i - x_j)\right)\\
        &= \prod_i F'(x_i) = \Norm_{L/K}(F'(x))
    \end{align*}%
    \endgroup%
    (car les \( F'(x_i) \) sont les conjugués de \( F'(x) \)).

    En particulier, appliquons \eqref{eq:disc;cas particulier} au cas où \( F(X) \) est un \emph{trinôme} \( X^n + aX + b \) (\( a,b \in K \)). Posons \( y = F'(x) \) ; on a
    \[
        y = n x^{n-1} + a = -(n-1)a - n b x^{-1}
    \]
    (car \( x^n + ax + b = 0 \), d'où \( n x^{n-1} = -n a - n b x^{-1} \)). On en tire \( x = -nb(y+(n-1)a)^{-1} \). Le polynôme minimal de \( y \) sur \( K \) est le numérateur de \( F(-nb(y+(n-1)a)^{-1}) \) ; tous calculs faits, c'est \( (Y+(n-1)a)^n - na(Y+(n-1)a)^{n-1} + (-1)^n n^n b^{n-1} \). La norme de \( y \) est \( (-1)^n \) fois le terme constant de ce polynôme, donc
    \[
        n^n b^{n-1} + (-1)^{n-1} (n-1)^{n-1} a^n.
    \]
    D'où
    \begin{equation}\label{disc;cas particulier de cas particulier}
        \D(1, x, \cdots, x^{n-1}) = [n^n b^{n-1} + (-1)^{n-1} (n-1)^{n-1} a^n] (-1)^{\frac{n(n-1)}{2}}.
    \end{equation}
    Pour \( n = 2 \) (resp. \( 3 \)) on retrouve le \( 4b-a^2 \) (resp. \( -4a^3-27b^2 \)) bien connu.

\section{Terminologie des corps de nombres}\label{sec:terminologie des corps de nombres}

    On appelle \emph{corps de nombres algébriques} (ou \emph{corps de nombres}) toute extension de degré fini (et donc algébrique) de \( \Q \). Étant donné un corps de nombre \( K \), le degré \( [K:Q] \) s'appelle le \emph{degré} de \( K \) ; un corps de nombres de degré \( 2 \) (resp. \( 3 \)) s'appelle un \emph{corps quadratique} (\cf.~\cref{sec:entiers des corps quadratiques}) (resp. un \emph{corps cubique}\index{corps cubique}). Un corps de nombres est de caractéristique~\( 0 \).

    Étant donné un corps de nombres \( K \), les éléments de \( K \) qui sont entiers sur \( \Z \)\,, s'appellent les \emph{entiers}\index{entier d'un corps de nombres} de \( K \). Ils forment un \emph{sous-anneau} \( A \) de \( K \) (\cref{cor:éléments entiers forment un anneau} \cref[de]{prop:entier;tower}), qui est un \( \Z \)‑module \emph{libre} de rang \( [K:\Q] \) (\cref{cor:structure de fermeture intégrale dans une extension fini;cas principal} \cref[de]{thm:structure de fermeture intégrale dans une extension fini}). Les discriminants des bases du \( \Z \)‑module \( A \) diffèrent par un élément inversible de \( \Z \) (\cref{def:discriminant d'un anneau sur un sous-anneau}), qui est même un carré (\cref{prop:disc;changement de base}) ; cet élément ne peut donc être que \( +1 \), de sorte que les discriminants des bases du \( \Z \)‑module \( A \) sont tous \emph{égaux} ; leur valeur commune s'appelle le \emph{discriminant absolu}\index{discriminant absolu (d'un corps de\IndexLinebreak nombres)}, ou \emph{discriminant}, de~\( K \).

    Comme un corps de nombres \( K \) détermine de façon unique l'anneau \( A \) des entiers de \( K \), on fait souvent l'abus de langage consistant à attribuer à \( K \) des notions relatives à \( A \) ; ainsi lorsqu'on parle d'idéaux (ou d'unités) de \( K \), il s'agit d'idéaux (ou d'unités) de~\( A \).

\section{Corps cyclotomiques}\label{sec:corps cyclotomiques}

    On appelle \emph{corps cyclotomique}\index{corps cyclotomique}\index{cyclotomique (corps)} tout corps de nombres engendré sur \( \Q \) par des racines de l'unité. Étant donné un nombre premier \( p \), nous désignerons par \( z \) une racine primitive \( p \)‑ième de l'unité (dans \( \C \) par exemple), et nous allons étudier le corps cyclotomique \( \Q[z] \). Le nombre \( z \) est racine du polynôme \( X^p - 1 \) ; comme \( z \) est \( \neq 1 \), il est aussi racine du polynôme \( \frac{X^p-1}{X-1} = X^{p-1} + X^{p-2} + \cdots + X + 1 \), appelé \emph{polynôme cyclotomique}\index{cyclotomique (polynome)@cyclotomique (polynôme)}. Il n'est nullement évident que ce polynôme est irréductible sur \( \Q \) (ce qui revient à dire que le corps \( \Q[z] \) est de degré \( p-1 \)). Pour le démontrer nous aurons besoin du ---
    \begin{theorem-with-name}{Critère d'Eisenstein}\index{Eisenstein (critere d')@Eisenstein (critère d')}
        Soient \( A \) un anneau principal, \( p \in A \) un élément premier de \( A \) et \( F(X) = X^n + a_{n-1} X^{n-1} + \cdots + a_1 X + a_0 \) un élément de \( A[X] \) tel que \( p \) divise tous les \( a_i \) (\( 0 \leqslant i \leqslant n-1 \)), mais que \( p^2 \) ne divise pas \( a_0 \). Alors \( F(X) \) est irréductible sur le corps des fractions \( K \) de~\( A \).
    \end{theorem-with-name}
    \begin{proof}
        Suppons en effet que l'on ait \( F = G \cdot H \) avec \( G, H \in K[X] \), \( G \) et \( H \) \emph{unitaires}. Les racines de \( F \) sont \emph{entières} sur \( A \). Or toute racine de \( G \) (resp. \( H \)) est racine de \( F \), donc est entière sur \( A \). Or les coefficients de \( G \) (resp. \( H \)) sont des sommes de produits des racines de \( G \) (resp. \( H \)) ; ils sont donc entiers sur \( A \) (\cref{cor:éléments entiers forment un anneau;pré} \cref[de]{prop:entier;tower}). Comme \( A \) est principal, donc intégralement clos (\cref{exp:anneau principal est intégralement clos}), on a \( G \in A[X] \) et \( H \in A[X] \).

        Soient alors \( \overline{F}, \overline{G}, \overline{H} \) les images de \( F, G, H \) dans \( (A/Ap)[X] \) ; on a \( \overline{F} = \overline{G} \cdot \overline{H} \). D'après l'hypothèse sur les \( a_i \)\,, on a \( \overline{F} = X^n \). Comme \( A/Ap \) est un anneau \emph{intègre}, la décomposition \( X^n = \overline{G} \cdot \overline{H} \) est nécessairement de la forme \( X^n = X^q \cdot X^{n-q} \) (car \( \overline{G} \) et \( \overline{H} \) sont unitaires) ; d'où \( \overline{G} = X^q \) et \( \overline{H} = X^{n-q} \). Si \( G \) et \( H \) sont tous deux non constants, on en déduit que \( p \) divise les termes constants de \( G \) et \( H \) ; donc \( p^2 \) divise le terme constant \( a_0 \) de \( F \), contrairement à l'hypothèse. Donc \( G \) ou \( H \) est constant, et \( F \) est irréductible.
    \end{proof}

    \begin{quote}
        \textsc{\textbf{Exemple}}.
        Le polynôme \( X^3 - 2X + 6 \) est irréductible sur \( \Q \) (prendre \( p = 2 \), \( A = \Z \)).
    \end{quote}

    \begin{theorem}\label{thm:polynôme cyclotomique est irréductible}
        Pour tout nombre premier \( p \), le polynôme cyclotomique \( X^{p-1} + X^{p-2} + \cdots + X + 1 \) est irréductible dans \( \Q[X] \).
    \end{theorem}
    \begin{proof}
        Posons en effet \( X = Y+1 \). On a
        \begin{align*}
            X^{p-1} + \cdots + 1 = \frac{X^p-1}{X-1} &= \frac{(Y+1)^p-1}{Y}\\
            &= Y^{p-1} + \sum_{j = p-1}^1 \binom{p}{j} Y^{j-1} \eqqcolon F_1(Y).
        \end{align*}
        Or \( p \) divise tous les coefficients binômiaux \( \binom{p}{j} \), mais \( p^2 \) ne divise pas le terme constant \( \binom{p}{1} = p \). Donc \( F_1(Y) \) est irréductible par le critère d'Eisenstein, donc aussi le polynôme cyclotomique.
    \end{proof}

    Soit toujours \( z \) une racine primitive \( p \)‑ième de l'unité. Il résulte \cref[de]{thm:polynôme cyclotomique est irréductible} que le corps \( \Q[z] \) est de degré \( p-1 \) ; donc \( (1, z, \cdots, z^{p-2}) \) est une base de \( \Q[z] \) sur \( \Q \). Nous allons étudier l'anneau des entiers de \( \Q[z] \) et montrer que c'est \( \Z[z] \).

    Pour cela nous aurons besoin de calculer quelques \emph{traces et normes} (on écrira \( \Tr(x) \) et \( \Norm(x) \) au lieu de \( \Tr_{\Q[z]/\Q}(x) \) et \( \Norm_{\Q[z]/\Q}(x) \)). Notons que les conjugués de \( z \) sur \( \Q \) sont les \( z^j \) (\( j = 1, \cdots, p-1 \)) (\cref{thm:polynôme cyclotomique est irréductible}).

    L'irréductibilité du polynôme cyclotomique donne aussitôt :
    \begin{equation}\label{eq:traces de z et 1}
        \Tr(z) = -1 \quad \text{ et } \quad \Tr(1) = p-1.
    \end{equation}
    D'où \( \Tr(z^j) = -1 \) pour \( j = 1, \cdots, p-1 \), et donc
    \begin{equation}\label{eq:traces de 1-z^j}
        \Tr(1-z) = \Tr(1-z^2) = \cdots = \Tr(1-z^{p-1}) = p.
    \end{equation}
    D'autre part le calcul fait dans \cref{thm:polynôme cyclotomique est irréductible} montre que \( \Norm(z-1) = (-1)^{p-1} p \), d'où \( \Norm(1-z) = p \) ; comme la norme de \( 1-z \) est le produit des conjugués de \( 1-z \), on a donc
    \begin{equation}\label{eq:produit des conjugués de 1-z}
        p = (1-z)(1-z^2)\cdots(1-z^{p-1}).
    \end{equation}
    Notons \( A \) l'anneau des entiers de \( \Q[z] \). Il contient évidemment \( z \) et ses puissances. On va montrer qu'on a
    \begin{equation}\label{eq:A(1-z) et pZ}
        A(1-z) \cap \Z = p\Z.
    \end{equation}
    En effet on a \( p \in A(1-z) \) d'après \eqref{eq:produit des conjugués de 1-z}, d'où \( A(1-z) \cap \Z \supset p\Z \) ; comme \( p\Z \) est idéal maximal de \( \Z \)\,, la relation \( A(1-z) \cap \Z \neq p\Z \) entraînerait \( A(1-z) \cap \Z = \Z \)\,, et \( 1-z \) serait inversible dans \( A \) ; ses conjugués \( 1-z^j \) le seraient alors aussi, et donc \( p \) également d'après \eqref{eq:produit des conjugués de 1-z} ; ainsi \( \frac{1}{p} \) serait entier sur \( \Z \)\,, ce qui est absurde (\cref{exp:anneau principal est intégralement clos}).

    Montrons enfin que, pour tout \( y \in A \), on a
    \begin{equation}\label{eq:Tr(y(1-z))}
        \Tr(y(1-z)) \in p\Z.
    \end{equation}
    En effet chaque conjugué \( y_j(1-z^j) \) de \( y(1-z) \) est multiple (dans \( A \)) de \( 1-z^j \), lequel est multiple de \( 1-z \) car
    {\[\setlength{\abovedisplayskip}{.5\baselineskip}%
        1-z^j = (1-z)(1+z+\cdots+z^{j-1}) ~;
    \]}%
    comme la trace est la somme des conjugués, on a donc
    \[
        \Tr(y(1-z)) \in A(1-z) ~;
    \]
    d'où, \eqref{eq:Tr(y(1-z))}, d'après \eqref{eq:A(1-z) et pZ}, car la trace d'un entier est dans \( \Z \) (\cref{cor:trace et norme d'élément entier} \cref[de]{prop:trace et norme d'élément entier}).

    Ceci étant, nous sommes en mesure de déterminer l'anneau des entiers de \( \Q[z] \).

    \begin{theorem}\label{thm:base d'un corps cyclotomique}
        Soient \( p \) un nombre premier et \( z \) une racine primitive \( p \)‑ième de l'unité (dans \( \C \)). Alors l'anneau \( A \) des entiers du corps cyclotomique \( \Q[z] \) est \( \Z[z] \), et \( (1,z,\cdots,z^{p-2}) \) est une base du \( \Z \)‑module~\( A \).
    \end{theorem}
    \begin{proof}
        En effet soit \( x = a_0 + a_1 z + \cdots + a_{p-2}z^{p-2} \) (\( a_i \in \Q \)) un élément de \( A \). On~a~alors
        \[
            x(1-z) = a_0(1-z) + a_1(z-z^2) + \cdots + a_{p-2}(z^{p-2}-z^{p-1}).
        \]
        En prenant le trace, il résulte de \eqref{eq:traces de z et 1} et de \eqref{eq:traces de 1-z^j} que
        \[
            \Tr(x(1-z)) = a_0 \Tr(1-z) = a_0 p.
        \]
        D'où, en utilisant \eqref{eq:Tr(y(1-z))}, \( p a_0 \in p\Z \) et \( a_0 \in \Z \). Comme \( z^{-1} = z^{p-1} \), on a \( z^{-1} \in A \), d'où \( (x-a_0)z^{-1} = a_1 + a_2 z + \cdots + a_{p-2}z^{p-3} \in A \) ; en appliquant la première partie du raisonnement à cet élément, on voit que \( a_1 \in \Z \). Par applications successives de ce procédé, on voit que chaque \( a_i \in \Z \).
    \end{proof}

    \begin{remark}
        Ce qui a été fait dans cette section s'étend sans peine aux corps cyclotomiques \( \Q[t] \) où \( t \) est une racine primitive \( p^r \)‑ième de l'unité (\( p \) premier). Un tel corps est de degré \( p^{r-1}(p-1) \), et son anneau des entiers est \( \Z[t] \). Le polynôme minimal de \( t \) sur \( \Q \) est
        \[
            X^{p^{r-1}(p-1)} + X^{p^{r-1}(p-2)} + \cdots + X^{p^{r-1}} + 1 = \frac{X^{p^r} - 1}{X^{p^{r-1}} - 1}.
        \]
    \end{remark}

\counterwithout{equation}{section}
\cleardoublepage
\phantomsection
\addcontentsline{toc}{section}{\emph{Appendice}.\texorpdfstring{~~}{ }Le corps \texorpdfstring{\( \C \)}{C} des nombres complexes est algébriquement clos}
\markright{\emph{Appendice}.~~Le corps \( \C \) des nombres complexes est algébriquement clos}
\section*{Appendice}

    \begin{flushleft}
        \textbf{Le corps \( \C \) des nombres complexes est algébriquement clos}
    \end{flushleft}

    Étant donné un corps \( K \), considérons les propriétés suivantes :
    \begin{enumerate-alph}
        \item \label{condition d'être algébriquement clos;cond1} Tout polynôme de degré \( > 0 \) sur \( K \) est un produit de polynômes du premier degré.
        \item \label{condition d'être algébriquement clos;cond2} Tout polynôme de degré \( > 0 \) sur \( K \) admet une racine dans \( K \).
    \end{enumerate-alph}
    Il est clair que \ref{condition d'être algébriquement clos;cond1} implique \ref{condition d'être algébriquement clos;cond2}. Réciproquement, si \ref{condition d'être algébriquement clos;cond2} est vraie, si \( P(X) \) est un polynôme de degré \( d \geqslant 1 \) sur \( K \) et si \( a \in K \) est une racine de \( P(X) \), alors \( P(X) \) est multiple de \( X-a \), et une récurrence sur le degré \( d \) montre que \ref{condition d'être algébriquement clos;cond1} est vraie. Un corps \( K \) joiussant des propriétés équivalentes \ref{condition d'être algébriquement clos;cond1} et \ref{condition d'être algébriquement clos;cond2} est dit \emph{algébriquement clos}.

    Nous allons montrer que \( \C \) (\( = \R[\i] \), \( \i^2 = -1 \)) est algébriquement clos par une méthode essentiellement due à Lagrange. Nous utiliserons uniquement les faits suivants :
    \begin{enumerate}
        \item \label{faits utilisés pour montrer C est algébriquement clos;fait1} Tout polynôme de degré impair sur \( \R \) admet une racine dans \( \R \) ; ceci est un cas particulier facile du théorème des valeurs intermédiaires.
        \item \label{faits utilisés pour montrer C est algébriquement clos;fait2} Tout polynôme du second degré sur \( \C \) a ses racines dans \( \C \) ; le calcul élémentaire sur \textquote{\( ax^2+bx+c \)} nous ramène à montrer que tout \( z = a + \i b \in \C \) (\( a,b \in \R \)) admet une racine carrée dans \( \C \) ; or \( (x + \i y)^2 = a + \i b \) (\( x,y \in R \)) équivaut à \( x^2-y^2 = a \), \( 2xy = b \), d'où \( a^2 + b^2 = (x^2+y^2)^2 \) et \( x^2+y^2 = \sqrt{a^2 + b^2} \) ; on en déduit les valeurs de \( x^2 \) et \( y^2 \), d'où \( x \) et \( y \).
        \item \label{faits utilisés pour montrer C est algébriquement clos;fait3} Étant donné un polynôme non constant \( P(X) \in K[X] \), il existe une extension \( K' \) de \( K \) telle que \( P(X) \) se décompose en facteurs du premier degré dans \( K'[X] \) ; ceci a été très facilement démontré dans \cref{prop:existence d'une extension algébrique telle que le polynôme se décompose} (démonstration quasi indépendante de ce qui la précède ; il suffit de savoir que, si \( F(X) \) est irréductible, \( K[X]/(F(X)) \) est un corps, et ensuite de faire une récurrence).
        \item \label{faits utilisés pour montrer C est algébriquement clos;fait4} Les relations entre coefficients et racines d'un polynôme.
        \item \label{faits utilisés pour montrer C est algébriquement clos;fait5} Le fait qu'un polynôme symétrique \( G(X_1, \cdots, X_n) \in K[X_1, \cdots, X_n] \) est un polynôme par rapport aux fonctions symétriques élémentaires \( \sum X_i \), \( \sum X_i X_j \), \( \cdots \), \( X_1 \cdots X_n \) des~\( X_i \).
    \end{enumerate}
    Ceci étant on a :
    \begin{theorem*}
        Le corps \( \C \) des nombres complexes est algébriquement clos.
    \end{theorem*}
    \begin{proof}
        Nous utiliserons la propriété \ref{condition d'être algébriquement clos;cond2}, que tout polynôme non constant \( P(X) \in \C[X] \) admet une racine dans \( \C \). En considérant \( F(X) = P(X) \conj{P}(X) \) (\( \conj{P} \) : polynôme dont les coefficients sont les complexes conjugués des coefficients correspondants de \( P \)) on se ramène au cas d'un polynôme à coefficients réels : en effet, si \( a \in \C \) est une racine de \( F(X) \), alors ou bien \( a \) est racine de \( P(X) \), ou bien \( a \) est racine de \( \conj{P}(X) \) et alors \( \conj{a} \) est racine de \( P(X) \). Ceci étant nous mettrons le degré de \( F(X) \) (\( \in \R[X] \)) sous la forme \( d = 2^n q \) où \( q \) est impair. Nous procéderons par récurrence sur l'\emph{exposant} \( n \) de \( 2 \). Pour \( n = 0 \), \( d \) est impair et \( F(X) \) a une racine dans \( \R \) (\cf.~\ref{faits utilisés pour montrer C est algébriquement clos;fait1}). Supposons \( n \geqslant 1 \). Par \ref{faits utilisés pour montrer C est algébriquement clos;fait3} il existe une extension \( K' \) de \( \C \) et \( x_1, \cdots, x_d \in K' \) tels que \( F(X) = \prod_{i=1}^d (X-x_i) \) (en supposant \( F(X) \) unitaire, ce qui est loisible). Soit \( c \) un élément arbitraire de \( \R \) ; considérons les éléments \( y_{ij} = x_i + x_j + c x_i x_j \) de \( K' \) (\( i \leqslant j \)) ; leur nombre est \( \frac{1}{2} d(d+1) = 2^{n-1} q (d+1) \) et \( q(d+1) \) est \emph{impair}. Le polynôme \( G(X) = \prod_{i \leqslant j} (X-y_{ij}) \) a pour coefficients des polynômes symétriques à coefficients réels en les \( x_i \) ; ce sont donc par \ref{faits utilisés pour montrer C est algébriquement clos;fait5} des polynômes à coefficients réels en les fonctions symétriques élémentaires des \( x_i \) ; ainsi les coefficients de \( G(X) \) sont réel par \ref{faits utilisés pour montrer C est algébriquement clos;fait4}. Comme son degré est de la forme \( 2^{n-1} \times \text{(impair)} \), l'hypothèse de récurrence montre qu'il admet une racine \( z_c \in \C \) ; l'un des \( y_{ij} \)\,, soit \( y_{i(c),j(c)} = x_{i(c)} + x_{j(c)} + c x_{i(c)} x_{j(c)} \) est donc égal à \( z_c \).

        Or, comme \( \R \) est \emph{infini} est l'ensemble des couples \( (i,j) \) (\( i \leqslant j \)) fini, il existe deux nombre réels distinctes \( c, c' \) tels que \( i(c) = i(c') \) et \( j(c) = j(c') \) ; notons \( r,s \) ces indices. Alors \( x_r + x_s + c x_r x_s = z_c \in \C \) et \( x_r + x_s + c' x_r x_s = z_{c'} \in \C \). Par conbinaisons linéaires on en déduit \( x_r + x_s \in \C \) et \( x_r x_s \in \C \). Alors, par \ref{faits utilisés pour montrer C est algébriquement clos;fait4}, \( x_r \) et \( x_s \) sont racines d'une équation du second degré à coefficients dans \( \C \). On a donc \( x_r, x_s \in \C \) par \ref{faits utilisés pour montrer C est algébriquement clos;fait2}. Ainsi \( F(x) \) a une racine dans \( \C \)\,, et le théorème est démontré.
    \end{proof}

\counterwithin{equation}{section}


\chapter{Anneaux nœthériens, anneaux de Dedekind}\label{chap:anneaux nœthériens;anneaux de Dedekind}

Le lecteur qui veut savoir pourquoi on a introduit les anneaux de Dedekind pourra se reporter au \cref{sec:anneaux de Dedekind}, et y lire l'exemple et la discussion qui suivent \cref{thm:anneau des entiers est de Dedekind}. Les anneaux nœthériens, dont nous étudions d'abord un minimum de propriétés, sont plus généraux que les anneaux de Dedekind ; Nous les introduisons pour placer ces propriétés dans leur cadre naturel de validité, et aussi parce qu'ils jouent un rôle fondamental dans d'autres applications de l'algèbre, en Géométrie Algébrique par exemple. Enfin la généralisation des anneaux nœthériens aux modules de même nom est un nouveau cas de \textquote{linéarisation}, méthode dont le lecteur a déjà constaté l'efficacité.

\section{Modules et anneaux nœthériens}

    On a démontré \cref[à]{thm:nœthérien} le résultat suivant :
    \begin{theorem}\label{thm:nœthérien'}
        Soient \( A \) un anneau, \( M \) un \( A \)‑module. Les conditions suivantes sont équivalents :
        \begin{enumerate}
            \item \label{thm:nœthérien;item1'} Toute famille non vide de sous-modules de \( M \) possède un élément maximal.
            \item \label{thm:nœthérien;item2'} Toute suite croissante de sous-modules de \( M \) est stationnaire.
            \item \label{thm:nœthérien;item3'} Tout sous-module de \( M \) est de type fini.
        \end{enumerate}
    \end{theorem}

    \begin{definition}[module nœthérien ; anneau nœthérien]\index{noetherien (anneau, module)@nœthérien (anneau, module)}
        Un \( A \)‑module \( M \) est dit \emph{nœthérien}\index{module!noetherien@nœthérien} s'il satisfait aux conditions équivalentes \cref[de]{thm:nœthérien'}. Un anneau \( A \) est dit \emph{nœthérien}\index{anneau!noetherien@nœthérien} si, considéré comme \( A \)‑module, c'est un module nœthérien.
    \end{definition}

    On a vu (\cref{cor:anneau principal;famille d'idéaux;élément maximal}) qu'un anneau principal est nœthérien.

    \begin{proposition}\label{prop:nœthérien par sous-module et module quotient}
        Soient \( A \) un anneau, \( E \) un \( A \)‑module, et \( E' \) un sous-module de \( E \). Pour que \( E \) soit nœthérien, il faut et il suffit que \( E' \) et \( E/E' \) soient nœthériens.
    \end{proposition}
    \begin{proof}
        Démontrons la nécessité. Supposons \( E \) nœthérien. L'ensemble ordonné des sous-modules de \( E' \) (resp. de \( E/E' \)) est isomorphe à l'ensemble ordonné des sous-modules de \( E \) contenus dans \( E' \) (resp. contenant \( E' \)). Ainsi \( E' \) et \( E/E' \) sont nœthériens par \ref{thm:nœthérien;item1'} ou \ref{thm:nœthérien;item2'}.

        Réciproquement, supposons \( E' \) et \( E/E' \) nœthériens. Soit \( (F_n)_{n \geqslant 0} \) une suite croissante de sous-modules de \( E \). Comme \( E' \) est nœthérien, il existe un entier \( n_0 \) tel que \( F_n \cap E' = F_{n+1} \cap E' \) pour tout \( n \geqslant n_0 \). Comme \( E/E' \) est nœthérien, il existe un entier \( n_1 \) tel que \( (F_n+E')/E' = (F_{n+1}+E')/E' \) pour tout \( n \geqslant n_1 \) ; alors on a \( F_n+E' = F_{n+1}+E' \). Prenons \( n \geqslant \sup(n_0,n_1) \), et montrons qu'on a \( F_n = F_{n+1} \) ; il suffit de voir que \( F_{n+1} \subset F_n \). Soit donc \( x \in F_{n+1} \) ; comme \( F_{n+1} + E' = F_n + E' \), il existe \( y \in F_n \) et \( z', z'' \in E' \) tels que \( x+z' = y+z'' \) ; alors \( x-y = z''-z' \in F_{n+1} \cap E' \) ; or \( F_{n+1} \cap E' = F_n \cap E' \) ; on a donc \( x-y \in F_n \)\,, d'où \( x \in F_n \) car \( y \in F_n \). Ainsi \( F_{n+1} = F_n \) pour tout \( n \geqslant \sup(n_0,n_1) \), et \( E \) est nœthérien d'après~\ref{thm:nœthérien;item2'}.
    \end{proof}

    \begin{corollary}\label{cor:module de type fini sur un anneau nœthérien}
        Soient \( A \) un anneau, \( E_1, \cdots, E_n \) des \( A \)‑modules nœthériens. Alors le \( A \)‑module produit \( \prod_{i=1}^n E_i \) est nœthérien.
    \end{corollary}
    \begin{proof}
        Pour \( n = 2 \), \( E_1 \) s'identifie au sous-module \( E_1 \times (0) \) de \( E_1 \times E_2 \)\,, et le quotient correspondant est isomorphe à \( E_2 \) ; d'où notre assertion par \cref{prop:nœthérien par sous-module et module quotient}. Le cas général s'ensuit par récurrence sur~\( n \).
    \end{proof}

    \begin{corollary}\label{cor:module de type fini sur anneau nœthérien est nœthérien}
        Soient \( A \) un anneau nœthérien et \( E \) un \( A \)‑module de type fini. Alors \( E \) est un \( A \)‑module nœthérien (et donc tous ses sous-modules sont de type fini).
    \end{corollary}
    \begin{proof}
        En effet (\cref{sec:quelques préliminaires sur les modules}), \( E \) est isomorphe à un module quotient \( A^n/R \) (\( n \) étant le cardinal d'un système générateur fini de \( E \)). Or \( A^n \) est nœthérien par \cref{cor:module de type fini sur un anneau nœthérien}, et \( A^n/R \) aussi par \cref{prop:nœthérien par sous-module et module quotient}.
    \end{proof}

\section{Application aux éléments entiers}

    \begin{proposition}\label{prop:anneau des entiers est nœthérien}
        Soient \( A \) un anneau nœthérien intégralement clos, \( K \) son corps des fractions, \( L \) une extension de degré fini \( n \) de \( K \), et \( A' \) la fermeture intégrale de \( A \) dans \( L \). On suppose \( K \) de caractéristique \( 0 \). Alors \( A' \) est un \( A \)‑module de type fini et un anneau nœthérien.
    \end{proposition}
    \begin{proof}
        En effet on sait que \( A' \) est un sous-module d'un \( A \)‑module libre de rang \( n \) (\cref{thm:structure de fermeture intégrale dans une extension fini}). Donc \( A' \) est une \( A \)‑module de type fini (\cref{cor:module de type fini sur anneau nœthérien est nœthérien} \cref[de]{prop:nœthérien par sous-module et module quotient}), et donc nœthérien (\ibid). D'autre part les idéaux de \( A' \) sont des cas particuliers de sous-\( A \)‑modules de \( A' \) ; ils satisfont donc à la condition maximale (\cref{thm:nœthérien}, \ref{thm:nœthérien;item1'}), de sorte que \( A' \) est un anneau nœthérien.
    \end{proof}

    \begin{example}
        L'anneau des entiers d'un corps de nombres \( L \) est \emph{nœthérien} (prendre \( A = \Z \)\,, \( K = \Q \)).
    \end{example}

\section{Quelques préliminaires sur les idéaux}\label{sec:quelques préliminaires sur les idéaux}

    Un idéal \( \ideal{p} \) d'un anneau \( A \) est dit \emph{premier}\index{ideal@idéal!premier}\index{premier (ideal)@premier (idéal)} si l'anneau quotient \( A/\ideal{p} \) est \emph{intègre}. Il revient au même de dire que les relations \( x \in A \setminus \ideal{p}, y \in A \setminus \ideal{p} \) impliquent \( xy \in A \setminus \ideal{p} \), ou encore que le complémentaire \( A \setminus \ideal{p} \) de \( \ideal{p} \) est stable pour la multiplication.

    Pour qu'un idéal \( \ideal{m} \) de \( A \) soit \emph{maximal}\index{ideal@idéal!maximal}\index{maximal (ideal)@maximal (idéal)} (c'est-à-dire maximal parmi les idéaux de \( A \) distincts de \( A \)), il faut et il suffit que \( A/\ideal{m} \) n'ait d'autres idéaux que lui-même et \( (0) \), c'est-à-dire que \( A/\ideal{m} \) soit un \emph{corps}. Ainsi \emph{tout idéal maximal est premier}. La réciproque est fausse, car l'idéal \( (0) \) de \( \Z \) est premier et non maximal.

    \begin{lemma}\label{lem:intersection d'un idéal premier et un sous-anneau est un idéal premier}
        Soient \( A \) un anneau, \( \ideal{p} \) un idéal premier de \( A \) et \( A' \) un sous-anneau de \( A \). Alors \( \ideal{p} \cap A' \) est un idéal premier de~\( A' \).
    \end{lemma}
    \begin{proof}
        En effet \( \ideal{p} \cap A' \) est le noyau de l'homomorphisme composé \( A' \to A \to A/\ideal{p} \), de sorte qu'on a un homomorphisme injectif \( A'/(\ideal{p} \cap A') \to A/\ideal{p} \). Or un sous-anneau d'un anneau intègre est intègre.
    \end{proof}

    Étant donnés deux idéaux \( \ideal{a} \) et \( \ideal{b} \) d'un anneau \( A \), on appelle \emph{produit}\index{produit d'ideaux@produit d'idéaux} de \( \ideal{a} \) et de \( \ideal{b} \), et on note \( \ideal{a}\ideal{b} \), non pas l'ensemble des produits \( ab \) où \( a \in \ideal{a} \) et \( b \in \ideal{b} \) (ensemble qui n'est pas en général un idéal), mais l'ensemble des \emph{somme finies} \( \sum a_i b_i \) de tels produits. On voit aussitôt que \( \ideal{a}\ideal{b} \) est un idéal de \( A \), et qu'on a
    \begin{equation}
        \ideal{a}\ideal{b} \subset \ideal{a} \cap \ideal{b}.
    \end{equation}
    \begin{quote}
        Il n'y a pas toujours égalité : dans un anneau principal le membre de gauche correspond au produit, et celui de droite au p.p.c.m.
    \end{quote}

    Le produit des idéaux est associative et commutatif, et \( A \) est élément neutre.

    \begin{quote}
        Étant données un \( A \)‑module \( E \), un sous-module \( F \), et un idéal \( \ideal{a} \) de \( A \), on définit de même le produit \( \ideal{a} F \) ; c'est un sous-module de~\( E \).
    \end{quote}

    \begin{lemma}\label{lem:idéal premier qui contient un produit d'idéaux contient l'un d'eux}
        Si un idéal premier \( \ideal{p} \) d'un anneau \( A \) contient un produit \( \ideal{a_1} \ideal{a_2} \cdots \ideal{a_n} \) d'idéaux, alors \( \ideal{p} \) contient l'un d'eux.
    \end{lemma}
    \begin{proof}
        En effet si \( \ideal{a_i} \not\subset \ideal{p} \) pour tout \( i \), il existe \( a_i \in \ideal{a_i} \) tel que \( a_i \notin \ideal{p} \). On a alors \( a_1 \cdots a_n \notin \ideal{p} \) vue que \( \ideal{p} \) est premier. Or \( a_1 \cdots a_n \in \ideal{a_1} \cdots \ideal{a_n} \). Contradiction.
    \end{proof}

    \begin{lemma}\label{lem:anneau nœthérien;tout idéal contient un produit d'idéaux premiers}
        Dans un anneau nœthérien, tout idéal contient un produit d'idéaux premiers. Dans un anneau nœthérien intègre \( A \), tout idéal non nul contient un produit d'idéaux premiers non nuls.
    \end{lemma}
    \begin{proof}
        Nous allons employer un raisonnement assez typique dans la théorie des anneaux nœthériens. Démontrons la seconde assertion (la démonstration de la première est analogue : il suffit de barrer trois fois \textquote{non nuls}). Raisonnons par l'absurde. La famille \( \Phi \) des idéaux non nuls de \( A \) qui ne contiennent aucun produit d'idéaux premiers non nuls, est alors \emph{non vide}. Comme \( A \) est nœthérien, \( \Phi \) admet un élément \emph{maximal} \( \ideal{b} \) (\cref{thm:nœthérien}, \ref{thm:nœthérien;item1'}). L'idéal \( \ideal{b} \) n'est pas premier, sinon il contiendrait le produit de la famille réduite à \( \ideal{b} \). Il existe donc \( x, y \in A \setminus \ideal{b} \) tels que \( xy \in \ideal{b} \). Alors les idéaux \( \ideal{b}+Ax \) et \( \ideal{b}+Ay \) contiennent strictement \( \ideal{b} \), donc n'appartiennent pas à \( \Phi \) vu le caractère maximal de \( \ideal{b} \) dans \( \Phi \). Ils contiennent donc des produits d'idéaux premiers non nuls :
        \[
            \ideal{b}+Ax \supset \ideal{p_1} \cdots \ideal{p_n}\,, \quad
            \ideal{b}+Ay \supset \ideal{q_1} \cdots \ideal{q_r}.
        \]
        Or, comme \( xy \in \ideal{b} \), on a
        \[
            (\ideal{b}+Ax)(\ideal{b}+AY) \subset \ideal{b} ~;
            \quad \text{ d'où } \quad
            \ideal{p_1} \cdots \ideal{p_n} \ideal{q_1} \cdots \ideal{q_r} \subset \ideal{b},
        \]
        contradiction.
    \end{proof}

    Enfin, soient \( A \) un anneau \emph{intègre} et \( K \) son corps des fractions. On appelle \emph{idéal fractionnaire}\index{fractionnaire (ideal)@fractionnaire (idéal)}\index{ideal fractionnaire@idéal fractionnaire} de \( A \) (ou de \( K \) par rapport à \( A \)) tout sous-\( A \)‑module \( I \) de \( K \) tel qu'il existe de \( d \in A \), \( d \neq 0 \) satisfaisant à \( I \subset d^{-1}A \) ; ceci revient à dire que les éléments de \( I \) ont un \textquote{dénominateur commun} \( d \in A \). Les idéaux ordinaires de \( A \) sont des idéaux fractionnaires (avec \( d = 1 \)) ; on les qualifie parfois d'\emph{idéaux entiers}\index{entier (ideal)@entier (idéal)}\index{ideal entier@idéal entier} s'il y a risque de~confusion.

    Tout sous-\( A \)‑module \emph{de type fini} \( I \) de \( K \) est un idéal fractionnaire ; en effet, si \( (x_1, \cdots, x_n) \) est un système générateur fini de \( I \), les \( x_i \) ont un dénominateur commun \( d \) (par exemple le produit des dénominateurs \( d_i \)\,, où \( x_i = a_i d_i^{-1} \) avec \( a_i, d_i \in A \)), et \( d \) sert de dénominateur commun à \( I \). Réciproquement, si \( A \) est \emph{nœthérien}, tout idéal fractionnaire \( I \) est un \( A \)‑module \emph{de type fini} : en effet on a \( I \subset d^{-1} A \), et \( d^{-1} A \) est un \( A \)‑module isomorphe à \( A \), donc nœthérien.

    On définit le \emph{produit} \( I I' \) de deux idéaux fractionnaires \( I, I' \) comme l'ensemble des sommes finies \( \sum x_i y_i \) où \( x_i \in I \) et \( y_i \in I' \). Si \( I \) et \( I' \) sont deux idéaux fractionnaires, de dénominateurs communs \( d \) et \( d' \), alors les ensembles
    \[
        I \cap I', \quad I+I', \quad I I'
    \]
    sont des \emph{idéaux fractionnaires} ; en effet ce sont évidemment des sous-\( A \)‑modules de \( K \), et ils admettent respectivement pour dénominateurs communs \( d \) (ou \( d' \)), \( d d' \) et~\( d d' \). Les idéaux fractionnaires non nuls de \( A \) forment un \emph{monoïde} commutatif pour la multiplication.

\section{Anneaux de Dedekind}\label{sec:anneaux de Dedekind}

    \begin{definition}[anneau de Dedekind]\index{anneau!de Dedekind}\index{Dedekind (anneau de)}
        Un anneau \( A \) est appelé un anneau de Dedekind s'il est nœthérien et intégralement clos (donc intègre), et si tout idéal premier non nul de \( A \) est maximal.
    \end{definition}

    L'anneau \( \Z \)\,, et plus généralement tout anneau principal, est un anneau de Dedekind. L'anneau des entiers d'un corps de nombres est un anneau de Dedekind, d'après \namecref{thm:anneau des entiers est de Dedekind} suivant :

    \begin{theorem}\label{thm:anneau des entiers est de Dedekind}
        Soient \( A \) un anneau de Dedekind, \( K \) son corps des fractions, \( L \) une extension de degré fini de \( K \) et \( A' \) la fermeture intégrale de \( A \) dans \( L \). On suppose \( K \) de caractéristique \( 0 \). Alors \( A' \) est un anneau de Dedekind et un \( A \)‑module de type fini.
    \end{theorem}
    \begin{proof}
        En effet \( A' \) est intégralement clos par construction, nœthérien et \( A \)‑module d'après \cref{prop:anneau des entiers est nœthérien}. Reste à montrer que tout idéal premier \( \ideal{p}' \neq (0) \) de \( A' \) est maximal. Or prenons un élément \( x \neq 0 \) de \( \ideal{p}' \) et une équation de dépendance intégrale de \( x \) sur \( A \), de degré minimum :
        \begin{equation}\label{eq:thm:anneau des entiers est de Dedekind;dépendance intégrale}
            x^n + a_{n-1}x^{n-1} + \cdots + a_1 x + a_0 = 0 \quad \text{(\( a_i \in A \))}.
        \end{equation}
        On a \( a_0 \neq 0 \), car sinon on simplifierait par \( x \), et on obtiendrait une équation de dépendance intégrale de degré \( n-1 \). Par \eqref{eq:thm:anneau des entiers est de Dedekind;dépendance intégrale}, on a \( a_0 \in A'x \cap A \subset \ideal{p}' \cap A \) ; on a donc \( \ideal{p}' \cap A \neq (0) \). Or \( \ideal{p}' \cap A \) est un idéal premier de \( A \) (\cref{lem:intersection d'un idéal premier et un sous-anneau est un idéal premier}) ; donc \( \ideal{p}' \cap A \) est un idéal maximal de \( A \), et \( A/(\ideal{p}' \cap A) \) est un corps. Mais \( A/(\ideal{p}' \cap A) \) s'identifie à un sous-anneau de \( A'/\ideal{p}' \), et \( A'/\ideal{p}' \) est \emph{entier} sur \( A/(\ideal{p}' \cap A) \) car \( A' \) est entier sur \( A \). Donc \( A'/\ideal{p}' \) est un corps (\cref{prop:anneau entier et corps}), de sorte que \( \ideal{p}' \) est maximal.
    \end{proof}

    L'intérêt des anneaux de Dedekind vient de ce que l'anneau des entiers d'un corps de nombres est un anneau de Dedekind, mais n'est pas toujours principal.

    \begin{quote}
        \textsc{\textbf{Exemple}}.
        Considérons l'anneau des entiers \( A = \Z[\sqrt{-5}] \) de \( \Q[\sqrt{-5}] \) (\cref{thm:entiers dans un corps quadratique}). On a
        \begin{equation}\label{eq:exemple d'une factorisation dans Z[sqrt{-5}]}
            (1+\sqrt{-5})(1-\sqrt{-5}) = 2 \cdot 3.
        \end{equation}
        Les normes des quatre facteurs sont respectivement \( 6 \), \( 6 \), \( 4 \) et \( 9 \) ; or \( 1+\sqrt{-5} \) ne peut avoir de diviseur non-trivial dans \( A \), car la norme d'un tel diviseur devrait être un diviseur non-trivial de \( 6 \), et que les équations
        \[
            a^2 + 5b^2 = 2
            \quad \text{ et } \quad
            a^2 + 5b^2 = 3
        \]
        n'ont pas de solution dans \( \Z \). Si \( A \) était principal, l'élément \( 1+\sqrt{-5} \), qui divise le produit \( 2 \cdot 3 \) par \eqref{eq:exemple d'une factorisation dans Z[sqrt{-5}]}, devrait diviser l'un de ses facteurs ; mais alors, en prenant les normes, \( 6 \) diviserait \( 4 \) ou \( 9 \), ce qui n'est pas.
    \end{quote}

    Historiquement, l'arithméticien Kummer (1810-1893) s'aperçut de la non principalité de certains anneaux d'entiers de corps de nombres (en fait, de corps cyclotomiques, ceci en liaison avec ses travaux sur l'équation de Fermat ; \cf.~\cref{sec:l'équation de Fermat;cas 2 et 4}). Pour obvier partiellement à cet inconvénient, lui et Dedekind (1831-1916) introduisirent la notion d'\emph{idéal}, et Dedekind étudia les anneaux qui portent maintenant son nom. L'intérêt majeur des anneaux principaux est l'unique décomposition en facteurs premiers. Dans les anneaux de Dedekind celle-ci est heureusement généralisée en une unique décomposition en \emph{idéaux premiers}, qui rend presque autant de services, et que nous allons maintenant décrire :
    \begin{theorem}\label{thm:anneau de Dedekind;idéal maximal est inversible}
        Soit \( A \) un anneau de Dedekind qui n'est pas un corps. Tout idéal maximal de \( A \) est inversible dans le monoïde des idéaux fractionnaires de~\( A \).
    \end{theorem}
    \begin{proof}
        Soit \( \ideal{m} \) un idéal maximal de \( A \) ; on a \( \ideal{m} \neq (0) \) car \( A \) n'est pas un corps. Posons
        \begin{equation}\label{eq:inverse d'un idéal fractionnaire}
            \ideal{m}' = \set{ x \in K \given x\ideal{m} \subset A }.
        \end{equation}
        Il est clair que \( \ideal{m}' \) est un sous-\( A \)‑module de \( K \), et qu'il admet pour dénominateur commun n'importe quel élément non nul de \( \ideal{m} \) ; donc \( \ideal{m}' \) est un idéal fractionnaire de \( A \). Il va nous suffire de montrer que \( \ideal{m}'\ideal{m} = A \). Or on a \( \ideal{m}'\ideal{m} \subset A \) d'après \eqref{eq:inverse d'un idéal fractionnaire} ; d'autre part il est clair que \( A \subset \ideal{m}' \) (car \( \ideal{m} \) est un idéal), d'où \( \ideal{m} = A \ideal{m} \subset \ideal{m}'\ideal{m} \). Comme \( \ideal{m} \) est maximal et que \( \ideal{m} \subset \ideal{m}'\ideal{m} \subset A \), on a soit \( \ideal{m}'\ideal{m} = A \), soit \( \ideal{m}'\ideal{m} = \ideal{m} \). Reste à montrer que \( \ideal{m}'\ideal{m} = \ideal{m} \) est impossible.

        Or, si \( \ideal{m}'\ideal{m} = \ideal{m} \) et si \( x \in \ideal{m}' \), on a \( x \ideal{m} \subset \ideal{m} \), d'où \( x^2 \ideal{m} \subset x \ideal{m} \subset \ideal{m} \), et \( x^n \ideal{m} \subset \ideal{m} \) pourtant \( n \in \N \) par récurrence. Ainsi n'importe quel élément non nul \( d \) de \( \ideal{m} \) sert de dénominateur commun à tous les \( x^n \), de sorte que \( A[x] \) est un idéal fractionnaire de \( A \). Comme \( A \) est nœthérien, \( A[x] \) est un \( A \)‑module de type fini (\cref{sec:quelques préliminaires sur les idéaux}, fin), donc \( x \) est \emph{entier} sur \( A \) (\cref{thm:conditions d'être entier}). Or \( A \) est intégralement clos ; on a donc \( x \in A \). Ainsi \( \ideal{m}'\ideal{m} = \ideal{m} \) implique que \( \ideal{m}' = A \). Reste à montrer que \( \ideal{m}' = A \) est impossible.

        En effet prenons un élément non nul \( a \in \ideal{m} \). L'idéal \( Aa \) contient un produit \( \ideal{p_1} \ideal{p_2} \cdots \ideal{p_n} \) d'idéaux premiers non nuls (\cref{lem:anneau nœthérien;tout idéal contient un produit d'idéaux premiers}) ; on peut supposer \( n \) minimum. On a \( \ideal{m} \supset Aa \supset \ideal{p_1} \cdots \ideal{p_n} \)\,, donc \( \ideal{m} \) contient l'un des \( \ideal{p_1} \) (\cref{lem:idéal premier qui contient un produit d'idéaux contient l'un d'eux}), \( \ideal{p_1} \) par exemple. Comme \( \ideal{p_1} \) est maximal par hypothèse, on a \( \ideal{m} = \ideal{p_1} \). Posons \( \ideal{b} = \ideal{p_2} \cdots \ideal{p_n} \) ; on a \( Aa \supset \ideal{m}\ideal{b} \), et \( Aa \not\supset \ideal{b} \) d'après le caractère minimal de \( n \). Il existe donc un élément \( b \in \ideal{b} \) tel que \( b \notin Aa \). Comme \( \ideal{m}\ideal{b} \subset Aa \), on a \( \ideal{m} b \subset Aa \), d'où \( \ideal{m} b a^{-1} \subset A \) ; d'après la définition \eqref{eq:inverse d'un idéal fractionnaire} de \( \ideal{m}' \), ceci prouve que \( ba^{-1} \in \ideal{m}' \). Or, comme \( b \notin Aa \), on a \( ba^{-1} \notin A \) ; d'où \( \ideal{m}' \neq A \).
    \end{proof}

    \begin{theorem}\label{thm:propriétés d'anneaux de Dedekind}
        Soient \( A \) un anneau de Dedekind, \( P \) l'ensemble des idéaux premiers non nuls de \( A \).
        \begin{enumerate}
            \item \label{thm:propriétés d'anneaux de Dedekind;item1} Tout idéal fractionnaire non nul \( \ideal{b} \) de \( A \) s'écrit, d'une façon et d'une seule, sous la forme
            \begin{equation}\label{eq:factorisation d'un idéal fractionnaire}
                \ideal{b} = \prod_{\ideal{p} \in P} \ideal{p}^{n_{\ideal{p}}(\ideal{b})}
            \end{equation}
            où les \( n_{\ideal{p}}(\ideal{b}) \) sont des entiers relatifs, presque tous nuls.
            \item \label{thm:propriétés d'anneaux de Dedekind;item2} Le monoïde des idéaux fractionnaires non-nuls de \( A \) est un groupe.
        \end{enumerate}
    \end{theorem}
    \begin{proof}
        Démontrons d'abord l'assertion d'\emph{existence} de \ref{thm:propriétés d'anneaux de Dedekind;item1}, c'est-à-dire que tout idéal fractionnaire \( \ideal{b} \) est produit de puissances (\( \geqslant 0 \) ou \( \leqslant 0 \)) d'idéaux premiers. Or il existe un élément non nul \( d \) de \( A \) tel que \( d \ideal{b} \subset A \) ; \ie. que \( d \ideal{b} \) soit un idéal entier de \( A \) ; ainsi \( \ideal{b} = (d \ideal{b}) \cdot (Ad)^{-1} \), et nous sommes ramenés au cas d'un idéal entier \( \ideal{b} \). Procédons comme dans \cref{lem:anneau nœthérien;tout idéal contient un produit d'idéaux premiers}, et considérons la famille \( \Phi \) des idéaux \( \neq (0) \) de \( A \) qui ne sont pas produits d'idéaux premiers. Supposons, par l'absurde, que \( \Phi \) soit non vide ; elle admet alors un élément maximal \( \ideal{a} \), car \( A \) est nœthérien. On a \( \ideal{a} \neq A \), car \( A \) est produit de la famille vide d'idéaux premiers. Alors \( \ideal{a} \) est contenu dans un idéal maximal \( \ideal{p} \), à savoir un élément maximal de la famille des idéaux non triviaux de \( A \) qui contiennent \( \ideal{a} \). Soit \( \ideal{p}' \) l'idéal (fractionnaire) inverse de \( \ideal{p} \). De \( \ideal{a} \subset \ideal{p} \) on déduit \( \ideal{a} \ideal{p}' \subset \ideal{p}\ideal{p}' = A \). Comme \( \ideal{p}' \supset A \), on a \( \ideal{a} \ideal{p}' \supset \ideal{a} \), et même \( \ideal{a}\ideal{p}' \neq \ideal{a} \) : en effet, si \( \ideal{a} \ideal{p}' = \ideal{a} \) et si \( x \in \ideal{p}' \), on aurait \( x\ideal{a} \subset \ideal{a} \), \( x^n\ideal{a} \subset \ideal{a} \) pour tout \( n \), \( x \) entier sur \( A \) et \( x \in A \) (comme dans \cref{thm:anneau de Dedekind;idéal maximal est inversible}) ; or ceci est impossible car \( \ideal{p}' \neq A \) (sinon \( \ideal{p}' = A \) et \( \ideal{p}\ideal{p}' = \ideal{p} \)). D'après le caractère maximal de \( \ideal{a} \) dans \( \Phi \), on a donc \( \ideal{a}\ideal{p}' \notin \Phi \), et \( \ideal{a}\ideal{p}' \) est un produit \( \ideal{p_1} \cdots \ideal{p_n} \) d'idéaux premiers. En multipliant par \( \ideal{p} \), on voit que \( \ideal{a} = \ideal{p} \ideal{p_1} \cdots \ideal{p_n} \). Ainsi tout idéal entier de \( A \) est produit d'idéaux premiers.

        Passons à l'assertion d'\emph{unicité} de \ref{thm:propriétés d'anneaux de Dedekind;item1}. Supposons qu'on ait \( \prod_{\ideal{p} \in P} \ideal{p}^{n(\ideal{p})} = \prod_{\ideal{p} \in P} \ideal{p}^{m(\ideal{p})} \), c'est-à-dire \( \prod_{\ideal{p} \in P} \ideal{p}^{n(\ideal{p})-m(\ideal{p})} = A \). Si les \( n(\ideal{p})-m(\ideal{p}) \) ne sont pas tous nuls, on sépare les exposants positifs et les exposants négatifs et on obtient :
        \begin{equation}
            \ideal{p_1}^{\alpha_1} \cdots \ideal{p_r}^{\alpha_r} = \ideal{q_1}^{\beta_1} \cdots \ideal{q_s}^{\beta_s}
        \end{equation}
        avec \( \ideal{p_i}, \ideal{q_j} \in P \), \( \alpha_i > 0 \), \( \beta_j > 0 \), \( \ideal{p_i} \neq \ideal{q_j} \) pour tous \( i,j \). Alors \( \ideal{p_1} \) contient \( \ideal{q_1}^{\beta_1} \cdots \ideal{q_s}^{\beta_s} \) ; il contient donc l'un des \( \ideal{q_j} \) (\cref{lem:idéal premier qui contient un produit d'idéaux contient l'un d'eux}), soit \( \ideal{p_1} \supset \ideal{q_1} \). Comme \( \ideal{p_1} \) et \( \ideal{q_1} \) sont tous deux maximaux, ceci implique \( \ideal{p_1} = \ideal{q_1} \), ce qui est contradictoire.

        Enfin \eqref{eq:factorisation d'un idéal fractionnaire} montre que \( \prod_{\ideal{p} \in P} \ideal{p}^{-n_{\ideal{p}}(\ideal{b})} \) est l'inverse de \( \ideal{b} \), ce qui prouve~\ref{thm:propriétés d'anneaux de Dedekind;item2}.
    \end{proof}

    \begin{remark}
        On vient de voir que le monoïde \( I(A) \) des idéaux fractionnaires non nuls d'un anneau de Dedekind \( A \) est un groupe. Or les idéaux fractionnaires \emph{principaux} (c.-à.-d. de la forme \( Ax \), \( x \in K^\times \)) forment un sous-groupe \( F(A) \) de \( I(A) \) (car \( (Ax)\cdot(Ay)^{-1} = Axy^{-1} \)). Le groupe quotient \( C(A) = I(A) / F(A) \) s'appelle le \emph{groupe des classes d'idéaux}\index{classes d'ideaux@classes d'idéaux} de \( A \). Pour que \( A \) soit principal, il faut et il suffit que \( C(A) \) soit réduit à son élément neutre.
    \end{remark}

    Terminons par un \emph{formulaire}, dans lequel \( n_{\ideal{p}}(\ideal{b}) \) désigne d'exposant de \( \ideal{p} \) dans la décomposition de \( \ideal{b} \) en produit d'idéaux premiers (\cf.~\eqref{eq:factorisation d'un idéal fractionnaire}).
    \begin{enumerate-Roman}\label{enum:formulaire d'exposant dans la décomposition en produit d'idéaux premiers}
        \item \label{enum:formulaire d'exposant dans la décomposition en produit d'idéaux premiers;item1} \( n_{\ideal{p}}(\ideal{a}\ideal{b}) = n_{\ideal{p}}(\ideal{a}) + n_{\ideal{p}}(\ideal{b}) \). \qquad (trivial)
        \item \label{enum:formulaire d'exposant dans la décomposition en produit d'idéaux premiers;item2} \( \ideal{b} \subset A \) \( \iff \) \( n_{\ideal{p}}(\ideal{b}) \geqslant 0 \) pour tout \( \ideal{p} \in P \).\\
        (\( \Rightarrow \) vu en cours de la démonstration \cref[de]{thm:propriétés d'anneaux de Dedekind} ; \( \Leftarrow \) trivial)
        \item \label{enum:formulaire d'exposant dans la décomposition en produit d'idéaux premiers;item3} \( \ideal{a} \subset \ideal{b} \) \( \iff \) \( n_{\ideal{p}}(\ideal{a}) \geqslant n_{\ideal{p}}(\ideal{b}) \) pour tout \( \ideal{p} \in P \).\\
        (en effet \( \ideal{a} \subset \ideal{b} \) équivaut à \( \ideal{a}\ideal{b}^{-1} \subset A \) ; on applique \ref{enum:formulaire d'exposant dans la décomposition en produit d'idéaux premiers;item1} et \ref{enum:formulaire d'exposant dans la décomposition en produit d'idéaux premiers;item2})
        \item \label{enum:formulaire d'exposant dans la décomposition en produit d'idéaux premiers;item4} \( n_{\ideal{p}}(\ideal{a}+\ideal{b}) = \inf(n_{\ideal{p}}(\ideal{a}),n_{\ideal{p}}(\ideal{b})) \).\\
        (car \( \ideal{a}+\ideal{b} \) est la borne supérieure de \( \ideal{a} \) et \( \ideal{b} \) pour l'inclusion des idéaux ; on applique alors \ref{enum:formulaire d'exposant dans la décomposition en produit d'idéaux premiers;item3})
        \item \label{enum:formulaire d'exposant dans la décomposition en produit d'idéaux premiers;item5} \( n_{\ideal{p}}(\ideal{a}\cap\ideal{b}) = \sup(n_{\ideal{p}}(\ideal{a}),n_{\ideal{p}}(\ideal{b})) \).\\
        (raison analogue ; attention au renversement des inégalités dans \ref{enum:formulaire d'exposant dans la décomposition en produit d'idéaux premiers;item3})
    \end{enumerate-Roman}

\section{Norme d'un idéal}\label{sec:norme d'un idéal}

    Tout au long de cette section, \( K \) désigne un corps de nombres, \( n \) son degré, et \( A \) l'anneau des entiers de \( K \). On écrit \( \Norm(x) \) au lieu de \( \Norm_{K/\Q}(x) \).

    \begin{proposition}\label{prop:norme d'un élément et cardinal d'un quotient}
        Si \( x \) est un élément non nul de \( A \), on a \( \abs{\Norm(x)} = \card(A/Ax) \).
    \end{proposition}
    \begin{quote}
        Notons que, comme \( x \in A \), on a \( \Norm(x) \in \Z \) (\cref{cor:trace et norme d'élément entier} \cref[de]{prop:trace et norme d'élément entier}), de sorte que la formule écrite a un sens.
    \end{quote}
    \begin{proof}
        On sait que \( A \) est un \( \Z \)‑module libre de rang \( n \) (\cref{sec:terminologie des corps de nombres}), et \( Ax \) un sous-\( \Z \)‑module de \( A \). Il est aussi de rang \( n \) car la multiplication par \( x \) est une bijection \( A \to Ax \). D'après \cref{thm:base adaptée}, il existe une base \( (e_1, \cdots, e_n) \) du \( \Z \)‑module \( A \), et des éléments \( c_i \) de \( \N \) tels que \( (c_1 e_1, \cdots, c_n e_n) \) soit une base de \( Ax \). Alors \( A/Ax \) est isomorphe à \( \prod_{i=1}^n \Z/c_i\Z \)\,, et son cardinal est \( c_1 c_2 \cdots c_n \). Notons \( u \) l'application \( \Z \)‑linéaire de \( A \) sur \( Ax \) définie par \( u(e_i) = c_i e_i \) pour \( i = 1, \cdots, n \) ; on a \( \det(u) = c_1 c_2 \cdots c_n \).

        D'autre part \( (xe_1, \cdots, xe_n) \) est aussi une base de \( Ax \) ; on a donc un automorphisme \( v \) du \( \Z \)‑module \( Ax \) tel que \( v(c_i e_i) = xe_i \) ; alors \( \det(v) \) est inversible dans \( \Z \)\,, d'où \( \det(v) = \pm 1 \). Mais alors \( v \circ u \) est la multiplication par \( x \), et son déterminant est, par définition, \( \Norm(x) \) (\cref{def:trace;norme;polynôme caractéristique}). Comme \( \det(v \circ u) = \det(v) \cdot \det(u) \), on en déduit \( \Norm(x) = \pm c_1 c_2 \cdots c_n = \pm \card(A/Ax) \).
    \end{proof}

    \begin{definition}[norme d'un idéal]\label{def:norme d'un idéal}
        Étant donné un idéal entier non nul \( \ideal{a} \) de \( A \), on appelle \emph{norme}\index{norme d'un idéal} de \( \ideal{a} \) et on note \( \Norm(\ideal{a}) \) le nombre \( \card(A/\ideal{a}) \).
    \end{definition}

    Notons que \( \Norm(\ideal{a}) \) est \emph{fini} ; en effet, si \( a \) est un élément non nul de \( \ideal{a} \), on a \( Aa \subset \ideal{a} \), et \( A/\ideal{a} \) s'identifie à un quotient de \( A/Aa \) ; d'où \( \card(A/\ideal{a}) \leqslant \card(A/Aa) \), qui est fini par \cref{prop:norme d'un élément et cardinal d'un quotient}. D'autre part celle-ci montre que, pour un idéal principal \( Ab \), on a \( \Norm(Ab) = \abs{\Norm(b)} \).

    \begin{proposition}\label{prop:multiplicativité des normes}
        Si \( \ideal{a} \) et \( \ideal{b} \) sont deux idéaux entiers non nuls de \( A \), on a \( \Norm(\ideal{a}\ideal{b}) = \Norm(\ideal{a}) \Norm(\ideal{b}) \).
    \end{proposition}
    \begin{proof}
        Par décomposition de \( \ideal{b} \) en produit d'idéaux maximaux (\cref{thm:propriétés d'anneaux de Dedekind}), il suffit de montrer qu'on a \( \Norm(\ideal{a}\ideal{m}) = \Norm(\ideal{a}) \Norm(\ideal{m}) \) pour \( \ideal{m} \) maximal. Comme \( \ideal{a}\ideal{m} \subset \ideal{a} \), on a \( \card(A/\ideal{a}\ideal{m}) = \card(A/\ideal{a}) \card(\ideal{a}/\ideal{a}\ideal{m}) \). Il suffit donc de prouver \( \card(\ideal{a}/\ideal{a}\ideal{m}) = \card(A/\ideal{m}) \). Or \( \ideal{a}/\ideal{a}\ideal{m} \) est un \( A \)‑module annulé par \( \ideal{m} \), donc un espace vectoriel sur \( A/\ideal{m} \). Ses sous-espaces vectoriels sont ses sous-\( A \)‑modules, et sont donc de la forme \( \ideal{q}/\ideal{a}\ideal{m} \) où \( \ideal{q} \) est un idéal tel que \( \ideal{a}\ideal{m} \subset \ideal{q} \subset \ideal{a} \). Or la formule \ref{enum:formulaire d'exposant dans la décomposition en produit d'idéaux premiers;item1} du \cref{sec:anneaux de Dedekind} montre qu'il n'a aucun idéal strictement compris entre \( \ideal{a}\ideal{m} \) et \( \ideal{a} \). Donc l'espace vectoriel \( \ideal{a}/\ideal{a}\ideal{m} \) est de dimension un sur \( A/\ideal{m} \). On a donc bien \( \card(\ideal{a}/\ideal{a}\ideal{m}) = \card(A/\ideal{m}) \).
    \end{proof}


\chapter{Classes d'idéaux, théorème des unités}\label{chap:classes d'idéaux;théorème des unités}

Le présent chapitre est consacré à deux importants théorèmes de finitude. Quelques outils d'Analyse (empruntés à la Topologie et à l'intégration dans \( \R^n \)) nous serons utiles.

\section{Préliminaires sur les groupes discrets de \texorpdfstring{\( \R^n \)}{R\textasciicircum n}}\label{sec:préliminaires sur les groupes discrets de R n}

    Un sous-groupe additif \( H \) de \( \R^n \) est discret si et seulement si, pour tout compact \( K \) de \( \R^n \), l'intersection \( H \cap K \) est finie. Un exemple typique de sous-groupe discret de \( \R^n \) est \( \Z^n \). Nous allons montrer que c'est à peut près le seul :
    \begin{theorem}\label{thm:structure d'un réseau sur R}
        Soit \( H \) un sous-groupe discret de \( \R^n \). Alors \( H \) est engendré (comme \( \Z \)‑module) par \( r \) vecteurs linéairement indépendants sur \( \R \) (d'où \( r \leqslant n \)).
    \end{theorem}
    \begin{proof}
        Choisissons, en effet, un système \( (e_1, \cdots, e_r) \) d'éléments de \( H \) qui sont linéairement indépendants sur \( \R \) et tels que \( r \) soit maximum. Soit
        \begin{equation}\label{eq:structure d'un réseau sur R;aux1}
            P = \set*{ \sum_{i=1}^r \alpha_i e_i \given 0 \leqslant \alpha_i \leqslant 1 } \subset \R^n
        \end{equation}
        le parallélotope construit sur ces vecteurs ; il est clair que \( P \) est compact, donc que \( P \cap H \) est fini. Soit alors \( x \in H \). Vu le caractère maximal de \( (e_i) \), \( x \) s'écrit \( x = \sum_{i=1}^r \lambda_i e_i \) avec \( \lambda_i \in \R \). Considérons alors, pour \( j \in \Z \)\,, l'élément
        \begin{equation}\label{eq:structure d'un réseau sur R;aux2}
            x_j = jx - \sum_{i=1}^r \floor{j\lambda_i} e_i
        \end{equation}
        (où \( \floor{\mu} \) désigne la partie entière de \( \mu \in \R \)). On a alors
        \[
            x_j = \sum_{i=1}^r (j \lambda_i - \floor{j\lambda_i}) e_i \,,
        \]
        d'où \( x_j \in P \), et \( x_j \in P \cap H \) par \eqref{eq:structure d'un réseau sur R;aux2}. Si l'on remarque que \( x = x_1 + \sum_{i=1}^r \floor{\lambda_i} e_i \)\,, on voit que le \( \Z \)‑module \( H \) est engendré par \( P \cap H \), et est donc \emph{de type fini}.

        D'autre part, comme \( P \cap H \) est fini et \( \Z \) infini, il existe deux entiers distincts \( j \) et \( k \) tels que \( x_j = x_k \). Il résulte alors de \eqref{eq:structure d'un réseau sur R;aux2} que, pour ces entiers, on a \( (j-k) \lambda_i = \floor{j\lambda_i} - \floor{k\lambda_i} \), ce qui montre que les \( \lambda_i \) sont \emph{rationnels}. Ainsi le \( \Z \)‑module \( H \) est engendré par un nombre \emph{fini} d'éléments, qui sont combinaisons linéaires à coefficients rationnels des \( (e_i) \). Soit \( d \) un dénominateur commun (\( d \in \Z \)\,, \( d \neq 0 \)) de ces coefficients ; on a alors \( d H \subset \sum_{i=1}^r \Z e_i \). Ainsi il existe une base \( (f_i) \) du \( \Z \)‑module \( \sum_{i=1}^r \Z e_i \) et des \( \alpha_i \in \Z \) tels que \( (\alpha_1 f_1, \cdots, \alpha_r f_r) \) engendrent \( dH \) (\cref{thm:base adaptée}). Comme le \( \Z \)‑module \( dH \) a même rang que \( H \) et que \( H \supset \sum_{i=1}^r \Z e_i \)\,, le rang de \( dH \) est \( \geqslant r \) ; il est donc égal à \( r \) et les \( \alpha_i \) sont non-nuls. Or les \( (f_i) \) sont, comme les \( (e_i) \), linéairement indépendants sur \( \R \). Donc \( dH \), et par conséquent \( H \), est engendré (sur \( \Z \)) par \( r \) éléments linéairement indépendants sur~\( \R \).
    \end{proof}

    \begin{theorem-with-name}{Exemple d'application}
        Soit \( t = (\theta_1, \cdots, \theta_n) \in \R^n \) tel que l'un au moins des \( \theta_i \) soit \emph{irrationnel}. Notons \( (e_1, \cdots, e_n) \) la base canonique de \( \R^n \), et \( H \) le sous-groupe de \( \R^n \) engendré sur \( \Z \) par \( (e_1, \cdots, e_n, t) \). Il n'est pas discret car sinon, d'après la démonstration \cref[de]{thm:structure d'un réseau sur R}, \( t \) serait combinaison linéaire rationnelle des \( e_i \). Donc, \( \varepsilon > 0 \) étant donné, il existe un élément non nul de \( H \) qui approach \( 0 \) à \( \varepsilon \) près ; donc il existe des entiers \( p_i \in \Z \)\,, \( q \in \N \)\,, \( q \neq 0 \) tels que \( \abs{q \theta_i - p_i} \leqslant \varepsilon \), donc que
        \[
            \abs*{\theta_i - \frac{p_i}{q}} \leqslant \frac{\varepsilon}{q}.
        \]
        Notons que l'intercalation simplette de \( \theta_i \) entre deux multiples consécutifs de \( \frac{1}{q} \) donnerait seulement l'approximation \( \smash{ \abs*{\theta_i - \frac{n_i}{q}} \leqslant \frac{1}{2q} } \) (\( n_i \in \Z \)).

        Ce que nous venons d'exposer est un des premiers résultats de la très riche théorie de l'approximation des nombres irrationnels par des nombres rationnels ; pour plus de détails sur cette théorie, voir Koksma, \textquote{Diophantische Approximationen}, Berlin (Springer), 1936.
    \end{theorem-with-name}

    \begin{definition}[réseau]\index{reseau (dans Rn)@réseau (dans \( \R^n \))}
        Un sous-groupe discret de rang \( n \) de \( \R^n \) est appelé un réseau de \( \R^n \).
    \end{definition}

    D'après \cref{thm:structure d'un réseau sur R}, un réseau est engendré sur \( \Z \) par une base de \( \R^n \), qui est alors une \( \Z \)‑base dudit réseau. Pour chaque \( \Z \)‑base \( e = (e_1, \cdots, e_n) \) d'un réseau \( H \), on désignera par \( P_e \) le parallélotope semi-ouvert \( P_e = \set*{ \sum_{i=1}^n \alpha_i e_i \given 0 \leqslant \alpha_i < 1 } \) ; ainsi tout point de \( \R^n \) est congru modulo \( H \) à un point et un seul de \( P_e \) (on dit alors que \( P_e \) est un \emph{domaine fondamental}\index{domaine fondamental} pour \( H \)). Nous noterons \( \mu \) la \emph{mesure de Lebesgue} dans \( \R^n \) ; ainsi, pour tout partie intégrable \( S \) de \( \R^n \), \( \mu(S) \) désignera sa mesure (que nous appellerons aussi son volume).

    \begin{lemma}
        Le volume \( \mu(P_e) \) est indépendant de la base \( e \) choisie pour~\( H \).
    \end{lemma}
    \begin{proof}
        En effet, soit \( f = (f_1, \cdots, f_n) \) une autre base de \( H \). On a \( f_i = \sum_{j=1}^n \alpha_{ij}e_j \) avec \( \alpha_{ij} \in \Z \). L'effet bien connu d'une transformation linéaire sur les volumes montre qu'on a \( \mu(P_f) = \abs{\det(\alpha_{ij})} \mu(P_e) \). Or, comme c'est un déterminant de changement de base, \( \det(\alpha_{ij}) \) est inversible dans \( \Z \)\,, donc vaut \( \pm 1 \). Ainsi \( \mu(P_f) = \mu(P_e) \).
    \end{proof}

    Le volume de l'un quelconque des \( P_e \) est appelé le \emph{volume}\index{volume d'un reseau@volume d'un réseau} du réseau \( H \) et est noté \( v(H) \) (le mot \textquote{volume} est ici un abus de langage, car \( \mu(H) = 0 \) ; peut-être vaudrait-il mieux dire \textquote{maille} du réseau \( H \) ?).

    \begin{theorem}[Minkowski]\label{thm:Minkowski}
        Soient \( H \) un réseau de \( \R^n \) et \( S \) un sous-ensemble intégrable de \( \R^n \) tels que \( \mu(S) > v(H) \). Il existe alors deux éléments \( x, y \) de \( S \) distincts tels que \( x-y \in H \).
    \end{theorem}
    \begin{proof}
        En effet, soient \( e = (e_1, \cdots, e_n) \) une \( \Z \)‑base de \( H \), et \( P_e \) le parallélotope semi-ouvert construit sur \( e \). Comme \( P_e \) est un domaine fondamental pour \( H \), \( S \) est la réunion disjointe des \( S \cap (h + P_e) \) (\( h \in H \)) ; d'où
        \begin{equation}\label{eq:th de Minkowski;aux}
            \mu(S) = \sum_{h \in H} \mu(S \cap (h+P_e)).
        \end{equation}
        Comme \( \mu \) est invariante par translation, on a
        \[
            \mu(S \cap (h+P_e)) = \mu((-h+S) \cap P_e).
        \]
        Or les ensembles \( (-h+S) \cap P_e \) (\( h \in H \)) ne peuvent être deux à deux disjoints car, sinon, \( \mu(P_e) \geqslant \sum_{h \in H} \mu((-h+S) \cap P_e) \), contrairement à \eqref{eq:th de Minkowski;aux} et à l'hypothèse \( \mu(P_e) = v(H) < \mu(S) \). Il existe donc deux éléments distincts \( h, h' \) de \( H \) tels que \( P_e \cap (-h+S) \cap (-h'+S) \neq\nobreak \emptyset \). On a donc des éléments \( x, y \) de \( S \) tels que \( -h + x = -h' + y \), d'où \( x-y = h-h' \in H \), et \( x \neq y \) car \( h \neq h' \).
    \end{proof}

    \begin{corollary}\label{cor:Minkowski}
        Soient \( H \) un réseau de \( \R^n \), et \( S \) une partie intégrable, symétrique par rapport à \( 0 \) et convexe de \( \R^n \). On suppose qu'une des relations suivantes est vraie :
        \begin{enumerate}
            \item \label{cor:Minkowski;item1} on a \( \mu(S) > 2^n v(H) \) ;
            \item \label{cor:Minkowski;item2} on a \( \mu(S) \geqslant 2^n v(H) \) et \( S \) est compacte.
        \end{enumerate}
        Alors \( S \cap H \) contient un point autre que \( 0 \).
    \end{corollary}
    \begin{proof}
        Dans le cas \ref{cor:Minkowski;item1}, on applique \cref{thm:Minkowski} à \( S' = \frac{1}{2} S \) (car \( \mu(S') = \frac{1}{2^n} \mu(S) > v(H) \)) ; il existe deux points distincts \( z, y \) de \( S' \) tels que \( y-z \in H \) ; alors \( x = y-z = \frac{1}{2}(2y+(-2z)) \) est un point de \( S \) (car \( S \) est symétrique et convexe), qui répond à la question. Dans le cas \ref{cor:Minkowski;item2}, on applique le cas \ref{cor:Minkowski;item1} à \( (1+\varepsilon)S \) (\( \varepsilon > 0 \)) ; en posant \( H' = H \setminus \set{0} \), on voit \( H' \cap (1+\varepsilon)S \) est non vide, et est fini car compact et discret. Alors \( \bigcap_{\varepsilon>0} H' \cap (1+\varepsilon)S \) est non vide ; un élément de cette intersection appartient à \( \bigcap_{\varepsilon>0} (1+\varepsilon)S \), ensemble qui est égal à \( S \) vu que \( S \) est compact.
    \end{proof}

    \begin{quote}
        L'hypothèse de compacité est nécessaire dans \ref{cor:Minkowski;item2}, comme le montrent le parallélotope ouvert \( \set*{ \sum_{i=1}^n \lambda_i e_i \given -1 < \lambda_i < 1 } \) et le réseau de base~\( (e_i) \).
    \end{quote}

\section{Le plongement canonique d'un corps de nombres}\label{sec:le plongement canonique d'un corps de nombres}

    Soient \( K \) un corps de nombres et \( n \) son degré. On a vu (\cref{thm:nombre de prolongements et degré}) qu'on a \( n \) isomorphismes distincts \( \sigma_i \colon K \to \C \). On en a exactement \( n \), car le polynôme minimal d'un élément primitif de \( K \) sur \( \Q \) (\cref{cor:théorème de l'élément primitif} \cref[de]{thm:nombre de prolongements et degré}) n'a que \( n \) racines dans \( \C \). Soit \( \alpha \colon \C \to \C \) le passage au nombre complexe conjugué ; alors, pour tout \( i \), \( \alpha \circ \sigma_i \) est l'un des \( \sigma_j \)\,, et est égal à \( \sigma_i \) si et seulement si \( \sigma_i(K) \subset \R \). Notons \( r_1 \) le nombre des indices \( i \) tels que \( \sigma_i(K) \subset \R \) ; alors les autre indices sont en nombres \emph{pair} \( 2r_2 \)\,, et on a
    \begin{equation}\label{eq:r1 et r2}
        r_1 + 2r_2 = n.
    \end{equation}
    Nous numéroterons les \( \sigma_i \) de sorte que \( \sigma_i(K) \subset \R \) pour \( 1 \leqslant i \leqslant r_1 \) et que \( \sigma_{j+r_2}(x) = \conj{\sigma_j(x)} \) pour \( r_1 + 1 \leqslant j \leqslant r_1+r_2 \) ; ainsi les \( r_1 + r_2 \) premiers \( \sigma_i \) déterminent les \( r_2 \) autres. Pour \( x \in K \), nous poserons
    \begin{equation}
        \sigma(x) = (\sigma_1(x), \cdots, \sigma_{r_1+r_2}(x)) \in \R^{r_1} \times \C^{r_2}.
    \end{equation}
    Nous appellerons \( \sigma \) le \emph{plongement canonique}\index{plongement canonique d'un\IndexLinebreak corps de nombres} de \( K \) dans \( \R^{r_1} \times \C^{r_2} \) ; c'est un homomorphisme injectif pour les structures d'anneaux. Nous identifierons souvent \( \R^{r_1} \times \C^{r_2} \) à \( \R^n \) (\cf.~\eqref{eq:r1 et r2}). Les notations \( \sigma, K, n, r_1, r_2 \) serons utilisées dans toute la suite de cette~section.

    \begin{proposition}\label{prop:volume par det}
        Si \( M \) est un sous-\( \Z \)‑module libre de rang \( n \) de \( K \), et si \( (x_i)_{1 \leqslant i \leqslant n} \) est une \( \Z \)‑base de \( M \), alors \( \sigma(M) \) est un réseau de \( \R^n \), dont le volume est donné par
        \begin{equation}\label{eq:volume par det}
            v(\sigma(M)) = 2^{-r_2} \abs*{\det_{1 \leqslant i,j \leqslant n} (\sigma_i(x_j))}.
        \end{equation}
    \end{proposition}
    \begin{proof}
        En effet, pour \( i \) fixé, les composantes de \( \sigma(x_i) \) par rapport à la base canonique de \( \R^n \) sont données par
        \begin{equation}\label{eq:composantes de sigma}
            \sigma_1(x_i), \cdots, \sigma_{r_1}(x_i), \Re(\sigma_{r_1+1}(x_i)), \Im(\sigma_{r_1+1}(x_i)), \cdots, \Re(\sigma_{r_1+r_2}(x_i)), \Im(\sigma_{r_1+r_2}(x_i))
        \end{equation}
        où \( \Re \) et \( \Im \) désignent la partie réelle et la partie imaginaire. Calculons le déterminant \( D \) dont la \( i \)‑ième colonne est \eqref{eq:composantes de sigma} ; en utilisant les formules \( \Re(z) = \frac{1}{2}(z+\conj{z}) \) et \( \Im(z) = \frac{1}{2\i}(z-\conj{z}) \) (\( z \in \C \)) et la linéarité par rapport aux lignes, on obtient \( D = \pm (2\i)^{-r_2} \det(\sigma_j(x_i)) \). Comme les \( x_i \) forment une base de \( K \) sur \( \Q \) on a \( \det(\sigma_j(x_i)) \neq 0 \) (\cref{prop:disc et det;K-isomorphismes}), et donc \( D \neq 0 \). Ainsi les vecteurs \( \sigma(x_i) \) sont linéairement indépendants dans \( \R^n \), de sorte que le \( \Z \)‑module qu'ils engendrent (à savoir \( \sigma(M) \)) est un réseau de \( \R^n \). Le calcul de \( D \) fait ci-dessus montre que son volume est bien donné par~\eqref{eq:volume par det}.
    \end{proof}

    \begin{proposition}\label{prop:volume;anneau et idéal}
        Soient \( d \) le discriminant absolu de \( K \), \( A \) son anneau des entiers, et \( \ideal{a} \) un idéal entier non nul de~\( A \). Alors \( \sigma(A) \) et \( \sigma(\ideal{a}) \) sont des réseaux, et on a :
        \begin{equation}\label{eq:volume;anneau et idéal}
            v(\sigma(A)) = 2^{-r_2} \abs{d}^{1/2}, \quad
            v(\sigma(\ideal{a})) = 2^{-r_2} \abs{d}^{1/2} \Norm(\ideal{a}).
        \end{equation}
    \end{proposition}
    \begin{proof}
        En effet on sait que \( A \) et \( \ideal{a} \) sont des \( \Z \)‑modules libres de rang \( n \), de sorte qu'on peut appliquer \cref{prop:volume par det}. D'autre part, si \( (x_i) \) est une \( \Z \)‑base de \( A \), on a \( d = \det(\sigma_i(x_j))^2 \) (\cref{prop:disc et det;K-isomorphismes}) ; d'où la première formule de \eqref{eq:volume;anneau et idéal}. La seconde s'en déduit en remarquant que \( \sigma(\ideal{a}) \) est un sous-groupe d'indice \( \Norm(\ideal{a}) \) de \( \sigma(A) \) (\cref{def:norme d'un idéal}), et qu'on obtient donc un domaine fondamental pour \( \sigma(\ideal{a}) \) par réunion disjointe de \( \Norm(\ideal{a}) \) domaines fondamentaux pour \( \sigma(A) \).
    \end{proof}

\section{Finitude du groupe des classes d'idéaux}\label{sec:finitude du groupe des classes d'idéaux}

    \begin{proposition}\label{prop:Minkowski}
        Soient \( K \) un corps de nombres, \( n \) son degré, \( r_1 \) et \( r_2 \) les entiers définis au début du~\cref{sec:le plongement canonique d'un corps de nombres}, \( d \) son discriminant absolu, et \( \ideal{a} \) un idéal entier non nul de \( K \). Alors \( \ideal{a} \) contient un élément non nul \( x \) tel que
        \vspace{-.2\baselineskip}%
        \begin{equation}\label{eq:prop;Minkowski;constant et inégalité}
            \abs{\Norm_{K/\Q}(x)} \leqslant \left(\frac{4}{\pi}\right)^{r_2} \frac{n!}{n^n} \; \abs{d}^{1/2} \Norm(\ideal{a}).
        \end{equation}
    \end{proposition}
    \begin{proof}
        En effet soit \( \sigma \) le plongement canonique de \( K \) dans \( \R^{r_1} \times \C^{r_2} \) (\cref{sec:le plongement canonique d'un corps de nombres}). Soient \( t \) un nombre réel \( > 0 \) et \( B_t \) l'ensemble des \( (y_1, \cdots, y_{r_1}, z_1, \cdots, z_{r_2}) \in \R^{r_1} \times \C^{r_2} \) tels~que
        \begin{equation}\label{eq:prop;Minkowski;aux1}
            \sum_{i=1}^{r_1} \abs{y_i} + 2 \sum_{j=1}^{r_2} \abs{z_j} \leqslant t.
        \end{equation}
        Alors \( B_t \) est un ensemble compact, convexe et symétrique par rapport à \( 0 \), dont on verra en appendice que le volume est
        \vspace{-.2\baselineskip}%
        \begin{equation}\label{eq:prop;Minkowski;aux2}
            \mu(B_t) = 2^{r_1} \left(\frac{\pi}{2}\right)^{r_2} \frac{t^n}{n!}.
        \end{equation}
        Choisissons \( t \) tel que \( \mu(B_t) = 2^n v(\sigma(\ideal{a})) \), c'est-à-dire tel que
        \[
            2^{r_1} \left(\frac{\pi}{2}\right)^{r_2} \frac{t^n}{n!} = 2^{n-r_2} \abs{d}^{1/2} \Norm(\ideal{a})
        \]
        (\cref{prop:volume;anneau et idéal}), ou encore \( t^n = 2^{n-r_1} \pi^{-r_2} n! \abs{d}^{1/2} \Norm(\ideal{a}) \). D'après \cref{cor:Minkowski} \cref[de]{thm:Minkowski}, il existe un élément non nul \( x \) de \( \ideal{a} \) tel que \( \sigma(x) \in B_t \). Évaluons sa norme \( \abs{\Norm(x)} = \prod_{i=1}^{r_1} \abs{\sigma_i(x)} \prod_{j=r_1+1}^{r_1+r_2} \abs{\sigma_j(x)}^2 \). L'inégalité de la moyenne géométrique montre qu'on a
        \vspace{-.2\baselineskip}%
        \[\qquad
            \abs{\Norm(x)} \leqslant \left[ \frac{1}{n} \sum_{i=1}^{r_1} \abs{\sigma_i(x)} + \frac{2}{n} \sum_{j=r_1+1}^{r_1+r_2} \abs{\sigma_j(x)} \right]^n \leqslant \frac{t^n}{n^n} \quad \text{(par \eqref{eq:prop;Minkowski;aux1})}.
        \]
        D'où \( \abs{\Norm(x)} \leqslant \frac{1}{n^n} 2^{n-r_1} \pi^{-r_2} n! \abs{d}^{1/2} \Norm(\ideal{a}) \), ce qui équivaut à \eqref{eq:prop;Minkowski;constant et inégalité} vu que \mbox{\( r_1 + 2r_2 = n \)}.
    \end{proof}

\vspace{-.5\baselineskip}
    \begin{corollary}\label{cor:Minkowski;existence d'un tel idéal}
        Avec les mêmes notations, toute classe d'idéaux de \( K \) (\cref{sec:anneaux de Dedekind}) contient un idéal entier \( \ideal{b} \) tel que
        \begin{equation}\label{eq:cor de Minkowski;existence d'un tel idéal}
            \abs{\Norm(\ideal{b})} \leqslant \left(\frac{4}{\pi}\right)^{r_2} \frac{n!}{n^n} \; \abs{d}^{1/2}.
        \end{equation}
    \end{corollary}
    \begin{proof}
        En effet soit \( \ideal{a}' \) un idéal de la classe donnée. Par homothétie on peut supposer que \( \ideal{a} = \ideal{a}'\,{}^{-1} \) est un idéal entier. Prenons un élément non nul \( x \) de \( \ideal{a} \) tel que \eqref{eq:prop;Minkowski;constant et inégalité} soit vraie. Alors \( \ideal{b} = x \ideal{a}^{-1} \) est un idéal entier de la classe donnée, dont la norme satisfait \eqref{eq:cor de Minkowski;existence d'un tel idéal} en vertu de la multiplicativité des normes (\cref{prop:multiplicativité des normes}).
    \end{proof}

    \begin{corollary}\label{cor:Minkowski;discriminant absolu}
        Soient \( K \) un corps de nombres, \( n \) son degré, et \( d \) son discriminant absolu. Alors pour \( n \geqslant 2 \) on~a
        \[
            \abs{d} \geqslant \frac{\pi}{3} \left(\frac{3\pi}{4}\right)^{n-1},
        \]
        et \( n/(\log \abs{d}) \) est majoré par une constante indépendante de \( K \).
    \end{corollary}
    \begin{proof}
        En effet, soit \( \ideal{b} \) un idéal entier non nul satisfaisant l'inégalité \eqref{eq:cor de Minkowski;existence d'un tel idéal} ; comme \( \Norm(\ideal{b}) \geqslant 1 \), on a \( \abs{d}^{1/2} \geqslant \left(\frac{\pi}{4}\right)^{r_2} \frac{n^n}{n!} \). Or \( \frac{\pi}{4} < 1 \) et \( 2r_2 \leqslant n \) ; on a donc \( \abs{d} \geqslant a_n \) où \( a_n = \left(\frac{\pi}{4}\right)^n \frac{n^{2n}}{(n!)^2} \). Or on a \( a_2 = \frac{\pi^2}{4} \) et \( \frac{a_{n+1}}{a_n} = \frac{\pi}{4} \left(1+\frac{1}{n}\right)^{2n} = \frac{\pi}{4} (1+2+\text{terms positifs}) \) (par la formule du binôme) \( \geqslant \frac{3\pi}{4} \). D'où, pour \( n \geqslant 2 \), \( \abs{d} \geqslant \frac{\pi^2}{4} \left(\frac{3\pi}{4}\right)^{n-2} \), ce qui donne l'inégalité annoncée. La majoration uniforme de \( n/(\log \abs{d}) \) s'ensuit en prenant les logarithmes.
    \end{proof}

    \begin{theorem}[Hermite-Minkowski]
        Pour tout corps de nombres \( K \neq \Q \)\,, le discriminant absolu \( d \) de \( K \) est \( \neq \pm 1 \).
    \end{theorem}
    \begin{proof}
        En effet, d'après \cref{cor:Minkowski;discriminant absolu}, on a \( \abs{d} \geqslant \frac{\pi}{3} \left(\frac{3\pi}{4}\right)^{n-1} \) et \( \frac{\pi}{3} > 1 \), \( \frac{3\pi}{4} > 1 \) ; d'où \( \abs{d} > 1 \).
    \end{proof}

    \begin{theorem}[Dirichlet]\label{thm:Dirichlet}
        Pour tout corps de nombres \( K \), le groupe des classes d'idéaux de \( K \) est fini (\cref{sec:anneaux de Dedekind}).
    \end{theorem}
    \begin{proof}
        En effet, en vertu \cref[de]{cor:Minkowski;existence d'un tel idéal} \cref[à]{prop:Minkowski}, il suffit de montrer que l'ensemble des idéaux entiers \( \ideal{b} \) de \( K \), dont la norme est un entier donné \( q \), est fini. Or, pour un tel idéal \( \ideal{b} \), on a \( \card(A/\ideal{b}) = q \) (\cref{sec:norme d'un idéal}), d'où \( q \in \ideal{b} \), car, dans un groupe, l'ordre d'un élément divise l'ordre du groupe. Ainsi nos idéaux \( \ideal{b} \) sont parmi ceux qui contiennent \( Aq \), et ces derniers sont en nombre fini (\cref{sec:anneaux de Dedekind}, \ref{enum:formulaire d'exposant dans la décomposition en produit d'idéaux premiers;item3} ; ou finitude de \( A/Aq \)).
    \end{proof}

    \begin{theorem}[Hermite]\label{thm:Hermite}
        Dans \( \C \) il n'y a qu'un nombre fini de corps de nombres de discriminant \( d \) donné.
    \end{theorem}
    \begin{proof}
        En effet, d'après \cref{cor:Minkowski;discriminant absolu} \cref[à]{prop:Minkowski}, le degré d'un tel corps est alors majoré. Nous pouvons donc supposer \( n \) donné, ainsi que les entiers \( r_1 \) et \( r_2 \). Soit \( K \) un tel corps.

        Dans \( \R^{r_1} \times \C^{r_2} \), considérons l'ensemble \( B \) suivant :
        \begin{enumerate-alph}
            \item \label{thm de Hermite;cas1} Si \( r_1 > 0 \), \( B \) est l'ensemble des \( (y_1, \cdots, y_{r_1}, z_1, \cdots, z_{r_2}) \in \R^{r_1} \times \C^{r_2} \) tels que
            \begin{equation}\label{eq:thm de Hermite;cas1}
                \begin{aligned}
                    \abs{y_1} \leqslant 2^n \left(\frac{\pi}{2}\right)^{-r_2} \abs{d}^{1/2}, ~ \abs{y_i} &\leqslant \frac{1}{2} \text{ pour } i = 2, \cdots, r_1 ;\\
                    \abs{z_j} &\leqslant \frac{1}{2} \text{ pour } j = 1, \cdots, r_2.
                \end{aligned}
            \end{equation}
            \item \label{thm de Hermite;cas2} Si \( r_1 = 0 \), \( B \) est l'ensemble des \( (z_1, \cdots, z_{r_2}) \in \C^{r_2} \) tels que
            \begin{equation}\label{eq:thm de Hermite;cas2}
                \begin{split}
                    \abs{z_1-\conj{z_1}} \leqslant 2^n \frac{8}{\pi} \left(\frac{\pi}{2}\right)^{-r_2} \abs{d}^{1/2}&, ~ \abs{z_1+\conj{z_1}} \leqslant \frac{1}{2} ;\\
                    \abs{z_j} &\leqslant \frac{1}{2} \text{ pour } j = 2, \cdots, r_2.
                \end{split}
            \end{equation}
        \end{enumerate-alph}
        Alors \( B \) est un ensemble compact, convexe, symétrique par rapport à \( 0 \), dont le volume est tout juste \( 2^n 2^{-r_2} \abs{d}^{1/2} \)~\footnote{Le calcul, très simple, de ce volume se fait en remarquant que \( B \) est un produit d'intervalles, de disques, et d'un rectangle dans le cas \ref{thm de Hermite;cas2}.}. En notant \( \sigma \) le plongement canonique de \( K \) (\cref{sec:le plongement canonique d'un corps de nombres}), \cref{prop:volume;anneau et idéal} et \cref{cor:Minkowski} \cref[de]{thm:Minkowski} montrent qu'il existe un entier \( x \neq 0 \) de \( K \) tel que \( \sigma(x) \in B \).

        Montrons que \( x \) est un élément \emph{primitif} de \( K \) sur \( \Q \). En effet, dans le cas \ref{thm de Hermite;cas1}, \eqref{eq:thm de Hermite;cas1} montre qu'on a \( \abs{\sigma_i(x)} \leqslant \frac{1}{2} \) pour \( i \neq 1 \) ; comme \( \abs{\Norm(x)} = \prod_{i=1}^n \abs{\sigma_i(x)} \) est un entier \( \neq 0 \) (\cref{cor:trace et norme d'élément entier} \cref[de]{prop:trace et norme d'élément entier}), on en déduit \( \abs{\sigma_1(x)} \geqslant 1 \), d'où \( \sigma_1(x) \neq \sigma_i(x) \) pour tout \( i \neq 1 \) ; or, si \( x \) n'était pas primitif, \( \sigma_1(x) \) coïnciderait avec l'un des \( \sigma_i(x) \) pour \( i \neq 1 \) (\cref{prop:trace;norme;polynôme caractéristique;extension}). Dans le cas \ref{thm de Hermite;cas2}, on voit de même qu'on a \( \abs{\sigma_1(x)} = \abs{\conj{\sigma_1(x)}} \geqslant 1 \), d'où \( \sigma_1(x) \neq \sigma_j(x) \) lorsque \( \sigma_j \) est distinct de \( \sigma_1 \) et \( \conj{\sigma_1} \) ; de plus, \eqref{eq:thm de Hermite;cas2} montre que la partie réelle \( \abs{\Re(\sigma_1(x))} \) est \( \leqslant \frac{1}{4} \), de sorte que \( \sigma_1(x) \) n'est pas réel et que \( \sigma_1(x) \neq \conj{\sigma_1(x)} \) ; comme dans le cas \ref{thm de Hermite;cas1} on en conclut que \( x \) est primitif.

        Maintenant les formules \eqref{eq:thm de Hermite;cas1} et \eqref{eq:thm de Hermite;cas2} montrent que les conjugués \( \sigma_i(x) \) de \( x \) sont \emph{bornés}, donc aussi les fonctions symétriques élémentaires des \( \sigma_i(x) \), c'est-à-dire les coefficients du polynôme minimal de \( x \). Comme ce sont des éléments de \( \Z \) (\cref{cor:trace et norme d'élément entier} \cref[de]{prop:trace et norme d'élément entier}), ils ne sont donc susceptibles que d'un nombre fini de valeurs. D'où un nombre fini de polynômes minimaux possibles pour \( x \), et par conséquent seulement un nombre fini de valeurs possibles de \( x \) dans \( \C \). Comme \( x \) engendre \( K \), \namecref{thm:Hermite} est démontré.
    \end{proof}

\section{Le théorème des unités}

    Par abus de langage, on appelle \emph{unités}\index{unites (d'un corps de nombres)@unités (d'un corps de nombres)} d'un corps de nombres \( K \) les éléments inversibles de l'anneau \( A \) des entiers de \( K \). Ces unités forment un groupe multiplicatif, noté \( A^\times \). Le résultat suivant nous sera utile.

    \begin{proposition}\label{prop:unité est de norme +-1}
        Soient \( K \) un corps de nombres, et \( x \in K \). Pour que \( x \) soit une unité de \( K \), il faut et il suffit que \( x \) soit un entier de \( K \), de norme~\( \pm 1 \).
    \end{proposition}
    \begin{proof}
        En effet, si \( x \) est une unité de \( K \), \( \Norm(x) \) et \( \Norm(x^{-1}) \) sont des éléments de \( \Z \)\,, dont le produit est \( \Norm(1) = 1 \) ; on a donc \( \Norm(x) = \pm 1 \). Réciproquement, soit \( x \) un entier de \( K \) de norme \( \pm 1 \) ; son équation caractéristique s'écrit \( x^n + a_{n-1} x^{n-1} + \cdots + a_1 x \pm 1 = 0 \) avec \( a_i \in \Z \) (\cref{sec:normes et traces}) ; alors \( \pm (x^{n-1} + \cdots + a_1) \) est l'inverse de \( x \), et est un entier de \( K \) ; donc \( x \) est une unité de~\( K \).
    \end{proof}

    \begin{theorem}[Dirichlet]\label{thm:unité}
        Soient \( K \) un corps de nombres, \( n \) son degré, \( r_1 \) et \( r_2 \) les entiers définis au \cref{sec:le plongement canonique d'un corps de nombres}, et \( r = r_1 + r_2 - 1 \). Le groupe \( A^\times \) des unités de \( K \) est isomorphe à \( \Z^r \times G \), où \( G \) est un groupe cyclique fini, formé par les racines de l'unité contenues dans~\( K \).
    \end{theorem}
    \begin{proof}
        Nous montrerons d'abord que \( A^\times \) est un groupe commutatif de type fini, et nous calculerons ensuite son rang. Considérons le plongement canonique (\cref{sec:le plongement canonique d'un corps de nombres}) \( x \mapsto (\sigma_1(x), \cdots, \sigma_{r_1+r_2}(x)) \) de \( K \) dans \( \R^{r_1} \times \C^{r_2} \), et l'application
        \begin{equation}
            x \mapsto L(x) \coloneqq (\log\abs{\sigma_1(x)}, \cdots, \log\abs{\sigma_{r_1+r_2}(x)})
        \end{equation}
        de \( K^\times \) dans \( \R^{r_1+r_2} \) ; c'est un homomorphisme (\ie. \( L(xy) = L(x)+L(y) \)), que nous appellerons le \emph{plongement logarithmique} de \( K^\times \). Soit \( B \) une partie compacte de \( \R^{r_1+r_2} \) ; montrons que l'ensemble \( B' \) des unités \( x \in A^\times \) telles que \( L(x) \in B \) est \emph{fini}. En effet, comme \( B \) est borné, il existe un nombre réel \( \alpha > 1 \) tel que, pour tout \( x \in B' \), on ait
        \[
            \frac{1}{\alpha} \leqslant \abs{\sigma_i(x)} \leqslant \alpha \quad \text{(\( i = 1, \cdots, n \))} ~;
        \]
        les fonctions symétriques élémentaire des \( \sigma_i(x) \) sont alors bornées en module ; comme ce sont des éléments de \( \Z \) (car \( x \in A \)), elles ne peuvent prendre qu'un nombre fini de valeurs ; il n'y a donc qu'un nombre fini de polynômes caractéristiques possibles pour \( x \), d'où un nombre fini de valeurs pour \( x \). La finitude de \( B' \) entraîne aussitôt les conséquences suivantes :
        \begin{enumerate-alph}
            \item Le noyau \( G \) de la restriction de \( L \) à \( A^\times \) est un groupe fini. Il est donc formé de racines de l'unité, et est \emph{cyclique} (\cref{thm:sous-groupe fini du groupe multiplicatif d'un corps}). Toute \emph{racine de l'unité} contenue dans \( K \) appartient d'ailleurs à ce noyau, car c'est un entier de \( K \), et que \( \abs{\sigma_i(x)}^q = \abs{\sigma_i(x^q)} = \abs{1} = 1 \) implique \( \abs{\sigma_i(x)} = 1 \).
            \item L'image \( L(A^\times) \) est un sous-groupe discret de \( \R^{r_1+r_2} \) (\cref{sec:préliminaires sur les groupes discrets de R n}), et est donc un \( \Z \)‑module libre de rang \( s \leqslant r_1+r_2 \) (\cref{thm:structure d'un réseau sur R}). Comme \( L(A^\times) \) est libre, \( A^\times \) est isomorphe à \( G \times L(A^\times) = G \times \Z^{s} \). Il nous reste à montrer que le rang \( s \) de \( L(A^\times) \) est égal à \( r_1+r_2-1 \).
        \end{enumerate-alph}

        L'inégalité \( s \leqslant r_1+r_2-1 \) est facile. En effet, pour \( x \in A^\times \), la relation \( \pm 1 = \Norm(x) = \prod_{i=1}^n \sigma_i(x) = \prod_{i=1}^{r_1} \sigma_i(x) \prod_{j=r_1+1}^{r_1+r_2} \sigma_j(x) \conj{\sigma_j(x)} \) (\cref{prop:unité est de norme +-1}) implique que le vecteur \( L(x) = (y_1, \cdots, y_{r_1+r_2}) \) appartient à l'hyperplan \( W \) d'équation
        \begin{equation}\label{eq:thm de Dirichlet;aux;hyperplan}
            \sum_{i=1}^{r_1} y_i + 2\sum_{j=r_1+1}^{r_1+r_2} y_j = 0 ~;
        \end{equation}
        ainsi \( L(A^\times) \) est un sous-groupe discret de \( W \), d'où \( s \leqslant r_1+r_2-1 \).

        Reste à montrer que \( L(A^\times) \) contient \( r = r_1+r_2-1 \) vecteurs linéairement indépendants, ce qui va être plus délicat. Il s'agit de montrer que, pour toute forme linéaire \( f \neq 0 \) sur \( W \), il existe une unité \( u \) telle que \( f(L(u)) \neq 0 \). Comme la projection de \( W \) sur \( \R^r \) est un isomorphisme (par \eqref{eq:thm de Dirichlet;aux;hyperplan}), on peut écrire, pour tout \( y = (y_1, \cdots, y_{r+1}) \in W \subset \R^{r+1} \),
        \begin{equation}\label{eq:thm de Dirichlet;aux;projection}
            f(y) = c_1 y_1 + \cdots + c_r y_r ~\text{ avec } c_i \in \R.
        \end{equation}
        Fixons un nombre réel \( \alpha \) suffisamment grand, plus précisément
        \[
            \alpha \geqslant 2^n \left(\frac{1}{4\pi}\right)^{r_2} \abs{d}^{1/2}.
        \]
        Pour tout système \( \lambda = (\lambda_1, \cdots, \lambda_r) \) de \( r \) nombre réels \( > 0 \), soit \( \lambda_{r+1} \) le nombre réel \( > 0 \) tel que \( \prod_{i=1}^{r_1} \lambda_i \prod_{j=r_1+1}^{r_1+r_2} \lambda_j^2 = \alpha \). Dans \( \R^{r_1} \times \C^{r_2} \) l'ensemble \( B \) des \( (y_1, \cdots, y_{r_1}, z_1, \cdots, z_{r_2}) \) (\( y_i \in \R \)\,, \( z_j \in \C \)) tels que \( \abs{y_i} \leqslant \lambda_i \) et \( \abs{z_j} \leqslant \lambda_j \) est compact, convexe, symétrique par rapport à \( 0 \), et son volume est \( \prod_{i=1}^{r_1} 2\lambda_i \prod_{j=r_1+1}^{r_1+r_2} \pi\lambda_j^2 = 2^{r_1}\pi^{r_2} \alpha \geqslant 2^n 2^{-r_2} \abs{d}^{1/2} \). Donc, d'après \cref{prop:volume;anneau et idéal} et \cref{cor:Minkowski} \cref[de]{thm:Minkowski}, il existe un entier \( x_{\lambda} \neq 0 \) de \( K \) tel que \( \sigma(x_{\lambda}) \in B \) ; autrement dit on a \( \abs{\sigma_i(x_{\lambda})} \leqslant \lambda_i \) pour \( i = 1, \cdots, n \) (en posant \( \lambda_{j+r_2} = \lambda_j \) pour \( j = r_1+1, \cdots, r_1+r_2 \)). Comme \( x_{\lambda} \) est un entier, on a
        \[
            1 \leqslant \abs{\Norm(x_{\lambda})} = \prod_{i=1}^n \abs{\sigma_i(x_{\lambda})} \leqslant \prod_{i=1}^{r_1} \lambda_i \prod_{j=r_1+1}^{r_1+r_2} \lambda_j^2 = \alpha.
        \]
        D'autre part, pour tout \( i \), on a
        \[
            \abs{\sigma_i(x_{\lambda})} = \abs{\Norm(x_{\lambda})} \prod_{j \neq i} \abs{\sigma_j(x_{\lambda})} \geqslant \prod_{j \neq i} \lambda_j^{-1} = \lambda_i \alpha^{-1}.
        \]
        \vspace{-.9\baselineskip}\par\noindent%
        D'où \( \lambda_i \alpha^{-1} \leqslant \abs{\sigma_i(x_{\lambda})} \leqslant \lambda_i \) pour tout \( i \), de sorte qu'on a
        \begin{equation}\label{eq:thm de Dirichlet;aux;estimation1}
            0 \leqslant \log \lambda_i - \log \abs{\sigma_i(x_{\lambda})} \leqslant \log \alpha.
        \end{equation}
        \vspace{-1.2\baselineskip}\par\noindent%
        D'après \eqref{eq:thm de Dirichlet;aux;projection}, on a donc
        \vspace{-.2\baselineskip}%
        \begin{equation}\label{eq:thm de Dirichlet;aux;estimation2}
            \abs*{ f(L(x_{\lambda})) - \sum_{i=1}^r c_i \log \lambda_i } \leqslant \left( \sum_{i=1}^r \abs{c_i} \right) \log \alpha.
        \end{equation}
        Notons \( \beta \) une constante majorant strictement le second membre de \eqref{eq:thm de Dirichlet;aux;estimation2}, et, pour tout entier \( h > 0 \), choisissons \( r \) nombres réels \( \lambda_{i,h} > 0 \) (\( i = 1, \cdots, r \)) tels que \( \sum_{i=1}^r c_i \log \lambda_{i,h} = 2 \beta h \) ; posons \( \lambda(h) = (\lambda_{1,h}, \cdots, \lambda_{r,h}) \), et soit \( x_h \) l'entier \( x_{\lambda(h)} \) correspondant. D'après \eqref{eq:thm de Dirichlet;aux;estimation2} on a \( \abs{f(L(x_h)) - 2 \beta h} < \beta \), d'où
        \begin{equation}\label{eq:thm de Dirichlet;aux;estimation3}
            (2h-1)\beta < f(L(x_h)) < (2h+1)\beta.
        \end{equation}
        Il résulte de \eqref{eq:thm de Dirichlet;aux;estimation3} que les nombres \( f(L(x_{\lambda})) \) (\( h \geqslant 0 \)) sont tous \emph{distincts}. D'autre part, comme \( \abs{\Norm(x_h)} \leqslant \alpha \), les idéaux \( Ax_h \) sont en nombre \emph{fini} (\cf.~\cref{sec:finitude du groupe des classes d'idéaux}, démonstration \cref[de]{thm:Dirichlet}). Il existe donc deux indices distincts \( h \) et \( k \) tels que \( A x_h = A x_k \)\,, d'où une unité \( u \) de \( A \) telle que \( x_k = u x_h \). On a alors (comme \( f \) est linéaire) \( f(L(u)) = f(L(x_k)) - f(L(x_h)) \neq 0 \), et \( u \) est l'unité cherchée.
    \end{proof}

    \begin{remark}
        \Cref{thm:unité} (appelé \textquote{théorème des unités}) montre qu'il existe \( r \) (\( = r_1+r_2-1 \)) unités \( (u_i) \) de \( K \) telles que toute unité \( u \) de \( K \) s'écrive, d'une façon et d'une seule, sous la forme
        \vspace{-.2\baselineskip}%
        \begin{equation}
            u = z u_1^{n_1} \cdots u_r^{n_r}
        \end{equation}
        avec \( n_i \in \Z \) et \( z \) racine de l'unité. Alors \( (u_i) \) s'appelle un \emph{système d'unités fondamentales}\index{unites fondamentales@unités fondamentales} de~\( K \).
    \end{remark}

    \subsection*{Exemple des corps cyclotomiques}
        Soient \( p \) un nombre premier \( \neq 2 \), \( z \) une racine primitive \( p \)‑ième de l'unité dans \( \C \)\,, et \( K \) le corps cyclotomique \( \Q[z] \) (\cf.~\cref{sec:corps cyclotomiques}) ; on a \( [K:\Q] = p-1 \) (\cref{thm:polynôme cyclotomique est irréductible}). Comme aucun conjugué de \( z \) dans \( \C \) n'est réel, on a \( r_1 = 0 \), \( 2 r_2 = p-1 \), d'où \( r = (p-3)/2 \).

\section{Unités des corps quadratiques imaginaires}

    Soit \( K \) un corps quadratique imaginaire (\cref{sec:entiers des corps quadratiques}). On a alors \( r_1 = 0 \), \( 2r_2 = 2 \), \( r_2 = 1 \) et \( r_1+r_2-1 = 0 \). Ainsi les seules unités de \( K \) sont les racines de l'unité contenues dans~\( K \) (\cref{thm:unité}) ; celles forment un groupe \emph{fini cyclique}. Un petit calcul va nous redonner ce résultat, et le préciser.

    Soit \( K = \Q[\sqrt{-m}] \), où \( m \) est un entier \( > 0 \) sans facteurs carrés. Rappelons que les unités \( K \) sont les entiers de norme \( \pm 1 \) de \( K \) (\cref{prop:unité est de norme +-1}).

    \begin{enumerate}
        \item Si \( m \equiv 1 \) ou \( 2 \pmod*{4} \), l'anneau des entiers de \( K \) est \( \Z + \Z\sqrt{-m} \) (\cref{thm:entiers dans un corps quadratique}). Pour \( x = a + b\sqrt{-m} \) (\( a,b \in \Z \)), on a
        \[
            \Norm(x) = a^2 + m b^2 \geqslant 0.
        \]
        Donc, pour que \( x \) soit une unité, il faut et il suffit que \( a^2+mb^2 = 1 \). Si \( m \geqslant 2 \), ceci implique \( b = 0 \) et \( a = \pm 1 \), d'où \( x = \pm 1 \). Si \( m=1 \), outre les solutions \( x = \pm 1 \), il y a les solutions \( a = 0 \), \( b = \pm 1 \), \( x = \pm \i \) (\( \i^2 = -1 \)).
        \item Si \( m \equiv 3 \pmod*{4} \), l'anneau des entiers de \( K \) est \( \Z + \Z\frac{1+\sqrt{-m}}{2} \) (\cref{thm:entiers dans un corps quadratique}). Pour \( x = a + \frac{b}{2}(1+\sqrt{-m}) \) (\( a,b \in \Z \)), on a
        \vspace{-.2\baselineskip}%
        \[
            \Norm(x) = \left(a+\frac{b}{2}\right)^2 + \frac{m b^2}{4}.
        \]
        Donc, pour que \( x \) soit une unité, il faut et il suffit que \( (2a+b)^2+mb^2 = 4 \). Si \( m \geqslant 7 \), ceci implique \( b = 0 \), d'où \( (2a)^2 = 4 \), \( a = \pm 1 \), et \( x = \pm 1 \). Si \( m=3 \), on obtient en outre les solutions \( b = \pm 1 \), d'où \( (2a \pm 1)^2 = 1 \), c'est-à-dire \( x = \frac{1}{2}(\pm 1 \pm \sqrt{-3}) \) (les signes \( \pm \) étant indépendants).
    \end{enumerate}

    En résumé nous avons obtenu le résultat suivant :
    \begin{proposition}
        Si \( K \) est un corps quadratique imaginaire, le groupe \( G \) des unités de \( K \) est formé de \( +1 \) et de \( -1 \), sauf dans les deux cas suivants :
        \begin{enumerate}
            \item si \( K = \Q[\i] \) (\( \i^2 = -1 \)), \( G \) est formé des racines quatrièmes de l'unité \( \i, -1, -\i, 1 \).
            \item si \( K = \Q[\sqrt{-3}] \), \( G \) est formé des racines sixièmes de l'unité \( \left(\frac{1+\sqrt{-3}}{2}\right)^j \), \( j = 0, \cdots, 5 \).
        \end{enumerate}
    \end{proposition}

\section{Unités des corps quadratiques réels}

    Cette section va être nettement plus amusante que la précédente. Soit \( K \) un corps quadratique réel. Avec les notations habituelles, on a \( r_1 = 2 \), \( r_2 = 0 \), d'où \( r = r_1+r_2-\nobreak 1 =\nobreak 1 \). Le théorème des unités (\cref{thm:unité}) montre que le groupe des unités de \( K \) est isomorphe au produit de \( \Z \) par le groupe des racines de l'unité contenues dans~\( K \). Comme \( K \) admet un plongement dans \( \R \)\,, celles-ci sont \( 1 \) et \( -1 \). Donc, en supposant \( K \) plongé dans \( \R \)\,, on a :
    \begin{proposition}
        Les unités positives d'un corps quadratique réel \( K \subset \R \) forment un groupe (multiplicatif) isomorphe à \( \Z \).
    \end{proposition}
\noindent
    Ce groupe admet donc un seul générateur \( > 1 \) ; on l'appelle l'\emph{unité fondamentale}\index{unites fondamentales@unités fondamentales} de~\( K \).

    Soit \( K = \Q[\sqrt{d}] \) où \( d \) est un entier \( \geqslant 2 \) sans facteurs carrés, et soit \( x = a + b \sqrt{d} \) (\( a,b \in \Q \)) une unité de \( K \) ; les nombres \( x, x^{-1}, -x, -x^{-1} \) sont des unités de \( K \), et, comme \( \Norm(x) = (a+b\sqrt{d})(a-b\sqrt{d}) = \pm 1 \) (\cref{prop:unité est de norme +-1}), ces quatre nombres sont \( \pm a \pm b\sqrt{d} \). Pour \( x \neq \pm 1 \), un seul des quatre nombres \( x, x^{-1}, -x, -x^{-1} \) est \( > 1 \), et c'est le plus grand des quatre. Donc \emph{les unités \( > 1 \) de \( K \) sont les unités de la forme \( a+b\sqrt{d} \) avec \( a, b > 0 \)}.
    \begin{enumerate}
        \item \label{unités des corps quadratiques réels;cas1} Supposons d'abord \( d \equiv 2 \) ou \( 3 \pmod*{4} \). Alors l'anneau des entiers de \( K \) est \( \Z + \Z\sqrt{d} \) (\cref{thm:entiers dans un corps quadratique}). Comme les unités de \( K \) sont les entiers de norme \( \pm 1 \) (\cref{prop:unité est de norme +-1}), les unités \( > 1 \) de \( K \) sont les nombres \( a+b\sqrt{d} \) avec \( a,b \in \Z \)\,, \( a, b > 0 \) tels que
        \begin{equation}\label{eq:Pell-Fermat}
            a^2 - db^2 = \pm 1.
        \end{equation}
        On voit donc que les solutions \textquote{en nombres entiers naturels} \( (a,b) \) de l'équation \eqref{eq:Pell-Fermat} (dite \textquote{\emph{équation de Pell-Fermat}})\index{equation de Pell-Fermat@équation de Pell-Fermat} s'obtiennent comme suit : on prend l'unité fondamentale \( a_1 + b_1 \sqrt{d} \) de \( K \), et on pose
        \begin{equation}\label{eq:sol de Pell-Fermat;unité fondamentale}
            a_n + b_n \sqrt{d} = (a_1 + b_1 \sqrt{d})^n \quad \text{(\( n \geqslant 1 \)) ;}
        \end{equation}
        la suite \( (a_n, b_n) \) fournit alors \emph{toutes les solutions} de \eqref{eq:Pell-Fermat}.
        \begin{remark}
            \begin{enumerate}
                \item Il résulte de \eqref{eq:sol de Pell-Fermat;unité fondamentale} que \( b_{n+1} = a_1 b_n + b_1 a_n \) ; comme \( a_1, b_1, a_n, b_n > 0 \), la suite \( (b_n) \) est strictement croissante. Ainsi, afin de \emph{calculer} explicitement l'unité fondamentale \( a_1 + b_1 \sqrt{d} \), on peut écrire la suite des \( d b^2 \) (\( b \in \N \)\,, \( b \geqslant 1 \)) et s'arrêter au premier nombre \( d b_1^2 \) de cette suite qui diffère d'un carré \( a_1^2 \) par \( \pm 1 \) ; alors \( a_1 + b_1 \sqrt{d} \) est l'unité fondamentale de \( K \). Par exemple, pour \( d = 7 \), la suite \( d b^2 \) est \( 7 \), \( 28 \), \( 63 = 64 - 1 = 8^2 - 1 \) ; on a donc \( b_1 = 3 \), \( a_1 = 8 \) et l'unité fondamentale de \( \Q[\sqrt{7}] \) est \( 8 + 3\sqrt{7} \). On voit de même que les unités fondamentales de \( \Q[\sqrt{2}] \), \( \Q[\sqrt{3}] \) et \( \Q[\sqrt{6}] \) sont \( 1+\sqrt{2} \), \( 2+\sqrt{3} \), \( 5+2\sqrt{6} \). Il y a d'autres procédés de calcul, plus rapides, de l'unité fondamentale, liés à la théorie des fractions continues.
                \item Si l'unité fondamentale est de norme \( 1 \), les \( (a_n,b_n) \) sont tous solutions de \eqref[']{eq:Pell-Fermat}~\( a^2-db^2=1 \) ; alors \eqref['']{eq:Pell-Fermat}~\( a^2-db^2=-1 \) n'a pas de solutions. Si l'unité fondamentale est de norme \( -1 \), les solutions de \eqref[']{eq:Pell-Fermat} sont les \( (a_{2n}, b_{2n}) \), et celles de \eqref['']{eq:Pell-Fermat} sont les \( (a_{2n+1}, b_{2n+1}) \). Le premier cas se produit pour \( d = 3 \), \( d = 6 \) et \( d = 7 \), le second pour \( d = 2 \) et \( d = 10 \).
            \end{enumerate}
        \end{remark}
        \item \label{unités des corps quadratiques réels;cas2} Supposons maintenant \( d \equiv 1 \pmod*{4} \). Les entiers de \( K = \Q[\sqrt{d}] \) sont alors les nombres \( \frac{1}{2}(a+b\sqrt{d}) \) avec \( a,b \in \Z \) de même parité (\cref{thm:entiers dans un corps quadratique}). Donc, si \( \frac{1}{2}(a+b\sqrt{d}) \) est une unité de \( K \), on a (\cref{prop:unité est de norme +-1}) :
        \begin{equation}\label{eq:Pell-Fermat;m=4}
            a^2 - db^2 = \pm 4.
        \end{equation}
        Réciproquement, pour toute solution \( (a,b) \) en nombres entiers de \eqref{eq:Pell-Fermat;m=4}, \( \frac{1}{2}(a+b\sqrt{d}) \) est un entier de \( K \) (car sa trace est \( a \), et sa norme est \( \pm 1 \) par \eqref{eq:Pell-Fermat;m=4}), et donc une unité de \( K \). Comme dans \ref{unités des corps quadratiques réels;cas1} on voir donc que, si \( \frac{1}{2}(a_1+b_1\sqrt{d}) \) désigne l'unité fondamentale de \( K \), les solutions \( (a,b) \) de \eqref{eq:Pell-Fermat;m=4} en nombres entiers \( > 0 \) forment la suite \( (a_n, b_n) \) (\( n \geqslant 1 \)) définie par
        \begin{equation}
            a_n+b_n\sqrt{d} = 2^{1-n} (a_1+b_1\sqrt{d})^n.
        \end{equation}
        Le calcul de \( a_1+b_1\sqrt{d} \) peut s'effectuer comme dans \ref{unités des corps quadratiques réels;cas1} ; les unités fondamentales de \( \Q[\sqrt{5}] \), \( \Q[\sqrt{13}] \) et \( \Q[\sqrt{17}] \) sont \( \frac{1}{2}(1+\sqrt{5}) \), \( \frac{1}{2}(3+\sqrt{13}) \), \( 4+\sqrt{17} \) ; ces trois unités sont de norme \( -1 \). Pour le choix du signe \( \pm \) dans \eqref{eq:Pell-Fermat;m=4}, on a les mêmes résultats que dans le cas~\ref{unités des corps quadratiques réels;cas1}.
        \begin{remark}
            Dans le cas \( d \equiv 1 \pmod*{4} \), les solutions de l'équation de Pell-Fermat proprement dite
            \[
                a^2 - db^2 = \pm 1
            \]
            correspond aux unités \( a+b\sqrt{d} \) (\( a,b > 0 \)) de l'anneau \( B = \Z[\sqrt{d}] \), qui est un sous-anneau de l'anneau \( A \) des entiers de \( K \). Or les unités \( > 0 \) de \( B \) forment un sous-groupe \( G \) du groupe des unités positives de \( A \). Soit \( u = \frac{1}{2}(a+b\sqrt{d}) \) l'unité fondamentale de \( K \). Si \( a \) et \( b \) sont tous deux \emph{pairs}, on a \( u \in B \), de sorte que \emph{\( G \) est formé des puissances de \( u \)} (c'est le cas si \( d = 17 \)). Si \( a \) et \( b \) sont tous deux \emph{impairs}, on a \( u^3 \in B \) : en effet on a \( 8u^3 = a(a^2+3b^2 d)+b(3a^2+b^2 d)\sqrt{d} \) ; comme \( a^2-db^2 = \pm 4 \), on a \( a^2+3b^2 d = 4(b^2 d \pm 1) \), qui est multiple de \( 8 \) car \( b \) et \( d \) sont impairs ; de même \( 3a^2+b^2 d = 4(a^2 \pm 1) \), encore multiple de \( 8 \) car \( a \) est impair. Dans ce cas \emph{\( G \) est formé des puissances de \( u^3 \)} (en effet on a \( u^2 \notin B \), sinon \( u = u^3/u^2 \in B \)) ; c'est le cas pour \( d = 5 \) (resp. \( d = 13 \)), et alors \( u^3 = 2+\sqrt{5} \) (resp. \( u^3 = 18+5\sqrt{13} \)).
        \end{remark}
    \end{enumerate}

\section{Une généralisation du théorème des unités}\label{sec:une généralisation du théorème des unités}

    \begin{proposition}
        Soit \( A \) un anneau qui est un \( \Z \)‑module de type fini. Alors le groupe multiplicatif \( A^\times \) des éléments inversibles de \( A \) est un groupe commutatif de type fini.
    \end{proposition}
    \begin{quote}
        Pour un groupe commutatif \( G \), \textquote{de type fini} veut dire \textquote{de type fini pour la structure de \emph{\( \Z \)‑module} de \( G \)}. Un sous-groupe d'un groupe commutatif de type fini est de type fini (\cref{cor:module de type fini sur anneau nœthérien est nœthérien} \cref[de]{prop:nœthérien par sous-module et module quotient}). Notons que \( A \) est un anneau \emph{nœthérien}, car les idéaux de \( A \) sont des sous-\( \Z \)‑modules de~\( A \).
    \end{quote}
    \begin{proof}
        Nous traiterons d'abord le cas où \( A \) est \emph{intègre}. Si son corps des fractions \( K \) est de caractéristique \( 0 \), c'est un \( \Q \)‑espace vectoriel de type fini, donc un corps de nombres. D'autre part \( A \) est entier sur \( \Z \) (car c'est un \( \Z \)‑module de type fini, \cf.~\cref{thm:conditions d'être entier}), donc est un sous-anneau de l'anneau \( B \) des entiers de \( K \) ; alors \( A^\times \subset B^\times \) et \( B^\times \) est de type fini par le théorème des unités (\cref{thm:unité}). Si \( K \) est de caractéristique \( p \neq 0 \), \( K \) est une extension finie de \( \F_p \)\,, donc un corps fini ; alors \( A^\times \) est fini.

        Passons maintenant au cas où \( A \) est \emph{réduit}\index{anneau!reduit@réduit}\index{reduit (anneau)@réduit (anneau)} (ce qui veut dire, par définition, que \( 0 \) est le seul élément nilpotent de \( A \)). Nous aurons besoin du lemme suivant.

        \begin{lemma*}
            Dans un anneau nœthérien réduit \( A \), l'idéal \( (0) \) est intersection finie d'idéaux \mbox{premiers}.
        \end{lemma*}
        \begin{proof}[Demonstration du lemme]
            En effet on sait que, dans un anneau nœthérien, tout idéal contient un produit d'idéaux premiers (\cref{lem:anneau nœthérien;tout idéal contient un produit d'idéaux premiers}). Or \( (0) \) est le plus petit des idéaux, et donc \emph{est} produit d'idéaux premiers : \( (0) = \ideal{p_1}^{n_1} \cdots \ideal{p_q}^{n_q} \). Soit alors \( x \in \ideal{p_1} \cap \cdots \cap \ideal{p_q} \). On a \( x^{n_1 + \cdots + n_q} \in \ideal{p_1}^{n_1} \cdots \ideal{p_q}^{n_q} = (0) \), donc \( x^{n_1 + \cdots + n_q} = 0 \) ; d'où \( x = 0 \) car \( A \) est réduit. On a donc \( (0) = \ideal{p_1} \cap \cdots \cap \ideal{p_q} \).
        \end{proof}

        Ceci étant, on a \( (0) = \ideal{p_1} \cap \cdots \cap \ideal{p_q} \)\,, les \( \ideal{p_i} \) étant des idéaux premiers. Donc l'homomorphisme canonique \( \varphi \colon A \to \prod_{i=1}^q A/\ideal{p_i} \) est injectif. Or un élément d'un anneau produit est inversible si et seulement si ses composantes sont toutes inversibles, de sorte que \( \left(\prod_i A/\ideal{p_i}\right)^\times = \prod_i (A/\ideal{p_i})^\times \). D'après le cas intègre, chaque \( (A/\ideal{p_i})^\times \) est de type fini, donc aussi \( \prod_i (A/\ideal{p_i})^\times \) et par conséquent \( \varphi(A^\times) \) (on rappelle que \( \Z \) est nœthérien). Ainsi \( A^\times \) est de type fini, vu que \( \varphi \) est injectif.

        Passons enfin au \emph{cas général}. Notons que l'ensemble \( \ideal{n} \) des éléments nilpotents de \( A \) est un \emph{idéal}, car \( x^p = 0 \), \( y^q = 0 \) et \( a \in A \) entraînent \( (x+y)^{p+q-1} = 0 \) et \( (ax)^p = 0 \). D'autre part il existe un entier \( s \) tel que \( \ideal{n}^s = (0) \) : en effet, \( A \) étant nœthérien, \( \ideal{n} \) admet un système générateur fini \( (x_1, \cdots, x_r) \) avec \( x_i^{q_i} = 0 \) pour tout \( i \) ; alors, pour \( s = q_1 + \cdots + q_r \)\,, tout monôme de degré \( s \) en les \( x_i \) est nul, de sorte que \( \ideal{n}^s = (0) \). Nous procéderons par récurrence sur \( s \). Le cas \( s = 1 \) est le cas réduit, déjà traité. Supposons donc \( s > 1 \), et notons \( \varphi \) l'homomorphisme canonique \( \varphi \colon A \to A/\ideal{n}^{s-1} \). On~a \( \varphi(A^\times) \subset (A/\ideal{n}^{s-1})^\times \), de sorte que \( \varphi(A^\times) \) est un groupe de type fini (par l'hypothèse de récurrence ; en effet, l'ensemble \( \ideal{n}' \) des éléments nilpotents de \( A/\ideal{n}^{s-1} \) n'est autre que \( \varphi(\ideal{n}) \), d'où \( \ideal{n}'\,{}^{s-1} = (0) \) dans \( A/\ideal{n}^{s-1} \)). D'autre part le noyau de la restriction de \( \varphi \) à \( A^\times \) est contenu dans \( 1+\ideal{n}^{s-1} \), et lui est même \emph{égal} car, comme \( s > 1 \), on a \( (\ideal{n}^{s-1})^2 \subset \ideal{n}^s = (0) \), et tout élément \( 1+x \) de \( 1+\ideal{n}^{s-1} \) est inversible vu que \( (1+x)(1-x) = 1-x^2 = 1 \). Reste à montrer que le groupe multiplicatif \( 1+\ideal{n}^{s-1} \) est de type fini. Or, comme \( (\ideal{n}^{s-1})^2 = (0) \), on a \( (1+x)(1+y) = 1+x+y \) pour \( x,y \in \ideal{n}^{s-1} \), de sorte que \( x \mapsto 1+x \) est un isomorphisme du groupe additif \( \ideal{n}^{s-1} \) sur le groupe multiplicatif \( 1+\ideal{n}^{s-1} \). Mais, comme \( A \) est un \( \Z \)‑module de type fini, il en est de même de~\( \ideal{n}^{s-1} \).
    \end{proof}

    \begin{quote}
        En utilisant des méthodes empruntées à la Géométrie Algébrique, on montre que pour tout anneau \emph{réduit} \( B \) de la forme \( B = \Z[x_1, \cdots, x_n] \) (c'est-à-dire engendré sur \( \Z \)\,, en tant qu'anneau, par un nombre fini d'éléments), le groupe \( B^\times \) des éléments inversibles de \( B \) est de type fini (\cite{samuel1966}).
    \end{quote}

\counterwithout{equation}{section}
\cleardoublepage
\phantomsection
\addcontentsline{toc}{section}{\emph{Appendice}.\texorpdfstring{~~}{ }Un calcul de volume}
\markright{\emph{Appendice}.~~Un calcul de volume}
\section*{Appendice}

\enlargethispage*{0pt}
\medskip
    \begin{flushleft}
        \textbf{Un calcul de volume}
    \end{flushleft}
\medskip

    \begin{proposition*}
        Soient \( r_1, r_2 \in \N \)\,, \( n = r_1+2r_2 \)\,, \( t \in \R \) et \( B_t \) l'ensemble des \( (y_1, \cdots, y_{r_1}, z_1, \cdots, z_{r_2}) \in \R^{r_1} \times \C^{r_2} \) tels que
        \begin{equation}\label{eq:prop;Minkowski;aux1'}
            \sum_{i=1}^{r_1} \abs{y_i} + 2 \sum_{j=1}^{r_2} \abs{z_j} \leqslant t.
        \end{equation}
        Alors, pour la mesure de Lebesgue, \( \mu \), on a \( \mu(B_t) = 0 \) pour \( t < 0 \), et
        \begin{equation}\label{eq:prop;Minkowski;aux2'}\hspace{5em}
            \mu(B_t) = 2^{r_1} \left(\frac{\pi}{2}\right)^{r_2} \frac{t^n}{n!} \quad \text{ pour } t \geqslant 0.
        \end{equation}
    \end{proposition*}
    \begin{proof}
        On peut se borner au cas \( t \geqslant 0 \), car, pour \( t < 0 \), on a \( B_t = \emptyset \) et \( \mu(B_t) = 0 \). Posons \( \mu(B_t) = V(r_1, r_2, t) \) et procédons par double récurrence sur \( r_1 \) et \( r_2 \). On a \( V(1,0,t) = 2t \) (segment \( [-t,t] \)) et \( V(0,1,t) = \frac{\pi t^2}{4} \) (disque de rayon \( \frac{t}{2} \)), ce qui est conforme à~\eqref{eq:prop;Minkowski;aux2'}.

        Passons de \( r_1 \) à \( r_1 + 1 \). L'ensemble \( B_t \subset \R \times \R^{r_1} \times \C^{r_2} \) correspondant à \( r_1 + 1 \) et \( r_2 \) est défini par
        {\[\setlength{\abovedisplayskip}{.5\baselineskip}%
        \setlength{\belowdisplayskip}{.5\baselineskip}%
            \abs{y} + \sum_{i=1}^{r_1} \abs{y_i} + 2 \sum_{j=1}^{r_2} \abs{z_j} \leqslant t \quad \text{(\( y \in \R \))}.
        \]}%
        La formule d'intégration \textquote{par tranches} nous donne
        {\[\setlength{\abovedisplayskip}{.5\baselineskip}%
        \setlength{\belowdisplayskip}{.5\baselineskip}%
            V(r_1+1,r_2,t) = \int_{\R} V(r_1,r_2,t-\abs{y}) \odif{y} = \int_{-t}^{+t} V(r_1,r_2,t-\abs{y}) \odif{y}.
        \]}%
        D'après l'hypothèse de récurrence, il vient
        {\[\setlength{\abovedisplayskip}{.5\baselineskip}%
        \setlength{\belowdisplayskip}{.5\baselineskip}%
            V(r_1+1,r_2,t) = 2 \int_0^t 2^{r_1} \left(\frac{\pi}{2}\right)^{r_2} \frac{(t-y)^n}{n!} \odif{y} = 2^{r_1+1} \left(\frac{\pi}{2}\right)^{r_2} \frac{t^{n+1}}{(n+1)!}\,,
        \]}%
        ce qui est encore conforme à~\eqref{eq:prop;Minkowski;aux2'}.

        Passons enfin de \( r_2 \) à \( r_2+1 \). L'ensemble \( B_t \subset \R^{r_1} \times \C^{r_2} \times \C \) correspondant à \( r_1 \) et \( r_2 + 1 \) est défini par
        \[
            \sum_{i=1}^{r_1} \abs{y_i} + 2 \sum_{j=1}^{r_2} \abs{z_j} + 2\abs{z} \leqslant t \quad \text{(\( z \in \C \))}.
        \]
        La formule d'intégration \textquote{par tranches} nous donne ici
        \[
            V(r_1,r_2+1,t) = \int_{\C} V(r_1,r_2,t-2\abs{z}) \odif{\mu(z)} = \int_{\abs{z}\leqslant\frac{t}{2}} V(r_1,r_2,t-2\abs{z}) \odif{\mu(z)}
        \]
        où \( \odif{\mu(z)} \) désigne la mesure de Lebesgue sur \( \C \). En posant \( z = \rho \e^{\i\theta} \) (\( \rho \in \R_{+} \)\,, \( 0 \leqslant \theta \leqslant 2\pi \)), on a \( \odif{\mu(z)} = \rho \odif{\rho,\theta} \). En utilisant l'hypothèse de récurrence, il vient alors
        \begingroup
        \allowdisplaybreaks
        \begin{align*}
            V(r_1,r_2+1,t) &= \int_0^{t/2} \int_0^{2\pi} 2^{r_1} \left(\frac{\pi}{2}\right)^{r_2} \frac{(t-2\rho)^n}{n!} \rho \odif{\rho,\theta}\\
            &= 2^{r_1} \left(\frac{\pi}{2}\right)^{r_2} \frac{2\pi}{n!} \int_0^{t/2} (t-2\rho)^n \rho \odif{\rho}.
        \end{align*}%
        \endgroup%
        On calcule \( \int_0^{t/2} (t-2\rho)^n \rho \odif{\rho} \) en posant \( 2\rho = x \) et en intégrant par parties ; on trouve que cette intégrale vaut \( \frac{t^{n+2}}{4(n+1)(n+2)} \). D'où
        \[
            V(r_1,r_2+1,t) = 2^{r_1} \left(\frac{\pi}{2}\right)^{r_2+1} \frac{t^{n+2}}{(n+2)!} \,,
        \]
        ce qui est conforme à~\eqref{eq:prop;Minkowski;aux2'} car \( r_1 + 2(r_2+1) = n+2 \).
    \end{proof}

\counterwithin{equation}{section}


\chapter{Décomposition des idéaux premiers dans une extension}

Soient \( K \) un corps de nombres, \( A \) l'anneau des entiers de \( K \), \( L \) une extension de degré fini de \( K \), et \( B \) la fermeture intégrale de \( A \) dans \( L \) (qui n'est autre que l'anneau des entiers de \( L \)). Étant donné un idéal premier \( \ideal{p} \neq (0) \) de \( A \), l'idéal \( B\ideal{p} \) qu'il engendre dans \( B \) n'est pas en général premier ; il se décompose donc en produit d'idéaux premiers (\cref{thm:propriétés d'anneaux de Dedekind}), soit \( B\ideal{p} = \prod_i \ideal{P_i}^{e_i} \). Dans ce chapitre nous nous proposons d'étudier cette décomposition. Le cas où \( B \) est un \( A \)‑module \emph{libre} (par exemple celui où \( A \) est un anneau principal ; \cf.~\cref{cor:structure de fermeture intégrale dans une extension fini;cas principal} \cref[de]{thm:structure de fermeture intégrale dans une extension fini}) est particulièrement simple. Nous exposerons au \cref{sec:préliminaires sur les anneaux de fractions} une technique permettant de se ramener à~ce~cas.

\enlargethispage*{\baselineskip}
\section{Préliminaires sur les anneaux de fractions}\label{sec:préliminaires sur les anneaux de fractions}

    \begin{definition}[anneau de fractions de \( A \) par rapport à \( S \)]\index{anneau de fractions}\index{fractions (anneau de)}
        Soient \( A \) un anneau intègre, \( K \) son corps des fractions, et \( S \) une partie de \( A \), stable pour la multiplication, ne contenant pas \( 0 \) et contenant \( 1 \). On appelle anneau de fractions de \( A \) par rapport à \( S \), et on note \( S^{-1} A \), l'ensemble des éléments \( \frac{a}{s} \in K \) avec \( a \in A \) et \( s \in S \).
    \end{definition}

    C'est un anneau commutatif (car \( \frac{a}{s} + \frac{a'}{s'} = \frac{s'a+sa'}{ss'} \) et \( \frac{a}{s} \cdot \frac{a'}{s'} = \frac{aa'}{ss'} \)) ; il contient \( A \) (car \( 1 \in S \)). Si \( S \) est l'ensemble des éléments non nuls de \( A \), on a \( S^{-1} A = K \). Si \( S \) est réduit à \( 1 \), ou s'il est formé d'éléments inversibles de \( A \), on a \( S^{-1} A = A \).

    \begin{proposition}\label{prop:localisation;correspondant}
        Soient \( A \) un anneau intègre, \( S \) une partie multiplicativement stable de \( A \) contenant \( 1 \) et ne contenant pas \( 0 \), et \( A' = S^{-1} A \).
        \begin{enumerate}
            \item \label{prop:localisation;correspondant;item1} Pour tout idéal \( \ideal{b}' \) de \( A' \), on a \( (\ideal{b}' \cap A) A' = \ideal{b}' \), de sorte que \( \ideal{b}' \mapsto {b}' \cap A \) est une injection croissante (pour l'inclusion) de l'ensemble des idéaux de \( A' \) dans celui des idéaux de~\( A \).
            \item \label{prop:localisation;correspondant;item2} L'application \( \ideal{p}' \mapsto \ideal{p}' \cap A \) est un isomorphisme de l'ensemble ordonné (par inclusion) des idéaux premiers de \( A' \) sur celui des idéaux premiers \( \ideal{p} \) de \( A \) tels que \( \ideal{p} \cap S = \emptyset \). L'application réciproque est \( \ideal{p} \mapsto \ideal{p}A' \).
        \end{enumerate}
    \end{proposition}
    \begin{proof}
        Démontrons \ref{prop:localisation;correspondant;item1}. Si \( \ideal{b}' \) est un idéal de \( A' \), on a \( \ideal{b}' \cap A \subset \ideal{b}' \), d'où \( (\ideal{b}' \cap A) A' \subset \ideal{b}' \) car \( \ideal{b}' \) est un idéal. Pour démontrer l'inclusion opposée, soit \( x \in \ideal{b}' \) ; on a \( x = \frac{a}{s} \) avec \( a \in A \) et \( s \in S \) ; or \( sx \in \ideal{b}' \) car \( A \subset A' \) et que \( \ideal{b}' \) est un idéal ; d'où \( a \in \ideal{b}' \) et \( a \in \ideal{b}' \cap A \). Alors \( x = \frac{1}{s} \cdot a \in A' (\ideal{b}' \cap A) \). D'où \( \ideal{b}' \subset A'(\ideal{b}' \cap A) \) et \( \ideal{b}' = A'(\ideal{b}' \cap A) \). Cette formule assure l'injectivité de l'application \( \varphi \colon \ideal{b}' \mapsto {b}' \cap A \), car on a une application \( \theta \colon \ideal{b} \mapsto A' \ideal{b} \) telle que \( \theta \circ \varphi = \text{identité} \). La croissance de \( \varphi \) est évidente. Ceci démontre~\ref{prop:localisation;correspondant;item1}.

        Passons à \ref{prop:localisation;correspondant;item2}. Si \( \ideal{p}' \) est un idéal premier de \( A' \), alors \( \ideal{p} = \ideal{p}' \cap A \) est un idéal premier de \( A \) (\cref{lem:intersection d'un idéal premier et un sous-anneau est un idéal premier}) ; de plus on a \( \ideal{p} \cap S = \emptyset \) car, si \( s \in \ideal{p} \cap S \), on a \( s \in \ideal{p}' \) et \( 1 = \frac{1}{s} \cdot s \in A' \ideal{p}' = \ideal{p}' \), ce qui est absurde. Inversement, soit \( \ideal{p} \) un idéal premier de \( A \) tel que \( \ideal{p} \cap S = \emptyset \) ; nous allons montrer que \( \ideal{p} A' \) est un idéal premier de \( A' \) et qu'on a \( \ideal{p} A' \cap A = \ideal{p} \). Notons d'abord que \emph{\( \ideal{p} A' \) est l'ensemble des \( \frac{p}{s} \) avec \( p \in \ideal{p} \) et \( s \in S \)} : en effet tout élément \( x \) de \( \ideal{p} A' \) s'écrit \( x = \sum_{i=1}^n \frac{a_i}{s_i}p_i \) (\( a_i \in A \), \( s_i \in S \), \( p_i \in \ideal{p} \)), donc \( x = \sum_i \frac{b_i}{s} p_i \) par réduction à un même dénominateur (\( b_i \in S \), \( s \in S \)) et donc \( x = \frac{p}{s} \) avec \( p = \sum b_i p_i \in \ideal{p} \). On en déduit que \( 1 \notin \ideal{p} A' \), car \( \ideal{p} \cap S = \emptyset \) et qu'on ne peut donc avoir \( 1 = \frac{p}{s} \) avec \( p \in \ideal{p} \) et \( s \in S \). Montrons que l'idéal \( \ideal{p} A' \) est premier : soient \( \frac{a}{s} \in A' \) et \( \frac{b}{t} \in A' \) tels que \( \frac{a}{s} \cdot \frac{b}{t} \in \ideal{p} A' \) ; on a alors \( \frac{a}{s} \cdot \frac{b}{t} = \frac{p}{u} \) avec \( p \in \ideal{p} \) et \( u \in S \) ; d'où \( abu = pst \in \ideal{p} \) ; comme \( \ideal{p} \cap S = \emptyset \), on a \( u \notin \ideal{p} \), d'où \( ab \in \ideal{p} \) (car \( \ideal{p} \) est premier) ; ainsi \( a \) ou \( b \) appartient à \( \ideal{p} \), de sorte que \( \frac{a}{s} \) ou \( \frac{b}{t} \) appartient à \( \ideal{p} A' \). Montrons enfin que \( \ideal{p} = \ideal{p} A' \cap A \) ; l'inclusion \( \ideal{p} \subset \ideal{p} A' \cap A \) est évidente ; inversement, si \( x \in \ideal{p} A' \cap A \), on a \( x = \frac{p}{s} \) (\( p \in \ideal{p} \), \( s \in S \)) car \( x \in \ideal{p} A' \) ; d'où \( sx = p \in \ideal{p} \) ; comme \( s \notin \ideal{p} \) (on a \( \ideal{p} \cap S = \emptyset \)) et que \( \ideal{p} \) est premier, on en déduit \( x \in \ideal{p} \). Ceci étant, les formules \( \ideal{p} = \ideal{p} A' \cap A \) et \( \ideal{p}' = A'(\ideal{p}' \cap A) \) montrent que les applications \( \varphi \colon \ideal{p}' \mapsto \ideal{p}' \cap A \) et \( \theta \colon \ideal{p} \mapsto \ideal{p} A' \) (restreintes aux idéaux premiers décrits dans l'énoncé) sont des bijections réciproques l'une de l'autre, car leurs composées dans les deux sens sont des applications identiques. Leur croissance est évidente.
    \end{proof}

    \begin{corollary}\label{cor:localisation d'un anneau nœthérien est nœthérien}
        Si \( A \) est un anneau nœthérien intègre, tout anneau de fractions \( S^{-1}A \) est nœthérien.
    \end{corollary}
    \begin{proof}
        En effet, l'ensemble des idéaux de \( S^{-1} A \) s'applique, de façon injective et croissante, dans celui des idéaux de \( A \) (\cref{prop:localisation;correspondant}, \ref{prop:localisation;correspondant;item1}) ; il satisfait donc aussi à la condition maximale.
    \end{proof}

    \begin{proposition}\label{prop:fermeture de la localisation est localisé de la fermeture intégrale}
        Soient \( R \) un anneau intègre, \( A \) un sous-anneau de \( R \), \( S \) une partie multiplicativement stable de \( A \) avec \( 1 \in S \) et \( 0 \neq S \), et \( B \) la fermeture intégrale de \( A \) dans \( R \). Alors la fermeture intégrale de \( S^{-1} A \) dans \( S^{-1} R \) est \( S^{-1} B \).
    \end{proposition}
    \begin{proof}
        En effet tout élément de \( S^{-1} B \) s'écrit \( \frac{b}{s} \) avec \( b \in B \) et \( s \in S \) ; on a une équation de dépendance intégrale \( b^n + a_{n-1} b^{n-1} + \cdots + a_0 = 0 \) avec \( a_i \in A \) ; en divisant par \( s^n \) on obtient \( \left(\frac{b}{s}\right)^n + \frac{a_{n-1}}{s} \left(\frac{b}{s}\right)^{n-1} + \cdots + \frac{a_0}{s^n} = 0 \), ce qui montre que \( \frac{b}{s} \) est entier sur \( S^{-1} A \). Inversement, soit \( \frac{x}{s} \) (\( x \in R \), \( s \in S \)) un élément de \( S^{-1} R \) entier sur \( S^{-1} A \) ; on a une équation de dépendance intégrale \( \left(\frac{x}{s}\right)^n + \frac{a_{n-1}}{t_{n-1}} \left(\frac{x}{s}\right)^{n-1} + \cdots + \frac{a_0}{t_0} = 0 \) (\( a_i \in A \), \( t_i \in S \)) ; en multipliant par \( (t_0 t_1 \cdots t_{n-1})^n \) on voit que \( x t_0 t_1 \cdots t_{n-1} / s \) est entier sur \( A \), donc est élément de \( B \) ; ainsi \( \frac{x}{s} = \frac{1}{t_0 t_1 \cdots t_{n-1}} \cdot \frac{x t_0 t_1 \cdots t_{n-1}}{s} \) est élément de~\( S^{-1} B \).
    \end{proof}

    \begin{corollary}\label{cor:localisation d'un anneau intégralement clos est intégralement clos}
        Si \( A \) est un anneau intégralement clos, tout anneau de fractions \( S^{-1} A \) est intégralement~clos.
    \end{corollary}
    \begin{proof}
        En effet on prend pour \( R \) le corps des fractions de \( A \).
    \end{proof}

    \begin{proposition}\label{cor:localisation d'un anneau de Dedekind est de Dedekind}
        Si \( A \) est un anneau de Dedekind, tout anneau de fractions \( S^{-1} A \) est un anneau de Dedekind.
    \end{proposition}
    \begin{proof}
        En effet \( S^{-1} A \) est nœthérien (\cref{cor:localisation d'un anneau nœthérien est nœthérien} \cref[de]{prop:localisation;correspondant}) et intégralement clos (\cref{cor:localisation d'un anneau intégralement clos est intégralement clos} \cref[de]{prop:fermeture de la localisation est localisé de la fermeture intégrale}). De plus, comme on \textquote{perd} des idéaux premiers en passant de \( A \) à \( S^{-1} A \) (\cref{prop:localisation;correspondant}, \ref{prop:localisation;correspondant;item2}), tout idéal premier non nul de \( S^{-1} A \) est maximal.
    \end{proof}

    \begin{proposition}\label{prop:localisation où S=A-p}
        Soient \( A \) un anneau de Dedekind, \( \ideal{p} \) un idéal premier non nul de \( A \) et \( S = A \setminus \ideal{p} \). Alors \( S^{-1} A \) est un anneau principal, et il existe un élément premier \( p \) de \( S^{-1} A \) tel que les seuls idéaux non nuls de \( S^{-1} A \) soient les \( (p^n)_{n \geqslant 0} \).
    \end{proposition}
    \begin{proof}
        En effet, comme \( \ideal{p} \) est le seul idéal premier \( \neq (0) \) de \( A \) contenu dans~\( \ideal{p} \), c'est-à-dire disjoint de \( S \), le seul idéal premier non nul de \( S^{-1} A \) est \( \ideal{P} = \ideal{p} S^{-1} A \) (\cref{prop:localisation;correspondant}, \ref{prop:localisation;correspondant;item2}). Comme \( S^{-1} A \) est un anneau de Dedekind (\cref{cor:localisation d'un anneau de Dedekind est de Dedekind}), ses seuls idéaux non nuls sont les \( \ideal{P}^n \) (\( n \geqslant 0 \)). Prenons alors \( p \in \ideal{P} \setminus \ideal{P}^2 \) ; l'idéal \( (p) \) qu'il engendre est contenu dans \( \ideal{P} \) mais pas dans \( \ideal{P}^2 \) ; donc nécessairement \( (p) = \ideal{P} \) ; les seuls idéaux non nuls de \( S^{-1} A \) sont ainsi les \( (p^n) \), et \( S^{-1} A \) est principal.
    \end{proof}

    \begin{proposition}\label{prop:quotient et localisation}
        Soient \( A \) un anneau intègre, \( S \) une partie multiplicativement stable par \( A \) (\( 1 \in S \), \( 0 \notin S \)) et \( \ideal{m} \) un idéal maximal de \( A \) tel que \( \ideal{m} \cap S = \emptyset \). Alors
        \[
            S^{-1} A / \ideal{m} S^{-1} A \simeq A/\ideal{m}.
        \]
    \end{proposition}
    \begin{proof}
        Plus précisément l'homomorphisme composé \( A \to S^{-1} A \to S^{-1} A / \ideal{m} S^{-1} A \) a pour noyau \( \ideal{m} S^{-1} A \cap A = \ideal{m} \) (\cref{prop:localisation;correspondant}, \ref{prop:localisation;correspondant;item2}), d'où une injection \( \varphi \colon A/\ideal{m} \to S^{-1} A / \ideal{m} S^{-1} A \). Reste à montrer que \( \varphi \) est surjective. Or soit \( x = \frac{a}{s} \in S^{-1} A \) (\( a \in A \), \( s \in S \)) ; comme \( s \notin \ideal{m} \) (on a \( \ideal{m} \cap S = \emptyset \)) et comme \( \ideal{m} \) est maximal, \( s \) est inversible modulo \( \ideal{m} \) et il existe \( b \in A \) tel que \( bs \equiv 1 \pmod*{\ideal{m}} \) ; alors \( \frac{a}{s} - ab = \frac{a}{s} (1-bs) \in \ideal{m} S^{-1} A \), de sorte que l'image par \( \varphi \) de la classe de \( ab \) est égale à la classe de \( \frac{a}{s} = x \).
    \end{proof}

\section{Décomposition d'un idéal premier dans une extension}\label{sec:décomposition d'un idéal premier dans une extension}

    Dans cette section, on désigne par \( A \) un anneau de Dedekind de caractéristique \( 0 \), par \( K \) son corps des fractions, par \( L \) une extension de degré fini \( n \) de \( K \), et par \( B \) la fermeture intégrale de \( A \) dans \( L \). On rappelle que \( B \) est un anneau de Dedekind (\cref{thm:anneau des entiers est de Dedekind}).

    \medskip
    Soit \( \ideal{p} \) un idéal premier non nul de \( A \). Alors \( B \ideal{p} \) est un idéal de \( B \) dont on a une décomposition
    \begin{equation}\label{eq:décomposition d'idéal dans une extension}
        B \ideal{p} = \prod_{i=1}^q \ideal{P_i}^{e_i}
    \end{equation}
    où les \( \ideal{P_i} \) sont des idéaux premiers de \( B \), deux à deux distincts, et où les \( e_i \) sont des entiers~\( \geqslant 1 \).

    \begin{proposition}\label{prop:facteurs premiers dans la décomposition d'idéal dans une extension}
        Les \( \ideal{P_i} \) sont exactement les idéaux premiers \( \ideal{Q} \) de \( B \) tels que \( \ideal{Q} \cap A = \ideal{p} \).
    \end{proposition}
    \begin{proof}
        En effet, pour un idéal premier \( \ideal{Q} \) de \( B \), la relation \( \ideal{Q} \cap A = \ideal{p} \) équivaut à \( \ideal{Q} \supset \ideal{p} B \) (\( \Rightarrow \) évident ; \( \Leftarrow \) car \( \ideal{Q} \cap A \) est un idéal premier de \( A \) et que \( \ideal{p} \) est maximal). \nameCref{prop:facteurs premiers dans la décomposition d'idéal dans une extension} résulte alors du formulaire des anneaux de Dedekind (\cref{sec:anneaux de Dedekind}).
    \end{proof}

    Ainsi \( A/\ideal{p} \) s'identifie à un sous-anneau de \( B/\ideal{P_i} \). Ces deux anneaux sont des corps. Comme \( B \) est un \( A \)‑module de type fini (\cref{thm:anneau des entiers est de Dedekind}), \( B/\ideal{P_i} \) est un espace vectoriel de dimension finie sur \( A/\ideal{p} \) ; nous noterons \( f_i \) cette dimension, et l'appellerons le \emph{degré résiduel}\index{degre residuel@degré résiduel}\index{residuel (degre)@résiduel (degré)} de \( \ideal{P_i} \) sur \( A \). L'exposant \( e_i \) dans \eqref{eq:décomposition d'idéal dans une extension} s'appelle l'\emph{indice de ramification}\index{indice de ramification}\index{ramification (indice de)} de \( \ideal{P_i} \) sur \( A \). Notons enfin qu'on a \( B \ideal{p} \cap A = \ideal{p} \) (\( \supset \) évidente ; \( \subset \) résulte de \( \ideal{P_i} \cap A = \ideal{p} \)), de sorte que \( B/B\ideal{p} \) est un espace vectoriel sur \( A/\ideal{p} \), de dimension finie comme ci-dessus.

    \begin{theorem}\label{thm:sum e f = n}
        Avec les notations précédentes on a
        \begin{equation}\label{eq:sum e f = n}
            \sum_{i=1}^q e_i f_i = [ B/B\ideal{p} : A/\ideal{p} ] = n.
        \end{equation}
    \end{theorem}
    \begin{proof}
        La première égalité est facile. Considérons la suite d'idéaux
        \[
            B \supset \ideal{P_1} \supset \ideal{P_1}^2 \supset \cdots \supset \ideal{P_1}^{e_1} \supset \ideal{P_1}^{e_1} \ideal{P_2} \supset \cdots \supset \ideal{P_1}^{e_1} \ideal{P_2}^{e_2} \supset \cdots \supset \ideal{P_1}^{e_1} \cdots \ideal{P_q}^{e_q} = B\ideal{p}.
        \]
        Deux termes consécutifs sont de la forme \( \ideal{B} \) et \( \ideal{B}\ideal{P_i} \) ; or, comme il n'y a pas d'idéaux strictement compris entre \( \ideal{B} \) et \( \ideal{B}\ideal{P_i} \)\,, \( \ideal{B} / \ideal{B}\ideal{P_i} \) est un espace vectoriel de dimension~\( 1 \) sur \( B/\ideal{P_i} \) (\cf.~démonstration \cref[de]{prop:multiplicativité des normes}) ; c'est donc un espace vectoriel de dimension \( f_i \) sur \( A/\ideal{p} \). Or, dans la suite ci-dessus, il y a \( e_i \) quotients de termes consécutifs de la forme \( \ideal{B} / \ideal{B}\ideal{P_i} \) avec \( i \) donné. Au total la dimension \( [ B/B\ideal{p} : A/\ideal{p} ] \) est égale à la somme des dimensions de ces quotients, donc à~\( \sum_{i=1}^q e_i f_i \).

        La seconde égalité est facile aussi dans le cas où \( B \) est un \( A \)‑module libre, en particulier lorsque \( A \) est \emph{principal} (\cref{cor:structure de fermeture intégrale dans une extension fini;cas principal} \cref[de]{thm:structure de fermeture intégrale dans une extension fini}) : en effet une base \( (x_1, \cdots, x_n) \) du \( A \)‑module \( B \) donne, par réduction modulo \( B \ideal{p} \), une base de \( B/B\ideal{p} \) sur \( A/\ideal{p} \). Nous allons nous ramener à ce cas en considérant la partie multiplicativement stable \( S = A \setminus \ideal{p} \) de \( A \) et les anneaux de fractions \( A' = S^{-1} A \) et \( B' = S^{-1} B \). On sait que \( A' \) est un anneau principal dont \( \ideal{p}A' \) est le seul idéal maximal (\cref{prop:localisation où S=A-p}), et que \( B' \) est la fermeture intégrale de \( A' \) dans \( L \) (\cref{prop:fermeture de la localisation est localisé de la fermeture intégrale}). Par le cas principal, on a donc \( [ B/B\ideal{p} : A/\ideal{p} ] = n \). Considérons alors la décomposition de l'idéal \( \ideal{p} B' \) dans l'anneau de Dedekind \( B' \) : de \( \ideal{p} B = \prod_{i=1}^q \ideal{P_i}^{e_i} \) on déduit \( \ideal{p} B' = \prod_{i=1}^q (B'\ideal{P_i})^{e_i} \). Comme \( \ideal{P_i} \cap A = \ideal{p} \) (\cref{prop:facteurs premiers dans la décomposition d'idéal dans une extension}), on a \( \ideal{P_i} \cap S = \emptyset \) et \( B' \ideal{P_i} \) est un idéal premier non nul de \( B' \) (\cref{prop:localisation;correspondant}, \ref{prop:localisation;correspondant;item2}). La première partie de la démonstration nous donne donc
        \[
            [ B'/\ideal{p}B' : A'/\ideal{p}A' ] = \sum_{i=1}^q e_i [ B'/B'\ideal{P_i} : A'/\ideal{p}A' ].
        \]
        Or on a \( A'/\ideal{p}A' \simeq A/\ideal{p} \) et \( B'/B'\ideal{P_i} \simeq B/\ideal{P_i} \) (\cref{prop:quotient et localisation}). D'où, en combinant nos égalités, \( n = [ B'/\ideal{p}B' : A'/\ideal{p}A' ] = \sum_{i=1}^q e_i f_i \)\,, ce qui achève de démontrer \eqref{eq:sum e f = n}.
    \end{proof}

    \begin{proposition}\label{prop:décomposition de B/Bp}
        Avec les mêmes notations, l'anneau \( B/B\ideal{p} \) est isomorphe à \( \prod_{i=1}^q B/\ideal{P_i}^{e_i} \).
    \end{proposition}
    \begin{proof}
        En effet, comme \( \ideal{P_i} \) est le seul idéal maximal de \( B \) qui contienne \( \ideal{P_i}^{e_i} \)\,, on a \( \ideal{P_i}^{e_i} + \ideal{P_j}^{e_j} \) pour \( i \neq j \). On applique alors \eqref{eq:décomposition d'idéal dans une extension} et \cref{lem:reste chinois}.
    \end{proof}

    \subsection*{Exemple des corps cyclotomiques}
        Soient \( p \) un nombre premier, et \( z \in \C \) une racine primitive \( p^r \)‑ième de l'unité. Les racines \( p^r \)‑ièmes de l'unité, dans \( \C \)\,, sont alors les \( z^j \) (\( j = 1, \cdots, p^r \)) ; parmi elles les racines primitives sont les \( z^j \) telles que \( j \) ne soit pas multiple de \( p \), et sont donc au nombre de
        \[
            \varphi(p^r) = p^r - p^{r-1} = p^{r-1}(p-1)
        \]
        (\cf.~\cref{sec:racine de l'unité dans un corps}). Ces racines primitives \( p^r \)‑ièmes de l'unité sont les racines du polynôme cyclotomique
        \begin{equation}\label{eq:polynôme cyclotomique;p sp r}
            F(X) = \frac{X^{p^r}-1}{X^{p^{r-1}}-1} = X^{p^{r-1}(p-1)} + X^{p^{r-1}(p-2)} + \cdots + X^{p^{r-1}} + 1.
        \end{equation}

        Nous nous proposons de redémontrer ici qu'on a \( [ \Q[z] : \Q ] = p^{r-1}(p-1) \), c'est-à-dire que \( F(X) \) est irréductible (\cf.~\cref{sec:corps cyclotomiques}). Posons \( e = p^{r-1}(p-1) \), et soient \( z_1, \cdots, z_e \) les racines primitives \( p^r \)‑ièmes de l'unité. Comme le terme constant de \( F(X+1) \) est \( p \), on~a
        \[
            \prod_{j=1}^e (z_j-1) = \pm p.
        \]
        Soit \( B \) l'anneau des entiers de \( \Q[z] \) ; on a évidemment \( z_j \in B \), et aussi \( z_j-1 \in B(z_k-1) \) pour tout \( j,k \) car \( z_j \) est une puissance \( z_k^q \) de \( z_k \) et qu'on a \( z_k^q-1 = (z_k-1)(z_k^{q-1} + \cdots + z_k + 1) \) ;  ainsi tous les idéaux \( B(z_k-1) \) sont égaux. On a donc \( B p = B(z_1-1)^e \).

        Or écrivons \( B p = \prod_{i=1}^q \ideal{P_i}^{e_i} \) où les \( \ideal{P_i} \) sont des idéaux premiers de \( B \). Les \( e_i \) sont donc tous multiple de \( e \). Mais on a \( e \geqslant [ \Q[z] : \Q ] \) (par \eqref{eq:polynôme cyclotomique;p sp r}), d'où \( e \geqslant \sum_{i=1}^q e_i f_i \) (\cref{thm:sum e f = n}). De ces inégalités en sens contraire on déduit que \( q = 1 \), \( e = e_1 \)\,, \( f_1 = 1 \) et \( [ \Q[z] : \Q ] = e \). En résumé :
        \begin{enumerate}
            \item \( [ \Q[z] : \Q ] = e = p^{r-1}(p-1) \).
            \item \( B(z_1-1) \) est un idéal premier de \( B \), de degré résiduel \( 1 \).
            \item \( Bp = B(z_1-1)^e \).
        \end{enumerate}

\section{Discriminant et ramification}\label{sec:discriminant et ramification}

    Avec les notations du \cref{sec:décomposition d'un idéal premier dans une extension} (soit \( B\ideal{p} = \prod_{i=1}^q \ideal{P_i}^{e_i} \)) on dit qu'un idéal premier \( \ideal{p} \) de \( A \) \emph{se ramifie}\index{ramifie (se ramifie)} dans \( B \) (ou dans \( L \)) si l'un des indices de ramification \( e_i \) est \( \geqslant 2 \). Au moyen de la théorie du discriminant (\cref{sec:discriminant}), nous allons déterminer les idéaux premiers de \( A \) qui se ramifient dans \( B \), et voir en particulier qu'ils sont en \emph{nombre fini}. Quelques lemmes sur les discriminants nous serons utiles.

    \begin{lemma}\label{lem:discriminant de produit et produit de discriminant;idéal}
        Soient \( A \) un anneau, \( B_1, \cdots, B_q \) des anneaux contenant \( A \) et qui sont des \( A \)‑modules libres de rang fini, et \( B = \prod_{i=1}^q B_i \) leur anneau produit. Alors \( \ideal{D}_{B/A} = \prod_{i=1}^q \ideal{D}_{B_i/A} \) (\cf.~\cref{def:discriminant d'un anneau sur un sous-anneau})
    \end{lemma}
    \begin{proof}
        En effet une récurrence sur \( q \) nous ramène au cas \( q = 2 \). Soient alors \( (x_1, \cdots, x_m) \), \( (y_1, \cdots, y_n) \) des bases de \( B_1 \) et \( B_2 \) sur \( A \). Avec l'identification classique de \( B_1 \) et \( B_2 \) avec \( B_1 \times (0) \) et \( (0) \times B_2 \)\,, \( (x_1, \cdots, x_m, y_1, \cdots, y_n) \) est une base de \( B = B_1 \times B_2 \) sur \( A \). On a \( x_i y_j = 0 \) par définition de la structure d'anneau produit, d'où \( \Tr(x_i y_j) = 0 \). Ainsi le discriminant \( \D(x_1, \cdots, x_m, y_1, \cdots, y_n) \) s'écrit :
        \[\def\arraystretch{1.5}
            \abs*{
                \begin{array}{c;{1pt/1pt}c}
                    \Tr(x_i x_{i'}) & 0 \\\hdashline[1pt/1pt]
                    0 & \Tr(y_j y_{j'})
                \end{array}
            }.
        \]
        Il vaut donc \( \det(\Tr(x_i x_{i'})) \cdot \det(\Tr(y_j y_{j'})) \), d'où
        \[
            \D(x_1, \cdots, x_m, y_1, \cdots, y_n) = \D(x_1, \cdots, x_m) \D(y_1, \cdots, y_n).
            \qedhere
        \]
    \end{proof}

    \begin{lemma}\label{lem:discriminant d'image est image de discriminant}
        Soient \( A \) un anneau, \( B \) un anneau contenant \( A \) et admettant une base finie \( (x_1, \cdots, x_n) \), et \( \ideal{a} \) un idéal de \( A \). Pour \( x \in B \) notons \( \overline{x} \) la classe de \( x \) dans \( B/\ideal{a}B \). Alors \( (\overline{x_1}, \cdots, \overline{x_n}) \) est une base de \( B/\ideal{a}B \) sur \( A/\ideal{a} \) et on a
        \begin{equation}\label{eq:discriminant d'image est image de discriminant}
            \D(\overline{x_1}, \cdots, \overline{x_n}) = \overline{\D(x_1, \cdots, x_n)}.
        \end{equation}
    \end{lemma}
    \begin{proof}
        En effet, soit \( x \in B \) ; si la matrice de la multiplication par \( x \), par rapport à la base \( (x_i) \) est \( (a_{ij}) \) (\( (a_{ij}) \in A \)), la matrice de la multiplication par \( \overline{x} \) dans la base \( (\overline{x_i}) \) est \( (\overline{a_{ij}}) \). On a donc \( \Tr(\overline{x}) = \overline{\Tr(x)} \). D'où \( \Tr(\overline{x_i} \cdot \overline{x_j}) = \overline{\Tr(x_i x_j)} \), et donc \eqref{eq:discriminant d'image est image de discriminant} en prenant les déterminant.
    \end{proof}

    \begin{lemma}\label{lem:condition d'être réduite par le discriminant}
        Soient \( K \) un corps fini ou de caractéristique \( 0 \), et \( L \) une \( K \)‑algèbre de dimension finie sur \( K \). Pour que \( L \) soit réduite, il faut et il suffit que \( \ideal{D}_{L/K} \neq (0) \).
    \end{lemma}
    \begin{proof}
        Supposons d'abord \( L \) non réduite, et soit \( x \in L \) un élément nilpotent non nul. On pose \( x_1 = x \) et on complète ce début de base en une base \( (x_1, \cdots, x_n) \) de~\( L \) sur~\( K \). Alors \( x_1 x_j \) est nilpotent, et ainsi la multiplication par \( x_1 x_j \) est un endomorphisme nilpotent ; donc toute les valeurs propres de celui-ci sont nulles, d'où \( \Tr(x_1 x_j) = 0 \). La matrice \( (\Tr(x_i x_j)) \) a donc une ligne nulle, de sorte que son discriminant \( \D(x_1, \cdots, x_n) \) est nul. D'où \( \ideal{D}_{L/K} = (0) \).

        Réciproquement, supposons \( L \) réduite. Alors l'idéal \( (0) \) de \( L \) est intersection finie d'idéaux premiers, \( (0) = \bigcap_{i=1}^q \ideal{P_i} \) (\cref{sec:une généralisation du théorème des unités}, lemme). Comme \( L/\ideal{P_i} \) est une algèbre intègre de dimension finie sur \( K \), c'est un corps (\cref{prop:anneau entier et corps}). Donc \( \ideal{P_i} \) est un idéal maximal de \( L \), de sorte que \( \ideal{P_i} + \ideal{P_j} = L \) pour \( i \neq j \). Ainsi \( L \) est isomorphe au produit \( \prod_{i=1}^q L/\ideal{P_i} \) (\cref{lem:reste chinois}). On a donc \( \ideal{D}_{L/K} = \prod_{i=1}^q \ideal{D}_{(L/\ideal{P_i})/K} \) (\cref{lem:discriminant de produit et produit de discriminant;idéal}). Or \( \ideal{D}_{(L/\ideal{P_i})/K} \neq (0) \) car \( K \) est fini ou de caractéristique \( 0 \) (\cref{prop:disc et det;K-isomorphismes}). D'où \( \ideal{D}_{L/K} \neq (0) \).
    \end{proof}

    \begin{definition}[idéal discriminant]\label{def:idéal discriminant}
        Soient \( K \) et \( L \) deux corps de nombres avec \( K \subset L \), \( A \) et \( B \) les anneaux des entiers de \( K \) et \( L \). On appelle \emph{idéal discriminant}\index{discriminant (ideal)@discriminant (idéal)}\index{ideal discriminant@idéal discriminant} de \( B \) sur \( A \) (ou de~\( L \) sur~\( K \)), et on note \( \ideal{D}_{B/A} \) ou \( \ideal{D}_{L/K} \) l'idéal de \( A \) engendré par les discriminants des bases de~\( L \) sur~\( K \) qui sont contenues dans~\( B \).
    \end{definition}

    \begin{remark}
        \begin{itemize}
            \item Si \( (x_1, \cdots, x_n) \) est une base de~\( L \) sur~\( K \) contenue dans \( B \), on a \( \Tr_{L/K}(x_i x_j) \in A \) (\cref{cor:trace et norme d'élément entier} \cref[de]{prop:trace et norme d'élément entier}), d'où \( \D(x_1, \cdots, x_n) \in A \). Ainsi \( \ideal{D}_{B/A} \) est un idéal \emph{entier} de \( A \). Il est \emph{non nul} par \cref{prop:disc et det;K-isomorphismes}.
            \item Lorsque \( B \) est un \( A \) module \emph{libre} (par exemple si \( A \) est principal) on a déjà défini l'idéal discriminant \( \ideal{D}_{B/A} \) comme étant engendré par \( \D(e_1, \cdots, e_n) \) où \( (e_1, \cdots, e_n) \) est une base de \( B \) sur \( A \) (\cref{def:discriminant d'un anneau sur un sous-anneau}). Il coïncide avec celui défini ci-dessus car, pour toute base \( (x_i) \) de~\( L \) sur~\( K \) contenue dans \( B \), on a \( x_i = \sum_i a_{ij} e_i \) avec \( a_{ij} \in A \), d'où \( \D(x_1, \cdots, x_n) = \det(a_{ij})^2 \D(e_1, \cdots, e_n) \) (\cref{prop:disc;changement de base}).
        \end{itemize}
    \end{remark}

    \begin{theorem}\label{thm:condition d'être ramifié}
        Avec les notations \cref[de]{def:idéal discriminant}, pour qu'un idéal premier \( \ideal{p} \) de \( A \) se ramifie dans \( B \), il faut et il suffit qu'il contienne l'idéal discriminant \( \ideal{D}_{B/A} \). Les idéaux premiers de \( A \) qui se ramifient dans \( B \) sont en nombre fini.
    \end{theorem}
    \begin{proof}
        La seconde assertion résulte de la première car on a vu que \( \ideal{D}_{B/A} \neq (0) \). Démontrons celle-ci. Comme \( B/\ideal{p}B \simeq \prod_{i=1}^q B/\ideal{P_i}^{e_i} \) (\cref{prop:décomposition de B/Bp}), \textquote{\( \ideal{p} \) se ramifie} équivaut à \textquote{\( B/\ideal{p}B \) non réduit}, c'est-à-dire à \textquote{\( \ideal{D}_{(B/\ideal{p}B)/(A/\ideal{p})} = (0) \)} car \( A/\ideal{p} \) est un corps fini (\cref{lem:condition d'être réduite par le discriminant}). Or posons \( S = A \setminus \ideal{p} \), \( A' = S^{-1} A \), \( B' = S^{-1} B \) et \( \ideal{p}' = \ideal{p} A' \). Alors \( A' \) est un anneau principal (\cref{prop:localisation où S=A-p}), \( B' \) est un \( A' \)‑module libre, et on a \( A/\ideal{p} \simeq A'/\ideal{p}' \) et \( B/\ideal{p} \simeq B'/\ideal{p}' \) (\cref{prop:quotient et localisation}). Donc, en désignant par \( (e_1, \cdots, e_n) \) une base de \( B' \) sur \( A' \), la relation \( \ideal{D}_{(B/\ideal{p}B)/(A/\ideal{p})} = (0) \) équivaut à \( \D(e_1, \cdots, e_n) \in \ideal{p}' \) (\cref{lem:discriminant d'image est image de discriminant}). Ceci étant, si \( \D(e_1, \cdots, e_n) \in \ideal{p}' \) et si \( (x_1, \cdots, x_n) \) est une base de~\( L \) sur~\( K \) contenant dans \( B \), on a \( x_i = \sum a_{ij}' e_j \) avec \( a_{ij}' \in A' \) (car \( B \subset B' \)), d'où \( \D(x_1, \cdots, x_n) = \det(a_{ij}')^2 \D(e_1, \cdots, e_n) \in \ideal{p}' \) ; comme \( \ideal{p}' \cap A = \ideal{p} \) (\cref{prop:localisation;correspondant}, \ref{prop:localisation;correspondant;item2}), on en déduit \( \D(x_1, \cdots, x_n) \in \ideal{p} \) et \( \ideal{D}_{B/A} \subset \ideal{p} \). Réciproquement, si \( \ideal{D}_{B/A} \subset \ideal{p} \), on a \( \D(e_1, \cdots, e_n) \in \ideal{p}' \) car on peut écrire \( e_i = \frac{y_i}{s} \) avec \( y_i \in B \) et \( s \in S \) pour \( 1 \leqslant i \leqslant n \) ; ainsi
        \[
            \D(e_1, \cdots, e_n) = s^{-2n} \D(y_1, \cdots, y_n) \in A' \ideal{D}_{B/A} \subset A'\ideal{p} = \ideal{p}'.
            \qedhere
        \]
    \end{proof}

\smallskip
    \subsection*{Exemple des corps quadratiques}
        Prenons \( K = \Q \) et \( L = \Q[\sqrt{d}] \) où \( d \) est un entier sans facteurs carrés (\cref{sec:entiers des corps quadratiques}).
        \begin{enumerate}
            \item Si \( d \equiv 2 \) ou \( 3 \pmod*{4} \), \( (1, \sqrt{d}) \) est une base de l'anneau des entiers de \( L \). Comme \( \Tr(1) = 2 \), \( \Tr(\sqrt{d}) = 0 \) et \( \Tr(d) = 2d \), on a \( \D(1, \sqrt{d}) = 4d \). Les nombres premiers qui se ramifient dans \( L \) sont donc \( 2 \) et les diviseurs premiers de~\( d \).
            \item Si \( d \equiv 1 \pmod*{4} \), \( \bigl(1, \frac{1+\sqrt{d}}{2}\bigr) \) est une base de l'anneau des entiers de \( L \). On a \( \Tr(1) = 2 \), \( \Tr\bigl(\frac{1+\sqrt{d}}{2}\bigr) = 1 \) et
            \[
                \Tr\left(\left(\frac{1+\sqrt{d}}{2}\right)^2\right) = \Tr\left(\frac{d+1}{4}+\frac{1}{2}\sqrt{d}\right) = \frac{d+1}{2}.
            \]
            D'où \( \D\bigl(1, \frac{1+\sqrt{d}}{2}\bigr) = 2 \cdot \frac{d+1}{2} - 1 = d \). Les nombres premiers qui se ramifient dans \( L \) sont donc les diviseurs de~\( d \).
        \end{enumerate}

        On remarquera qu'un corps quadratique \( \Q[\sqrt{d}] \) est uniquement déterminé par son discriminant \( D \) : en effet, si \( D \equiv 0 \pmod*{4} \) on a \( d = D/4 \) avec \( d \equiv 2 \) ou \( 3 \pmod*{4} \) et si \( D \equiv 1 \pmod*{4} \) on a \( d = D \) ; \( D \equiv 2 \) ou \( 3 \pmod*{4} \) est impossible. On notera ainsi le discriminant d'un corps quadratique n'est pas un entier arbitraire.

    \subsection*{Exemple des corps cyclotomiques}
        Soient \( p \) un nombre premier, \( z \in \C \) une racine primitive \( p \)‑ième de l'unité, et \( L = \Q[z] \) le corps cyclotomique correspondant. On sait que l'anneau \( B \) des entiers de \( L \) admet \( (1, z, \cdots, z^{p-2}) \) pour base sur \( \Z \) (\cref{thm:base d'un corps cyclotomique}), et que le polynôme minimal \( F(X) \) de \( z \) sur \( \Q \) satisfait à \( (X-1)F(X) = X^p-1 \) (\cref{thm:polynôme cyclotomique est irréductible}). On va calculer le discriminant \( \ideal{D}_{B/\Z} \) en utilisant la formule \( \D(1, z, \cdots, z^{p-2}) = \Norm(F'(z)) \) (la formule \eqref{eq:disc;cas particulier}). Par dérivation de \( (X-1)F(X) = X^p-1 \), on obtient \( (z-1)F'(z) = p z^{p-1} \) (car \( F(z) = 0 \)). Or \( \Norm(p) = p^{p-1} \), \( \Norm(z) = \pm 1 \), \( \Norm(z-1) = \pm p \) (\cref{sec:corps cyclotomiques}). On a donc
        \begin{equation}
            \D(1, z, \cdots, z^{p-2}) = \pm p^{p-2}.
        \end{equation}
        Il s'ensuit que \( p \) est le \emph{seul} nombre premier qui se ramifie dans \( \Q[z] \). Le résultat suivant est quelquefois utile pour déterminer l'anneau des entiers d'un corps de nombres :
        \begin{proposition}
            Soient \( L \) un corps de nombres de degré \( n \) sur \( \Q \)\,, et \( (x_1, \cdots, x_n) \) des entiers de \( L \) formant une base de \( L \) sur \( \Q \). Si le discriminant \( D(x_1, \cdots, x_n) \) est sans facteurs carrés, alors \( (x_1, \cdots, x_n) \) est une base sur \( \Z \) de l'anneau \( B \) des entiers de~\( L \).
        \end{proposition}
        \begin{proof}
            Em effet, si \( (e_1, \cdots, e_n) \) est une base de \( B \) sur \( \Z \)\,, on a \( x_i = \sum_{j=1}^n a_{ij} e_j \) avec \( a_{ij} \in \Z \). D'où \( \D(x_1, \cdots, x_n) = \det(a_{ij})^2 \D(e_1, \cdots, e_n) \). Comme \( \D(x_1, \cdots, x_n) \) est sans facteurs carrés, on en déduit \( \det(a_{ij}) = \pm 1 \), ce qui implique que \( (x_1, \cdots, x_n) \) est aussi une base de \( B \) sur \( \Z \).
        \end{proof}

        \begin{quote}
            L'exemple des corps cyclotomiques (pour \( p \geqslant 5 \)), ou des corps quadratiques, montre que la condition suffisante ci-dessus n'est nullement nécessaire.
        \end{quote}

        \begin{example}
            Le polynôme \( X^3-X-1 \) (resp. \( X^3+X+1 \), \( X^3+10X+1 \)) est \emph{irréductible} sur \( \Q \) ; sinon, en effet, il aurait un facteur du premier degré, donc une racine \( x \in \Q \) ; on aurait alors \( x \in \Z \) car le polynôme est unitaire ; comme son terme constant est \( 1 \), on aurait même \( x = \pm 1 \) (car tout facteur de \( x \) divise ce terme constant) ; or ceci n'est pas. Donc, en désignant par \( x \in \C \) une racine dudit polynôme, le corps \( L = \Q[x] \) est un \emph{corps cubique} (\ie. de degré \( 3 \)). Ainsi \( (1, x, x^2) \) est une base de \( L \) sur \( \Q \)\,, et \( x \) évidemment un entier de~\( L \). Or, d'après la formule \eqref{disc;cas particulier de cas particulier}, on a \( \D(1,x,x^2) = -4+27 = 23 \) (resp. \( 31 \), \( 4027 \)), qui est un nombre premier. Donc \( (1, x, x^2) \) est une base sur \( \Z \) de l'anneau des entiers~de~\( L \).
        \end{example}

\section{Décomposition d'un nombre premier dans un corps quadratique}\label{sec:décomposition d'un nombre premier dans un corps quadratique}

    Soient \( d \in \Z \) un entier sans facteurs carrés, \( L \) le corps quadratique \( L = \Q[\sqrt{d}] \), \( B \) l'anneau des entiers de \( L \), et \( p \) un nombre premier. On va étudier la décomposition en idéaux premiers de l'idéal~\( pB \).

    La formule \( \sum_{i=1}^q e_i f_i = 2 \) (\cref{thm:sum e f = n}) montre qu'on a \( q \leqslant 2 \) et que seuls trois cas peuvent se produire :
    \begin{enumerate}
        \item \( q = 2 \), \( e_1 = e_2 = 1 \), \( f_1 = f_2 = 1 \) ; on dit alors que \( p \) est \emph{décomposé}\index{decompose (nombre premier)@décomposé (nombre premier)} dans~\( L \) ;
        \item \( q = 1 \), \( e_1 = 1 \), \( f_1 = 2 \) ; on dit alors que \( p \) est \emph{inerte}\index{inerte (nombre premier)} dans~\( L \) ;
        \item \( q = 1 \), \( e_1 = 2 \), \( f_1 = 1 \) ; ceci veut dire que \( p \) \emph{se ramifie}\index{ramifie (nombre premier)} dans~\( L \).
    \end{enumerate}
    Examinons d'abord le cas où \( p \) est \emph{impair}. On sait (\cref{sec:entiers des corps quadratiques}) que \( B = \Z + \Z \sqrt{d} \) ou \( B = \Z + \Z \bigl(\frac{1+\sqrt{d}}{2}\bigr) \) suivant la valeur de \( d \). Mais, si on prend les classes de \( B \) modulo \( Bp \), on voit, dans le second cas, que \( a+b\bigl(\frac{1+\sqrt{d}}{2}\bigr) \) (avec \( b \) impair) est congru à \( a + (b+p) \bigl(\frac{1+\sqrt{d}}{2}\bigr) \), qui est élément de \( \Z + \Z \sqrt{d} \). Donc, dans tous les cas, on a
    \[
        B/Bp \simeq (\Z + \Z \sqrt{d}) / (p).
    \]
    Or \( \Z + \Z \sqrt{d} \simeq \Z[X]/(X^2-d) \). D'où
    \[
        B/Bp \simeq \Z[X]/(p,X^2-d) \simeq (\Z[X]/(p))/(X^2-d) \simeq \F_p[X]/(X^2-\overline{d}),
    \]
    où \( \overline{d} \) désigne la classe de \( d \) modulo \( p \). Or l'assertion que \( p \) est décomposé (resp. est inerte, resp. se ramifie) dans \( B \) signifie que \( B/Bp \) est produit de deux corps (resp. est un corps, resp. a des éléments nilpotents) (\cf.~\cref{prop:décomposition de B/Bp}) ; ceci signifie donc que, dans \( \F_p[X] \), le polynôme \( X^2-\overline{d} \) est produit de deux facteurs distincts du premier degré (resp. est irréductible, resp. est un carré) ; or ceci se produit si \( \overline{d} \) est un carré non nul dans \( \F_p \) (resp. n'est pas un carré dans \( \F_p \)\,, resp. est nul dans \( \F_p \)). Lorsque \( \overline{d} \) est un carré non nul dans \( \F_p \) (resp. n'est pas un carré dans \( \F_p \)), on dit que \( d \) est un \emph{résidu quadratique}\index{residu quadratique@résidu quadratique} (resp. un \emph{non-résidu}\index{non-residu@non-résidu}) modulo~\( p \).

    Traitons maintenant le cas \( p = 2 \). Si \( d \equiv 2, 3 \pmod*{4} \), on a \( B = \Z + \Z \sqrt{d} \), d'où, comme plus haut, \( B/2B \simeq \F_2[X]/(X^2-\overline{d}) \) ; or \( X^2 - \overline{d} \) vaut \( X^2 \) ou \( X^2 + 1 = (X+1)^2 \), et est donc un carré ; ainsi \( 2 \) se ramifie dans \( B \). Si \( d \equiv 1 \pmod*{4} \), \( \frac{1+\sqrt{d}}{2} \) admet \( X^2-X-\frac{d-1}{4} \) pour polynôme minimal, d'où, comme plus haut, \( B/2B \simeq \F_2[X]/(X^2-X-\delta) \) où \( \delta \) est la classe modulo \( 2 \) de \( \frac{d-1}{4} \) ; pour \( d \equiv 1 \pmod*{8} \) on a \( \delta = 0 \) et \( X^2-X-\delta = X(X-1) \), de sorte que \( 2 \) est décomposé ; pour \( d \equiv 5 \pmod*{8} \), on a \( \delta = 1 \) et \( X^2-X-\delta = X^2+X+1 \) est irréductible dans \( \F_2[X] \), de sorte que \( 2 \) est inerte.

    En résumé, on a démontré les résultats suivants :
    \begin{proposition}\label{prop:décomposition d'un nombre premier dans un corps quadratique}
        Soit \( L = \Q[\sqrt{d}] \) un corps quadratique, où \( d \in \Z \) est sans facteurs carrés.
        \begin{enumerate}
            \item \label{prop:décomposition d'un nombre premier dans un corps quadratique;cas1} Sont décomposés dans \( L \), les nombres premiers impairs \( p \) tels que \( d \) soit résidu quadratique modulo \( p \), et \( 2 \) si \( d \equiv 1 \pmod*{8} \) ;
            \item \label{prop:décomposition d'un nombre premier dans un corps quadratique;cas2} Sont inertes dans \( L \), les nombres premiers impairs \( p \) tels que \( d \) soit non-résidu modulo \( p \), et \( 2 \) si \( d \equiv 5 \pmod*{8} \) ;
            \item \label{prop:décomposition d'un nombre premier dans un corps quadratique;cas3} Se ramifient dans \( L \), les diviseurs premiers impairs de \( d \), et \( 2 \) si \( d \equiv 2 \) ou \( 3 \pmod*{4} \).
        \end{enumerate}
    \end{proposition}
    L'assertions \ref{prop:décomposition d'un nombre premier dans un corps quadratique;cas3} a déjà été démontrée dans un exemple du~\cref{sec:discriminant et ramification}.

\section{Loi de réciprocité quadratique}\label{sec:réciprocité quadratique}

    Étant donnés un nombre premier \emph{impair} \( p \) et un entier \( d \) premier à \( p \), nous avons introduit au \cref{sec:décomposition d'un nombre premier dans un corps quadratique} la locution \textquote{\( d \) est un résidu quadratique modulo \( p \)} (resp. \textquote{\( d \) est un non-résidu modulo \( p \)}) comme signifiant que la classe de \( d \) modulo \( p \) est un carré (resp. un non-carré) dans \( \F_p^\times \). Nous introduisons ici le \emph{symbole de Legendre}\index{Legendre (symbole de)}\index{symbole de Legendre} \( \legendre{d}{p} \) ainsi~défini.
    \vspace{-.3\baselineskip}
    \begin{equation}
        \begin{cases}
            \displaystyle\legendre{d}{p} = +1 & \text{si \( d \) est résidu quadratique modulo \( p \)}.\\[1.5em]
            \displaystyle\legendre{d}{p} = -1 & \text{si \( d \) est non-résidu modulo \( p \)}.
        \end{cases}
    \end{equation}

    Bien entendu \( \legendre{d}{p} \) n'est défini que pour \( d \) premier à \( p \), c'est-à-dire pour \( d \in \Z \setminus p\Z \). Le groupe multiplicatif \( \F_p^\times \) étant cyclique d'ordre pair \( p-1 \) (\cref{thm:corps fini}), les carrés en forment un sous-groupe \( (\F_p^\times)^2 \) d'indice \( 2 \), et \( \F_p^\times/(\F_p^\times)^2 \) est isomorphe à \( \set{+1,-1} \). Ainsi le symbole de Legendre s'obtient en composant les homomorphismes
    \[
        \Z \setminus p\Z \to \F_p^\times \to \F_p^\times/(\F_p^\times)^2 \simeq \set{+1,-1}.
    \]
    On a donc la formule de multiplicativité
    \begin{equation}\label{eq:multiplicativité du symbole de Legendre}
        \legendre{ab}{p} = \legendre{a}{p} \legendre{b}{p}.
    \end{equation}

    \begin{proposition}[Critère d'Euler]\label{prop:critère d'Euler}\index{Euler (critere d')@Euler (critère d')}
        Si \( p \) est un nombre premier impair, et si \( a \in \Z \setminus p\Z \)\,, on a
        \[
            \legendre{a}{p} \equiv a^{\frac{p-1}{2}} \pmod{p}.
        \]
        \vspace{-\baselineskip}
    \end{proposition}
    \begin{proof}
        En effet notons \( w \) une racine primitive modulo \( p \) (\cref{sec:corps finis}) ; on a \( a \equiv w^j \pmod*{p} \) avec \( 0 \leqslant j \leqslant p-2 \) car la classe \( \overline{w} \) de \( w \) est un générateur de \( \F_p^\times \). Il est clair que \textquote{\( a \) résidu quadratique} équivaut à \textquote{\( j \) pair} ; on a donc \( \legendre{a}{p} = (-1)^j \). D'autre part, \( \F_p^\times \) a un seul élément d'ordre \( 2 \), à savoir \( \overline{w}^{(p-1)/2} \), et celui-ci est égal à \( -1 \) car son carré est \( 1 \) ; donc, dans \( \Z \)\,, on a \( -1 \equiv w^{(p-1)/2} \pmod*{p} \). Ainsi
        \vspace{-.3\baselineskip}
        \[
            \legendre{a}{p} = (-1)^j \equiv w^{j\cdot\frac{p-1}{2}} \equiv a^{\frac{p-1}{2}} \pmod*{p}.
            \qedhere
        \]
    \end{proof}

    Nous allons maintenant démontrer un célèbre résultat, qui montre que les propriétés de congruences modulo deux nombres premiers impairs distincts ne sont pas indépendantes.

    \begin{theorem}[\textquote{Loi de réciprocité quadratique de Legendre-Gauss}]\label{thm:loi de réciprocité quadratique}\index{loi de reciprocite quadratique@loi de réciprocité quadratique}
        Si \( p \) et \( q \) sont deux nombre premiers impairs distincts, on a
        \[
            \legendre{p}{q} \legendre{q}{p} = (-1)^{\frac{(p-1)(q-1)}{4}}.
        \]
    \end{theorem}
    \vspace{-.5\baselineskip}
    \begin{proof}
        En effet considérons, dans une extension convenable de \( \F_q \)\,, une racine primitive \( p \)‑ième de l'unité \( w \). Comme \( w^p = 1 \), la notation \( w^x \) a un sens pour \( x \in \F_p \). Nous écrirons aussi le symbole de Legendre \( \legendre{x}{p} \) pour \( x \in \F_p^\times \)\,, car \( \legendre{d}{p} \) ne dépend évidemment que de la classe de \( d \) modulo \( p \). Pour \( a \in \F_p^\times \)\,, considérons la \textquote{\emph{somme de~Gauss}}\index{Gauss (somme de)}\index{somme de Gauss} :
        \begin{equation}
            \tau(a) = \sum_{x \in \F_p^\times} \legendre{x}{p} w^{ax}.
        \end{equation}
        C'est un élément d'une extension de \( \F_q \). Posant \( ax = y \), on~a
        \[
            \tau(a) = \sum_{y \in \F_p^\times} \legendre{ya^{-1}}{p} w^y = \legendre{a^{-1}}{p} \sum_{y \in \F_p^\times} \legendre{y}{p} w^y
        \]
        (par \eqref{eq:multiplicativité du symbole de Legendre}), d'où
        \vspace{-.2\baselineskip}%
        \begin{equation}\label{eq:somme de Gauss;relation entre a et 1}
            \tau(a) = \legendre{a}{p} \tau(1).
        \end{equation}
        D'autre part, comme on calcule en caractéristique \( q \) et que \( \legendre{x}{p} \in \F_q \)\,, on a \( \tau(1)^q = \sum_{x \in \F_p^\times} \legendre{x}{p}^q w^{qx} \), d'où, en identifiant \( q \) à sa classe modulo \( p \),
        \vspace{-.2\baselineskip}%
        \begin{equation}\label{eq:somme de Gauss;relation entre q et 1}
            \tau(1)^q = \tau(q).
        \end{equation}
        \vspace{-1.2\baselineskip}\par\noindent%
        Calculons maintenant \( \tau(1)^2 \). On~a
        \vspace{-.2\baselineskip}%
        \[
            \tau(1)^2 = \sum_{x,y \in \F_p^\times} \legendre{x}{p} \legendre{y}{p} w^{x+y}.
        \]
        \vspace{-1.2\baselineskip}\par\noindent%
        Posant \( y = tx \), il vient
        \vspace{-.2\baselineskip}%
        \[
            \tau(1)^2 = \sum_{x,t \in \F_p^\times} \legendre{x}{p}^2 \legendre{t}{p} w^{x(1+t)} = \sum_{x,t \in \F_p^\times} \legendre{t}{p} w^{x(1+t)} = \sum_{t \in \F_p^\times} \left[ \legendre{t}{p} \sum_{x \in \F_p^\times} w^{x(1+t)} \right].
        \]
        Si \( w^{1+t} \neq 1 \), on a \( \sum_{j=0}^{p-1} (w^{1+t})^j = 0 \) par la formule de la progression géométrique, car \( (w^{1+t})^p = 1 \) ; d'où \( \sum_{x \in \F_p^\times} w^{x(1+t)} = -1 \). Si \( w^{1+t} = 1 \), on a \( \sum_{x \in \F_p^\times} w^{x(1+t)} = p-1 \) ; ce dernier cas n'a lieu que pour \( t = -1 \), car \( w \) est une racine primitive \( p \)‑ième de l'unité. On~a~donc
        {\[\setlength{\abovedisplayskip}{.5\baselineskip}%
        \setlength{\belowdisplayskip}{.5\baselineskip}%
            \tau(1)^2 = \legendre{-1}{p}(p-1) - \sum_{\substack{t \in \F_p^\times \\ t \neq -1}} \legendre{t}{p}.
        \]}%
        Comme il y a autant de carrés que de non-carrés dans \( \F_p^\times \)\,, on a \( \sum_{t \in \F_p^\times} \legendre{t}{p} = 0 \), d'où
        \vspace{-.25\baselineskip}%
        \[
            \tau(1)^2 = \legendre{-1}{p}(p-1) + \legendre{-1}{p} = \legendre{-1}{p} p.
        \]
        Par le critère d'Euler (\cref{prop:critère d'Euler}), on a donc
        \vspace{-.25\baselineskip}%
        \begin{equation}\label{eq:somme de Gauss;carré en 1}
            \tau(1)^2 = (-1)^{\frac{p-1}{2}} p.
        \end{equation}
        Enfin, par \eqref{eq:somme de Gauss;relation entre a et 1} et \eqref{eq:somme de Gauss;relation entre q et 1}, on a \( \tau(1)^q = \tau(q) = \legendre{q}{p} \tau(1) \). Comme \( \tau(1) \) est non nul par \eqref{eq:somme de Gauss;carré en 1}, on simplifie : \( \tau(1)^{q-1} = \legendre{q}{p} \). Par \eqref{eq:somme de Gauss;carré en 1} encore on~a
        \vspace{-.3\baselineskip}%
        \[
            \legendre{q}{p} = (\tau(1)^2)^{\frac{q-1}{2}} = (-1)^{\frac{(p-1)(q-1)}{4}} p^{\frac{q-1}{2}}.
        \]
        \vspace{-.7\baselineskip}\par\noindent%
        Comme \( p^{\frac{q-1}{2}} = \legendre{p}{q} \) (\cref{prop:critère d'Euler}) et que \( \legendre{p}{q} = \legendre{p}{q}^{-1} \), la loi de réciprocité est démontrée.
    \end{proof}

    \begin{proposition}[\textquote{formules complémentaires}]\label{prop:réciprocité quadratique;formules complémentaires}
        Si \( p \) est un nombre premier impair on a
        \begin{enumerate}
            \item \label{prop:réciprocité quadratique;formules complémentaires;item1} \( \legendre{-1}{p} = (-1)^{\frac{p-1}{2}} \).
            \item \label{prop:réciprocité quadratique;formules complémentaires;item2} \( \legendre{2}{p} = (-1)^{\frac{p^2-1}{8}} \).
        \end{enumerate}
    \end{proposition}
    \begin{proof}
        En effet \ref{prop:réciprocité quadratique;formules complémentaires;item1} est un cas particulier du critère d'Euler (\cref{prop:critère d'Euler}). Démontrons donc \ref{prop:réciprocité quadratique;formules complémentaires;item2}. Notons d'abord que, comme les carrés de \( 1,3,5,7 \pmod*{8} \) sont \( 1,1,1,1 \), on a \( p^2 \equiv 1 \pmod*{8} \), et la formule écrite a donc un sens. Remarquons ensuite que, dans le groupe \( H = \set{1,3,5,7} \) des éléments inversibles de \( \Z/8\Z \)\,, \( \set{1,7} \) est un sous-groupe \( H' \) d'indice \( 2 \) ; posons \( \theta(x) = 1 \) pour \( x \in H' \) et \( \theta(x) = -1 \) pour \( x \in H \setminus H' \), de sorte qu'on a \( \theta(xy) = \theta(x)\theta(y) \) pour \( x,y \in H \). Soit alors \( w \) une racine primitive \( 8 \)‑ième de l'unité dans une extension de \( \F_p \). Comme dans \cref{thm:loi de réciprocité quadratique}, considérons, pour \( a \in H \), la \textquote{\emph{somme de Gauss}}
        \begin{equation}
            \tau(a) = \sum_{x \in H} \theta(x) w^{ax}.
        \end{equation}
        Comme dans \cref{thm:loi de réciprocité quadratique} on a \( \tau(a) = \theta(a) \tau(1) \) et \( \tau(1)^p = \tau(p) \) (en identifiant \( p \) à sa classe modulo \( 8 \)). D'après la définition de \( \theta(x) \), on~a
        \vspace{-.2\baselineskip}%
        {\allowdisplaybreaks%
        \begin{align*}
            \tau(1) &= w - w^3 - w^5 + w^7 = (1-w^2)(w-w^5)\\
            &= w(1-w^2)(1-w^4) = 2w(1-w^2)
        \end{align*}}%
        \vspace{-1.2\baselineskip}\par\noindent%
        (car \( w^8=1 \) et \( w^4=-1 \)) ; d'où
        \vspace{-.2\baselineskip}%
        \[
            \tau(1)^2 = 4w^2(1-2w^2+w^4) = -8w^4 = 8.
        \]
        \par\vspace{-.2\baselineskip}\noindent\setstretch{1.2}%
        Comme dans \cref{thm:loi de réciprocité quadratique}, on en déduit \( \tau(1)^p = \tau(p) = \theta(p)\tau(1) \) ; d'où, en simplifiant, \( \theta(p) = (\tau(1)^2)^{(p-1)/2} = 8^{(p-1)/2} = \legendre{8}{p} \) (\cref{prop:critère d'Euler}) \( = \legendre{2}{p}^3 = \legendre{2}{p} \). On a donc \( \legendre{2}{p} = \theta(p) \). Or on constate par calcul direct, pour \( x = 1,3,5,7 \) (ou, plus efficacement, pour \( x = 1,3,-3,-1 \)) qu'on a \( \theta(x) = (-1)^{\smash{(x^2-1)}/8} \), et que \( (-1)^{\smash{(x^2-1)}/8} \) ne dépend que de la classe de \( x \) modulo~\( 8 \).
    \end{proof}

    \begin{theorem-with-name}{Exemple d'application}
        La loi de réciprocité et les formules complémentaires permettent de calculer le symbole de Legendre par réductions successives. Calculons ainsi \( \legendre{23}{59} \), sans avoir à écrire la longue table des carrés modulo \( 59 \). On~a
        \begin{align*}
            \legendre{23}{59} &= (-1)^{11 \cdot 29}\legendre{59}{23} = - \legendre{13}{23} = -(-1)^{6 \cdot 11} \legendre{23}{13} = - \legendre{10}{13}\\
            &= -\legendre{-3}{13} = -\legendre{-1}{13} \legendre{3}{13} = -(-1)^6 \legendre{3}{13}\\
            &= - (-1)^{6 \cdot 1} \legendre{13}{3} = -\legendre{1}{3} = -1.
        \end{align*}
        Donc \( 23 \) n'est pas un carré modulo~\( 59 \).
    \end{theorem-with-name}

\section{Théorème des deux carrés}\label{sec:théorème des deux carrés}

    Nous allons appliquer \cref{prop:décomposition d'un nombre premier dans un corps quadratique} au corps \( L = \Q[\i] \) où \( \i^2=-1 \). Comme \( -1 \equiv 3 \pmod*{4} \), l'anneau \( B \) des entiers de \( L \) est \( \Z + \Z \i \) ; on l'appelle l'\emph{anneau des entiers de Gauss}\index{entier de Gauss}\index{Gauss (entier de)} ; son discriminant est \( -4 \) (l'exemple du \cref{sec:discriminant et ramification}). Si \( p \) est un nombre premier impair, et si \( u \) est un générateur du groupe cyclique \( \F_p^\times \)\,, on a \( -1 = u^{(p-1)/2} \). Donc \( -1 \) est un carré dans \( \F_p \) si et seulement si \( \frac{p-1}{2} \) est pair. D'où la classification :
    \begin{itemize}
        \item \( 2 \) se ramifie dans \( \Q[i] \) ;
        \item les nombres premiers de la forme \( 4k+1 \) sont décomposés ;
        \item les nombres premiers de la forme \( 4k+3 \) sont inertes.
    \end{itemize}

    Le résultat suivant va nous être utile :
    \begin{proposition}\label{prop:anneau des entiers de Gauss est principal}
        L'anneau \( B = \Z + \Z \i \) des entiers de Gauss est principal.
    \end{proposition}
    \begin{proof}
        Écrasons, en effet, cette mouche avec un gros pavé. Avec les notations du \cref{sec:finitude du groupe des classes d'idéaux}, on a \( n=2 \), \( r_1=0 \), \( r_2=1 \) et \( d = -4 \). Donc (\cref{cor:Minkowski;existence d'un tel idéal} \cref[de]{prop:Minkowski}), toute classe d'idéaux de \( B \) contient un idéal entier de norme \( \leqslant \frac{4}{\pi} \cdot \frac{2}{4} \abs{4}^{1/2} = \frac{4}{\pi} \), donc contient l'idéal unité \( B \) (qui est le seul idéal entier de norme~\( 1 \)) car \( \frac{4}{\pi} < 2 \). Ainsi tout idéal de \( B \) est équivalent à l'idéal principal \( B \), et est donc principal.
    \end{proof}

    \begin{quote}
        \emph{Esquisse} de démonstration élémentaire : comme les points de \( B \) forment un quadrillage de \( \C \)\,, un peu de géométrie montre que, pour tout \( x \in \Q[\i] \), il existe \( z \in B \) tel que \( \Norm(x-z) = \abs{x-z}^2 \leqslant \bigl(\frac{1}{\sqrt{2}}\bigr)^2 = \frac{1}{2} < 1 \) ; alors, si \( \ideal{a} \) est un idéal non nul de \( B \), on choisit dans \( \ideal{a} \) un élément non nul \( u \) de norme minimale (NB : cette norme est un entier \( > 0 \)) ; pour \( v \in \ideal{a} \) on approche \( \frac{v}{u} \) par un \( z \in B \) tel que \( \Norm(\frac{v}{u}-z) < 1 \) ; alors \( \Norm(v-zu) < \Norm(u) \), d'où \( v-zu = 0 \) car \( v-zu \in \ideal{a} \) ; par conséquent \( v \in Bu \) et \( \ideal{a} = Bu \). On notera l'analogie avec le processus de division euclidienne dans~\( \Z \).
    \end{quote}

    \smallskip
    \begin{proposition}[Fermat]\label{prop:Fermat;deux carrés;4n+1}
        Tout nombre premier \( p \equiv 1 \pmod*{4} \) est somme de deux carrés (\ie. est de la forme \( p = a^2+b^2 \) avec \( a,b \in \N \)).
    \end{proposition}
    \begin{proof}
        En effet \( Bp \) se décompose en un produit \( \ideal{p_1} \ideal{p_2} \) d'idéaux premier distincts. D'où \( p^2 = \Norm(Bp) = \Norm(\ideal{p_1}) \Norm(\ideal{p_2}) \) (\cref{prop:multiplicativité des normes}). Comme les normes de~\( \ideal{p_1} \) et de~\( \ideal{p_2} \) sont distinctes de \( i \), on a nécessairement
        \[
            \Norm(\ideal{p_1}) = \Norm(\ideal{p_2}) = p.
        \]
        Or, \( \ideal{p_1} \) est un idéal principal \( B(a+b\i) \) (\( a,b \in \Z \)) (\cref{prop:anneau des entiers de Gauss est principal}) ; d'où, en prenant les normes, \( p = \Norm(a+b\i) = a^2 + b^2 \).
    \end{proof}

    \smallskip
    \begin{theorem}
        Soient \( x \) un entier naturel, et \( x = \prod_p p^{v_p(x)} \) sa décomposition en facteurs premiers. Pour que \( x \) soit somme de deux carrés, il faut et il suffit que, pour tout \( p \equiv 3 \pmod*{4} \), l'exposant \( v_p(x) \) soit pair.
    \end{theorem}
    \begin{proof}
        Pour démontrer la suffisance, remarquons qu'une somme de deux carrés \( a^2+b^2 \) est la norme \( \Norm(a+b\i) \) d'un élément de \( B \) ; par la multiplicativité des normes, l'ensemble \( S \) des sommes de deux carrés est donc stable par multiplication. Comme \( 2 = 1^2+1^2 \in S \) et que tout carré est élément de \( S \) (\( x^2 = x^2 + 0^2 \)), il résulte alors \cref[de]{prop:Fermat;deux carrés;4n+1} que notre condition est suffisante.

        Réciproquement, soient \( x = a^2 + b^2 \) une somme de deux carrés (\( a,b \in \N \)) et \( p \) un nombre premier \( \equiv 3 \pmod*{4} \). On a vu que l'idéal \( Bp \) de \( B \) est premier. Or on a \( x = a^2+b^2 = (a+b\i)(a-b\i) \). Soit \( n \) l'exposant de \( Bp \) dans la décomposition de \( B(a+b\i) \) en facteurs premiers. Comme \( Bp \) est stable par l'automorphisme \( \sigma \colon u+\i v \mapsto u-\i v \) de \( B \), et que \( \sigma(a+\i b) = a- \i b \), l'exposant de \( Bp \) dans la décomposition de \( B(a-\i b) \) est aussi \( n \) ; dans celle de \( B(a^2+b^2) \), l'exposant de \( Bp \) est donc \( 2n \). Comme aucun nombre premier distinct de \( p \) n'appartient à \( Bp \) (car \( Bp \cap \Z = p\Z \)), on a \( v_p(x) = 2n \) et \( v_p(x) \) est pair.
    \end{proof}

\section{Théorème des quatre carrés}\label{sec:théorème des quatre carrés}

    Dans cette section, nous nous proposons de démontrer \namecref{thm:Lagrange;somme des quatre carrés} suivant :
    \begin{theorem}[Lagrange]\label{thm:Lagrange;somme des quatre carrés}
        Tout entier naturel est somme de quatre carrés.
    \end{theorem}

    La méthode employée est analogue à celle du \cref{sec:théorème des deux carrés} : au lieu de l'anneau des entiers de Gauss, nous travaillerons dans un anneau de \emph{quaternions}\index{quaternions} convenablement choisi.

    Commençons par définir les quaternions. Étant donné un anneau \( A \), nous noterons \( (1,\i,\j,\k) \) la base canonique du \( A \)‑module \( A^4 \), et nous définissons une multiplication par :
    \begin{equation}\label{eq:quaternion}
        \begin{cases}
            \enspace \text{\( 1 \) est élément unité},\\
            \enspace \i^2 = \j^2 = \k^2 = -1,\\
            \enspace \i\j = -\j\i = \k, ~ \j\k=-\k\j = \i, ~ \k\i = -\i\k = \j.
        \end{cases}
    \end{equation}
    Nous étendons cette multiplication aux éléments \( a+b\i+c\j+d\k \) de \( A^4 \) par linéarité ; la distributivité est alors évidente. Quant à l'associativité, il suffit de la vérifier sur les éléments de base : ainsi
    \[
        \i(\j\k) = \i^2 = -1 = \k^2 = (\i\j)\k ~;
    \]
    les formule où figure \( 1 \) étant évidentes, il reste \( 3^3-1 = 26 \) formules à vérifier ; le lecteur patient et incrédule en réduira en le nombre par permutations, et vérifiera cells qui restent ; les autres croirons l'auteur sur parole. Muni de cette multiplication, \( A^4 \) est donc un \emph{anneau non nécessairement commutatif}, et même une \emph{\( A \)‑algèbre}, qu'on appelle l'\emph{anneau des quaternions} sur \( A \), et qu'on note \( \H(A) \) (\( \H \) en honneur de W. R. Hamilton, inventeur des quaternions).

    Étant donné un quaternion \( z = a+b\i+c\j+d\k \) sur \( A \) (on écrit \( a \) au lieu de \( a \cdot 1 \)), on appelle \emph{quaternion conjugué} de \( z \), et on note \( \conj{z} \) le quaternion \( \conj{z} = a-b\i-c\j-d\k \).

    \begin{lemma}
        On a \( \conj{z+z'} = \conj{z}+\conj{z'} \), \( \conj{z z'} = \conj{z'}\cdot\conj{z} \) et \( \conj{\conj{z}} = z \) ; en termes plus savants, \( z \mapsto \conj{z} \) est un antiautomorphisme involutif de~\( \H(A) \).
    \end{lemma}
    \begin{proof}
        La première et la troisième formule sont évidentes. Pour la seconde on est ramenés par linéarité à montrer qu'on a \( \conj{xy} = \conj{y} \cdot \conj{x} \) lorsque \( x,y \in \set{1,\i,\j,\k} \). Or c'est clair si \( x = 1 \) ou si \( y = 1 \). Si \( x = y = \i \) on a \( \conj{xy} = \conj{-1} = -1 \) et \( \conj{y} \cdot \conj{x} = (-\i)(-\i) = \i^2 = -1 \). Si \( x = \i \) et \( y = \j \) on a \( \conj{xy} = \conj{\k} = -\k \) et \( \conj{y} \cdot \conj{x} = (-\j)(-\i) = \j\i = -\k \). Les autres vérifications s'en déduisent par permutation.
    \end{proof}

    Étant donné un quaternion \( z \) sur \( A \), on appelle \emph{norme réduite} de \( z \), et on note \( \Norm(z) \) le quaternion~\( z \conj{z} \).

    \begin{lemma}\label{lem:norme des quaternions}
        \begin{enumerate}
            \item \label{lem:norme des quaternions;item1} Étant donné un quaternion \( z = a+b\i+c\j+d\k \) sur \( A \), on a \( \Norm(z) = a^2+b^2+c^2+d^2 \) (quatre carrés !), donc \( \Norm(z) \in A \).
            \item \label{lem:norme des quaternions;item2} Étant donnés deux quaternions \( z, z' \) sur \( A \), on a \( \Norm(zz') = \Norm(z)\Norm(z') \).
        \end{enumerate}
    \end{lemma}
    \begin{proof}
        Pour \ref{lem:norme des quaternions;item1}, on développe \( (a+b\i+c\j+d\k)(a-b\i-c\j-d\k) \) : par \eqref{eq:quaternion} les termes \textquote{rectangles} disparaissent, et il reste \( a^2+b^2+c^2+d^2 \). On voit aussi que \( z \conj{z} = \conj{z} z \). Alors
        \[
            \Norm(zz') = zz' \cdot \conj{zz'} = zz' \,\conj{z'} \,\conj{z} = z \Norm(z') \conj{z} = z \conj{z} \Norm(z')
        \]
        (car l'élément \( \Norm(z') \) de \( A \) est permutable à tout quaternion) ; d'où \( \Norm(zz') = \Norm(z)\Norm(z') \).
    \end{proof}

    % \vspace{-\baselineskip}
    \begin{quote}
        \Cref{lem:norme des quaternions} montre que, dans un anneau \( A \) (commutatif), l'ensemble des normes réduites de quaternions, c'est-à-dire des sommes de \( 4 \) carrés, est stable pour la multiplication.
    \end{quote}

    Cela étant nous considérerons, dans \( \H(\Q) \), le sous-anneau non-commutatif \( \H(\Z) \) et l'ensemble \( \H \) des \textquote{\emph{quaternions d'Hurwitz}}\index{quaternions d'Hurwitz} \( a+b\i+c\j+d\k \) où \( a,b,c,d \) sont, ou bien tous les quatre dans \( \Z \)\,, ou bien tous les quatre dans \( \frac{1}{2}+\Z \).

    \begin{lemma}\label{lem:quaternions d'Hurwitz}
        \begin{enumerate}
            \item \label{lem:quaternions d'Hurwitz;item1} L'ensemble \( \H \) des quaternions d'Hurwitz est un sous-anneau non-commutatif de \( \H(\Q) \) contenant \( \H(\Z) \), et stable par \( z \mapsto \conj{z} \).
            \item \label{lem:quaternions d'Hurwitz;item2} Pour tout \( z \in \H \) on a \( z+\conj{z} \in \Z \) et \( \Norm(z) = z\conj{z} \in \Z \).
            \item \label{lem:quaternions d'Hurwitz;item3} Pour que \( z \in \H \) soit inversible, il faut et il suffit que \( \Norm(z) = 1 \).
            \item \label{lem:quaternions d'Hurwitz;item4} Tout idéal à gauche (resp. à droite) \( a \) de \( \H \) est principal (\ie. est de la forme \( \H z \) (resp. \( z \H \))).
        \end{enumerate}
    \end{lemma}
    \begin{proof}
        Pour \ref{lem:quaternions d'Hurwitz;item1}, toutes les assertions sont évidentes, sauf la stabilité de \( \H \) pour la multiplication. Pour celle-ci il suffit de vérifier que, si on pose \( u = \frac{1}{2}(1+\i+\j+\k) \), on a \( u \cdot 1 \), \( u \cdot \i \)\,, \( u \cdot \j \)\,, \( u \cdot \k \) et \( u^2 \in \H \) ; or \( u \cdot 1 = \frac{1}{2}(1+\i+\j+\k) \), \( u \cdot \i = \frac{1}{2}(-1+\i+\j-\k) \), \( u \cdot \j = \frac{1}{2}(-1-\i+\j+\k) \), \( u \cdot \k = \frac{1}{2}(-1+\i-\j+\k) \) ; d'où, par addition \( 2u^2 = \frac{1}{2}(-2+2\i+2\j+2\k) \) et \( u^2 \in \H \).

        Pour \ref{lem:quaternions d'Hurwitz;item2}, si
        \[
            z = \frac{1}{2}+a + \left(\frac{1}{2}+b\right) \i + \left(\frac{1}{2}+c\right) \j + \left(\frac{1}{2}+d\right) \k \quad \text{(\( a,b,c,d \in \Z \))},
        \]
        on a \( z + \conj{z} = 1 + 2a \in \Z \)\,, et
        \[
            z \conj{z} = \left(\frac{1}{2}+a\right)^2 + \left(\frac{1}{2}+b\right)^2 + \left(\frac{1}{2}+c\right)^2 + \left(\frac{1}{2}+d\right)^2 \in \frac{4}{4} + \Z \subset \Z
        \]
        par \cref{lem:norme des quaternions}.

        Si \( z \) est inversible dans \( \H \)\,, et si \( z' \) est son inverse, on a
        \[
            \Norm(z)\Norm(z') = \Norm(z z') = 1 ~;
        \]
        comme \( \Norm(z) \) et \( \Norm(z') \) sont des entiers \( > 0 \) (\ref{lem:quaternions d'Hurwitz;item2} et \cref{lem:norme des quaternions}, \ref{lem:norme des quaternions;item1}), on a nécessairement \( \Norm(z) = 1 \). Réciproquement, si \( z \in \H \) et si \( \Norm(z) = 1 \), on a \( z \conj{z} = \conj{z} z = \Norm(z) = 1 \) et \( z \) est inversible car \( \conj{z} \in \H \) par \ref{lem:quaternions d'Hurwitz;item1}. Ceci démontre~\ref{lem:quaternions d'Hurwitz;item3}.

        Démontrons enfin \ref{lem:quaternions d'Hurwitz;item4}. Étant donné un quaternion \( x = a+b\i+c\j+d\k \in \H(\Q) \), il existe quatre entiers \( a',b'c',d' \in \Z \) tels que
        \[
            \abs{a-a'} \leqslant \frac{1}{2}, \quad
            \abs{b-b'} \leqslant \frac{1}{2}, \quad
            \abs{c-c'} \leqslant \frac{1}{2}, \quad
            \abs{d-d'} \leqslant \frac{1}{2} ~;
        \]
        posons \( z = a'+b'\i+c'\j+d'\k \) ; on a alors
        \[
            \Norm(x-z) = (a-a')^2 + (b-b')^2 + (c-c')^2 + (d-d')^2 \leqslant 4 \cdot \frac{1}{4} = 1.
        \]
        Il y a même inégalité stricte, sauf dans le cas où \( a,b,c,d \) sont tous dans \( \frac{1}{2} + \Z \) ; mais alors on a \( x \in \H \). Donc, pour tout quaternion \( x \in \H(\Q) \), il existe un quaternion d'Hurwitz \( z \in \H \) tel que \( \Norm(x-z) < 1 \) (c'est justement pour avoir l'inégalité stricte qu'on a introduit les quaternions d'Hurwitz, ceux de \( \H(\Z) \) n'auraient pas suffi). Ceci étant, soit \( \ideal{a} \) un idéal à gauche de \( \H \) ; pour montrer qu'il est principal, on peut supposer \( \ideal{a} \neq (0) \). Choisissons, dans \( \ideal{a} \), un élément non nul \( u \) de norme réduite minimale (il en existe, car ces normes sont des entiers \( > 0 \) par \ref{lem:quaternions d'Hurwitz;item2}) ; alors \( u \) est inversible dans \( \H(\Q) \), car son inverse est \( \conj{u} \Norm(u)^{-1} \) (ceci montre d'ailleurs que \( \H(\Q) \) est un corps non-commutatif). Pour \( y \in \ideal{a} \), formons \( y u^{-1} \in \H(\Q) \) et prenons un élément \( z \in \H \) tel que \( \Norm(yu^{-1}-z) < 1 \). Alors, par \cref{lem:norme des quaternions}, \ref{lem:norme des quaternions;item2}, on a \( \Norm(y-zu) = \Norm((yu^{-1}-z)u) < \Norm(u) \). Comme \( y-zu \in \ideal{a} \) et que \( \Norm(u) \) est minimal, on en déduit \( y-zu = 0 \), \( y \in \H u \) et~\( \ideal{a} = \H u \).
    \end{proof}

    Ceci étant, comme l'ensemble des sommes de quatre carrés dans \( \Z \) est multiplicativement stable (\cf.~\cref{lem:norme des quaternions}), \cref{thm:Lagrange;somme des quatre carrés} se ramène à la ---
    \begin{proposition}
        Tout nombre premier \( p \) est somme de quatre carrés.
    \end{proposition}
    \begin{proof}
        Comme \( 2 = 1^2+1^2+0^2+0^2 \), on peut supposer \( p \) impair. Comme \( p \) est permutable à tout quaternion, l'idéal à gauche \( \H p \) est bilatère ; on peut donc former l'anneau quotient \( \H / \H p \). Comme \( p \) est impair, tout \( z \in \H \) est congru modulo \( \H p \) à un élément de \( \H(\Z) \) (si les composantes de \( z \) sont toutes dans \( \frac{1}{2}+\Z \)\,, on forme \( z + p \cdot \frac{1}{2}(1+\i+\j+\k) \)) ; donc \( \H / \H p \) est isomorphe au quotient correspondant de \( \H(\Z) \), c'est-à-dire à~\( \H(\F_p) \).

        Or, comme la forme \( a^2+b^2+c^2+d^2 \) représente \( 0 \) dans \( \F_p \) (\cref{thm:Chevalley;corps fini} ; voir en remarque ci-dessous une démonstration directe), \( \H(\F_p) \) admet des éléments non nuls de norme réduite nulle ; un tel élément n'est pas inversible (\cref{lem:norme des quaternions}, \ref{lem:norme des quaternions;item2}), donc engendre un idéal à gauche non trivial. En revenant à \( \H \)\,, on voit que \( \H p \) est contenu dans un idéal à gauche \( \H z \) distinct de \( \H \) et de \( \H p \). Par conséquent on a \( p = z' z \) avec \( z,z' \in \H \) non inversibles. Alors \( p^2 = \Norm(p) = \Norm(z)\Norm(z') \), et, comme \( \Norm(z) \) et \( \Norm(z') \) sont des entiers \( > 1 \) (\cref{lem:quaternions d'Hurwitz}, \ref{lem:quaternions d'Hurwitz;item2} et \ref{lem:quaternions d'Hurwitz;item3}), on a \( \Norm(z) = \Norm(z') = p \).

        Posons \( z = a+b\i+c\j+d\k \) (\( a,b,c,d \in \Z \) ou \( \in \frac{1}{2}+\Z \)). Si \( a,b,c,d \in \Z \)\,, on a \( p = \Norm(z) = a^2+b^2+c^2+d^2 \), et on a gagné. Reste à montrer que, si \( a,b,c,d \in \frac{1}{2}+\Z \)\,, on peut se ramener au cas précédent en multipliant \( z \) par un élément de norme réduite \( 1 \) de \( \H \)\,, plus précisément par un élément de la forme \( \frac{1}{2}(\pm 1 \pm \i \pm \j \pm \k) \). En effet considérons la classe \( \eta \) de \( 2z \) dans \( \H(\Z)/4\H(\Z) \simeq \H(\Z/4\Z) \) ; comme \( \Norm(z) \in \Z \)\,, on a \( \Norm(2z) \in 4z \), d'où \( \Norm(\eta) = 0 \) et \( \eta \conj{\eta} = 0 \) ; \( \conj{\eta} \) est la classe d'un quaternion \( z' \) de la forme \( \pm 1 \pm \i \pm \j \pm \k \) ; alors \( u = \frac{1}{2}z' \in \H \)\,, \( u \) est de norme réduite \( 1 \), et, comme la classe de \( (2z)\cdot(2u) \) est nulle modulo \( 4 \), on a \( zu \in \H(\Z) \). Comme \( p = \Norm(z) = \Norm(zu) \), notre assertion est démontrée.
    \end{proof}

    \begin{remark}
        Voici une démonstration très élémentaire du fait que, sur un corps fini \( K \), la forme quadratique \( a^2+b^2+c^2+d^2 \) représente \( 0 \) (\ie. a un zéro non trivial dans \( K^4 \)). En prenant \( c = 1 \), \( d = 0 \), il suffit de montrer que l'équation \( a^2+b^2+1 = 0 \) a une solution dans \( K^2 \). Écrivons la \( b^2+1 = -a^2 \). En caractéristique \( 2 \), on peut prendre \( b = 0 \) et \( a = 1 \). Sinon, si \( q \) est le cardinal de \( K \), il y a \( \frac{q+1}{2} \) carrés dans \( K \) (\( 0 \) et les \( \frac{q-1}{2} \) carrés non nuls) ; donc l'ensemble \( T \) (resp. \( T' \)) des éléments de \( K \) de la forme \( b^2+1 \) avec \( b \in K \) (resp. de la forme \( -a^2 \) avec \( a \in K \)) a \( \frac{q+1}{2} \) éléments par translation (resp. symétrie). Comme \( \frac{q+1}{2} + \frac{q+1}{2} > q \), on a \( T \cap T' \neq \emptyset \), ce qui signifie que \( b^2+1 = -a^2 \) a une solution.
        \pushQED{\qed}\qedhere
    \end{remark}


\chapter{Extensions galoisiennes des corps de nombres}

\section{Théorie de Galois}

    Cette section est un complément à la théorie générale des corps commutatifs, \cf.~\cref{sec:éléments algébriques sur un corps;extensions algébriques}, \cref{sec:éléments conjugués;corps conjugués}, \cref{sec:normes et traces} et \cref{sec:discriminant}. Étant donnés un corps \( L \) et un ensemble \( G \) d'automorphismes de \( L \), l'ensemble des \( x \in L \) tels que \( \sigma(x) = x \) pour tout \( \sigma \in G \) est, comme on le voit aussitôt, un \emph{sous-corps} de \( L \), qu'on appelle le \emph{corps des invariants} de \( G \). D'autre part, étant donnée une extension \( L \) d'un corps \( K \), l'ensemble des \( K \)‑automorphismes de \( L \) est un \emph{groupe} pour la composition des applications.

    \begin{theorem}\label{thm:extension galoisienne}
        Soit \( L \) une extension de degré fini \( n \) d'un corps \( K \) fini ou de caractéristique \( 0 \). Les conditions suivantes sont équivalentes :
        \begin{enumerate}
            \item \label{thm:extension galoisienne;cond1} \( K \) est le corps des invariants du groupe \( G \) des \( K \)‑automorphismes de \( L \) ;
            \item \label{thm:extension galoisienne;cond2} pour tout \( x \in L \), le polynôme minimal de \( x \) sur \( K \) a toutes ses racines dans \( L \) ;
            \item \label{thm:extension galoisienne;cond3} \( L \) est engendrée par les racines d'un polynôme sur \( K \).
        \end{enumerate}
        Sous ces conditions, le groupe \( G \) des \( K \)‑automorphismes de \( L \) a \( n \) éléments.
    \end{theorem}
    \begin{proof}
        Montrons que \ref{thm:extension galoisienne;cond1} implique \ref{thm:extension galoisienne;cond2}. En effet, pour \( x \in L \), le polynôme \( \prod_{\sigma \in G} (X-\sigma(x)) \) est invariant par \( G \) (car tout \( \tau \in G \) permute entre eux ses facteurs)\footnote{La finitude de \( G \) résulte \cref[de]{thm:nombre de prolongements et degré}.}. Donc ses coefficients appartiennent à \( K \). Comme il admet \( x \) pour racine (\( 1 \in G \)), c'est un multiple du polynôme minimal de \( x \) sur \( K \) (\cref{sec:éléments algébriques sur un corps;extensions algébriques}, \ref{élément algébrique;item5}). D'où~\ref{thm:extension galoisienne;cond2}.

        Pour voir que \ref{thm:extension galoisienne;cond2} implique \ref{thm:extension galoisienne;cond3}, on prend un élément primitif \( x \) de~\( L \) sur~\( K \) (\cref{cor:théorème de l'élément primitif} \cref[de]{thm:nombre de prolongements et degré}). Son polynôme minimal sur \( K \) a toutes ses racines dans \( L \) par \ref{thm:extension galoisienne;cond2}, et celles-ci engendrent évidemment \( L \) sur~\( K \).

        Prouvons enfin que \ref{thm:extension galoisienne;cond3} implique \ref{thm:extension galoisienne;cond1}. Par hypothèse \( L \) est engendrée sur \( K \) par un nombre fini d'éléments \( (x^{(1)}, \cdots, x^{(q)}) \) et par tous leurs conjugués \( (x_{\,j}^{(i)}) \) (\cref{sec:éléments conjugués;corps conjugués}). Alors tout \( K \)‑isomorphisme \( \sigma \) de \( L \) dans une extension de~\( L \) envoie chacun de ces générateurs sur un autre ; on a donc \( \sigma(L) \subset L \), d'où \( \sigma(L) = L \) par l'algèbre linéaire car \( \sigma \) est une application \( K \)‑linéaire injective ; autrement dit \( \sigma \) est un \emph{\( K \)‑automorphisme} de~\( L \). Ainsi le groupe \( G \) des \( K \)‑automorphismes de~\( L \) a \( n \) éléments (par \cref{thm:nombre de prolongements et degré} et \cref{cor:théorème de l'élément primitif}). Soit alors \( x \in L \) invariant par \( G \) ; alors tout \( \sigma \in G \) est un \( K[x] \)‑automorphisme de \( L \) ; or (\cref{sec:éléments conjugués;corps conjugués}) il y a exactement \( [ L : K[x] ] \) \( K[x] \)‑isomorphismes de \( L \) dans une extension de \( L \) ; on a donc \( n \leqslant [ L : K[x] ] \), d'où \( n = [ L : K[x] ] \), \( K[x] = K \) et \( x \in K \). Ceci démontre~\ref{thm:extension galoisienne;cond1}. L'assertion \( \card(G) = n \) a été démontrée en cours de~route.
    \end{proof}

    \begin{definition}[extension galoisienne ; groupe de Galois ; extension abélienne / cyclique]\index{cyclique (extension)}\index{extension!abelienne@abélienne}\index{extension!cyclique}\index{extension!galoisienne}
        Si les conditions \cref[de]{thm:extension galoisienne}\index{galoisienne (extension)} sont satisfaites, on dit que \( L \) est une \emph{extension galoisienne} de \( K \), et que \( G \) est le \emph{groupe de Galois}\index{groupe de Galois}\index{Galois (groupe de)} de~\( L \) sur~\( K \). Si \( G \) est abélien (resp. \emph{cyclique}) on dit que \( L \) est une \emph{extension abélienne}\index{abelienne (extension)@abélienne (extension)} (resp. \emph{cyclique}) de~\( K \).
    \end{definition}

    \begin{corollary}[{\cref[de]{thm:extension galoisienne}}]\label{cor:extension galoisienne}
        Soient \( K \) un corps fini ou de caractéristique \( 0 \), \( L \) une extension de degré fini \( n \) de \( K \), et~\( H \) un groupe d'automorphismes de \( L \) admettant \( K \) pour corps d'invariants. Alors \( L \) est extension galoisienne de \( K \), et \( H \) est son groupe de Galois.
    \end{corollary}
    \begin{proof}
        En effet, pour \( x \in L \), le polynôme \( \prod_{\sigma \in H} (X-\sigma(x)) \) est invariant par \( H \), donc a ses coefficients dans \( K \), et est par conséquant multiple du polynôme minimal de \( x \) sur \( K \) ; ainsi, par \cref{thm:extension galoisienne}, \ref{thm:extension galoisienne;cond2}, \( L \) est extension galoisienne de \( K \). Si \( G \) désigne son groupe de Galois, on a \( H \subset G \) et \( \card(G) = n \) (\cref{thm:extension galoisienne}). Prenons alors un élément primitif \( x' \) de~\( L \) sur~\( K \) (\cref{cor:théorème de l'élément primitif} \cref[de]{thm:nombre de prolongements et degré}) et considérons le polynôme \( P(X) = \prod_{\sigma \in H} (X-\sigma(x')) \) ; comme plus haut il a ses coefficients dans~\( K \) et est multiple du polynôme minimal de \( x' \) sur \( K \) ; d'où \( n \leqslant \deg(P) \). Comme \( \deg(P) = \card(H) \leqslant \card(G) = n \), on en déduit \( H = G \).
    \end{proof}

    \begin{theorem}\label{thm:correspondance de Galois}
        Soient \( K \) un corps fini ou de caractéristique \( 0 \), \( L \) une extension galoisienne de \( K \), et \( G \) son groupe de Galois. À tout sous-groupe \( G' \) de \( G \) associons le corps des invariants \( k(G') \) de \( G' \), et à tout sous-corps \( K' \) de \( L \) contenant \( K \) associons le sous-groupe \( g(K') \subset G \) des \( K' \)‑automorphismes de~\( L \).
        \begin{enumerate}
            \item \label{thm:correspondance de Galois;item1} Les applications \( g \) et \( k \) sont des bijections réciproques l'une de l'autre, décroissantes pour les relations d'inclusion. De plus \( L \) est l'extension galoisienne de tout corps intermédiaire \( K' \) (\ie. \( K \subset K' \subset L \)).
            \item \label{thm:correspondance de Galois;item2} Pour qu'un corps intermédiaire \( K' \) soit extension galoisienne de \( K \), il faut et il suffit que \( g(K') \) soit un sous-groupe invariant de \( G \) ; alors le groupe de Galois de \( K' \) sur \( K \) s'identifie au groupe quotient \( G/g(K') \).
        \end{enumerate}
    \end{theorem}
    \begin{proof}
        En effet, pour tout corps intermédiaire \( K' \) et tout \( x \in L \), le polynôme minimal de \( x \) sur \( K' \) divise le polynôme minimal de \( x \) sur \( K \) ; il a donc toutes ses racines dans \( L \) par \cref{thm:extension galoisienne}, \ref{thm:extension galoisienne;cond2}, de sorte que \( L \) est extension galoisienne de \( K' \) par \cref{thm:extension galoisienne}, \ref{thm:extension galoisienne;cond2} encore. Ainsi \( K' \) est le corps des invariants du groupe \( g(K') \) des \( K' \)‑automorphismes de \( L \) (\cref{thm:extension galoisienne}, \ref{thm:extension galoisienne;cond1}) ; autrement dit \( k(g(K')) = K' \). Soit maintenant \( G' \) un sous-groupe de \( G \) ; alors \( G' \) est le groupe de Galois de \( L \) sur \( k(G') \) (\cref{cor:extension galoisienne}) ; autrement dit on a \( G' = g(k(G')) \). Les formules \( k(g(K')) = K' \) et \( g(k(G')) = G' \) montrent que \( k \) et \( g \) sont des bijections réciproques l'une de l'autre. Leur décroissance est évidente. Ceci démontre~\ref{thm:correspondance de Galois;item1}.

        Prouvons maintenant \ref{thm:correspondance de Galois;item2}. Soient \( K' \) un corps intermédiaire (\( K \subset K' \subset L \)). Pour \( x \in K \), les racines du polynôme minimal de \( x \) sur \( K \) sont les \( \sigma(x) \) (\( x \in G \)) ; d'après \cref{thm:extension galoisienne}, \ref{thm:extension galoisienne;cond2}, pour que \( K' \) soit extension galoisienne de \( K \), il faut et il suffit que \( \sigma(x) \in K' \) pour tout \( x \in K' \) et tout \( \sigma \in G \), c'est-à-dire que \( \sigma(K') \subset K' \) pour tout \( \sigma \in G \). Or, si \( \sigma(K') \subset K' \), si \( \tau \in g(K') \) et si \( x \in K' \), on a \( \sigma^{-1} \tau \sigma(x) = \sigma^{-1} \sigma(x) = x \), d'où \( \sigma^{-1} \tau \sigma \in g(K') \) ; autrement dit \textquote{\( K' \) galoisienne sur~\( K \)} implique \textquote{\( g(K') \) invariant dans~\( G \)}. Inversement, supposons \( g(K') \) invariant dans \( G \) ; si \( x \in K' \), si \( \sigma \in G \) et si \( \tau \in g(K') \), on a \( \tau \sigma(x) = \sigma \cdot \sigma^{-1} \tau \sigma(x) = \sigma(x) \) car \( \sigma^{-1} \tau \sigma \in g(K') \) et car \( x \in K' \) ; ainsi \( \sigma(x) \) est invariant par tout élément \( \tau \) de \( g(K') \), de sorte que \( \sigma(x) \in K' \) ; par conséquent \textquote{\( g(K') \) invariant dans \( G \)} implique \textquote{\( \sigma(K') \subset K' \)}, et donc que \( K' \) est galoisienne sur~\( K \).

        Déterminons enfin dans ce cas, le groupe de Galois de \( K' \) sur \( K \). Comme on a \( \sigma(K') \subset K' \) pour tout \( \sigma \in G \) (et même \( \sigma(K') = K' \) par l'algèbre linéaire), la restriction \( \restrict{\sigma}{K'} \) de \( \sigma \) à \( K' \) est un \( K \)‑automorphisme de \( K' \). On a alors un \textquote{homomorphisme de restriction} \( \sigma \mapsto \restrict{\sigma}{K'} \) de \( G \) dans le groupe de Galois \( H \) de \( K' \) sur \( K \) ; son noyau est évidemment \( g(K') \). Comme on a
        \begin{align*}
            \card(H) = [K':K] = [L:K][L:K']^{-1} &= \card(G) \cdot \card(g(K'))^{-1}\\
            &= \card(G/g(K')),
        \end{align*}
        cet homomorphisme est surjectif, et \( H \simeq G/g(K') \).
    \end{proof}

\smallskip
    \begin{example*}[extensions quadratiques]\index{extension!quadratique}\index{quadratique (extension)}
        Soient \( K \) un corps de caractéristique \( 0 \), et \( L \) une extension quadratique (\ie. de degré \( 2 \)) de \( K \). Comme au début du \cref{sec:entiers des corps quadratiques} on voit que \( L \) est de la forme \( K[x] \) où \( x \) est racine d'un polynôme \( X^2-d \) (\( d \in K \), \( d \) non carré dans \( K \)). Comme l'autre racine de ce polynôme est \( -x \), \( K \) admet un \( K \)‑automorphisme non trivial \( \sigma \) défini par \( \sigma(x) = -x \), \ie.,
        \begin{equation}
            \sigma(a+bx) = a-bx \quad \text{(\( a,b \in K \))}.
        \end{equation}
        On a \( \sigma^2 = 1 \) et \( K \) est le corps des invariants de \( \sigma \). Donc \( L \) est extension \emph{galoisienne} de \( K \), avec le groupe \emph{cyclique} \( \set{1, \sigma} \) pour groupe de Galois (\cref{thm:extension galoisienne} et \cref{cor:extension galoisienne}).
    \end{example*}

    \begin{example*}[extensions cyclotomiques]\label{ex:extensions cyclotomiques}\index{cyclotomique (extension)}\index{extension!cyclotomique}
        Soient \( K \) un corps de caractéristique \( 0 \), \( z \) une racine primitive \( n \)‑ième de l'unité dans une extension de \( K \), et \( L = K(z) \) ; on dit alors que \( L \) est une extension \emph{cyclotomique} de~\( K \). Le polynôme minimal \( F(X) \) de \( z \) sur \( K \) divise \( X^n - 1 \) (\cref{sec:éléments algébriques sur un corps;extensions algébriques}, \ref{élément algébrique;item5}), donc ses racines sont des racines \( n \)‑ièmes de l'unité et donc des puissances de \( z \) (\cref{sec:racine de l'unité dans un corps}). Ainsi \( L \) est extension \emph{galoisienne} de \( K \) par \cref{thm:extension galoisienne}, \ref{thm:extension galoisienne;cond3}.

        Soit \( G \) son groupe de Galois ; tout \( \sigma \in G \) est déterminé par \( \sigma(z) \), qui est une puissance \( z^{j(\sigma)} \) de \( z \) où \( j(\sigma) \) est bien déterminé modulo \( n \). Pour \( \sigma, \tau \in G \) on a \( \sigma\tau(z) = \sigma(z^{j(\tau)}) = \sigma(z)^{j(\tau)} = z^{j(\sigma)j(\tau)} \), d'où \( j(\sigma\tau) \equiv j(\sigma)j(\tau) \pmod*{n} \). Autrement dit, on peut considérer \( \sigma \mapsto j(\sigma) \) comme un \emph{homomorphisme} \( G \to (\Z/n\Z)^\times \). Comme \( j(\sigma) \) détermine \( \sigma \) de façon unique, cet homomorphisme est \emph{injectif} et \( G \) est abélien. Ainsi \emph{toute extension cyclotomique est abélienne}. Si \( n \) est premier, cette extension est même \emph{cyclique}, car \( G \) est isomorphe à un sous-groupe de \( (\Z/n\Z)^\times = \F_n^\times \) (\cref{thm:corps fini}, \ref{thm:corps fini;item2}).
        \begin{quote}
            Comme tout sous-groupe d'un groupe abélien est invariant, tout corps intermédiaire \( K' \) d'une extension cyclotomique \( L \) de \( K \) est extension galoisienne (et même abélienne) de~\( K \) (\cref{thm:correspondance de Galois}, \ref{thm:correspondance de Galois;item2}). En particulier tout sous-corps d'un corps cyclotomique est extension abélienne de~\( \Q \). Réciproquement on démontre (théorème de Kronecker-Weber) que toute extension abélienne de~\( \Q \) est un sous-corps d'un corps cyclotomique.
        \end{quote}
        On notera que, avec les notations précédentes, l'automorphisme \( \sigma \) \emph{élève toutes les racines \( n \)‑ièmes de l'unité à la puissance \( j(\sigma) \)}, car celles-ci sont des puissances de \( z \). Ainsi \( \sigma \mapsto j(\sigma) \) est indépendant du choix de~\( z \).
    \end{example*}

    \begin{example*}[corps finis]\label{ex:galois;corps finis}
        Soit \( \F_q \) un corps fini (\( q = p^s \) avec \( p \) premier) ; toute extension de degré fini de \( \F_q \) est de la forme \( \F_{q^n} \) ; son degré est \( n \) (\cref{sec:corps finis}). Or on sait que \( \sigma \colon x \mapsto x^q \) est un automorphisme de \( \F_{q^n} \) (\cref{prop:(x+y)^p = x^p+y^p}). Pour tout \( x \in \F_{q^n} \)\,, on a \( \sigma^j(x) = x^{q^j} \), d'où \( \sigma^n = 1 \) car \( \F_{q^n} \) est l'ensemble des \( x \) tels que \( x^{q^n} = x \) (\cref{thm:corps fini}, \cref{thm:corps fini;item3}). D'autre part, pour \( 1 \leqslant j \leqslant n-1 \), on a \( \sigma^j \neq 1 \) car \( \F_{q^j} \neq \F_{q^n}. \) Donc \( \set{1, \sigma, \cdots, \sigma^{n-1}} \) est un groupe cyclique d'ordre \( n \). Ainsi, d'après \cref{cor:extension galoisienne} \cref[à]{thm:extension galoisienne}, \( \F_{q^n} \) est une extension cyclique de degré \( n \) de \( \F_q \)\,, et son groupe de Galois a un générateur privilégié, à savoir \( x \mapsto x^q \), qu'on appelle l'\emph{automorphisme de Frobenius}\index{automorphisme de\IndexLinebreak Frobenius}\index{Frobenius (automorphisme\IndexLinebreak de)}.
    \end{example*}

\section{Groupe de décomposition et groupe d'inertie}\label{sec:groupe de décomposition et groupe d'inertie}

    Dans cette section, \( A \) désigne un anneau de Dedekind, \( K \) son corps des fractions qu'on suppose de caractéristique \( 0 \), \( K' \) une extension galoisienne de \( K \), \( n \) son degré, \( G \) son groupe de Galois, et \( A' \) la fermeture intégrale de \( A \) dans~\( K' \).

    \medskip
    En appliquant \( \sigma \in G \) à une équation de dépendance intégrale (sur \( A \)) d'un élément \( x \in A' \), on voit que \( \sigma(x) \in A' \) ; donc :
    \begin{equation}\label{eq:la fermeture intégrale est stable par le groupe de Galois}
        \text{\itshape \( A' \) est stable par \( G \), \ie, \( \sigma(A') = A' \) pour tout \( \sigma \in G \)}.
    \end{equation}

    \begin{quote}
        En fait on a seulement démontré \( \sigma(A') \subset A' \) ; amis on a alors aussi \( \sigma^{-1}(A') \subset A' \) d'où \( A' = \sigma \sigma^{-1}(A) \subset \sigma(A) \). Nous omettrons souvent dans la suite ce facile complément de raisonnement.
    \end{quote}

    D'autre part, si \( \ideal{p} \) est un idéal maximal de \( A \) et \( \ideal{p}' \) un idéal maximal de \( A' \) tel que \( \ideal{p}' \cap A = \ideal{p} \) (c'est-à-dire figurant dans la décomposition de \( A' \ideal{p} \) en idéaux premiers ; \cf.~\cref{prop:facteurs premiers dans la décomposition d'idéal dans une extension}), on a évidemment \( \sigma(\ideal{p}') \cap A = \ideal{p} \), et \( \sigma(\ideal{p}') \) figure dans la décomposition de \( A'\ideal{p} \) avec le \emph{même} exposant que \( \ideal{p}' \). Nous dirons que \( \ideal{p}' \) et \( \sigma(\ideal{p}') \) sont des idéaux premiers \emph{conjugués}\index{conjugues (ideaux premiers)@conjugués (idéaux premiers)} de \( A' \). Nous allons montrer qu'il n'y en a pas d'autres dans la décomposition de \( A'\ideal{p} \) :
    \begin{proposition}\label{prop:décomposition dans une extension;conugués}
        Si \( \ideal{p} \) est un idéal maximal de \( A \), les idéaux maximaux \( \ideal{p_i}' \) de \( A' \) figurant dan la décomposition de \( A'\ideal{p} \) (\ie. tels que \( \ideal{p_i}' \cap A = \ideal{p} \)) sont deux à deux conjugués, et ont le même degré résiduel \( f \) et le même indice de ramification \( e \) ; ainsi \( A'\ideal{p} = \left(\prod_{i=1}^g \ideal{p_i}'\right)^e \), et~\( n = efg \).
    \end{proposition}
    \begin{proof}
        L'assertion sur l'indice de ramification et le degré résiduel est évidente, car un automorphisme \( \sigma \) conserve \emph{toutes} les relations algébriques. La formule \( n = efg \) est alors un cas particulier de \( \sum_{e_i f_i} = n \) (\cref{thm:sum e f = n}). Ceci étant, soit \( \ideal{p}' \) l'un des~\( \ideal{p_i}' \)\,, et supposons qu'un autre des \( \ideal{p_i}' \)\,, que nous noterons \( \ideal{q}' \), ne soit pas conjugué de~\( \ideal{p}' \). Comme \( \ideal{q}' \) et \( \sigma(\ideal{p}') \) (\( \sigma \in G \)) sont maximaux et distincts, on a \( \sigma(\ideal{p}') \subset \ideal{q}' \). Or on a \namecref{lem:évitement des idéaux premiers} suivant :
        \begin{lemma}[lemme d'évitement des idéaux premiers]\label{lem:évitement des idéaux premiers}
            Soient \( R \) un anneau, \( \ideal{p}_1, \cdots, \ideal{p}_q \) une famille finie d'idéaux premiers de \( R \), et \( \ideal{b} \) un idéal de \( R \) tel que \( \ideal{b} \not\subset \ideal{p_i} \) pour tout \( i \). Alors il existe \( b \in \ideal{b} \) tel que \( b \notin \ideal{p_i} \) pour tout~\( i \).
        \end{lemma}
        \begin{proof}
            En effet, en supprimant les \( \ideal{p_i} \) non maximaux dans \( \set{\ideal{p_1}, \cdots, \ideal{p_q}} \), on peut supposer qu'on a \( \ideal{p_j} \not\subset \ideal{p_i} \) pour tout \( i \neq j \) ; soit alors \( x_{ij} \in \ideal{p_j} \) tel que \( x_{ij \notin \ideal{p_i}} \). D'autre part, comme \( \ideal{p} \not\subset \ideal{p_i} \)\,, il existe \( a_i \in \ideal{b} \) tel que \( a_i \notin \ideal{p_i} \). Posons alors \( b_i = a_i \prod_{j \neq i} x_{ij} \) ; on a \( b_i \in \ideal{b} \), \( b_i \in \ideal{p_j} \) pour \( j \neq i \), et \( b_i \notin \ideal{p_i} \) car \( \ideal{p_i} \) est premier. Alors \( b = b_1 + \cdots + b_q \) répond à la question, car \( b \in \ideal{b} \) et que, pour tout \( i \), on a \( \sum_{j \neq i} b_j \in \ideal{p_i} \)\,, \( b_i \notin \ideal{p_i} \) et donc \( b \notin \ideal{p_i} \).
        \end{proof}
        Ceci étant, \namecref{lem:évitement des idéaux premiers} montre qu'on a un élément \( x \in \ideal{q}' \) tel que \( x \notin \sigma(\ideal{p}') \) pour tout \( \sigma \in G \). Considérons alors \( \Norm(x) = \prod_{\tau \in G} \tau(x) \) (\cref{prop:trace;norme;polynôme caractéristique;extension}) ; comme \( \tau(x) \in A' \) pour tout \( \tau \in G \) (par \eqref{eq:la fermeture intégrale est stable par le groupe de Galois}) on a \( \Norm(x) \in \ideal{q}' \), d'où \( \Norm(x) \in \ideal{q}' \cap A = \ideal{p} \) ; d'autre part on a \( x \notin \tau^{-1}(\ideal{p}') \), d'où \( \tau(x) \notin \ideal{p}' \) pour tout \( \tau \in G \) ; comme \( \ideal{p}' \) est premier, on en déduit \( \Norm(x) \notin \ideal{p}' \), ce qui contredit \( \Norm(x) \in \ideal{p} \).
    \end{proof}

    Ceci étant, soit \( \ideal{p}' \) l'un des idéaux maximaux de \( A' \) tels que \( \ideal{p}' \cap A = \ideal{p} \). Les \( \sigma \in G \) tels que \( \sigma(\ideal{p}') = \ideal{p}' \) forment un sous-groupe \( D \) de \( G \), qu'on appelle le \emph{groupe de décomposition}\index{decomposition (groupe de)@décomposition (groupe de)}\index{groupe de decomposition@groupe de décomposition} de \( \ideal{p}' \). Si \( g \) est le nombre des conjugués de \( \ideal{p}' \) on a donc
    \begin{equation}\label{eq:card D = ef}
        g = \card(G) \cdot \card(D)^{-1}
        \quad \text{ ou } \quad
        \card(D) = n/g = ef.
    \end{equation}

    Pour \( \sigma \in D \), les relations \( \sigma(A') = A' \) et \( \sigma(\ideal{p}') = \ideal{p}' \) montrent que \( \sigma \) définit, par passage au quotient, un automorphisme \( \overline{\sigma} \) de \( A'/\ideal{p}' \) (en effet \( x \equiv y \pmod*{\ideal{p}'} \) entraîne \( \sigma(x) \equiv \sigma(y) \pmod*{\ideal{p}'} \)). Il est clair que \( \overline{\sigma} \) est un \( (A/\ideal{p}) \)‑automorphisme. L'application \( \sigma \mapsto \overline{\sigma} \) est un \emph{homomorphisme} de groupes, dont le \emph{noyau} est l'ensemble \( I \) des \( \sigma \in D \) tels que \( \sigma(x) - x \in \ideal{p}' \) pour tout \( x \in A' \) ; ainsi \( I \) est un \emph{sous-groupe invariant} de \( D \), qu'on appelle le \emph{groupe d'inertie}\index{groupe d'inertie}\index{inertie (groupe d')} de~\( \ideal{p}' \).

    \begin{proposition}\label{prop:ramification et groupe d'inertie}
        Avec les mêmes notations, on suppose \( A/\ideal{p} \) fini ou de caractéristique \( 0 \). Alors \( A'/\ideal{p}' \) est extension galoisienne de degré \( f \) de \( A/\ideal{p} \), et \( \sigma \mapsto \overline{\sigma} \) est un homomorphisme surjectif de \( D \) sur son groupe de Galois. De plus \( \card(I) = e \).
    \end{proposition}
    \begin{proof}
        En effet soient \( K_{D} \) le corps des invariants de \( D \), \( A_D = A' \cap K_D \) la fermeture intégrale de \( A \) dans \( K_D \) et \( \ideal{p}_D \) l'idéal premier \( \ideal{p}' \cap A_D \). D'après \cref{prop:décomposition dans une extension;conugués} et la définition de \( D \), \( \ideal{p}' \) est le seul facteur premier de \( A'\ideal{p}_D \) ; posons \( A' \ideal{p}_D = \ideal{p}'\,{}^{e'} \), et notons \( f' \) le degré résiduel \( [ A'/\ideal{p}' : A_D/\ideal{p}_D ] \). D'après \cref{thm:sum e f = n,thm:correspondance de Galois} et \eqref{eq:card D = ef}, on a
        \[
            e'f' = [K':K_D] = \card(D) = ef.
        \]
        Comme \( A/\ideal{p} \subset A_D/\ideal{p}_D \subset A'/\ideal{p}' \) on a \( f' \leqslant f \) ; comme \( \ideal{p}A_D \subset \ideal{p}_D \)\,, on a \( e' \leqslant e \) ; joint à \( e'f' = ef \), ceci montre qu'on a \( e = e' \) et \( f = f' \), d'où :
        \begin{equation}\label{eq:isomorphe entre les quotients;normal et celui avec D}
            A/\ideal{p} \simeq A_D/\ideal{p}_D.
        \end{equation}
        Ceci étant, soit \( \overline{x} \) un élément primitif de \( A'/\ideal{p}' \) sur \( A/\ideal{p} \), et soit \( x \in A' \) un représentant de \( \overline{x} \). Soit \( X^r + a_{r-1} X^{r-1} + \cdots + a_0 \) le polynôme minimal de \( x \) sur \( K_D \) ; on a \( a_i \in A_D \) (\cref{cor:trace et norme d'élément entier} \cref[de]{prop:trace et norme d'élément entier}) ; l'ensemble de ses racines est celui des \( \sigma(x) \) avec \( \sigma \in D \). Le polynôme \textquote{réduit} \( X^r + \overline{a_{r-1}} X^{r-1} + \cdots + \overline{a_0} \) a ses coefficients dans \( A/\ideal{p} \) (par \eqref{eq:isomorphe entre les quotients;normal et celui avec D}) et l'ensemble de ses racines et celui des \( \overline{\sigma}(\overline{x}) \) avec \( \sigma \in D \). Il en résulte d'abord que \( A'/\ideal{p}' \) contient tous les conjugués de \( \overline{x} \) sur \( A/\ideal{p} \), et \( A'/\ideal{p}' \) est donc extension \emph{galoisienne} de \( A/\ideal{p} \) (\cref{thm:extension galoisienne}, \ref{thm:extension galoisienne;cond3}). Il en résulte d'autre part que, comme tout conjugué de \( \overline{x} \) sur \( A/\ideal{p} \) est un \( \overline{\sigma}(\overline{x}) \), tout \( (A/\ideal{p}) \)‑automorphisme de \( A'/\ideal{p}' \) est un \( \overline{\sigma} \). Ainsi le groupe de Galois de \( A'/\ideal{p}' \) sur \( A/\ideal{p} \) s'identifie à \( D/I \) ; comme son ordre est \( [A'/\ideal{p}':A/\ideal{p}] = f \), on a \( \card(D)/\card(I) = f \), d'où \( \card(I) = e \) d'après~\eqref{eq:card D = ef}.
    \end{proof}

    \begin{corollary}\label{cor:ramification et groupe d'inertie}
        Pour que \( \ideal{p} \) ne se ramifie pas dans \( A' \), il faut et il suffit que le groupe d'inertie \( I \) soit réduit à l'identité.
    \end{corollary}

    \begin{remark}
        Si on note \( D_{\ideal{p}'} \) et \( I_{\ideal{p}'} \) les groupes de décomposition et d'inertie d'idéal maximal \( \ideal{p}' \), ceux de \emph{conjugué} \( \sigma(\ideal{p}') \) sont
        \begin{equation}
            D_{\sigma(\ideal{p}')} = \sigma D_{\ideal{p}'} \sigma^{-1}, \quad
            I_{\sigma(\ideal{p}')} = \sigma I_{\ideal{p}'} \sigma^{-1}.
        \end{equation}
        En effet, pour \( \tau \in D_{\ideal{p}'} \)\,, on a \( \sigma \tau \sigma^{-1} \cdot \sigma(\ideal{p}') = \sigma\tau(\ideal{p}') = \sigma(\ideal{p}') \), d'où \( \sigma D_{\ideal{p}'} \sigma^{-1} \subset D_{\sigma(\ideal{p}')} \) ; d'autre part, pour \( \tau' \in D_{\sigma(\ideal{p}')} \)\,, on a \( (\sigma^{-1} \tau' \sigma)(\ideal{p}') = \sigma^{-1} \tau' (\sigma(\ideal{p}')) = \sigma^{-1} \sigma(\ideal{p}') = \ideal{p}' \), d'où \( \sigma^{-1} D_{\sigma(\ideal{p}')} \sigma \subset D_{\ideal{p}'} \) et l'inclusion opposée s'ensuit. De même, pour \( \tau \in I_{\ideal{p}'} \) et \( x \in A' \), on a \( \sigma\tau\sigma^{-1}(x) - x = \sigma\tau(\sigma^{-1}(x)) - \sigma\sigma^{-1}(x) = \sigma(\tau(\sigma^{-1}(x))-\sigma^{-1}(x)) \in \sigma(\ideal{p}') \), d'où \( \sigma I_{\ideal{p}'} \sigma^{-1} \subset I_{\sigma(\ideal{p}')} \) ; pour l'inclusion opposée, soit \( \tau' \in I_{\sigma(\ideal{p}')} \), on a alors \( \sigma^{-1} \tau' \sigma(x) - x = \sigma^{-1}(\tau'(\sigma(x))-\sigma(x)) \in \sigma^{-1} \sigma(\ideal{p}') = \ideal{p}' \), d'où \( \sigma^{-1} \tau' \sigma \in I_{\ideal{p}'} \) et \( I_{\sigma(\ideal{p}')} \subset \sigma I_{\ideal{p}'} \sigma^{-1} \).

        Lorsque \( K' \) est une extension \emph{abélienne} de \( K \), les groupes \( D_{\sigma(\ideal{p}')} \) (resp. \( I_{\sigma(\ideal{p}')} \)) (\( \sigma \in G \)) sont donc tous \emph{égaux}, et ne dépendent que de l'idéal \( \ideal{p} \) du petit anneau (\cref{prop:décomposition dans une extension;conugués}).
    \end{remark}

\section{Cas des corps de nombres ; l'automorphisme de Frobenius}

    Ce qui précède s'applique aux corps de nombres et à leurs anneaux d'entiers ; en effet ces corps sont de caractéristique \( 0 \), et les corps résiduels de ces anneaux sont~finis.

    Conservons les notations précédentes (\( K \subset K' \) corps de nombres, \( K' \) galoisien sur \( K \), groupe \( G \), anneaux \( A \) et \( A' \)). Soit \( \ideal{p} \) un idéal maximal de \( A \) qui \emph{ne se ramifie pas} dans \( A' \), et soit \( \ideal{p}' \) un facteur premier de \( A'\ideal{p} \). Alors le groupe d'inertie de \( \ideal{p}' \) est réduit à l'identité (\cref{cor:ramification et groupe d'inertie} \cref[de]{prop:ramification et groupe d'inertie}), et son groupe de décomposition \( D \) est donc canoniquement isomorphe au groupe de Galois de \( A'/\ideal{p}' \) sur \( A/\ideal{p} \) (\cref{prop:ramification et groupe d'inertie}). Mais ce dernier est cyclique, avec un générateur privilégié \( \overline{\sigma} \colon x \mapsto x^q \) où \( q = \card(A/\ideal{p}) \) (\cref{ex:galois;corps finis}). Donc \( D \) est lui aussi \emph{cyclique}, avec un générateur privilégié \( \sigma \) tel que \( \sigma(x) \equiv x^q \pmod*{\ideal{p}'} \) pour tout \( x \in A' \). Ce générateur s'appelle encore l'\emph{automorphisme de Frobenius}\index{automorphisme de\IndexLinebreak Frobenius}\index{Frobenius (automorphisme\IndexLinebreak de)} de \( \ideal{p}' \) ; on le note souvent \( (\ideal{p}',K'/K) \).

    Pour \( \tau \in G \) on a, comme dans la remarque à la fin du \cref{sec:groupe de décomposition et groupe d'inertie},
    \begin{equation}
        (\tau(\ideal{p}'),K'/K) = \tau \cdot (\ideal{p}',K'/K) \cdot \tau^{-1}.
    \end{equation}
    En particulier, si \( K' \) est extension \emph{abélienne}, \( (\ideal{p}',K'/K) \) ne dépend que de l'idéal \( \ideal{p} \) de~\( A \) ; alors on le note parfois \( \legendre{K'/K}{\ideal{p}} \).

    \begin{proposition}\label{prop:Frobenius et corps intermédiaire}
        Avec les hypothèses et notations précédentes, soit \( F \) un corps intermédiaire \( (K \subset F \subset K') \) ; notons \( f \) le degré résiduel de \( \ideal{p}' \cap F \) sur \( K \). Alors
        \begin{enumerate}
            \item \label{prop:Frobenius et corps intermédiaire;item1} on a \( (\ideal{p}',K'/F) = (\ideal{p}',K'/K)^f \) ;
            \item \label{prop:Frobenius et corps intermédiaire;item2} si \( F \) est galoisien sur \( K \), la restriction de \( (\ideal{p}',K'/K) \) à \( F \) est égale à \( (\ideal{p}' \cap F, F/K) \).
        \end{enumerate}
    \end{proposition}
    \begin{proof}
        En effet, posons \( \sigma = (\ideal{p}',K'/K) \). Par définition on a \( \sigma(\ideal{p}') = \ideal{p}' \) et \( \sigma(x) \equiv x^q \pmod*{\ideal{p}'} \) pour tout \( x \in A' \) (ici \( q = \card(A/\ideal{p}) \)). On a donc \( \sigma^f(\ideal{p}') = \ideal{p}' \) et \( \sigma^f(x) \equiv x^{q^f} \pmod*{\ideal{p}'} \) pour tout \( x \in A' \) ; par définition de \( f \), \( q^f \) est le cardinal du corps résiduel \( (A' \cap F)/(\ideal{p}' \cap F) \). De plus le groupe de décomposition de \( \ideal{p}' \) sur \( F \) est évidemment un sous-groupe du groupe de décomposition \( D \) de \( \ideal{p}' \) sur \( K \), et est d'ordre
        \[
            [ A'/\ideal{p}' : (A' \cap F)/(\ideal{p}' \cap F) ] = f^{-1}[ A'/\ideal{p}' : A/\ideal{p} ] = f^{-1} \cdot \card(D)
        \]
        par \eqref{eq:card D = ef} ; comme \( D \) est cyclique et engendré par \( \sigma \), son seul sous-groupe d'ordre \( f^{-1} \cdot \card(D) \) est engendré par \( \sigma^f \). Ceci démontre~\ref{prop:Frobenius et corps intermédiaire;item1}.

        Supposons maintenant \( F \) galoisien sur \( K \), et notons \( \sigma' \) la restriction de \( \sigma \) à \( F \) (\cref{thm:extension galoisienne}, \ref{thm:extension galoisienne;cond2}). Comme \( \sigma(\ideal{p}') = \ideal{p}' \), on a \( \sigma(\ideal{p}' \cap F) = \ideal{p}' \cap F \) et \( \sigma' \) appartient au groupe de décomposition de \( \ideal{p}' \cap F \) sur \( K \). De plus on a évidemment \( \sigma'(x) \equiv x^q \pmod*{\ideal{p}' \cap F} \) pour tout \( x \in A' \cap F \), avec \( q = \card(A/\ideal{p}) \). Ceci démontre~\ref{prop:Frobenius et corps intermédiaire;item2}.
    \end{proof}

\section{Application aux corps cyclotomiques}\label{sec:application aux corps cyclotomiques}

    Nous allons utiliser ce qui précède pour démontrer un résultat qui généralise l'irréductibilité du polynôme cyclotomique, et en donne une troisième démonstration (\cf.~\cref{thm:polynôme cyclotomique est irréductible} et l'exemple du \cref{sec:décomposition d'un idéal premier dans une extension}).

    \begin{theorem}\label{thm:corps cyclotomique}
        Soit \( z \) une racine primitive \( n \)‑ième de l'unité dans \( \C \). Alors :
        \begin{enumerate}
            \item \label{thm:corps cyclotomique;item1} Aucun nombre premier \( p \) qui ne divise pas \( n \) ne se ramifie dans \( \Q[z] \) ;
            \item \label{thm:corps cyclotomique;item2} \( \Q[z] \) est extension abélienne de \( \Q \)\,, de degré \( \varphi(n) \) et de groupe de Galois isomorphe à \( (\Z/n\Z)^\times \).
        \end{enumerate}
    \end{theorem}
    \begin{proof}
        En effet soient \( F(X) \) le polynôme minimal de \( z \) sur \( \Q \)\,, et \( d \) son degré (on~a \( d = [\Q[z]:\Q] \)). Le polynôme \( F(X) \) est un diviseur de \( X^n-1 \), soit \( X^n-1 = F(X)G(X) \). On a \( \D(1, z, \cdots, z^{d-1}) = \pm \Norm(F'(z)) \) (la formule \eqref{eq:disc;cas particulier}) ; or de \( n X^{n-1} = F'(X)G(X) + F(X)G(X) \), on tire \( n z^{n-1} = F'(z)G(z) \) ; comme \( z \) est une unité de \( \Q[z] \) et est donc de norme \( \pm 1 \), on en déduit en prenant les normes que \( \Norm(F'(z)) \) divise \( n^d \). Enfin, comme \( z \) est un entier de \( \Q[z] \), le discriminant absolu de \( \Q[z] \) divise \( \D(1, z, \cdots, z^{d-1}) \) et donc \( n^d \). Ainsi, d'après \cref{thm:condition d'être ramifié}, aucun nombre premier \( p \) qui ne divise pas \( n \) ne se ramifie dans \( \Q[z] \). Ceci démontre~\ref{thm:corps cyclotomique;item1}.

        Pour \ref{thm:corps cyclotomique;item2} rappelons (\cref{ex:extensions cyclotomiques}) que \( \Q[z] \) est extension abélienne de \( \Q \) et qu'on a un homomorphisme injectif \( j \) du groupe de Galois \( G \) de \( \Q[z] \) sur \( \Q \) dans \( (\Z/n\Z)^\times \) ; plus précisément l'élément \( \sigma \in G \) élève toutes les racines \( n \)‑ièmes de l'unité à la puissance \( j(\sigma) \). Soit alors \( p \) un nombre premier qui ne divise pas \( n \) ; par~\ref{thm:corps cyclotomique;item1}, l'automorphisme de Frobenius \( \legendre{\Q[z]/\Q}{p} \) est défini ; notons le \( \sigma_p \). En notant \( A \) l'anneau des entiers de \( \Q[z] \) et \( \ideal{p} \) un facteur premier quelconque de \( Ap \), on a par définition \( \sigma_p \equiv x^p \pmod*{\ideal{p}} \) pour tout \( x \in A \). En particulier, en posant \( j = j(\sigma_p) \), on a \( z^j \equiv z^p \pmod*{\ideal{p}} \). Or on a
        \[
            \prod_{\substack{0 \leqslant r \leqslant n-1 \\ r \,\not\equiv\, p \pmod*{n}}} (z^p-z^r) = P'(z^p) = n z^{p(n-1)},
        \]
        où \( P(X) = X^n-1 = \prod_{0 \leqslant r \leqslant n-1} (X-z^r) \) ; comme \( n \) est premier à \( p \), que \( \ideal{p} \cap \Z = p\Z \)\,, et que \( z \) est inversible, on en déduit qu'on a
        \[
            \prod_{\substack{0 \leqslant r \leqslant n-1 \\ r \,\not\equiv\, p \pmod*{n}}} (z^p-z^r) \notin \ideal{p}.
        \]
        La relation \( z^j \equiv z^p \pmod*{\ideal{p}} \) implique donc que \( j \) est la classe de \( p \) modulo \( n \). Ainsi \( j(G) \) contient les classes modulo \( n \) de tous les nombres premiers \( p \) qui ne divisent pas \( n \), et donc, par multiplicativité, les classes de tous les entier premiers à \( n \) ; autrement dit \( j(G) = (\Z/n\Z)^\times \) et ceci démontre~\ref{thm:corps cyclotomique;item2}.
    \end{proof}

\section{Nouvelle démonstration de la loi de réciprocité quadratique}

    Soient \( q \) un nombre premier \emph{impair}, et \( K \) le corps cyclotomique engendré par une racine primitive \( q \)‑ième de l'unité dans \( \C \). Le groupe de Galois \( G \) de \( K \) sur \( \Q \) est isomorphe à \( \F_q^\times \) (\cref{thm:corps cyclotomique}, \ref{thm:corps cyclotomique;item2}), donc est \emph{cyclique} d'ordre pair \( q-1 \). Il admet donc un sous-groupe \( H \) d'indice \( 2 \) et un seul, qui correspond au sous-groupe des carrés \( (\F_q^\times)^2 \). Ainsi \( K \) contient un sous-corps \emph{quadratique} \( F \) et un seul (\cref{thm:correspondance de Galois}, \ref{thm:correspondance de Galois;item2}). Aucun nombre premier \( p \neq q \) ne se ramifie dans \( F \), car, sinon, il se ramifierait dans \( K \), contrairement \cref[à]{thm:corps cyclotomique}, \ref{thm:corps cyclotomique;item1}. Le calcul du discriminant d'un corps quadratique (l'exemple du \cref{sec:discriminant et ramification}) montre qu'on a nécessairement \( F = \Q[\sqrt{q}] \) si \( q \equiv 1 \pmod*{4} \), et \( F = \Q[\sqrt{-q}] \) si \( q \equiv 3 \pmod*{4} \) ; en posant \( q^* = (-1)^{\smash{\frac{q-1}{2}}} q \), on a en tous cas \( F = \Q[\sqrt{q^*}] \).

    Soit \( p \) un nombre premier distinct de \( q \). Notons \( \sigma_p \) l'automorphisme de Frobenius \( \legendre{K/\Q}{p} \) (\cf.~\cref{sec:application aux corps cyclotomiques}). Sa restriction à \( F \) est \( \legendre{F/\Q}{p} \) (\cref{prop:Frobenius et corps intermédiaire}, \ref{prop:Frobenius et corps intermédiaire;item2}) ; c'est l'identité si \( \sigma_p \in H \), c'est-à-dire si l'exposant \( j(\sigma_p) = \) classe de \( p \pmod*{q} \) (\cf.~\cref{sec:application aux corps cyclotomiques}) est un \emph{carré} dans \( \F_q^\times \) ; c'est l'automorphisme non identique dans le cas contraire. Autrement dit, en identifiant le groupe de Galois \( G/H \) de \( F \) sur \( \Q \) à \( \set{+1,-1} \), on a
    \begin{equation}\label{eq:quadratique reciprocité;nouvelle démonstration;aux1}
        \legendre{F/\Q}{p} = \legendre{p}{q}
    \end{equation}
    par définition du symbole de Legendre \( \legendre{p}{q} \) (\cref{sec:réciprocité quadratique}).

    D'autre part les résultats sur la décomposition du nombre premier \( p \) dans \( F = \Q[\sqrt{q^*}] \) (\cref{sec:décomposition d'un nombre premier dans un corps quadratique}) donnent d'autres renseignements sur \( \legendre{F/\Q}{p} \). Par définition c'est l'identité si \( p \) est décomposé dans \( F \), et l'automorphisme non identique si \( p \) est inerte. D'après \cref{prop:décomposition d'un nombre premier dans un corps quadratique}, on a donc, si \( p \) est \emph{impair},
    \begin{equation}\label{eq:quadratique reciprocité;nouvelle démonstration;aux2}
        \legendre{F/\Q}{p} = \legendre{q^*}{p}.
    \end{equation}
    En comparant \eqref{eq:quadratique reciprocité;nouvelle démonstration;aux1} et \eqref{eq:quadratique reciprocité;nouvelle démonstration;aux2}, on obtient \( \legendre{p}{q} = \legendre{q^*}{p} = \legendre{-1}{p}^{\frac{q-1}{2}} \legendre{q}{p} \) ; or \( \legendre{-1}{p} = (-1)^{\frac{p-1}{2}} \) par le très élémentaire critère d'Euler (\cref{prop:critère d'Euler}). D'où
    \[
        \legendre{p}{q} = (-1)^{\frac{(p-1)(q-1)}{4}} \legendre{q}{p}
    \]
    et on retrouve la loi de réciprocité quadratique (\cref{thm:loi de réciprocité quadratique}).

    Pour \( p = 2 \), rappelons que \( 2 \) est décomposé dans \( F \) si \( q^* \equiv 1 \pmod*{8} \) et est inerte si \( q^* \equiv 5 \pmod*{8} \). On a donc
    \begin{equation}\label{eq:quadratique reciprocité;nouvelle démonstration;aux3}
        \legendre{F/\Q}{2} = (-1)^{\frac{q^2-1}{8}}.
    \end{equation}
    En comparant \eqref{eq:quadratique reciprocité;nouvelle démonstration;aux1} et \eqref{eq:quadratique reciprocité;nouvelle démonstration;aux3}, on obtient
    \[
        \legendre{2}{q} = (-1)^{\frac{q^2-1}{8}},
    \]
    ce qui est la \textquote{formule complémentaire} difficile (\cref{prop:réciprocité quadratique;formules complémentaires}, \ref{prop:réciprocité quadratique;formules complémentaires;item2}).


\counterwithout{equation}{section}

\cleardoublepage
\phantomsection
\addcontentsline{toc}{chapter}{Compléments sans démonstrations}
\chapter*{Compléments sans démonstrations}

Nous donnons ici, sans démonstration, quelques compléments à ce qui a été fait dans le texte. Il s'agit de questions très liées à celles traitées dans le texte, et dont le degré de profondeur et de difficulté est tout à fait analogue ; elles n'ont pas été incluses afin de garder à ce livre une taille raisonnable, et aussi parce qu'on les trouve traitées dans d'autres ouvrages (voir, par exemple, le chap.\,V de \cite{zariski1958}, directement lisible après ce livre).

L'auteur a reculé devant la tâche consistant à donner, sans démonstration, une description du développement ultérieur de la théorie des nombres (adèles, corps de classes, fonctions zêta et série \( L \), arithmétiques des algèbres simples, théorie analytique, formes quadratiques, \etc.). Il renvoie par cela aux \textquote{Lectures supplémentaires} après la Bibliographie.

\section*{Formules de transitivité}

    Étant donnés trois corps emboîtés, \( K \subset L \subset M \), chacun extension de \emph{degré fini} du précédent, on a les applications \textquote{trace} :
    \[
        \Tr_{L/K} \colon L \to K, \quad
        \Tr_{M/L} \colon M \to L, \quad
        \Tr_{M/K} \colon M \to K,
    \]
    et les applications \textquote{normes} analogues (\cref{sec:normes et traces}). Alors, pour \( x \in M \), on a
    \begin{equation}
        \begin{cases}
            \, \Tr_{M/K}(x) = \Tr_{L/K}(\Tr_{M/L}(x))\\
            \, \Norm_{M/K}(x) = \Norm_{L/K}(\Norm_{M/L}(x))
        \end{cases}
        \!\!\!\!\!.
    \end{equation}

\section*{Norme relative d'un idéal}

    Étant donnés deux corps de nombres emboîtés, \( K \subset K' \), et un idéal (entier ou fractionnaire) \( \ideal{a}' \) de \( K' \), l'idéal de \( K \) engendré par les \( \Norm_{K'/K}(x) \) (\( x \in \ideal{a}' \)) s'appelle la \emph{norme relative} de \( \ideal{a}' \), et se note \( \Norm_{K'/K}(\ideal{a}') \) ou \( \Norm(\ideal{a}') \). Si \( \ideal{a}' \) est un idéal principal \( (a') \) on~a
    \begin{equation}
        \Norm_{K'/K}((a')) = (\Norm_{K'/K}(a')).
    \end{equation}
    Si \( K = \Q \) on retrouve la notion exposée au \cref{sec:norme d'un idéal} : si \( \ideal{a}' \) est un idéal entier de \( K' \), et si \( A' \) est l'anneau des entiers de \( K' \), on~a
    \begin{equation}
        \Norm_{K'/\Q}(\ideal{a}') = \card(A'/\ideal{a}') \Z.
    \end{equation}
    Revenons au cas général ; si \( \ideal{a} \) est un idéal de \( K \), et si \( n = [K':K] \) on~a
    \begin{equation}
        \Norm_{K'/K}(A'\ideal{a}) = \ideal{a}^n.
    \end{equation}
    Si \( \ideal{a}' \) et \( \ideal{b}' \) sont des idéaux de \( K' \), on a la formule de multiplicativité :
    \begin{equation}
        \Norm_{K'/K}(\ideal{a}'\ideal{b}') = \Norm_{K'/K}(\ideal{a}') \Norm_{K'/K}(\ideal{b}').
    \end{equation}
    Enfin, si \( \ideal{p}' \) est un idéal premier de \( K' \), si \( \ideal{p} = \ideal{p}' \cap K \), et si \( f \) est le degré résiduel de \( \ideal{p}' \) sur \( K \), on~a
    \begin{equation}
        \Norm_{K'/K}(\ideal{p}') = \ideal{p}^f.
    \end{equation}

    Avec trois corps de nombres emboîtés \( K \subset K' \subset K'' \), on a la formule de transitivité suivante, où \( \ideal{a}'' \) désigne un idéal de~\( K'' \) :
    \begin{equation}
        \Norm_{K''/K}(\ideal{a}'') = \Norm_{K'/K}(\Norm_{K''/K'}(\ideal{a}'')).
    \end{equation}
    Dans la même situation, la notion de norme relative d'un idéal permet de donner une formule de transitivité des discriminants (où \( \ideal{D}_{K'/K} \) désigne le discriminant de \( K' \) sur \( K \) ; \cf.~\cref{def:idéal discriminant}) :
    \begin{equation}
        \ideal{D}_{K''/K} = \Norm_{K'/K}(\ideal{D}_{K''/K'}) \cdot \ideal{D}_{K'/K}^{[K'':K']}.
    \end{equation}
    Tout ceci se généralise à un anneau de Dedekind \( A \) et à sa fermeture intégrale \( A' \) dans une extension de degré du corps des fractions de~\( A \).

\section*{La différente}

    Ce qui suit s'applique à un anneau de Dedekind \( A \) et à la fermeture intégrale de \( A \) dans une extension de degré fini de son corps des fractions. Pour alléger, nous nous bornerons au cas des corps de nombres.

    Soient \( K \subset K' \) deux corps de nombres emboîtés, \( A \) et \( A' \) leurs anneaux d'entiers. On dit qu'un idéal maximal \( \ideal{p}' \) de \( A' \) est \emph{ramifié} sur \( A \) (ou sur \( K \)) si son indice de ramification sur \( A \) est \( > 1 \). Alors l'idéal maximal \( \ideal{p} = \ideal{p}' \cap A \) de \( A \) \emph{se ramifie} dans \( A' \) (\cref{sec:discriminant et ramification}). Il résulte aisément du \cref{thm:condition d'être ramifié} que les idéaux maximaux de \( A' \) qui sont ramifiés sur \( A \) sont \emph{en nombre fini}. Nous allons caractériser un idéal \( \ideal{d}_{K'/K} \) de \( A' \), la \textquote{\emph{différente}} de \( K' \) sur \( K \) tel que ces idéaux maximaux soient exactement ceux qui contiennent \( \ideal{d}_{K'/K} \) (noter l'analogie avec \cref{thm:condition d'être ramifié}).

    On démontre d'abord que l'ensemble des \( x \in K' \) tels que
    \begin{equation}
        \Tr_{K'/K}(xA') \subset A
    \end{equation}
    est un idéal fractionnaire \( \ideal{C} \) de \( A' \) ; on l'appelle la \emph{codifférente} de \( K' \) sur \( K \) ; par définition la \emph{différente} \( \ideal{d}_{K'/K} \) est l'idéal inverse \( \ideal{C}^{-1} \). C'est un idéal \emph{entier} non nul de \( A' \). On démontre qu'il est engendré par les \( F'(x) \), où \( x \) parcourt \( A' \) et où \( F \) désigne le polynôme minimal de \( x \) sur \( K \). En particulier, si \( A' \) est de la forme \( A[y] \) (ce qui n'est pas toujours le cas), et si \( G \) est le polynôme minimal de \( y \) sur \( K \), alors la différente \( \ideal{d}_{K'/K} \) est l'idéal principal de \( A' \) engendré par \( G'(y) \).

    Les idéaux premiers non nuls de \( A' \) qui sont ramifiés sur \( A \) sont ceux qui contiennent \( \ideal{d}_{K'/K} \). Plus précisément soit
    \begin{equation}
        \ideal{d}_{K'/K} = \prod_i \ideal{p_i}'\,{}^{m_i} \quad \text{(\( m_i > 0 \))}
    \end{equation}
    la décomposition de la différente en idéaux premiers, et soit \( e_i \) l'indice de ramification de \( \ideal{p_i}' \) sur \( A \). Alors les idéaux premiers non nuls de \( A' \) qui sont ramifiés sur \( A \) sont les \( \ideal{p_i}' \)\,, et on a \( m_i \geqslant e_i - 1 \) pour tout \( i \). De plus on a \( m_i = e_i - 1 \) si et seulement si \( e_i \) est premier à la caractéristique du corps résiduel \( A'/\ideal{p_i}' \).

    La différente \( \ideal{d}_{K'/K} \) (idéal de \( A' \)) et le discriminant \( \ideal{D}_{K'/K} \) (idéal de \( A \)) sont liés par la relation suivante :
    \begin{equation}
        \ideal{D}_{K'/K} = \Norm_{K'/K}(\ideal{d}_{K'/K})
    \end{equation}
    (\cf.~la formule \eqref{eq:disc;cas particulier}). Ainsi la donné de la différente est plus précise que celle du discriminant.

    Enfin, étant donnés trois corps de nombres emboîtés \( K \subset K' \subset K'' \), on a la formule de transitivité suivante pour les différente :
    \begin{equation}
        \ideal{d}_{K''/K} = \ideal{d}_{K''/K'} \cdot \ideal{d}_{K'/K}.
    \end{equation}


\backmatter

\phantomsection
\addcontentsline{toc}{chapter}{\bibname}
\begin{thebibliography}{}
    \thispagestyle{plain}
    \bibitem{bourbaki1} \textsc{N. Bourbaki}. \textit{Algèbre} (Paris, Hermann). Surtout chap.\,V en ce qui concerne les corps, chap.\,VI en ce qui concerne la divisibilité, et chap.\,VII en ce qui concerne les modules sur les anneaux principaux.
    \bibitem{bourbaki2} \textsc{N. Bourbaki}. \textit{Algèbre commutative} (\ibid). Surtout chap.\,V en ce qui concerne les éléments entiers, et chap.\,VII en ce qui concerne les anneaux de Dedekind et factoriels. Théorie très complète et générale des anneaux de fractions au chap.\,II. Bon exposé de la théorie des valuations au chap.\,VI.
    \bibitem{cartan1962} \textsc{H. Cartan}. \textit{Théorie élémentaire des fonctions analytiques} (Paris, Hermann, 1962).
    \bibitem{choquet1963} \textsc{G. Choquet}. \textit{Cours d'analyse} (Paris, Masson, 1963).
    \bibitem{lang1962} \textsc{S. Lang}. \textit{On quasi-algebraic closure} (Ann. of Math., 55 (1962), 373-390).
    \bibitem{samuel1966} \textsc{P. Samuel}. \textit{À propos du théorème des unités} (Bull. Sci. Math., 90 (1966), 89-96) .
    \bibitem{samuel1964} \textsc{P. Samuel}. \textit{Anneaux factoriels} (Publ. Soc. Mat. São Paulo, 1964).
    \bibitem{terjanian1966} \textsc{G. Terjanian}, \textit{Sur une conjecture de M. Artin} (C. R. Acad. Sci. Paris, 1966).
    \bibitem{wiles1994} \textsc{A. Wiles}, \textit{Modular elliptic curves and Fermat's last theorem} (Ann. of Math., 141 (1995), no.\,3, 443-551).
    \bibitem{zariski1958} \textsc{O. Zariski} et \textsc{P. Samuel}. \textit{Commutative Algebra}, Vol. I (Van Nostrand, Princeton, 1958). Chap.\,II sur les corps, chap.\,IV sur les anneaux nœthériens, chap.\,V sur les éléments entiers et les anneaux de Dedekind.
\end{thebibliography}


\chapter{Lectures supplémentaires}

\begin{hangparas}{1.2em}{1}
    \textsc{E. Artin}. \textit{Theory of algebraic numbers} (G. Striker, Schildweg 12, Göttingen, Allemagne-1957) (Donne le pas à la théorie des valuations ; très élégant ; nombreux exemples).

    \medskip
    \textsc{H. Hasse}. \textit{Zahlentheorie} (Akademie Verlag, Berlin, 1949) (massif et très complet).

    \medskip
    \textsc{H. Hasse}. \textit{Vorlesungen über Zahlen theorie} (Springer, 1964) (décrit de multiples aspects de la théorie des nombres).

    \medskip
    \textsc{G. H. Hardy} et \textsc{E. M. Wright}. \textit{An introduction to the theory of numbers} (Clarendon Press-Oxford, 1965) (profond et attrayant; remarquable sens esthétique dans le choix des sujets).

    \medskip
    \textsc{E. Hecke}. \textit{Vorlesungen über die Theorie der algebraischen Zahlen} (Chelsea, New York, 1948) (un classique, très efficace et complet).

    \medskip
    \textsc{S. Lang}. \textit{Algebraic Numbers} (Addison-Wesley, 1964) (petit livre très dense et concentré).

    \medskip
    \textsc{S. Lang}. \textit{Diophantine Geometry} (Interscience Tract n\textsuperscript{o}\,11, J. Wiley, New York, 1962) (orienté vers les équations diophantiennes; expose très nettement leur lien avec la Géométrie algébrique).

    \medskip
    \textsc{O'Meara}. \textit{Introduction to quadratic forms} (Springer, 1963) (exposé très efficace de la théorie des nombres algébriques, suivi d'une de ses plus belles applications et motivations).

    \medskip
    \textsc{J. P. Serre}. \textit{Corps locaux} (Hermann, Paris, 1962) (l'accent porte ici sur les corps \( p \)‑adiques ; exposé très clair et lucide des méthodes algébriques les plus récentes de la Théorie des nombres ; très riche de contenu ; nombreux exemples).

    \medskip
    \textsc{E. Artin} et \textsc{J. Tate}. \textit{Class-field theory} (Math. Dept. Harvard University) (l'exposé Ie plus mod erne de la fameuse théorie du \textquote{corps de classes}, c'est-à-dire des extensions abéliennes des corps de nombres).

    \medskip
    \textsc{A. Weil}. \textit{Basic number theory} (Springer, 1967) (utilise la fructueuse méthode des adèles, et traite parallèlement corps de nombres et corps de fonctions).

    \medskip
    \textsc{Z. I. Borovic} et \textsc{I. R. Safarevic}. \textit{Théorie des nombres} (Gauthiers Villars, 1966) (très complet; excellents chapitres sur les méthodes analytiques, complexes et \( p \)‑adiques ; nombreuses tables numériques).
\end{hangparas}


\printindex


\end{document}
